\documentclass[11pt,b5paper,notitlepage]{article}
\usepackage[b5paper, margin={0.5in,0.65in}]{geometry}

\usepackage{amsmath,amscd,amssymb,amsthm,mathrsfs,amsfonts,layout,indentfirst,graphicx,caption,mathabx, stmaryrd,appendix,calc,imakeidx,upgreek,appendix} % mathabx for \widecheck
%\usepackage{ulem} %wave underline
\usepackage[dvipsnames]{xcolor}
\usepackage{palatino}  %template
\usepackage{slashed} % Dirac operator
\usepackage{mathrsfs} % Enable using \mathscr
%\usepackage{eufrak}  another template/font
\usepackage{extarrows} % long equal sign, \xlongequal{blablabla}
\usepackage{enumitem} % enumerate label change e.g. [label=(\alph*)]  shows (a) (b) 

\usepackage{fancyhdr} % date in footer
\usepackage{verbatim}
\usepackage{halloweenmath}
\usepackage{simpler-wick}





\usepackage{tikz-cd}
\usepackage[nottoc]{tocbibind}  
\title{Erratum to ``Convergence of Sewing Conformal Blocks"}
\author{Bin Gui}
\date{September 2024}

\begin{document}

\maketitle

In Thm 6.3, one should add the extra assumption that $\mathfrak X$ admits local coordinates $\eta_\bullet$. This condition ensures that $\mathscr W_{\mathfrak X}(\mathbb W_\bullet)\simeq \mathbb W_\bullet\otimes\mathscr O_{\mathcal B}$, and hence each stalk of $\mathscr W_{\mathfrak X}(\mathbb W_\bullet)$ is generated by the global sections of $\mathscr W_{\mathfrak X}(\mathbb W_\bullet)$.

\end{document}
