\documentclass[10pt]{amsart}
\usepackage{amsmath,amscd,amssymb,amsthm,mathrsfs,amsfonts,layout,indentfirst,graphicx,caption,mathabx,  tikz, stmaryrd,appendix,epstopdf}
\usepackage{palatino}  %template
\usepackage{slashed} % Dirac operator
\usepackage{mathrsfs} % Enable using \mathscr
%\usepackage{eufrak}  another template/font

\allowdisplaybreaks  %allow aligns to break between pages
\usepackage{latexsym}
\usepackage{chngcntr}
\usepackage[colorlinks,linkcolor=blue,anchorcolor=blue,citecolor=red,linktocpage]{hyperref}

%\setcounter{tocdepth}{1}	 hide subsections in the content

\usepackage{fullpage}
\counterwithin{figure}{section}

\pagestyle{plain}

\captionsetup[figure]
{
	labelsep=none	
}


\theoremstyle{definition}
\newtheorem{df}{Definition}[section]
\newtheorem{eg}[df]{Example}
\newtheorem{conj}[df]{Conjecture}
\newtheorem{rem}[df]{Remark}
\newtheorem{ass}[df]{Assumption}
\newtheorem{cv}[df]{Convention}
\newtheorem{st}{Step}
\newtheorem{iss}{Issue}[section]
\theoremstyle{plain}
\newtheorem{thm}[df]{Theorem}
\newtheorem{pp}[df]{Proposition}
\newtheorem{co}[df]{Corollary}
\newtheorem{lm}[df]{Lemma}
\newtheorem{cond}{Condition}
\renewcommand{\thecond}{\Alph{cond}} % "letter-numbered" theorems



%\substack   multiple lines under sum
%\underset{b}{a}   b is under a


% Remind: \overline{L_0}


\newcommand{\tr}{\mathrm{t}} %transpose
\newcommand{\End}{\mathrm{End}} %endomorphism
\newcommand{\id}{\mathrm{id}}
\newcommand{\Hom}{\mathrm{Hom}}
\newcommand{\Conf}{\mathrm{Conf}}
\newcommand{\Res}{\mathrm{Res}}
\newcommand{\KZ}{\mathrm{KZ}}
\newcommand{\ev}{\mathrm{ev}}
\newcommand{\coev}{\mathrm{coev}}
\newcommand{\opp}{\mathrm{opp}}
\newcommand{\Rep}{\mathrm{Rep}}
\newcommand{\diag}{\mathrm{diag}}
\newcommand{\loc}{\mathrm{loc}}
\newcommand{\contr}{\mathrm{c}}
\newcommand{\uni}{\mathrm{u}}
\newcommand{\di}{\slashed d}
\newcommand{\Diff}{\mathrm{Diff}}
\newcommand{\PSU}{\mathrm{PSU}}
\newcommand{\Vir}{\mathrm{Vir}}


	\newcommand{\makeheading}[1]%
	{\hspace*{-\marginparsep minus \marginparwidth}%
		\begin{minipage}[t]{\textwidth}%
			{\large \bfseries #1}\\[-0.15\baselineskip]%
			\rule{\columnwidth}{1pt}%
		\end{minipage}}
\begin{document}\sloppy
\makeheading{Bin Gui \hfill Research Statement}
%%%%%%%%%%%%%%%%%%%%%%%%%%%%%%%%%%%%%%%%%%%%%%%%%
\vspace{5mm}

My research involves relationships between vertex operator algebras (VOAs)  and conformal nets, both of which are considered to be mathematical axiomatisations of  2d conformal field theories (CFTs). In both areas, many important and profound results have been achieved over the past 30 years, and deep connections with different branches of mathematics have also been discovered, namely number theory and modular forms, complex and algebraic geometry, low dimensional topology, 3d topological quantum field theory (3d TQFT), operator algebras, quantum groups, and infinite dimensional Lie algebras. Despite having the same physics backgrounds,  VOAs and conformal nets are studied  in rather different ways, the former being a branch of algebra, and the latter being rooted in the theory of functional analysis. 

My research fits into the  program of showing that unitary VOAs and conformal nets have equivalent representation tensor categories. This, in particular, implies that these two theories produce the same 3d TQFT and knot invariants. Based on the unitarity results proved in \cite{G unitary1,G unitary2}, I obtained, in \cite{G equivalence}, a general theorem on the equivalence of the modular tensor categories obtained from the two theories. This theorem is known to hold for all unitary minimal models, affine $\mathfrak {sl}_n$, affine $G_2$ (\cite{G G2}), and the tensor subcategory of  single valued representations of affine $\mathfrak {so}_{2n}$. Hopefully it can be applied to many more examples.



\section{Background}

Modular tensor categories (MTCs) are  semisimple tensor categories with  extra structures and constraints  (non-degenerate braidings,  twists, objects having left and right duals). They first arose in the study of rational CFT by Moore and Seiberg in \cite{MS poly} and \cite{MS CQ},  although the precise definition  was given by Turaev. 

MTCs lie at the center of quantum symmetries and quantum topology. In \cite{T MC} and \cite{T book}, Turaev showed, based on his work with Reshetikhin on conctruction of 3d TQFTs from quantum group representations (\cite{RT}), that one can obtain 3d TQFTs, invariants of 3-manifolds, and knot invariants from MTCs, and that certain representation categories of quantum groups at certain roots of unity provide examples of MTCs. These MTCs are in fact \emph{unitary} (see \cite{Wenzl unitary} or \cite{Xu unitary}), which partly means that the structural maps are unitary. Note that one major motivation of  Reshetikhin-Turaev construction is to derive a mathematical counterpart of Chern-Simons theory, a 3d TQFT studied by Witten in \cite{Witten Jones} and closely related to knot invariants, especially Jones polynomials. Jones' discovery of his knot invariants  in \cite{Jones polynomials}, interestingly enough, is based on his work on subfactors (\cite{Jones subfactors}), a branch of functional analysis and operator algebras. This suggests that unitary modular tensor categories (UMTCs) are closely related to subfactors, and it is true! Starting from a  finite index, finite depth subfactor $N\subset M$, one can construct a unitary fusion category generated by the bi-module ${}_NL^2(M)_N$ or ${}_ML^2(M)\boxtimes_{N}L^2(M)_M$, where the tensor product operation $\boxtimes$ is defined by \emph{Connes fusions} (\cite{Connes fusion}). The Drinfeld center of this unitary fusion category is a UMTC. Conversely, one can also construct subfactors from UMTCs (or from unitary fusion categories).


As for the relation between CFT and MTC, Moore and Seiberg's construction of MTCs from rational CFTs in \cite{MS poly} and \cite{MS CQ} was based on the "duality property" of chiral correlation functions (conformal blocks). This property is obvious from path integral quantization, so their work is on the physics level of rigor. A mathematical realization of their construction can start from either VOAs or conformal nets. However, difficulties arise in both approaches.

The definition of VOAs formalize the notions of field operators in chiral CFT as well as their algebraic operations (the operator product expansions). In this approach,  Moore and Seiberg's duality property can  be stated in the precise language of VOAs. Therefore, it is usually  not hard to compute such quantities  as fusion rules, quantum dimensions, and  $S$ matrices, and show that they coincide with the results computed by physicists. However, finding a mathematically rigorous proof of duality property is very hard, and this constitutes the main difficulty of constructing MTCs from rational VOAs. This problem was unsolved until 2005, when Huang proved the general modular invariance property for genus $1$ conformal blocks (\cite{H modular}). 

In the theory of conformal nets, one studies  unitary chiral CFTs under the framework of algebraic quantum field theory (AQFT) developed by Haag, Kastler, Doplicher, Roberts, etc. Here, rather than studying field operators localized at points, one investigates bounded operators localized in regions, and the ``closed" associative *-algebras (the von Neumann algebras) generated by them. The construction of MTCs from conformal nets uses, instead of Moore and Seiberg's duality property, the Connes fusion of bimodules introduced in \cite{Connes fusion}. Therefore, this construction is very closely related to subfactors and their fusion categories. Unfortunately, it is often not easy to calculate the fusion rules and the S-matrices in the MTCs of conformal nets, let alone checking that they coincide  with the computations of physicists! 





\section {Toward the equivalence of MTCs}

One fundamental idea to compute the invariants of conformal nets and subfactors is to relate them with  unitary vertex operator algebras (UVOAs). If such a relation is established,   the knowledge of UVOAs can help us  extract the information of conformal nets. Hence one wants to prove the following:

\begin{conj}\label{conj1}
If $V$ is a unitary VOA, and $\mathcal M_V$ is the conformal net constructed from $V$, then the MTC $\Rep(V)$ associated to $V$ is equivalent to the MTC $\Rep(\mathcal M_V)$ associated to $\mathcal M_V$.
\end{conj}

In order to prove this conjecture, we have to overcome several issues:

\begin{iss}\label{iss1}
How do we construct the conformal net $\mathcal M_V$ from $V$?
\end{iss}

\begin{iss}\label{iss2}
How do we define the functor $\mathfrak F:\Rep(V)\rightarrow \Rep(\mathcal M_V)$?
\end{iss}


\begin{iss}\label{iss3}
By Huang's result (\cite{H modularity}), the representation category $\Rep(V)$ of $V$ is a MTC. Is $\Rep(\mathcal M_V)$ also a MTC?
\end{iss}
\begin{iss}[Unitarity]\label{iss4}
If issue \ref{iss3} is solved, then the MTC $\Rep(\mathcal M_V)$ is manifestly unitary. Is  $\Rep(V)$ also  a unitary MTC?
\end{iss}

\begin{iss}[Existence of functorial isomorphisms]\label{iss5}
How do we define, for any pair of $V$-modules $W_i,W_j$, a functorial isomorphism $\mathfrak F(W_i)\boxtimes\mathfrak F(W_j)\rightarrow \mathfrak F(W_i\boxtimes W_j)$, which preserves the structural maps of the two categories (e.g. the associative maps, the braid operators, etc.)?
\end{iss}

Issue \ref{iss1} was treated systematically in \cite{CKLW}. Given a UVOA $V$, we have a vertex operator $Y$, which associates to each vector $v\in V$ a holomorphic field operator $Y(v,z)$ defined on $z\in\mathbb C^\times$. For any proper open interval $I\subset S^1$ and any $f\in C^\infty_c(I)$, one can define the \emph{smeared field} localized in $I$ to be $Y(v,f)=\int_{-\pi}^{\pi} Y(v,e^{i\theta})f(e^{i\theta})\frac{d\theta}{2\pi}$.
When $Y$ satisfies certain nice analytic conditions (the energy bounds conditions), the smeared fields are preclosed operators. Then the von Neumann algebra $\mathcal M_V(I)$ is defined to be the one generated by the smeared field operators localized in $I$. For many examples, the net of von Neumann algebras $\mathcal M_V$ was checked in \cite{CKLW} to satisfy all the axioms of conformal net.

Issue \ref{iss2} is not a very serious problem: One can show, for many  $V$, that all the representations are unitarizable, and that one can integrate the (unitary) representations of $V$ to those of $\mathcal M_V$.  Issue \ref{iss3} can be covered by issue \ref{iss5}, for if one can construct such functorial isomorphisms, then $\Rep(V)$ and $\Rep(\mathcal M_V)$ are equivalent as pre-modular tensor categories. In particular, the non-degeneracy of the S-matrices in $\Rep(V)$ implies the same property of the S-matrices in $\Rep(\mathcal M_V)$, which shows that $\Rep(\mathcal M_V)$ is modular.







\section{Unitarity of $\Rep(V)$}

It appeares that issue \ref{iss4} can be covered by \ref{iss5}. One might argue that if issue \ref{iss5} were solved, then  $\Rep(V)$, being equivalent to the UMTC $\Rep(\mathcal M_V)$, must also be unitary. However, as I observed in \cite{G unitary1,G unitary2} and \cite{G equivalence}, it is impossible to construct the functorial isomorphism in \ref{iss5} without proving the unitarity of $\Rep(V)$ first. The reason is that the Connes fusions   are not defined algebraically but analytically:  $\mathfrak F(W_i)\boxtimes\mathfrak F(W_j)$, as a Hilbert space, carries an inner product. The only way to construct a desired functorial isomorphism  is to define an inner product (a unitary structure) on each $\mathfrak F(W_i\boxtimes W_j)$, and then construct a homomorphism $\mathfrak F(W_i)\boxtimes\mathfrak F(W_j)\rightarrow \mathfrak F(W_i\boxtimes W_j)$ which preserves the inner products. And since the $\mathcal M_V$-module $\mathfrak F(W_i\boxtimes W_j)$ is integrated from the $V$-module $W_i\boxtimes W_j$, it is necessary to define an inner product on $W_i\boxtimes W_j$, under which $W_i\boxtimes W_j$ becomes a unitary $V$-module. This inner product cannot be defined arbitrarily: it should make $\Rep(V)$ a unitary MTC. In particular, the associativity maps
\begin{align*}
A:W_i\boxtimes(W_j\boxtimes W_k)\rightarrow (W_i\boxtimes W_j)\boxtimes W_k
\end{align*}
and the braid isomorphisms
\begin{align*}
\sigma_{i,j}:W_i\boxtimes W_j\rightarrow W_j\boxtimes W_i
\end{align*}
have to be unitary maps if the inner products are defined in this way.


Note that one can prove the unitarity of $\Rep(V)$ when $V$ is an affine VOA (i.e., when $V$ is generated by some affine simple Lie algebra at non-negative level). In fact, due to the work of Kazhdan-Lustzig (\cite{KL1,KL2,KL3,KL4}) and Finkelberg (\cite{Fink}), $\Rep(V)$ is equivalent to certain representation  tensor category of a quantum group at certain root of unity. The latter was shown to be unitary independently by Wenzl (\cite{Wenzl unitary}) and Xu (\cite{Xu unitary}). So $\Rep(V)$ must be unitary in this case. However, even for affine VOAs, this result is almost useless for solving conjecture \ref{conj1}. This is because their work does not tell us how to define explicitly the inner product on $W_i\boxtimes W_j$, without which the functorial isomorphisms are impossible to construct.

In \cite{G unitary1,G unitary2}, I showed that when the representations and a generating set of intertwining operators of $V$ satisfy certain energy bounds, then one can define an inner product on $W_i\boxtimes W_j$, under which the MTC $\Rep(V)$ becomes unitary. I showed in \cite{G equivalence} that this inner product is the right one for us the construct the functorial isomorphisms. In fact, one can always define a desired sesquilinear form on $W_i\boxtimes W_j$ using certain fusion relations of intertwining operators. The hard part is to show that this sesquilinear form is positive definite, i.e., it is an inner product. As the non-degeneracy follows from the rigidity of $\Rep(V)$, one only needs to check the positivity of these sesquilinear forms. The proof of the positivity constitutes the main part of my work \cite{G unitary1,G unitary2}. My result can be applied to many important examples, including unitary minimal models, affine $\mathfrak{sl}_n$, affine $\mathfrak {so}_{2n}$, and affine $G_2$. For these examples, issue \ref{iss4} is solved.


\section{Equivalence of $\Rep(V)$ and $\Rep(\mathcal M_V)$}



To  prove the equivalence of MTCs, it remains to solve issue \ref{iss5}. In \cite{G equivalence}, I showed that if $V$ satisfies certain energy bounds conditions which are slightly stricter than those in \cite{G unitary1,G unitary2}, then we can define  functorial isomorphisms $\mathfrak F(W_i)\boxtimes\mathfrak F(W_j)\rightarrow \mathfrak F(W_i\boxtimes W_j)$ which intertwine the associativity maps and the braid operators. Hence for these $V$, $\Rep(V)$ and $\Rep(\mathcal M_V)$ are equivalent.

The main idea in \cite{G equivalence} is the ``braiding fusion duality": For both categories, the fusion (tensor) structure is closely related to the braid relations. Therefore, if we can relate the braid relations in $\Rep(V)$ and  $\Rep(\mathcal M_V)$, then the fusion structures of these two tensor categories can be identified. 

The idea of braiding fusion duality was first used by Wassermann in  \cite{Wass} to compute the fusion rules of the conformal net of affine $\mathfrak {sl}_n$. This idea was further carried out by his student Loke and Toledano-Laredo to treat the cases of unitary minimal models (\cite{Loke}) and affine $\mathfrak {so}_{2n}$ (\cite{Toledano}) respectively. However, computing the fusion rules in $\Rep(\mathcal M_V)$ (and showing that they are the same as those in $\Rep(V)$) is not enough for showing that the tensor categories $\Rep(\mathcal M_V)$ and $\Rep(V)$ are equivalent. And unfortunately, the work of Wassermann, Loke, and Toledano-Laredo cannot be used directly to prove this equivalence. This is due to the following reasons:

(1) If an intertwining operator $\mathcal Y$ of $V$ is energy-bounded (i.e., bounded by some $s$-th power of the energy operator, where $s\geq0$), then  the smeared intertwining operators $\mathcal Y(w,f)$ localized in some open proper interval $I\subset S^1$ is preclosed. Therefore we can define the intertwining operators of $\mathcal M_V$ localized in $I$ by taking the closure of $\mathcal Y(w,f)$ and cutting off  unbounded parts. However, for affine $\mathfrak {sl}_n$ and affine $\mathfrak {so}_{2n}$ (and  for most UVOAs), checking that \emph{all} the intertwining operators of $V$ are energy bounded is very hard. It is more practical to prove that a \emph{generating set} of intertwining operators are energy-bounded. If $\mathcal Y$ lies outside this generating set, the work of Wassermann, Loke, and Toledano-Laredo does not show us how to construct intertwining operators of $\mathcal M_V$ from $\mathcal Y$. 

(2) In their work, the braid relations for intertwining operators that are required to show the equivalence of the fusion structures are lacking.

Both issues were treated in my work \cite{G equivalence}:

(a) If $\mathcal Y$ is not inside in the generating set of energy bounded intertwining operators of $V$, then we replace $\mathcal Y(w,f)$ by a product of smeared intertwining operators, each of which is a member of this generating set. Therefore, this product of smeared intertwining operators is also preclosed, and satisfies nice analytic properties.   The hard part of this approach is to compute the braiding and the adjoint relations for products of (smeared) intertwining operators.  

(b) The  braid relations required for showing the equivalence of the fusion structures can only be obtained if we define the inner product on $W_i\boxtimes W_j$  as in  \cite{G unitary1,G unitary2}. 

The result of \cite{G equivalence} can be applied to the tensor categories of unitary minimal models, affine $\mathfrak {sl}_{n}$, affine $G_2$ (\cite{G G2}), and the tensor subcategory of the single-valued representations of affine $\mathfrak {so}_{2n}$. So for all these examples, the equivalence of the ribbon tensor categories between VOAs and the corresponding conformal nets are established. Except for affine $\mathfrak{sl}_{2}$, people didn't know how to prove this equivalence previous to my work. (The affine $\mathfrak {sl}_2$ case can be proved using the result of \cite{Wass}, simply because the quantum dimensions of the vector representations are small enough for the fusion rules and the $T$-matrices to determine all the other structures of the tensor categories. See \cite{Hen15} pp.6-7. This argument no longer works for arbitrary $\mathfrak {sl}_n$ and other Lie type VOAs.) We expect that my result can be applied  to more examples.



\section{Proposed research}

$\alpha.$ Complete the project of solving conjecture \ref{conj1}. For some UVOAs, this can be achieved by checking that  they satisfy the conditions in \cite{G equivalence}. But some other examples (e.g. affine $\mathfrak {so}_{2n}$) fail to satisfy all these conditions.  This explains why the result in \cite{G equivalence} can only be applied to a tensor subcategory of affine $\mathfrak {so}_{2n}$. 

$\beta$. The multi-interval problem. In $\Rep(\mathcal M_V)$, the Connes fusion of two $\mathcal M_V$-modules $\mathcal H_i,\mathcal H_j$ is defined by modding out the actions of the von Neumann algebra $\mathcal M_V(I)$ localized in a \emph{connected} open subset $I\subset S^1$. When $I$ is not connected, i.e., when $I$ is a disjoint union of several open intervals, we need to study the corresponding fusion category for $\mathcal M_V$, and find the counterpart for $V$.

$\gamma$. Higher genus correlation functions. This is related to the multi-interval problem. Moore and Seiberg's duality property was prove for genus $g=0,1$. A proof for higher genus duality property is still lacking, and the main difficulty lies in proving the convergence of sewed intertwining operators.

All these problems are related. One cannot fully understand conformal nets without  a comprehension of (unitary) VOAs, and vice versa.  I'm looking forward to working  with Yi-Zhi Huang and James Lepowsky, two experts in the field of vertex operator algebras and conformal field theory. 

 



\begin{thebibliography}{99}
\small
\bibitem[BPZ84]{BPZ}
Belavin, A.A., Polyakov, A.M. and Zamolodchikov, A.B., 1984. Infinite conformal symmetry in two-dimensional quantum field theory. Nuclear Physics B, 241(2), pp.333-380.

\bibitem[CKLW15]{CKLW}
Carpi, S., Kawahigashi, Y., Longo, R. and Weiner, M., 2015. From vertex operator algebras to conformal nets and back. arXiv preprint arXiv:1503.01260.



\bibitem[Con80]{Connes fusion}
Connes, A., 1980. On the spatial theory of von Neumann algebras. Journal of Functional Analysis, 35(2), pp.153-164.


\bibitem[Fin96]{Fink}
Finkelberg, M., 1996. An equivalence of fusion categories. Geometric And Functional Analysis, 6(2), pp.249-267.
	
	
\bibitem[Gui17a]{G unitary1}
Gui,  Unitarity of the modular tensor categories associated to unitary vertex operator algebras, I, arXiv:1711.02840.

\bibitem[Gui17b]{G unitary2}
Gui,  Unitarity of the modular tensor categories associated to unitary vertex operator algebras, II, arXiv:1712.04931.

\bibitem[Gui17c]{G G2}
Gui, Linear energy bounds for vector intertwining operators of affine $G_2$ vertex operator algebras, in preparation.

\bibitem[Gui17d] {G equivalence}
Gui, Equivalence of the modular tensor categories associated to unitary vertex operator algebras and conformal nets, in preparation.

\bibitem[Hen15]{Hen15}
Henriques, A., 2015. What Chern-Simons theory assigns to a point. arXiv preprint arXiv:1503.06254.

\bibitem[Hua05]{H modular}
Huang, Y.Z., 2005. Differential equations, duality and modular invariance. Communications in Contemporary Mathematics, 7(05), pp.649-706.

\bibitem[Hua08] {H modularity}
Huang, Y.Z., 2008. Rigidity and modularity of vertex tensor categories. Communications in contemporary mathematics, 10(supp01), pp.871-911.

\bibitem[Jon83] {Jones subfactors}
Jones, V.F., 1983. Index for subfactors. Inventiones mathematicae, 72(1), pp.1-25.

\bibitem[Jon85] {Jones polynomials}
Jones, V.F., 1985. A polynomial invariant for knots via von Neumann algebras. Bulletin of the American Mathematical Society, 12(1), pp.103-111.

\bibitem[KL93a]{KL1}
Kazhdan, D. and Lusztig, G., 1993. Tensor structures arising from affine Lie algebras. I. Journal of the American Mathematical Society, 6(4), pp.905-947.

\bibitem[KL93b] {KL2}
Kazhdan, D. and Lusztig, G., 1993. Tensor structures arising from affine Lie algebras. II. Journal of the American Mathematical Society, 6(4), pp.949-1011.

\bibitem[KL94a]{KL3}
Kazhdan, D. and Lusztig, G., 1994. Tensor structures arising from affine Lie algebras. III. Journal of the American Mathematical Society, 7(2), pp.335-381.

\bibitem[KL94b] {KL4}
Kazhdan, D. and Lusztig, G., 1994. Tensor structures arising from affine Lie algebras. IV. Journal of the American Mathematical Society, 7(2), pp.383-453.


\bibitem[KZ84]{KZ}
Knizhnik, V.G. and Zamolodchikov, A.B., 1984. Current algebra and Wess-Zumino model in two dimensions. Nuclear Physics B, 247(1), pp.83-103.

\bibitem[Loke94] {Loke}
Loke, T.M., 1994. Operator algebras and conformal field theory of the discrete series representations of Diff ($S^1$) (Doctoral dissertation, University of Cambridge).


\bibitem[MS88] {MS poly}
Moore, G. and Seiberg, N., 1988. Polynomial equations for rational conformal field theories. Physics Letters B, 212(4), pp.451-460.

\bibitem[MS89] {MS CQ}
Moore, G. and Seiberg, N., 1989. Classical and quantum conformal field theory. Communications in Mathematical Physics, 123(2), pp.177-254.

\bibitem[RT91] {RT}
Reshetikhin, N. and Turaev, V.G., 1991. Invariants of 3-manifolds via link polynomials and quantum groups. Inventiones mathematicae, 103(1), pp.547-597.

\bibitem[Tol04] {Toledano}
Toledano-Laredo, V., 2004. Fusion of positive energy representations of lspin (2n). arXiv preprint math/0409044.

\bibitem[Tur92] {T MC}
Turaev, V.G., 1992. Modular categories and 3-manifold invariants. International Journal of Modern Physics B, 6(11n12), pp.1807-1824.

\bibitem[Tur16] {T book}
Turaev, V.G., 2016. Quantum invariants of knots and 3-manifolds (Vol. 18). Walter de Gruyter GmbH \& Co KG.


\bibitem[Was98]{Wass}
Wassermann, A., 1998. Operator algebras and conformal field theory III. Fusion of positive energy representations of LSU (N) using bounded operators. Inventiones mathematicae, 133(3), pp.467-538.

\bibitem[Wen98] {Wenzl unitary}
Wenzl, H., 1998. C* tensor categories from quantum groups. Journal of the American Mathematical Society, 11(2), pp.261-282.

\bibitem[Wit89] {Witten Jones}
Witten, E., 1989. Quantum field theory and the Jones polynomial. Communications in Mathematical Physics, 121(3), pp.351-399.

\bibitem[Xu98]{Xu unitary}
Xu, F., 1998. Standard $\lambda$-lattices from quantum groups. Inventiones mathematicae, 134(3), pp.455-487.
	
\end{thebibliography}

\end{document}
