% !TeX spellcheck = en_US
% !TEX program = pdflatex
\documentclass[12pt,b5paper,notitlepage]{article}
\usepackage[b5paper, margin={0.5in,0.65in}]{geometry}
%\usepackage{fullpage}
\usepackage{amsmath,amscd,amssymb,amsthm,mathrsfs,amsfonts,layout,indentfirst,graphicx,caption,mathabx, stmaryrd,appendix,calc,imakeidx,upgreek} % mathabx for \wtidecheck
%\usepackage{ulem} %wave underline
\usepackage[dvipsnames]{xcolor}
\usepackage{palatino}  %template

\usepackage{slashed} % Dirac operator
\usepackage{mathrsfs} % Enable using \mathscr
%\usepackage{eufrak}  another template/font
\usepackage{extarrows} % long equal sign, \xlongequal{blablabla}
\usepackage{enumitem} % enumerate label change e.g. [label=(\alph*)]  shows (a) (b) 


%%%%%%%%%%%%%%%%%%%%%%%%%%%%%%

%\usepackage{fontspec}
%\setmainfont{Palatino Linotype}
%\usepackage{emoji}


% emoji, use lualatex  remove \usepackage{palatino}

%%%%%%%%%%%%%


\usepackage{CJK}   % Chinese package





\usepackage{csquotes} % \begin{displayquote}   \begin{displaycquote}  for quotation
\usepackage{epigraph}   %\epigraph{}{}  for quotation
%\pmb  mandatory math bold 

\usepackage{fancyhdr} % date in footer

%\usepackage{soul}  %\ul underline break line automatically

\usepackage{ulem}  % \uline  underline break line   also    \uwave

\usepackage{relsize} % use \mathlarger \larger \text{\larger[2]$...$} to enlarge the size of math symbols

\usepackage{verbatim}  % comment environment


\usepackage{halloweenmath} % Interesting halloween math symbols

%%%%%%%%%%%%%%%%%%%%%%%%%%%%%%
\usepackage{tcolorbox}
\tcbuselibrary{theorems}
% box around equations   \tcboxmath
%%%%%%%%%%%%%%%%%%%%%%%%%%%%%%%%%%





%%%%%%%%%%%%%%%%%%%%%%%%%%%%%
% circled colon and thick colon \hcolondel and \colondel

\usepackage{pdfrender}

\newcommand*{\hollowcolon}{%
	\textpdfrender{
		TextRenderingMode=Stroke,
		LineWidth=.1bp,
	}{:}%
}

\newcommand{\hcolondel}[1]{%
	\mathopen{\hollowcolon}#1\mathclose{\hollowcolon}%
}
\newcommand{\colondel}[1]{%
	\mathopen{:}#1\mathclose{:}%
}

%%%%%%%%%%%%%%%%%%%%%%%%%%%%%%%%






\usepackage{tikz}
\usetikzlibrary{fadings}
\usetikzlibrary{patterns}
\usetikzlibrary{shadows.blur}
\usetikzlibrary{shapes}

\usepackage{tikz-cd}
\usepackage[nottoc]{tocbibind}   % Add  reference to ToC


\makeindex


% The following set up the line spaces between items in \thebibliography
\usepackage{lipsum}  
\let\OLDthebibliography\thebibliography
\renewcommand\thebibliography[1]{
	\OLDthebibliography{#1}
	\setlength{\parskip}{0pt}
	\setlength{\itemsep}{2pt} 
}


%\hyperref{page.10}{...}

\allowdisplaybreaks  %allow aligns to break between pages
\usepackage{latexsym}
\usepackage{chngcntr}
\usepackage[colorlinks,linkcolor=blue,anchorcolor=blue, linktocpage,
%pagebackref
]{hyperref}
\hypersetup{ urlcolor=cyan,
	citecolor=[rgb]{0,0.5,0}}


\setcounter{tocdepth}{1}	 %hide subsections in the content


\counterwithin{figure}{section}

%\counterwithin*{footnote}{section}   %% Footnote numbering is recounted from the beginning of each subsection



\pagestyle{plain}

\captionsetup[figure]
{
	labelsep=none	
}













\theoremstyle{definition}
\newtheorem{df}{Definition}[section]
\newtheorem{eg}[df]{Example}
\newtheorem{exe}[df]{Exercise}
\newtheorem{rem}[df]{Remark}
\newtheorem{obs}[df]{Observation}
\newtheorem{ass}[df]{Assumption}
\newtheorem{cv}[df]{Convention}
\newtheorem{prin}[df]{Principle}
\newtheorem{nota}[df]{Notation}
\newtheorem*{axiom}{Axiom}
\newtheorem{coa}[df]{Theorem}
\newtheorem{srem}[df]{$\star$ Remark}
\newtheorem{seg}[df]{$\star$ Example}
\newtheorem{sexe}[df]{$\star$ Exercise}
\newtheorem{sdf}[df]{$\star$ Definition}
\newtheorem{concl}[df]{Conclusion}



\newtheorem{prob}{\color{red}Problem}[section]
%\renewcommand*{\theprob}{{\color{red}\arabic{section}.\arabic{prob}}}
\newtheorem{sprob}[prob]{\color{red}$\star$ Problem}
%\renewcommand*{\thesprob}{{\color{red}\arabic{section}.\arabic{sprob}}}
% \newtheorem{ssprob}[prob]{$\star\star$ Problem}



\theoremstyle{plain}
\newtheorem{thm}[df]{Theorem}
\newtheorem{ccl}[df]{Conclusion}
\newtheorem{thd}[df]{Theorem-Definition}
\newtheorem{pp}[df]{Proposition}
\newtheorem{co}[df]{Corollary}
\newtheorem{lm}[df]{Lemma}
\newtheorem{sthm}[df]{$\star$ Theorem}
\newtheorem{slm}[df]{$\star$ Lemma}
\newtheorem{claim}[df]{Claim}
\newtheorem{spp}[df]{$\star$ Proposition}
\newtheorem{scorollary}[df]{$\star$ Corollary}


\newtheorem{cond}{Condition}
\newtheorem{Mthm}{Main Theorem}
\renewcommand{\thecond}{\Alph{cond}} % "letter-numbered" theorems
\renewcommand{\theMthm}{\Alph{Mthm}} % "letter-numbered" theorems


%\substack   multiple lines under sum
%\underset{b}{a}   b is under a


% Remind: \overline{L_0}



\usepackage{calligra}
\DeclareMathOperator{\shom}{\mathscr{H}\text{\kern -3pt {\calligra\large om}}\,}
\DeclareMathOperator{\sext}{\mathscr{E}\text{\kern -3pt {\calligra\large xt}}\,}
\DeclareMathOperator{\Rel}{\mathscr{R}\text{\kern -3pt {\calligra\large el}~}\,}
\DeclareMathOperator{\sann}{\mathscr{A}\text{\kern -3pt {\calligra\large nn}}\,}
\DeclareMathOperator{\send}{\mathscr{E}\text{\kern -3pt {\calligra\large nd}}\,}
\DeclareMathOperator{\stor}{\mathscr{T}\text{\kern -3pt {\calligra\large or}}\,}


\usepackage{aurical}
\DeclareMathOperator{\VVir}{\text{\Fontlukas V}\text{\kern -0pt {\Fontlukas\large ir}}\,}

\newcommand{\vol}{\text{\Fontlukas V}}
\newcommand{\dvol}{d~\text{\Fontlukas V}}

\usepackage{aurical}
\usepackage[T1]{fontenc}








\newcommand{\fk}{\mathfrak}
\newcommand{\mc}{\mathcal}
\newcommand{\wtd}{\widetilde}
\newcommand{\wht}{\widehat}
\newcommand{\wch}{\widecheck}
\newcommand{\ovl}{\overline}
\newcommand{\udl}{\underline}
\newcommand{\tr}{\mathrm{t}} %transpose
\newcommand{\Tr}{\mathrm{Tr}}
\newcommand{\End}{\mathrm{End}} %endomorphism
\newcommand{\idt}{\mathbf{1}}
\newcommand{\id}{\mathrm{id}}
\newcommand{\Hom}{\mathrm{Hom}}
\newcommand{\Conf}{\mathrm{Conf}}
\newcommand{\Res}{\mathrm{Res}}
\newcommand{\res}{\mathrm{res}}
\newcommand{\KZ}{\mathrm{KZ}}
\newcommand{\ev}{\mathrm{ev}}
\newcommand{\coev}{\mathrm{coev}}
\newcommand{\opp}{\mathrm{opp}}
\newcommand{\Rep}{\mathrm{Rep}}
\newcommand{\diag}{\mathrm{diag}}
\newcommand{\Dom}{\scr{D}}
\newcommand{\loc}{\mathrm{loc}}
\newcommand{\con}{\mathrm{c}}
\newcommand{\uni}{\mathrm{u}}
\newcommand{\ssp}{\mathrm{ss}}
\newcommand{\di}{\slashed d}
\newcommand{\Diffp}{\mathrm{Diff}^+}
\newcommand{\Diff}{\mathrm{Diff}}
\newcommand{\PSU}{\mathrm{PSU}(1,1)}
\newcommand{\Vir}{\mathrm{Vir}}
\newcommand{\Witt}{\mathrm{Witt}}
\newcommand{\Span}{\mathrm{Span}}
\newcommand{\pri}{\mathrm{p}}
\newcommand{\ER}{E^1(V)_{\mathbb R}}
\newcommand{\prth}[1]{( {#1})}
\newcommand{\bk}[1]{\langle {#1}\rangle}
\newcommand{\bigbk}[1]{\big\langle {#1}\big\rangle}
\newcommand{\Bigbk}[1]{\Big\langle {#1}\Big\rangle}
\newcommand{\biggbk}[1]{\bigg\langle {#1}\bigg\rangle}
\newcommand{\Biggbk}[1]{\Bigg\langle {#1}\Bigg\rangle}
\newcommand{\GA}{\mathscr G_{\mathcal A}}
\newcommand{\vs}{\varsigma}
\newcommand{\Vect}{\mathrm{Vec}}
\newcommand{\Vectc}{\mathrm{Vec}_{\mathbb C}}
\newcommand{\scr}{\mathscr}
\newcommand{\sjs}{\subset\joinrel\subset}
\newcommand{\Jtd}{\widetilde{\mathcal J}}
\newcommand{\gk}{\mathfrak g}
\newcommand{\hk}{\mathfrak h}
\newcommand{\xk}{\mathfrak x}
\newcommand{\yk}{\mathfrak y}
\newcommand{\zk}{\mathfrak z}
\newcommand{\pk}{\mathfrak p}
\newcommand{\hr}{\mathfrak h_{\mathbb R}}
\newcommand{\Ad}{\mathrm{Ad}}
\newcommand{\DHR}{\mathrm{DHR}_{I_0}}
\newcommand{\Repi}{\mathrm{Rep}_{\wtd I_0}}
\newcommand{\im}{\mathbf{i}}
\newcommand{\Co}{\complement}
%\newcommand{\Cu}{\mathcal C^{\mathrm u}}
\newcommand{\RepV}{\mathrm{Rep}^\uni(V)}
\newcommand{\RepA}{\mathrm{Rep}(\mathcal A)}
\newcommand{\RepN}{\mathrm{Rep}(\mathcal N)}
\newcommand{\RepfA}{\mathrm{Rep}^{\mathrm f}(\mathcal A)}
\newcommand{\RepAU}{\mathrm{Rep}^\uni(A_U)}
\newcommand{\RepU}{\mathrm{Rep}^\uni(U)}
\newcommand{\RepL}{\mathrm{Rep}^{\mathrm{L}}}
\newcommand{\HomL}{\mathrm{Hom}^{\mathrm{L}}}
\newcommand{\EndL}{\mathrm{End}^{\mathrm{L}}}
\newcommand{\Bim}{\mathrm{Bim}}
\newcommand{\BimA}{\mathrm{Bim}^\uni(A)}
%\newcommand{\shom}{\scr Hom}
\newcommand{\divi}{\mathrm{div}}
\newcommand{\sgm}{\varsigma}
\newcommand{\SX}{{S_{\fk X}}}
\newcommand{\DX}{D_{\fk X}}
\newcommand{\mbb}{\mathbb}
\newcommand{\mbf}{\mathbf}
\newcommand{\bsb}{\boldsymbol}
\newcommand{\blt}{\bullet}
\newcommand{\Vbb}{\mathbb V}
\newcommand{\Ubb}{\mathbb U}
\newcommand{\Xbb}{\mathbb X}
\newcommand{\Kbb}{\mathbb K}
\newcommand{\Abb}{\mathbb A}
\newcommand{\Wbb}{\mathbb W}
\newcommand{\Mbb}{\mathbb M}
\newcommand{\Gbb}{\mathbb G}
\newcommand{\Cbb}{\mathbb C}
\newcommand{\Nbb}{\mathbb N}
\newcommand{\Zbb}{\mathbb Z}
\newcommand{\Qbb}{\mathbb Q}
\newcommand{\Pbb}{\mathbb P}
\newcommand{\Rbb}{\mathbb R}
\newcommand{\Ebb}{\mathbb E}
\newcommand{\Dbb}{\mathbb D}
\newcommand{\Hbb}{\mathbb H}
\newcommand{\cbf}{\mathbf c}
\newcommand{\Rbf}{\mathbf R}
\newcommand{\wt}{\mathrm{wt}}
\newcommand{\Lie}{\mathrm{Lie}}
\newcommand{\btl}{\blacktriangleleft}
\newcommand{\btr}{\blacktriangleright}
\newcommand{\svir}{\mathcal V\!\mathit{ir}}
\newcommand{\Ker}{\mathrm{Ker}}
\newcommand{\Cok}{\mathrm{Coker}}
\newcommand{\Sbf}{\mathbf{S}}
\newcommand{\low}{\mathrm{low}}
\newcommand{\Sp}{\mathrm{Sp}}
\newcommand{\Rng}{\mathrm{Rng}}
\newcommand{\vN}{\mathrm{vN}}
\newcommand{\Ebf}{\mathbf E}
\newcommand{\Nbf}{\mathbf N}
\newcommand{\Stb}{\mathrm {Stb}}
\newcommand{\SXb}{{S_{\fk X_b}}}
\newcommand{\pr}{\mathrm {pr}}
\newcommand{\SXtd}{S_{\wtd{\fk X}}}
\newcommand{\univ}{\mathrm {univ}}
\newcommand{\vbf}{\mathbf v}
\newcommand{\ubf}{\mathbf u}
\newcommand{\wbf}{\mathbf w}
\newcommand{\CB}{\mathrm{CB}}
\newcommand{\Perm}{\mathrm{Perm}}
\newcommand{\Orb}{\mathrm{Orb}}
\newcommand{\Lss}{{L_{0,\mathrm{s}}}}
\newcommand{\Lni}{{L_{0,\mathrm{n}}}}
\newcommand{\UPSU}{\widetilde{\mathrm{PSU}}(1,1)}
\newcommand{\Sbb}{{\mathbb S}}
\newcommand{\Gc}{\mathscr G_c}
\newcommand{\Obj}{\mathrm{Obj}}
\newcommand{\bpr}{{}^\backprime}
\newcommand{\fin}{\mathrm{fin}}
\newcommand{\Ann}{\mathrm{Ann}}
\newcommand{\Real}{\mathrm{Re}}
\newcommand{\Imag}{\mathrm{Im}}
\newcommand{\cl}{\mathrm{cl}}
\newcommand{\Ind}{\mathrm{Ind}}
\newcommand{\Supp}{\mathrm{Supp}}
\newcommand{\Specan}{\mathrm{Specan}}
\newcommand{\red}{\mathrm{red}}
\newcommand{\uph}{\upharpoonright}
\newcommand{\Mor}{\mathrm{Mor}}
\newcommand{\pre}{\mathrm{pre}}
\newcommand{\rank}{\mathrm{rank}}
\newcommand{\Jac}{\mathrm{Jac}}
\newcommand{\emb}{\mathrm{emb}}
\newcommand{\Sg}{\mathrm{Sg}}
\newcommand{\Nzd}{\mathrm{Nzd}}
\newcommand{\Owht}{\widehat{\scr O}}
\newcommand{\Ext}{\mathrm{Ext}}
\newcommand{\Tor}{\mathrm{Tor}}
\newcommand{\Com}{\mathrm{Com}}
\newcommand{\Mod}{\mathrm{Mod}}
\newcommand{\nk}{\mathfrak n}
\newcommand{\mk}{\mathfrak m}
\newcommand{\Ass}{\mathrm{Ass}}
\newcommand{\depth}{\mathrm{depth}}
\newcommand{\Gode}{\mathrm{Gode}}
\newcommand{\Fbb}{\mathbb F}
\newcommand{\sgn}{\mathrm{sgn}}
\newcommand{\Aut}{\mathrm{Aut}}
\newcommand{\Modf}{\mathrm{Mod}^{\mathrm f}}
\newcommand{\codim}{\mathrm{codim}}
\newcommand{\card}{\mathrm{card}}
\newcommand{\dps}{\displaystyle}
\newcommand{\Int}{\mathrm{Int}}
\newcommand{\Nbh}{\mathrm{Nbh}}
\newcommand{\Pnbh}{\mathrm{PNbh}}
\newcommand{\Cl}{\mathrm{Cl}}
\newcommand{\diam}{\mathrm{diam}}
\newcommand{\eps}{\varepsilon}
\newcommand{\Vol}{\mathrm{Vol}}
\newcommand{\LSC}{\mathrm{LSC}}
\newcommand{\USC}{\mathrm{USC}}
\newcommand{\Ess}{\mathrm{Rng}^{\mathrm{ess}}}
\newcommand{\Jbf}{\mathbf{J}}
\newcommand{\SL}{\mathrm{SL}}
\newcommand{\GL}{\mathrm{GL}}
\newcommand{\Lin}{\mathrm{Lin}}
\newcommand{\ALin}{\mathrm{ALin}}
\newcommand{\bwn}{\bigwedge\nolimits}
\newcommand{\nbf}{\mathbf n}
\newcommand{\dive}{\mathrm{div}}
\newcommand{\QC}{\mathrm{QCoh}_{\mathrm L}}
\newcommand{\Coh}{\mathrm{Coh}_{\mathrm L}}
\newcommand{\rad}{\mathrm{rad}}
\newcommand{\SLF}{\mathrm{SLF}}
\newcommand{\op}{\mathrm{op}}
\newcommand{\trc}{\mathrm{tr}}
\newcommand{\Mob}{\textrm{Möb}}
\newcommand{\Poid}{{\mathrm P}^+(1,d)}
\newcommand{\xbf}{\mathbf x}
\newcommand{\ybf}{\mathbf y}
\newcommand{\ebf}{\mathbf e}
\newcommand{\SO}{\mathrm{SO}}
\newcommand{\Rc}{\mathbb R^{1,1}_{\mathrm{c}}}
\newcommand{\Cf}{\mathrm{Cf}}
\newcommand{\DiffS}{\mathrm{Diff}^+(\mathbb S^1)}
\newcommand{\WDS}{\widetilde{\mathrm{Diff}^+}(\mathbb S^1)}
\newcommand{\PU}{\mathrm{PU}}
\newcommand{\HV}{\mathcal H_{\mathbb V}}
\newcommand{\ek}{\mathfrak{e}}
\newcommand{\ointn}{\oint\nolimits}
\newcommand{\sd}{\slashed{d}}
\newcommand{\ac}{\mathrm{ac}}
\newcommand{\mr}{\mathring}
\newcommand{\MV}{\mathcal V}
\newcommand{\half}{\frac{1}{2}}








\usepackage{tipa} % wierd symboles e.g. \textturnh
\newcommand{\tipar}{\text{\textrtailr}}
\newcommand{\tipaz}{\text{\textctyogh}}
\newcommand{\tipaomega}{\text{\textcloseomega}}
\newcommand{\tipae}{\text{\textrhookschwa}}
\newcommand{\tipaee}{\text{\textreve}}
\newcommand{\tipak}{\text{\texthtk}}
\newcommand{\mol}{\upmu}
\newcommand{\dmol}{d\upmu}




\usepackage{tipx}
\newcommand{\tipxgamma}{\text{\textfrtailgamma}}
\newcommand{\tipxcc}{\text{\textctstretchc}}
\newcommand{\tipxphi}{\text{\textqplig}}















\numberwithin{equation}{section}




\title{Topics in Operator Algebras: Algebraic Conformal Field Theory}
\author{\sc{ Bin Gui}
	%\\
	%{\small \sc Yau Mathematical Sciences Center, Tsinghua University.}\\
	%{\small binguimath@gmail.com\qquad bingui@tsinghua.edu.cn}
}
\date{}

%\definecolor{mycolor}{RGB}{227,237,205} \pagecolor{mycolor}

\begin{document}\sloppy % avoid stretch into margins
	\pagenumbering{arabic}
	%\pagenumbering{gobble}
	\setcounter{page}{1}
	\setcounter{section}{-1}
	%\setcounter{equation}{6}



	






	%%%%%%%%%%%%%%%%%%%%%%%%%%%%%%%%%%%%%%%%%%%%%%%%%%%%%%%%%



	
	\maketitle
%\small   \hyperlink{page.7}{Last page of TOC}


%\hyperlink{beforeindex}{Last page before index}~~~~~~  
%\hypertarget{beforeindex}{}



%\noindent Sections on history include but are not limited to: 
%\ref{lb55} (point-set topology),  \ref{lb550} (integral theory, Fourier series), \ref{lb543} (Banach-Alaoglu, Hahn-Banach),  \ref{lb548} (quotient Banach spaces, Hahn-Banach), \ref{lb671} and most part of Ch. \ref{lb672} (Hilbert spaces, integral equations), \ref{lb733} (measurable sets), \ref{mc89} (Riesz-Fischer theorem, $L^p$-$L^q$ duality), \ref{lb896} (functional calculus, spectral theory)
%\normalsize
%\thispagestyle{empty}	 %remove page number of this page


%Contents hyperlinks: \hyperlink{page.2}{Page 2}, \hyperlink{page.3}{Page 3}

%%%%%%%%%%%%%%%%%%%%%%%%%%%%%
%\vspace{-0.5cm}

%\makeatletter
%\newcommand*{\toccontents}{\@starttoc{toc}}
%\makeatother
%\toccontents



	
% title and table of contents same page, no content title

%%%%%%%%%%%%%%%%%%%%%%%%%%%%%

\normalsize

%\hyperlink{beforeindex}{Current page of writing}~~~~~~ 

\tableofcontents



\newpage


\section{Notations}

$\Nbb=\{0,1,2,\dots\}$. $\Zbb_+=\{1,2,3,\dots\}$. $\Cbb^\times=\Cbb\setminus\{0\}$. $\Dbb_r=\{z\in\Cbb:|z|<r\}$. $\ovl\Dbb_r=\{z\in\Cbb:|z|\leq r\}$. $\Dbb_r^\times=\{z\in\Cbb:0<|z|<r\}$.


Unless otherwise stated, an \textbf{interval of \pmb{$\Sbb^1$}} \index{00@Interval of $\Sbb^1$} denotes a non-empty non-dense connected open subset of $\Sbb^1$.

If $X$ is a complex manifold, we let $\scr O(X)$ denote the set of holomorphic functions $f:X\rightarrow\Cbb$.

Unless otherwise stated, an \textbf{unbounded operator} $T:\mc H\rightarrow\mc K$ (where $\mc H,\mc K$ are Hilbert spaces) denotes a linear map from a dense linear subspace $\Dom(T)\subset \mc H$ to $\mc H$. $\Dom(T)$ is called the \textbf{domain} of $T$. We let $T^*$ be the adjoint of $T$. In practice, we are also interested in $T^*$ defined on a dense subspace of its domain $\Dom(T^*)$. We call its restriction a \textbf{formal adjoint} of $T$ and denote it by $T^\dagger$.


Given a Hilbert space $\mc H$, its inner product is denoted by $(\xi,\eta)\in\mc H^2\mapsto\bk{\xi|\eta}$. We assume that it is linear on the first variable and antilinear on the second one. (Namely, we are following mathematician's convention.) 


Whenever we write $\bk{\xi,\eta}$, we understand that it is linear on both variables. E.g. $\bk{\cdot,\cdot}$ denotes the pairing between a vector space and its dual space.


If $\mc H,\mc K$ are Hilbert spaces, we let
\begin{align}
\fk L (\mc H,\mc K)=\{\text{Bounded linear maps }\mc H\rightarrow\mc K\}\qquad\fk L(\mc H)=\fk L(\mc H,\mc H)
\end{align}
If $V,W$ are vector spaces, we let 
\begin{align}
\Hom (V,W)=\{\text{Linear maps }V\rightarrow W\}\qquad\End(V)=\Hom(V,V)
\end{align}
An unbounded operator $T:\mc H\rightarrow\mc H$ denotes a linear map $\Dom(T)\rightarrow\mc H$ where $\Dom(T)$ is a dense linear subspace of $\mc H$. We say that an unbounded operator $T$ is \textbf{continuous} if it is continuous with respect to the norms on the domain and the codomain. Thus, ``bounded" means continuous and $\Dom(T)=\mc H$.



If $z_\blt=(z_1,\dots,z_k)$ are mutually commuting formal variables, for each $n_\blt=(n_1,\dots,n_k)\in\Zbb^k$ we let
\begin{align*}
z_\blt^{n_\blt}=z_1^{n_1}\cdots z_k^{n_k}
\end{align*}

For each vector space $W$, we let
\begin{gather*}
W[[z_\blt]]=\Big\{\sum_{n_\blt\in\Nbb^k} w_{n_\blt}z_\blt^{n_\blt} \Big\}\qquad W[[z_\blt^{\pm1}]]=\Big\{\sum_{n_\blt\in\Zbb^k} w_{n_\blt}z_\blt^{n_\blt} \Big\}\\
W((z_\blt))=\Big\{\sum_{n_\blt\in\Zbb^k} w_{n_\blt}z_\blt^{n_\blt}:w_{n_\blt}=0\text{ when }n_1,\dots,n_k\ll 0 \Big\}\\
W[z_\blt]=W((z_\blt))\cap W((z_\blt^{-1}))=\text{polynomials of $z_\blt$ with $W$-coefficients}
\end{gather*}
where $w_{n_\blt}\in W$.


If $X$ is a set, the $n$-fold \textbf{configuration space} \index{00@Configuration space} $\Conf^n(X)$ \index{Conf@$\Conf^n(X)$} is
\begin{align}
\Conf^n(X)=\{(x_1,\dots,x_n)\in X:x_i\neq x_j\text{ if }i\neq j\}
\end{align}

\begin{df}
A map of complex vector spaces $T:V\rightarrow V'$ is called \textbf{antilinear} \index{antilinear} or \textbf{conjugate linear} if $T(a\xi+b\eta)=\ovl aT\xi+\ovl bT\eta$ for all $\xi,\eta\in V$ and $a,b\in\Cbb$. If $V$ and $V'$ are (complex) inner product spaces, we say that $T$ is \textbf{antiunitary} \index{00@Antiunitary} if it is am antiliear surjective and satisfies $\Vert T\xi\Vert=\Vert\xi\Vert$ for all $\xi\in V$, equivalently,
\begin{align}\label{eq14}
\bk{T\xi|T\eta}=\ovl{\bk{\xi|\eta}}\equiv\bk{\eta|\xi}
\end{align}
for all $\xi,\eta\in V$.
\end{df}

For each $n\in\Zbb$, we let $\ek_n\in C^\infty(\Sbb^1)$ be $\ek_n(z)=z^n$.







\newpage



\section{Introduction: PCT symmetry, Bisognano-Wichmann, Tomita-Takesaki}


Algebraic quantum field theory (AQFT) is a mathematically rigorous approach to QFT using the language of functional analysis and operator algebras. The main subject of this course is 2d \textbf{algebraic conformal field theory (ACFT)}, namely, 2d CFT in the framework of AQFT.


\subsection{}




Let $d\in\Zbb_+$. We first sketch the general picture of an $(1+d)$ dimensional Poincar\'e invariant QFT in the spirit of \textbf{Wightman axioms}. We consider Bosonic theory for simplicity. 

We let $\Rbb^{1,d}$ be the $(1+d)$-dimensional \textbf{Minkowski space}. So it is $\Rbb^{1+d}$ but with metric tensor
\begin{align}
ds^2=(dx^0)^2-(dx^1)^2-\cdots-(dx^d)^2
\end{align}
Here $x^0$ denotes the time coordinate, and $x^1,\dots,x^d$ denote the spatial coordinates.  The (restricted) \textbf{Poincar\'e group} \index{00@Poincar\'e group $\Poid$, restricted} is 
\begin{align*}
\Poid=\Rbb^{1,d}\rtimes\mathrm{SO}^+(1,d)
\end{align*}
Here, $\Rbb^{1,d}$ acts by translation on $\Rbb^{1,d}$. $\mathrm{SO}^+(1,d)$ is the (restricted) \textbf{Lorentz group}, \index{00@Lorentz group $\mathrm{SO}^+(1,d)$} the identity component of the (full) Lorentz group $\mathrm O(1,d)$ whose elements are invertible linear maps on $\Rbb^{1,d}$ preserving the Minkowski metric. 


\begin{rem}
Any $g\in \mathrm O(1,d)$ must have determinent $\pm1$. One can show that $\mathrm{SO^+}(1,d)$ is precisely the elements $g\in \mathrm O(1,d)$ such that $\det g=1$, and that $g$ \uwave{does not change the direction of time} (i.e., if $\mbf v=(v_0,\dots,v_d)\in\Rbb^{1,d}$ satisfies $v_0>0$, then the first component of $g\mbf v$ is $>0$). See \cite[Sec. I.2.1]{Haag}.
\end{rem}



\begin{df}
We say that $\xbf=(x_0,\dots,x_d),\ybf=(y_0,\dots,y_d)\in\Rbb^{1,d}$ are \textbf{spacelike (separated)} \index{00@Spacelike (separated)} if their Minkowski distance is negative, i.e.,
\begin{align*}
(x_0-y_0)^2<(x_1-y_1)^2+\cdots+(x_d-y_d)^2
\end{align*}
\end{df}




\subsection{}\label{lb1}






A Poincar\'e invariant QFT consists of the following data:
\begin{enumerate}[label=(\arabic*)]
\item We have a Hilbert space $\mc H$.
\item There is a (strongly continuous) projective unitary representation $U$ of $\Poid$ on $\mc H$. In particular, its restriction to the translation on the $k$-th component (where $k=0,1,\dots,d$) gives a one parameter unitary group $x^k\in\Rbb\mapsto \exp(\im x^k P_k)$ where $P_k$ is a self-adjoint operator on $\mc H$.
\item (Positive energy) The following are positive operators:
\begin{align*}
P_0\geq0\qquad (P_0)^2-(P_1)^2-\cdots-(P_d)^2\geq0
\end{align*}
The operator $P_0$ is called the \textbf{Hamiltonian} or the \textbf{energy operator}. $P_1,\dots,P_d$ are the momentum operators. $(P_0)^2-(P_1)^2-\cdots-(P_d)^2$ is the mass.
\item We have a collection of \textbf{(quantum) fields} $\scr Q$, where each $\Phi\in\scr Q$ is an operator-valued function on $\Rbb^{1,d}$. For each $\mbf x\in\Rbb^{1,d}$, $\Phi(x)$ is a ``linear operator on $\mc H$".  
\item {\textbf(Locality}) \index{00@Locality} If $\xbf_1,\xbf_2\in\Rbb^{1,d}$ are \uwave{spacelike} and $\Phi_1,\Phi_2\in\scr Q$, then
\begin{align}\label{eq18}
[\Phi_1(x_1),\Phi_2(x_2)]=0
\end{align}
\item (*-invariance) For each $\Phi\in\scr Q$, there exists $\Phi^\dagger\in\scr Q$ such that
\begin{align}\label{eq3}
\Phi(\xbf)^\dagger=\Phi^\dagger(\xbf)
\end{align}
Moreover, $\Phi^{\dagger\dagger}=\Phi$.
\item (Poincar\'e invariance) There is a distinguished unit vector\footnote{A unit vector denotes a vector with length $1$} $\Omega$, called the \textbf{vacuum vector}, such that
\begin{align*}
U(g)\Omega=\Omega\qquad\forall g\in\Poid
\end{align*}
Moreover, for each $g\in\Poid$ and $\Phi\in\scr Q$, we have
\begin{align}\label{eq2}
U(g)\Phi(\xbf)U(g)^{-1}=\Phi(g\xbf)
\end{align}
\item (Cyclicity) Vectors of the form
\begin{align}\label{eq4}
\Phi_1(\xbf_1)\cdots\Phi_n(\xbf_n)\Omega
\end{align}
(where $n\in\Nbb$, $\xbf_1,\dots,\xbf_n\in\Rbb^{1,d}$ are mutually spacelike, and $\Phi_1,\dots,\Phi_n\in\scr Q$) span a dense subspace of $\mc H$.
\end{enumerate}



\begin{rem}
In some QFT, there is a factor (a function of $\xbf$) before $\Phi(g\xbf)$ in the Poincar\'e invariance relation \eqref{eq2}. Similarly, there is a factor before $\Phi^\dagger(\xbf)$ in the $*$-invariance formula \eqref{eq3}. We will encounter these more general covariance property later. In this section, we content ourselves with the simplest case that the factors are $1$.
\end{rem}

\begin{rem}\label{lb7}
By the Poincar\'e invariance and the cyclicity, the action of $\Poid$ is uniquely determined by $\scr Q$ by
\begin{align}
U(g)\Phi_1(\xbf_1)\cdots\Phi_n(\xbf_n)\Omega=\Phi_1(g\xbf_1)\cdots\Phi_n(g\xbf_n)\Omega
\end{align}
\end{rem}


\subsection{}

Technically speaking, $\Phi(\xbf)$ can not be viewed as a linear operator on $\mc H$. It  cannot be defined even on a sufficiently large subspace of $\mc H$. One should think about \textbf{smeared fields} \index{00@Smeared field}
\begin{align}
\Phi(f)=\int_{\Rbb^{1,d}}\Phi(\xbf)f(\xbf)d\xbf
\end{align}
where $f\in C_c^\infty(\Rbb^{1,d})$. (In contrast, we call $\Phi(\xbf)$ a \textbf{pointed field}.) Then $\Phi(f)$ is usually a closable unbounded operator on $\mc H$ with dense domain $\Dom(\Phi(f))$. Moreover, $\Dom(\Phi(f))$ is preserved by any smeared operator $\Psi(g)$. Therefore, for any $f_1,\dots,f_n\in C_c^\infty(\Rbb^{1,d})$ the following vector can be defined in $\mc H$:
\begin{align}\label{eq1}
\Phi_1(f_1)\cdots\Phi_n(f_n)\Omega
\end{align}
The precise meaning of cyclicity in Subsec. \ref{lb1} means that vectors of the form \eqref{eq1} span a dense subspace of $\mc H$. Locality means that for $f_1,f_2\in C_c^\infty(\Rbb^{1,d})$ compactly supported in \uwave{spacelike} regions, on a reasonable dense subspace of $\mc H$ (e.g., the subspace spanned by \eqref{eq1}) we have
\begin{align}
[\Phi_1(f_1),\Phi_2(f_2)]=0
\end{align}
The $*$-invariance means that
\begin{align}
\bk{\Phi(f)\xi|\eta}=\bk{\xi|\Phi^\dagger(f)\eta}
\end{align}
for each $\xi,\eta$ in the this good subspace. 




\subsection{}

In the remaining part of this section,  if possible, we also understand $\Phi(\xbf)$ as a smeared operator $\Phi(f)$ where $f\in C_c^\infty(\Rbb^{1,d})$ satisfies $\int f=1$ and is supported in a small region containing $\xbf$. Thus, $\Phi(\xbf)$ can almost be viewed as a closable operator. Hence the expression \eqref{eq4} makes sense in $\mc H$.


We now explore the consequences of positive energy. As we will see, it implies that $\Phi_1(\xbf_1)\cdots\Phi_n(\xbf_n)\Omega$, a function of $\xbf_\blt$, can be analytically continued.

The fact that $P_0\geq0$ implies that when $t\leq0$, $e^{tP_0}$ is a bounded linear operator with operator norm $\leq1$. Therefore, if $\tau$ belongs to
\begin{align*}
\fk I=\{\Imag\tau\geq0\}
\end{align*}
then $e^{\im \tau P_0}=e^{\im \Real\tau}\cdot e^{-\Imag\tau}$ is bounded. Indeed, $\tau\in\fk I\mapsto e^{\im\tau P_0}$ is continuous, and is holomorphic on $\Int\fk I$. 

Let $\mbf e_0=(1,0,\dots,0)$. Let $\xbf_1,\dots,\xbf_n\in\Rbb^{1,d}$ be distinct. By the Poincar\'e covariance, the relation
\begin{align}\label{eq5}
e^{\im\tau P_0}\Phi_1(\xbf_1)\cdots\Phi_n(\xbf_n)\Omega=\Phi_1(\xbf_1+\tau e_0)\cdots\Phi_n(\xbf_n+\tau e_0)\Omega
\end{align}
holds for all real $\tau$. Moreover, the LHS is continuous on $\fk I$ and holomorphic on $\Int\fk I$. This suggests that the RHS of \eqref{eq5} can also be defined as an element of $\mc H$ when $\tau\in\fk I$. 


\subsection{}\label{lb3}


We shall further explore the question: for which $\xbf_i$ is in $\Cbb^d$ can $\Phi_1(\xbf_1)\cdots\Phi_n(\xbf_n)\Omega$ be reasonably defined as an element of $\mc H$? 

\begin{rem}
We expect that the smeared fields should be defined on any \textbf{$P_0$-smooth vectors}, i.e., vectors in $\bigcap_{k\geq0}\Dom(P_0^k)$. For each $r>0$, since one can find $C_{k,r}\geq0$ such that $\lambda^{2k}\leq C_{k,r}e^{2r\lambda}$ for all $\lambda\geq0$, we conclude that
\begin{align}
\Rng(e^{-rP_0})\equiv \Dom(e^{rP_0})\subset \bigcap_{k\geq0}\Dom(P_0^k)
\end{align}
\end{rem}



The above remark shows that $\Phi_1(\xbf_1)$, viewed as a smeared operator localized on a small neighborhood of $\xbf_1$, is definable on $e^{\im\zeta_2P_0}\Phi_2(\xbf_2)\Omega=\Phi_2(\zeta_2\ebf_0+\xbf_2)\Omega$ whenever $\Imag\zeta_2>0$. Thus, heuristically, $(\zeta_1,\zeta_2)\mapsto e^{\im\zeta_1P_0}\Phi_1(\xbf_1)e^{\im\zeta_2P_0}\Phi_2(\xbf_2)\Omega$ should also be holomorphic on
\begin{align*}
\{(\zeta_1,\zeta_2)\in\Cbb^2:\Imag\zeta_1,\Imag\zeta_2>0\}
\end{align*}
Repeating this procedure, we see that the holomorphicity holds for
\begin{align*}
e^{\im\zeta_1 P_0}\Phi_1(\xbf_1)e^{\im\zeta_2P_0}\Phi_2(\xbf_2)\cdots e^{\im\zeta_nP_0}\Phi_n(\xbf_n)\Omega
\end{align*}
when $\Imag\zeta_i>0$. By Poincar\'e covariance, the above expression equals
\begin{align*}
\Phi_1(\xbf_1+\zeta_1\ebf_0)\Phi_2(\xbf_2+(\zeta_1+\zeta_2)\ebf_0)\cdots\Phi_n(\xbf_n+(\zeta_1+\cdots+\zeta_n)\ebf_0)\Omega
\end{align*}
Therefore,
\begin{align}\label{eq6}
(\zeta_1,\dots,\zeta_n)\mapsto \Phi_1(\xbf_1+\zeta_1\ebf_0)\cdots\Phi_n(\xbf_n+\zeta_n\ebf_0)\in\mc H
\end{align}
should be holomorphic on $\{\zeta_\blt\in\Cbb^n:0<\Imag\zeta_1<\cdots<\Imag\zeta_n\}$.

By the locality axiom, the order of products of quantum fields can be exchanged. Thus, our expectation for a reasonable QFT includes the following condition:
\begin{concl}
Let $\xbf_1,\dots,\xbf_n\in\Rbb^{1,d}$. Then \eqref{eq6} is holomorphic on
\begin{subequations}\label{eq11}
\begin{align}\label{eq11a}
\{(\zeta_1,\dots,\zeta_n)\in\Cbb^n:\Imag\zeta_i>0,\text{ and }\Imag\zeta_i\neq\Imag\zeta_j\text{ if }i\neq j\}
\end{align}
Moreover, since \eqref{eq6} is also definable and continuous on
\begin{align}\label{eq11b}
\{(\zeta_1,\dots,\zeta_n)\in\Rbb^n:\text{$\xbf_1+\zeta_1\ebf_1,\dots,\xbf_n+\zeta_n\ebf_0$ are mutually spacelike}\}
\end{align}
\end{subequations}
we expect that the function \eqref{eq6} is continuous on the union of \eqref{eq11a} and \eqref{eq11b}.
\end{concl}

\subsection{}\label{lb2}


We have (informally) derived some consequences from the positivity of $P_0$.  Note that since $P_0\geq0$, we have $U(g)P_0U(g)^{-1}\geq0$ for each $g\in\SO^+(1,d)$. Since $P_0$ is the generator of the flow $t\in\Rbb\mapsto t\ebf_0\in\Rbb^{1,d}\subset\Poid$, $U(g)P_0U(g)^{-1}$ is the generator of the flow
\begin{align}
t\in\Rbb\mapsto g(t\ebf_0)g^{-1}=t\cdot g\ebf_0
\end{align}
Therefore, if $g\mbf e_0=(a_0,\dots,a_n)$, then
\begin{align}
U(g)P_0U(g)^{-1}=a_0P_0+\cdots+a_nP_n
\end{align}
Hence the RHS must be positive. But what are all the possible $g\ebf_0$?

\begin{rem}
One can show that the orbit of $\mbf \ebf_0=(1,0,\dots,0)$ under $\mathrm{SO}^+(1,d)$ is the upper hyperbola with diameter $1$, i.e., the set of all $(a_0,\dots,a_n)\in\Rbb^{1,d}$ satisfying
\begin{align}
a_0>0\qquad (a_0)^2-(a_1)^2-\cdots-(a_n)^2=1
\end{align}
\end{rem}

Thus $\sum_i a_iP_i\geq0$ for all such $a_\blt$. What are the consequences of this positivity?



\subsection{}\label{lb12}



To simplify the following discussions, we set $d=2$ and
\begin{align*}
t=x^0\qquad x=x^1
\end{align*}
We further set
\begin{align}
u=t-x\qquad v=t+x
\end{align}
so that
\begin{align}
t=\frac{u+v}2\qquad x=\frac{-u+v}2
\end{align}
The Minkowski metric becomes
\begin{align}
\boxed{~(dt)^2-(dx)^2=du\cdot dv~}
\end{align}
Then 
\begin{align}
(u,v)\text{ is spacelike to }(u',v')\qquad\Longleftrightarrow\qquad (u-u')(v-v')<0
\end{align}


\begin{figure}[h]
	\centering
	\includegraphics[height=2cm]{fig1.png}
	\caption{. The coordinates $u,v$}
\end{figure}


For each $\Phi\in\scr Q$, we write
\begin{align}
\wtd \Phi(u,v):=\Phi(t,x)=\Phi\big(\frac{u+v}2,\frac{-u+v}2\big)
\end{align}
We let $H_0$ and $H_1$ be the self-adjoint operators such that
\begin{align*}
H_0=P_0-P_1\qquad H_1=P_0+P_1
\end{align*}
so that they are the generators of the flow $t\mapsto (t,-t)$ and $t\mapsto (t,t)$.




\begin{rem}
Since $\Rbb^{1,d}$ is an abelian group, we know that $P_i$ commutes with $P_j$. Hence $H_0$ commutes with $H_1$. 
\end{rem}

\subsection{}\label{lb9}


The orbit of $\mbf e_0$ under $\SO^+(1,1)$ is the unit upper hyperbola $(x^0)^2-(x_1)^2=1,x^0>0$. Equivalently, it is $uv=1,u>0$. According to Subsec. \ref{lb2}, we conclude that $b_0H_0+b_1H_1\geq0$ for each $b_0,b_1$ satisfying $b_0b_1=1,b_0>0$ (equivalently, for each $b_0>0,b_1>0$). This implies
\begin{align}
H_0\geq0\qquad H_1\geq0
\end{align}
Therefore, similar to the argument in Subsec. \ref{lb3} (and specializing to the special case that $\xbf_1=\cdots=\xbf_n=0$), the holomorphicity of
\begin{align*}
(\zeta_\blt,\gamma_\blt)\mapsto e^{\im\zeta_1H_0+\im\gamma_1 H_1}\wtd\Phi_1(0)e^{\im\zeta_2H_0+\im\gamma_2 H_1}\wtd\Phi_2(0)\cdots e^{\im\zeta_nH_0+\im\gamma_nH_1}\wtd\Phi_n(0)\Omega
\end{align*}
on the region $\Imag\zeta_i>0,\Imag\gamma_i>0$, together with locality, implies:

\begin{concl}\label{lb4}
Let $\Phi_1,\dots,\Phi_n\in\scr Q$. Then
\begin{align}\label{eq8}
(u_1,v_1,\dots,u_n,v_n)\mapsto \wtd\Phi_1(u_1,v_1)\cdots\wtd\Phi(u_n,v_n)\Omega
\end{align}
is holomorphic on
\begin{subequations}\label{eq9}
\begin{align}\label{eq9a}
\{(u_\blt,v_\blt)\in\Cbb^{2n}:\Imag u_i>0 ,\Imag v_i>0,\Imag u_i\neq\Imag u_j,\Imag v_i\neq\Imag v_j\text{ if }i\neq j\}
\end{align}
and can be continuously extended to 
\begin{align}\label{eq9b}
\{(u_\blt,v_\blt)\in\Rbb^{2n}:(u_i-u_j)\cdot(v_i-v_j)<0\text{ if }i\neq j\}
\end{align}
\end{subequations}
\end{concl}

Rigorously speaking, the above mentioned ``continuity" of the extension should be understood in terms of distributions. Here, we ignore such subtlety and view pointed fields as smeared field in a small region.


\begin{comment}
\begin{rem}\label{lb5}
Recall that we have said that a pointed field should also be viewed as a smeared field localized on a small region. Otherwise, the RHS of \eqref{eq8} still belongs to $\mc H$ in the interior of \eqref{eq9}, but not so on the boundary. Thus, a more rigorous interpretation of Conc. \ref{lb4} should be as follows: If $O_1,\dots,O_n$ are spacelike separated regions of $\Rbb^2$ (i.e. for each $(u_i,v_i)\in O_i$ and $(u_j,v_j)\in O_j$ where $i\neq j$, we have \eqref{eq7}), and if $f_i\in C_c^\infty(O_i)$, then the function
\begin{align}\label{eq10}
\begin{aligned}
\int_{O_1}\cdots\int_{O_n}& \wtd\Phi_1(u_1+\tau_1,v_1+\sgm_1)\cdots\wtd\Phi(u_n+\tau_n,v_n+\sgm_n)\Omega \\
&\cdot f_1(u_1,v_1)\cdots f_n(u_n,v_n)\cdot  d(u_1,v_1)\cdots d(u_n,v_n)
\end{aligned}
\end{align}
of $(\tau_1,\sgm_1,\dots,\tau_n,\sgm_n)$ is continuous on \eqref{eq9} and holomorphic on its interior.
\end{rem}
\end{comment}




\subsection{}

%In the following, as in Rem. \ref{lb5}, the following claims about points fields should be viewed as claims about smeared fields supported in small regions, or should be translated into claims about smeared fields in a similar manner as in \eqref{eq10}.


We note that $\diag(-1,\pm 1)$ is not inside $\SO^+(1,1)$, since it reverses the time direction. Neither is $\diag(1,-1)$ in $\SO^+(1,1)$ because its determinant is negative. Consequently, the QFT is not necessarily symmetric under the following operations:
\begin{itemize}
\item \textbf{{\color{red}T}ime reversal} \index{00@Time reversal}  $t\mapsto -x$.
\item \textbf{{\color{red}P}arity transformation} \index{00@Parity transformation} $x\mapsto -x$.
\item \textbf{{\color{red}PT} transformation}  $(t,x)\mapsto (-t,-x)$, the combination of time and parity inversions.
\end{itemize} 
Mathematically, this means that the maps
\begin{gather*}
\Phi_1(t_1,x_1)\cdots \Phi_n(t_n,x_n)\Omega\quad\mapsto\quad \Phi_1(-t_1,x_1)\cdots \Phi_n(-t_n,x_n)\Omega\\
\Phi_1(t_1,x_1)\cdots \Phi_n(t_n,x_n)\Omega\quad\mapsto\quad \Phi_1(t_1,-x_1)\cdots \Phi_n(t_n,-x_n)\Omega\\
\Phi_1(t_1,x_1)\cdots \Phi_n(t_n,x_n)\Omega\quad\mapsto\quad \Phi_1(-t_1,-x_1)\cdots \Phi_n(-t_n,-x_n)\Omega
\end{gather*}
(where $(t_1,x_1),\dots,(t_n,x_n)$ are mutually spacelike) are not necessarily unitary. (Compare Rem. \ref{lb7}.) Simiarly, the QFT is not necessarily symmetric under \textbf{{\color{red}C}harge conjugation} $\Phi\mapsto\Phi^\dagger$, which means that the map
\begin{align*}
\Phi_1(t_1,x_1)\cdots \Phi_n(t_n,x_n)\Omega\quad\mapsto\quad &\Phi_n(t_n,x_n)^\dagger\cdots \Phi_1(t_1,x_1)^\dagger\Omega\\
=&\Phi_1^\dagger(t_1,x_1)\cdots\Phi_n^\dagger(t_n,x_n)\Omega
\end{align*}
is not necessarily (anti)unitary. However, as we shall explain, the combination of PCT transformations is actually unitary, and hence is a symmetry of the QFT. This is called the PCT theorem.

\subsection{}


To prove the PCT theorem, we shall first prove that the PT transformation, though not implemented by a unitary operator, is actually implemented by the analytic continuation of a one parameter unitary group.

\begin{df}
The one parameter group $s\mapsto \Lambda(s)\in\SO^+(1,1)$ defined by
\begin{align}
\Lambda(s)(u,v)=(e^{-s}u,e^sv) 
\end{align}
is called the \textbf{Lorentz boost}. \index{00@Lorentz boost $\Lambda$} Equivalently,
\begin{align}
\Lambda(s)\begin{bmatrix}
t\\
x
\end{bmatrix}
=\begin{bmatrix}
\cosh s&\sinh s\\
\sinh s&\cosh s
\end{bmatrix}
\begin{bmatrix}
t\\
x
\end{bmatrix}
\end{align}
\end{df}

Define the (open) \textbf{right wedge} $\mc W$ and \textbf{left wedge} $-\mc W$ by
\begin{align}
\mc W=\{(u,v)\in\Rbb^2:v>0,u<0\}=\{(t,x)\in\Rbb^{1,1}:-x<t<x\}
\end{align}


\begin{thm}[\textbf{PT theorem}]\label{lb8}
Let $(u_1,v_1),\dots,(u_n,v_n)\in \mc W$ be mutually spacelike (i.e. satisfying $(u_i-u_j)(v_i-v_j)<0$ if $i\neq j$), cf. Fig. \ref{lb6}. Let $\Phi_1,\dots,\Phi_n\in\scr Q$. Let $K$ be the self-adjoint generator of the Lorentz boost, i.e.,
\begin{align*}
U(\Lambda(s))=e^{\im sK}
\end{align*}
Then  $\Phi_1(\xbf_1)\cdots\Phi_n(\xbf_n)\Omega$ belongs to the domain of $e^{-\pi K}$, and
\begin{align}\label{eq16}
e^{-\pi K}\Phi_1(\xbf_1)\cdots\Phi_n(\xbf_n)\Omega=\Phi_1(-\xbf_1)\cdots\Phi_n(-\xbf_n)\Omega
\end{align}
\end{thm}


\begin{figure}[h]
	\centering
	\includegraphics[height=2cm]{fig2.png}
	\caption{.}\label{lb6}
\end{figure}


Equivalently, $\wtd\Phi_1(u_1,v_1)\cdots\wtd\Phi_n(u_n,v_n)\Omega$ belongs to the domain of $e^{-\pi K}$, and
\begin{align}\label{eq12}
e^{-\pi K}\wtd\Phi_1(u_1,v_1)\cdots\wtd\Phi_n(u_n,v_n)\Omega=\wtd\Phi_1(-u_1,-v_1)\cdots\wtd\Phi_n(-u_n,-v_n)\Omega
\end{align}
Note that the requirement that $(u_1,v_1),\dots,(u_n,v_n)\in\mc W$ are spacelike means, after relabeling the subscripts, that
\begin{align*}
0<v_1<\cdots<v_n\qquad 0<-u_1<\cdots<-u_n
\end{align*}
\begin{proof}
This theorem relies on the following fact that we shall prove rigorously in the future:
\begin{itemize}
\item[$\varstar$] Let $T\geq0$ be a self-adjoint operator on $\mc H$ with $\Ker(T)=0$. Let $r>0$. Then $\xi\in\mc H$ belongs to $\Dom(T^r)$ iff the function $s\in\Rbb\mapsto T^{\im s}\xi\in\mc H$ can be extended to a continuous function $F$ on 
\begin{align*}
\{z\in\Cbb:-r\leq\Imag z\leq0\}
\end{align*}
and holomorphic on its interior. Moreover, for such $\xi$ we have $F(-\im r)=T^r\xi$.
\end{itemize}
In fact, the function $F(z)$ is given by $z\mapsto T^z\xi$.

We shall apply this result to $T=e^{-K}$ and $r=\pi$. For that purpose, we must show that the $\mc H$-valued function of $s\in\Rbb$ defined by
\begin{align*}
e^{\im\pi s}\wtd\Phi_1(u_1,v_1)\cdots\wtd\Phi_n(u_n,v_n)\Omega=\wtd\Phi_1(e^{-s}u_1,e^sv_1)\cdots\wtd\Phi_n(e^{-s}u_n,e^sv_n)\Omega
\end{align*}
can be extended to a continuous function on
\begin{align*}
\{z\in\Cbb:0\leq\Imag z\leq \pi\}
\end{align*}
and holomorphic on its interior. 

In fact, we can construct this $\mc H$-valued function, which is
\begin{align*}
z\mapsto \wtd\Phi_1(e^{-z}u_1,e^zv_1)\cdots\wtd\Phi_n(e^{-z}u_n,e^zv_n)\Omega
\end{align*}
noting that the conditions in Conc. \ref{lb4} are fulfilled. In particular, the condition $0<\Imag<\pi$ is used to ensure that, since $u_i<0,v_i>0$, we have $\Imag (e^{-z}u_i)>0$ and $\Imag(e^zv_i)>0$ as required by \eqref{eq9a}. The value of this function at $z=\im\pi$ equals the RHS of \eqref{eq12}. Therefore the theorem is proved.
\end{proof}




\subsection{}




\begin{thm}[\textbf{PCT theorem}]\label{lb10}
We have an antiunitary map $\Theta:\mc H\rightarrow\mc H$, called the \textbf{PCT operator}, \index{00@PCT operator} such that
\begin{align}
\Theta\cdot \Phi_1(\xbf_1)\cdots\Phi_n(\xbf_n)\Omega=\Phi_1(-\xbf_1)^\dagger\cdots\Phi_n(-\xbf_n)^\dagger\Omega
\end{align}
for any $\Phi_1,\dots,\Phi_n\in\scr Q$ and mutually spacelike $\xbf_1,\dots,\xbf_n$.
\end{thm}


Equivalently, $\Theta$ is defined by
\begin{align}\label{eq13}
\Theta\cdot\wtd\Phi_1(u_1,v_1)\cdots\wtd\Phi_n(u_n,v_n)=\wtd\Phi_1(-u_1,-v_1)^\dagger\cdots \wtd\Phi_n(-u_n,-v_n)^\dagger\Omega
\end{align}

\begin{proof}
The existence of an antilinear isometry $\Theta$ satisfying \eqref{eq13} is equivalent to showing that (cf. \eqref{eq14})
\begin{align*}
\begin{aligned}
&\bk{\wtd\Phi_1(\ubf_1)\cdots\wtd\Phi_n(\ubf_n)\Omega|\wtd\Psi_1(\ubf_1')\cdots\wtd\Psi_k(\ubf_k')\Omega}\\
=&\bk{\wtd\Psi_1(-\ubf_1')^\dagger\cdots\wtd\Psi_k(-\ubf_k')^\dagger\Omega|\wtd\Phi_1(-\ubf_1)^\dagger\cdots\wtd\Phi_n(-\ubf_n)^\dagger\Omega}
\end{aligned}\tag{$\star$}\label{eq17}
\end{align*} 
if $\ubf_1,\dots\ubf_n$ are spacelike, and $\ubf_1',\dots\ubf_k'$ are spacelike. (We do not assume that, say, $\ubf_1$ and $\ubf_1'$ are spacelike.)

It suffices to prove this in the special case that $\ubf_1,\dots,\ubf_n,\ubf_1',\dots,\ubf_k'$ are mutually spacelike. Then the general case will follow that both sides of the above relation can be analytically continued to suitable regions as functions of $\ubf_1,\dots,\ubf_n$. For example, the fact that $H_0,H_1\geq0$ implies that
\begin{align*}
e^{\im\zeta H_0+\im\gamma H_1}\wtd\Phi_1(\ubf_1)\cdots\wtd\Phi_n(\ubf_n)\Omega=\wtd\Phi_1(\ubf_1+(\zeta,\gamma))\cdots\wtd\Phi_n(\ubf_n+(\zeta,\gamma))\Omega
\end{align*}
is continuous on $\{(\zeta,\gamma)\in\Cbb^2:\Imag\zeta\geq0,\Imag\gamma\geq0\}$ and holomorphic on its interior.

Set $\Gamma_j=\Psi_j^\dagger$. Then \eqref{eq17} is equivalent to
\begin{align*}
&\bk{\wtd\Phi_1(\ubf_1)\cdots\wtd\Phi_1(\ubf_n)\wtd\Gamma_1(\ubf_1')\cdots\wtd\Gamma_k(\ubf_k')\Omega|\Omega}\\
=&\bk{\wtd\Phi_1(-\ubf_1)\cdots\wtd\Phi_1(-\ubf_n)\wtd\Gamma_1(-\ubf_1')\cdots\wtd\Gamma_k(-\ubf_k')\Omega|\Omega}
\end{align*}
By the PT Thm. \ref{lb8}, this relation is equivalent to
\begin{align*}
&\bk{\wtd\Phi_1(\ubf_1)\cdots\wtd\Phi_1(\ubf_n)\wtd\Gamma_1(\ubf_1')\cdots\wtd\Gamma_k(\ubf_k')\Omega|\Omega}\\
=&\bk{e^{-\pi K}\wtd\Phi_1(\ubf_1)\cdots\wtd\Phi_1(\ubf_n)\wtd\Gamma_1(\ubf_1')\cdots\wtd\Gamma_k(\ubf_k')\Omega|\Omega}
\end{align*}
But this of course holds since $e^{-\pi K}\Omega=\Omega$ by Poincar\'e invariance.
\end{proof}



\subsection{}\label{lb11}

Combining the PT Thm. \ref{lb8} with the PCT Thm. \ref{lb10}, we conclude that $e^{-\pi K}$ is an injective positive operator, $\Theta$ is antinitary, and
\begin{subequations}\label{eq15}
\begin{align}
\Theta e^{-\pi K}A\Omega=A^\dagger\Omega
\end{align}
where $A$ is a product of spacelike separated field in $\mc W$. The rigorous statement should be that
\begin{align*}
A=\Phi_1(f_1)\cdots\Phi_n(f_n)
\end{align*}
where $\Phi_1,\dots,\Phi_n\in\scr Q$, and $f_i\in C_c^\infty(O_i)$ where $O_1,\dots,O_n\subset\mc W$ are open and mutually spacelike. If we let $\scr A(\mc W)$ be the $*$-algebra generated by all such $A$, then by the Poincar\'e invariance, for each $g\in\Poid$ we have
\begin{align*}
U(g)\scr A(\mc W)U(g)^{-1}=\scr A(g\mc W)
\end{align*}
In particular, since for the Lorentz boost $\Lambda$ we have $\Lambda(s)\mc W=\mc W$, we therefore have
\begin{align}
e^{\im sK}\scr A(\mc W)e^{-\im sK}=\scr A(\mc W)
\end{align}
for all $s\in\Rbb$. Since the PT transformation sends $\mc W$ to $-\mc W$, the definition of $\Theta$ clearly also implies
\begin{align}\label{eq15c}
\Theta\scr A(\mc W)\Theta^{-1}=\scr A(-\mc W)
\end{align}
Note that since $\mc W$ is local to $-\mc W$, we have $[\scr A(\mc W),\scr A(-\mc W)]=0$. Therefore, $\Theta\scr A(\mc W)\Theta$ is a subset of the (in some sense) commutant of $\scr A(\mc W)$.
\end{subequations}


\subsection{}



The set of formulas \eqref{eq15} is reminiscent of the Tomita-Takesaki theory, one of the deepest theories in the area of operator algebras. The setting is as follows. 

Let $\mc M$ be a von Neumann algebra on a Hilbert space $\mc H$. Namely, $\mc M$ is a $*$-subalgebra of $\fk L(\mc H)$ closed under the ``strong operator topology". (We will formally introduce von Neumann algebras in a later section.) Let $\Omega\in\mc H$ be a unit vector. Assume that $\Omega$ is \textbf{cyclic} (i.e. $\mc M\Omega$ is dense) and \textbf{separating} (i.e., if $x\in\mc M$ and $x\Omega=0$ then $x=0$) under $\mc M$. Then the \textbf{Tomita-Takesaki theorem} says that the linear map
\begin{align*}
S:\mc M\Omega\rightarrow\mc M\Omega\qquad x\Omega\mapsto x^*\Omega
\end{align*}
is antilinear and closable. Denote its closure also by $S$, and consider its polar decomposition $S=J\Delta^{\frac 12}$ where $\Delta$ is a positive closed operator, and $J$ is an antiunitary map. Then $\Delta$ is injective, we have $J^{-1}=J^*=J$, and
\begin{align*}
\Delta^{\im s}\mc M\Delta^{-\im s}=\mc M\qquad J\mc MJ=\mc M'
\end{align*}
where $\mc M'$ is the commutant $\{y\in\fk L(\mc H):xy=yx~(\forall x\in\mc M)\}$.We call $\Delta$ and $J$ respectively the \textbf{modular operator} and the \textbf{modular conjugation}.



\subsection{}



To relate the Tomita-Takesaki theory to QFT, one takes $\mc M$ to be $\fk A(\mc W)$, the von Neumann algebra generated by $\scr A(\mc W)$. Note that the elements of $\scr A(\mc W)$ are typically unbounded operators, whereas those of $\fk A(\mc W)$ are bounded. Thus, the meaning of ``the von Neumann algebra generated by a set of closed/closable operators" should be clarified. This is an important notion, and we will study it in a later section.


To apply the setting of Tomita-Takesaki, one should first show that the vacuum vector is cyclic and separating under $\fk A(\mc W)$. This is not an easy task, although it is relatively easier to show that $\Omega$ is cyclic and separating under $\scr A(\mc W)$. Moreover, we have
\begin{thm}[\textbf{Bisognano-Wichmann}]\index{00@Bisognano-Wichmann theorem}
Let $\Delta$ and $J$ be the modular operator and the modular conjugation of $(\fk A(\mc W),\Omega)$. Then $J=\Theta$ and $\Delta^{\frac 12}=e^{-\pi K}$.
\end{thm}
Since \eqref{eq15c} easily implies $\Theta\fk A(\mc W)\Theta^{-1}=\fk A(-\mc W)$, together with $J\mc M J^{-1}=\mc M'$ we obtain
\begin{align}
\fk A(\mc W)'=\fk A(-\mc W)
\end{align}
a version of \textbf{Haag duality}.\index{00@Haag duality}



One of the main goals of this course is to give a rigorous and self-contained proof of the PCT theorem, the Bisognano-Wichmann theorem, and the Haag duality for 2d chiral conformal field theories.




\subsection{}

For a general odd number $d>0$, the above results should be modified as follows. Let $K$ be the generator of the \textbf{Lorentz boost} \index{00@Lorentz boost}
\begin{align*}
\Lambda(s)=\left(
\begin{array}{c|c}
\begin{matrix}
\cosh s&\sinh s\\
\sinh s&\cosh s
\end{matrix}
&0\\
\hline
0&
\begin{matrix}
1&&\\
&\ddots&\\
&&1
\end{matrix}
\end{array}
\right)
\end{align*}
Let $\Lambda(\im\pi)=\diag(-1,-1,1,\dots,1)$, which does not belong to $\Poid$ since it reverses the time direction (although it has positive determinant). Then the PT Thm. \ref{lb8} should be modified by replacing \eqref{eq16} with 
\begin{align}
e^{-\pi K}\Phi_1(\xbf_1)\cdots\Phi_n(\xbf_n)\Omega=\Phi_1(\Lambda(\im\pi)\xbf_1)\cdots\Phi_n(\Lambda(\im\pi)\xbf_n)\Omega
\end{align} 

Let $\rho=\diag(1,1,-1,\dots,-1)$, which has determinant $1$ (since $d$ is odd) and hence belongs to $\SO^+(1,d)$. Then the PCT Thm. \ref{lb10} still holds verbatim. Let
\begin{align}
\mc W=\{(a_0,\dots,a_n)\in\Rbb^{1,d}:-a_1<a_0<a_1\}
\end{align}
Then the \textbf{Bisognano-Wichmann theorem} says that $e^{-\pi K}$ is the modular operator of $(\fk A(\mc W),\Omega)$, and $\Theta U(\rho)$ is the modular conjugation. 


We refer the readers to \cite[Sec. V.4.1]{Haag} and the reference therein for a detailed study.
 


\newpage


\section{2d conformal field theory}


\subsection{}

We look at a 2d \textbf{unitary full conformal field theory} (unitary full CFT) $\mc Q$ on the \textbf{space-compactified Minkowski space}\index{R@$\Rc$}
\begin{align*}
\pmb{\Rc}=\Rbb\times\Sbb^1\qquad \text{with metric tensor }(dt)^2-(dx)^2=dudv
\end{align*}
The space $\Rc$ describes the propagation of the closed string $\{0\}\times\Sbb^1$. Here, as in Subsec. \ref{lb12}, we write a general element of $\Rc$ as $\xbf=(t,x)$, and write
\begin{align*}
u=t-x\quad v=t+x\qquad\text{so that}\qquad t=\frac{u+v}2\quad x=\frac{-u+v}2
\end{align*}
The field operators are of the form $\Phi(\xbf)=\Phi(t,x)$. Recall that 
\begin{align*}
\wtd \Phi(u,v):=\Phi(t,x)=\Phi\big(\frac{u+v}2,\frac{-u+v}2\big)
\end{align*}
Identifying $\Rbb/2\pi\Zbb=\Sbb^1$ via $\exp$, a field $\Phi$ can be viewed as an ``operator valued function" on $\Rbb^{1,1}$ satisfying
\begin{align}\label{eq29}
\Phi(t,x+2\pi)=\Phi(t,x)\qquad\text{equivalently}\qquad \wtd\Phi(u,v)=\wtd\Phi(u-2\pi,v+2\pi)
\end{align}
The field operators are ``acting on" a Hilbert space $\mc H$ with vacuum vector $\Omega$.


Compared to the axioms for Poincar\'e invariant QFT in Subsec. \ref{lb1}, some changes should be made to describe a CFT. We still have the locality \eqref{eq18}. Instead of considering $\mathrm{P}^+(1,1)$ we must consider the group of orientation-preserving, time-direction preserving, and conformal (i.e. angle-preserving) transforms on $\Rc$. ``Conformal" means that the diffeomorphism $g:\Rc\rightarrow\Rc$ satisfies
\begin{align*}
g^*(dudv)=\lambda(u,v)dudv
\end{align*}
for a smooth function $\lambda:\Rc\rightarrow\Rbb_{>0}$. Our next goal is to classify such $g$.


\subsection{}

\begin{df}
We let $\Diffp(\Sbb^1)$ \index{Diff@$\Diffp(\Sbb^1)$} be the group of orientation-preserving diffeomorphisms of $\Sbb^1$. Equivalently, it is the group of smooth functions $f:\Sbb^1\rightarrow\Sbb^1$ whose lift $\wtd f:\Rbb\rightarrow\Rbb$ satisfies for all $x\in\Rbb$ that
\begin{gather}\label{eq19}
\wtd f(x+2\pi)=\wtd f(x)+2\pi\qquad \wtd f'(x)>0
\end{gather}
\end{df}

Note that by the basics of covering spaces, any element of $\Diffp(\Sbb^1)$ can be lifted to $\wtd f$ satisfying \eqref{eq19}. Conversely, if $\wtd f$ satisfies \eqref{eq19}, then $\wtd f$ gives rise to an injective smooth map $f:\Sbb^1\rightarrow\Sbb^1$. (Note that $\wtd f'>0$ implies that $\wtd f$ is strictly increasing.) Since $\wtd f'(x)>0$, the function $f$ is injective, and the inverse function theorem shows that the compact set $f(\Sbb^1)$ is open, and hence equals $\Sbb^1$. Thus $f\in\Diffp(\Sbb^1)$.


\begin{rem}
Note that $f$ uniquely determines $\wtd f$ up to an $2\pi\Zbb$-addition, i.e., both $\wtd f$ and $\wtd f+2n\pi$ (where $n\in\Zbb$) correspond to $f$. Therefore, if we let \pmb{$\WDS$} \index{DiffS@$\WDS$} be the topological group formed by all $\wtd f$ satisfying \eqref{eq19},\footnote{The topology is defined such that a net $\wtd f_\alpha$ converges to $\wtd f$ iff the $n$-th derivative $\wtd f_\alpha^{(n)}$ converges uniformly to $\wtd f^{(n)}$ for all $n\in\Nbb$.} then we have an exact sequence
\begin{align}\label{eq26}
1\longrightarrow \Zbb\longrightarrow \WDS\longrightarrow\DiffS\longrightarrow 1
\end{align}
where $\Zbb$ is freely generated by $x\in\Rbb\mapsto x+2\pi$. 

Note that the map $(\wtd f,t)\in\WDS\times[0,1]\mapsto \wtd f_t\in\WDS$ defined by
\begin{align*}
\wtd f_t(x)=(1-t)\wtd f(x)+tx
\end{align*}
shows that $\WDS$ is contractible (to the identity element) and hence simply-connected. (Therefore $\DiffS$ is connected.) We conclude that $\WDS$ is the universal cover of $\DiffS$. \hfill\qedsymbol
\end{rem}



\subsection{}
\begin{thm}\label{lb13}
Under the coordinates $(u,v)$, an orientation-preserving time-direction-preserving conformal transform $g$ of $\Rc$ is precisely of the form 
\begin{align}\label{eq20}
g(u,v)=(\alpha(u),\beta(v))
\end{align}
where $\alpha,\beta:\Rbb\rightarrow\Rbb$ belong to $\WDS$.
\end{thm}



\begin{proof}
Step 1. First, suppose that $g$ is of the form \eqref{eq20}. Then $g$ gives a well-defined smooth map $\Rc\rightarrow\Rc$ because
\begin{align}\label{eq21}
g(u-2\pi,v+2\pi)=g(u,v)+(-2\pi,2\pi)
\end{align}
One checks easily that $g$ is a diffeomorphism (with inverse given by $(\alpha^{-1}(u),\beta^{-1}(v))$) preserving the orientation and the time direction. Since $g^*dudv=\alpha'(u)\beta'(v)dudv$, $g$ is conformal.\\[-1ex]

Step 2. Conversely, choose an orientation preserving conformal transform $g$. We lift $g$ to a smooth conformal map $\Rbb^{1,1}\rightarrow\Rbb^{1,1}$ also denoted by $g=(\alpha,\beta)$. So $\alpha,\beta:\Rbb^{1,1}\rightarrow\Rbb$. Then, besides \eqref{eq21}, $g$ also satisfies:
\begin{gather*}
\partial_u\alpha\partial_u\beta=0\qquad \partial_v\alpha\partial_v\beta=0\tag{a}\label{eq22}\\
\partial_u\alpha\partial_v\beta+\partial_v\alpha\partial_u\beta>0 \tag{b}\label{eq23}\\
\partial_u\alpha\partial_v\beta-\partial_v\alpha\partial_u\beta>0 \tag{c}\label{eq24}
\end{gather*}
Here, \eqref{eq22} and \eqref{eq23} are due to the fact that
\begin{align*}
g^*(dudv)=(\partial_u\alpha du+\partial_v\alpha dv)(\partial_u\beta du+\partial_v\beta dv)
\end{align*}
equals $\lambda(u,v)dudv$ for some smooth $\lambda:\Rbb^{1,1}\rightarrow\Rbb_{>0}$. (So $\lambda$ is the LHS of \eqref{eq23}.) Since $g$ is orientation preserving, \eqref{eq24} follows from the computation
\begin{align*}
g^*(du\wedge dv)=(\partial_u\alpha\partial_v\beta-\partial_v\alpha\partial_u\beta)du\wedge dv
\end{align*}

Step 3. By \eqref{eq22}, at a given $p\in\Rbb^{1,1}$, if $\partial_u\alpha\neq0$, then $\partial_u\beta=0$. Conversely, if at $p$ we have $\partial_u\beta=0$, then \eqref{eq23} shows that $\partial_u\alpha\partial_v\beta>0$, and hence $\partial_u\alpha\neq0$. Thus
\begin{gather*}
\partial_u\alpha|_p\neq 0\qquad\Longleftrightarrow\qquad \partial_u\beta|_p=0\\
\partial_v\alpha|_p\neq 0\qquad\Longleftrightarrow\qquad \partial_v\beta|_p=0
\end{gather*}
where the second equivalence follows from the same argument. Therefore, the set of $p$ at which $\partial_v\alpha= 0$ is both open and closed, and hence must be either $\Rbb^{1,1}$ or $\emptyset$. Similarly, either $\partial_u\beta=0$ everywhere, or $\partial_u\beta\neq0$ everywhere.

Let us prove that
\begin{align*}
\partial_v\alpha=0\qquad \partial_u\beta=0
\end{align*}
everywhere. Suppose the first is not true. Then by the previous paragraph, we have  $\partial_v\alpha\neq0$ and $\partial_v\beta=0$ everywhere. Then \eqref{eq23} implies $\partial_v\alpha\partial_u\beta>0$, and \eqref{eq24} implies $-\partial_v\alpha\partial_u\beta>0$, impossible. So the first (and similarly the second) is true.\\[-1ex]

Step 4. Therefore, we can write $\alpha=\alpha(u)$ and $\beta=\beta(v)$, and we have $\alpha'\neq0$ and $\beta'\neq0$ everywhere. \eqref{eq23} implies that $\alpha'(u)\beta'(v)>0$ for all $u,v$. Thus, either $\alpha'>0$ and $\beta'>0$ everywhere, or $\alpha'<0$ and $\beta'<0$ everywhere. The latter cannot happen, since $g$ preserves the direction of time. Thus $\alpha'>0$ and $\beta'>0$ everywhere. Since $g$  satisfies \eqref{eq21}, we see that $\alpha$ satisfies \eqref{eq19}. Similarly $\beta$ satisfies \eqref{eq19}. This finishes the proof.
\end{proof}


\subsection{}

We let \pmb{$\Cf^+(\Rc)$} \index{CfR@$\Cf^+(\Rc)$} be the group of diffeomorphisms of $\Rc$ preserving the orientation and the time-direction. Then Thm. \ref{lb13} says that any $g\in\Cf^+(\Rc)$ can be represented by some $(\alpha,\beta)\in\WDS^2$. 

However, $(\alpha,\beta)$ is not uniquely determined by $g$. Indeed, in Step 2 of the proof of Thm. \ref{lb13} we have lifted $g$ to a smooth map on $\mbb R^{1,1}$. This lift is unique up to addition by $(-2\pi,2\pi)\Zbb$ in the $(u,v)$ coordinates (or $(0,2\pi)\Zbb$ in the $(t,x)$ coordinates). Thus, $(\alpha,\beta)$ are unique up to addition by $(-2\pi,2\pi)\Zbb$. This non-uniqueness can be ignored once we pass to $(\wch\alpha,\wch\beta)$, the projection of $(\alpha,\beta)$ into $\DiffS^2$. Thus, we have a well-defined (continuous) surjective group homomorphism $\Gamma:\Cf^+(\Rc)\rightarrow\DiffS\times\DiffS$ sending $g$ to $(\wch \alpha,\wch\beta)$. 

One checks easily that the kernel of this homomorphism is freely generated by $(2\pi,0)$ (equivalently, by $(0,2\pi)$) under the $(u,v)$ coordinates, equivalently, by $(\pi,\pi)$ under the $(t,x)$ coordinates. Therefore, we have an exact sequence of groups
\begin{align}
1\longrightarrow \Zbb\longrightarrow \Cf^+(\Rc)\xlongrightarrow{\Gamma}\DiffS^2\longrightarrow 1
\end{align}
Since $\Gamma$ is a covering map, we also have a covering map $\WDS^2\twoheadrightarrow \Cf^+(\Rc)$ such that the following diagram commutes
\begin{equation}\label{eq25}
\begin{tikzcd}
                       & \WDS^2 \arrow[d, two heads] \arrow[ld, two heads] \\
\Cf^+(\Rc) \arrow[r, two heads, "\Gamma"'] & \DiffS^2                                           
\end{tikzcd}
\end{equation}

\subsection{}



Since we require that $\mc Q$ is a CFT with Hilbert space $\mc H$, we must have a \textbf{strongly continuous projective unitary representation} \index{00@Projective unitary representation} $\mc U$ of $\Cf^+(\Rc)$. Namely,
\begin{align*}
\mc U:\Cf^+(\Rc)\rightarrow \PU(\mc H)
\end{align*}
is a continuous group homomorphism. Here, $\PU(\mc H)$ is the quotient group (with quotient topology) $U(\mc H)/\sim$ where $U(\mc H)$ is the group of unitary operators of $\mc H$ (equipped with the strong operator topology), and $U_1\simeq U_2$ iff $U_1=\lambda U_2$ for some $\lambda\in\Cbb$ such that $|\lambda|=1$. We suppress the adjectives ``strongly continuous" when no confusion arises.


By \eqref{eq25}, $\mc U$ can be lifted to a projective unitary representation of $\WDS^2$ on $\mc H$. Since $\WDS^2$ is simply connected, its projective unitary representations are (roughly) equivalent to the projective unitary representations of the Lie algebra of $\WDS\times\WDS$, which is $\Vect(\Sbb^1)\oplus\Vect(\Sbb^1)$ where $\Vect(\Sbb^1)$ \index{Vec@$\Vect(\Sbb^1)$} is the Lie algebra of smooth real vector fields of $\Sbb^1$. 

The elements of $\Vect(\Sbb^1)$ are of the form $f\partial_\theta$ where $f\in C^\infty(\Sbb^1,\Rbb)$ and $\partial_\theta$ is the unique vector field on $\Sbb^1$ that is pulled back by $\exp(\im \cdot):\Rbb\rightarrow\Sbb^1$ to $\partial_\theta\in\Vect(\Rbb^1)$ where $\theta$ is the standard coordinate of $\Rbb$ (sending $x$ to $x$). The Lie bracket of $\Vect(\Sbb^1)$ is the negative of the Lie derivative, i.e.
\begin{align*}
[f_1\partial_\theta,f_2\partial_\theta]_{\Vect(\Sbb^1)}=(-f_1\partial_\theta f_2+f_2\partial_\theta f_1)\partial_\theta
\end{align*}
The negative choice is due to the fact that the group action of $g\in\DiffS$ on $h\in C^\infty(\Sbb^1)$ is given by $h\circ g^{-1}$; however, the Lie derivative is defined by differentiating $g\mapsto h\circ g$.





\subsection{}

The complexification $\Vectc(\Sbb^1)$ \index{Vec@$\Vectc(\Sbb^1)$} of $\Vect(\Sbb^1)$ is the Lie algebra of all $f\partial_\theta$ where $f\in C^\infty(\Sbb^1)\equiv C^\infty(\Sbb^1,\Cbb)$. Let $z=e^{\im\theta}\in C^\infty(\Sbb^1)$, which is the inclusion map $\Sbb^1\hookrightarrow\Cbb$. Then we can define $\partial_z\in\Vectc(\Sbb^1)$ by 
\begin{align}\label{eq28}
\partial_z=\frac 1{\im z}\partial_\theta\qquad\text{so that}\qquad \partial_\theta=\im z\partial_z=\im e^{\im\theta}\partial_z
\end{align}
Then $\Vectc(\Sbb^1)$ is a $*$-Lie algebra, i.e., a complex Lie algebra equipped with an involution $\dagger$. For $\Vectc(\Sbb^1)$, the involution is defined by
\begin{align*}
(f\partial_\theta)^\dagger=-\ovl f\partial_\theta
\end{align*}
so that $\Vect(\Sbb^1)$ is precisely the set of all $\xk\in\Vectc(\Sbb^1)$ satisfying $\xk^\dagger=-\xk$. In particular, noting $\ovl z=z^{-1}$ on $\Sbb^1$, we have
\begin{align}
(\partial_z)^\dagger=z^2\partial_z
\end{align}

$\Vectc(\Sbb^1)$ contains a ``sufficiently large" $*$-Lie subalgebra, the \textbf{Witt algebra} \index{00@Witt algebra $\Witt$} $\Witt=\Span_\Cbb\{l_n:n\in\Zbb\}$, where \index{ln@$l_n=z^n\partial_z$}
\begin{align}
l_n=z^n\partial_z
\end{align}
One easily computes that $l_n^\dagger=l_{-n}$, and that
\begin{align*}
[l_m,l_n]=(m-n)l_{m+n}
\end{align*}
Projective unitary representations of $\WDS$ correspond to (honest) unitary representations of central extensions of $\WDS$, which (roughly) correspond to unitary representations of central extensions of $\Witt$. 




\subsection{}

It can be shown that the central extensions of $\Witt$ are equivalent to the \textbf{Virasoro algebra} \pmb{$\Vir$}. \index{00@Virasoro algebra $\Vir$} As a vector space, $\Vir$ has basis $\{C,L_n:n\in\Zbb\}$. These basis elements satisfy
\begin{subequations}
\begin{gather}
[L_m,L_n]=(m-n)L_{m+n}+\frac C{12}(m^3-m)\delta_{m+n,0}\qquad [L_n,C]=0\\
L_n^\dagger=L_{-n}\qquad C^\dagger=C 
\end{gather}  
\end{subequations}
Thus, the projective unitary representation of $\WDS\times\WDS$ on $\mc H$ can be described by a unitary representation of $\Vir\oplus\wht\Vir$. Here, $\wht\Vir=\{\wht C,\wht L_n:n\in\Zbb\}$ is isomorphic to $\Vir$.


One can decompose a (unitary full) CFT $\mc Q$ into a ``direct sum" of CFTs such that $C$ and $\wht C$ act as scalars $c,\wht c\in\Rbb$. (In fact, one can show that $c,\wht c\geq0$.) We call $(c,\wht c)$ the \textbf{central charge} \index{00@Central charge} of the CFT $\mc Q$. Since $L_0^\dagger=L_0$ and $\wht L_0^\dagger=\wht L_0$, one usually assume that $L_0,\wht L_0$ act as self-adjoint operators on $\mc H$.



\subsection{}

In a Poincar\'e invariant QFT, the vacuum vector is fixed by $\Poid$. However, in our CFT $\mc Q$, \uwave{the vacuum vector $\Omega$ is not fixed by $\Cf^+(\Rc)$}. In terms of $\Vir$, then $L_n\Omega$ is not necessarily zero for all $n$. This phenomenon  is related to the fact that an arbitrary one-parameter subgroup $t\in\Rbb\mapsto g_t\in  \WDS$, when each $g_t$ acts on $\Sbb^1$ and hence can be viewed as a map $g_t:\Sbb^1\rightarrow\Pbb^1$, does not have a sufficiently large domain for the analytic continuation $z\mapsto g_z$. 

On the other hand, we do have
\begin{align}
L_n\Omega=0\qquad\text{ if }n=-1,0,1
\end{align}
(and similarly $\wht L_0\Omega=\wht L_{\pm 1}\Omega=0$). These $L_0,L_{\pm1}$ span a Lie $*$-subalgebra
\begin{align*}
\fk{sl}(2,\Cbb)=\Span_\Cbb\{L_0,L_{\pm1}\}
\end{align*}
with skew-symmetric part \index{su11@$\fk{su}(1,1)$}
\begin{align}
\fk{su}(2):=\{\xk\in\fk{sl}(2,\Cbb):\xk^\dagger=-\xk\}=\Span_\Rbb\Big\{\im l_0,\frac{l_1-l_{-1}}2,\frac{\im(l_1+l_{-1})}2\Big\}
\end{align}
As we will see in the future, the one-parameter group generated by $(l_1-l_{-1})/2$ is related to the PCT symmetry of the CFT.



The Lie subgroup of $\WDS$ with Lie algebra $\fk{su}(2)$ is \pmb{$\UPSU$}, \index{PSU@$\PSU$ and $\UPSU$}, the universal cover of the \textbf{M\"obius group} $\PSU$ \index{00@M"obius group $\PSU$} whose elements are linear fractional transforms
\begin{align*}
z\in\Pbb^1\mapsto \frac{\alpha z+\beta}{\ovl\beta z+\ovl\alpha}\qquad \text{where }\alpha,\beta\in\Cbb,|\alpha|^2-|\beta|^2=1
\end{align*}
The condition $|\alpha|^2-|\beta|^2=1$ is to ensure that the transform sends $\Sbb^1$ to $\Sbb^1$. The exact sequence \eqref{eq26} restricts to
\begin{align}
1\longrightarrow \Zbb\longrightarrow \UPSU\longrightarrow\PSU\longrightarrow 1
\end{align}
where $\Zbb$ is freely generated by ``the anticlockwise rotation by $2\pi$". Thus, the projective action $\mc U(g)$ of any $g$ in $\UPSU\times\UPSU\subset\WDS\times\WDS$ fixes $\Omega$ up to $\Sbb^1$-multiplications. Now we choose $\mc U(g)$ to be the unique one such that $\mc U(g)\Omega=\Omega$. Then $\mc U$ gives an (honest) strongly-continuous unitary representation of $\UPSU\times\UPSU$ on $\mc H$ fixing $\Omega$.


\subsection{}


A field $\Phi\in\mc Q$ is called \textbf{chiral} (resp. \textbf{antichiral}) \index{00@Chiral/antichiral fields} if $\wtd\Phi$ depends only on $u$ (resp. $v$) but not on $v$ (resp. $u$). We let \pmb{$\mc V$} resp. \pmb{$\wht{\mc V}$} be the set of chiral resp. anti chiral fields. They can be viewed as algebraic structures. (We will say more about such structures in the future.) 


Let $\mc H_0$ (resp. $\wht{\mc H}_0$) be the closure of the subspace spanned by $\varphi(f_1)\cdots\varphi(f_n)\Omega$ where each $f_i\in C_c^\infty(\Rc)$ and $\varphi_i\in\mc V$ (resp. $\varphi_i\in\wht{\mc V}$). Then $\mc H_0$ can be viewed as a (unitary) representation of $\mc V$, called the \textbf{vacuum representation}. Clearly $\Omega\in\mc H_0\cap\wht{\mc H}_0$. 

A basic assumption of unitary full CFT is the existence of orthogonal decomposition
\begin{align}\label{eq27}
\mc H=\bigoplus_{i\in\fk I}\mc H_i\otimes\wht{\mc H}_i\qquad\supset\mc H_0\otimes\wht{\mc H}_0
\end{align}
where each $\mc H_i$ (resp. $\wht{\mc H}_i$) is an irreducible unitary representation of $\mc V$ (resp. $\wht{\mc V}$). Here, $\bigoplus$ could be a finite, or infinite discrete, or even continuous (i.e. a direct integral). A large class of important CFTs are called \textbf{rational CFTs} \index{00@Rational CFT}, which means that the direct sum is finite. Here, $\mc H_0$ is identified with $\mc H_0\otimes\Omega$ so that it can thus be viewed as a subspace of $\mc H$; similarly $\wht{\mc H}_0\simeq\Omega\otimes\wht{\mc H}_0$. Therefore, with respect to the decomposition \eqref{eq27}, the vacuum vector $\Omega\in\mc H$ can be written as $\Omega\otimes\Omega$. 


\subsection{}


From now on, we slightly change our notation a bit: 
\begin{cv}
An element of $\WDS$ is not viewed as a function $\wtd f:\Rbb\rightarrow\Rbb$, but rather a multivalued smooth function $f:\Sbb^1\rightarrow\Sbb^1$ related to the original $\wtd f$ by
\begin{align*}
f(e^{\im\theta})=\wtd f(\theta)
\end{align*}
Following this convention, and similar to \eqref{eq28}, we define
\begin{align}
f'(e^{\im\theta})\equiv\partial_z f(e^{\im\theta})=\frac{\wtd f'(\theta)}{\im e^{\im\theta}}
\end{align}
Similarly, for each $\Phi\in\mc Q$, we let
\begin{align}
\mathring{\Phi}(e^{\im u},e^{\im v})\xlongequal{\mathrm{def}}\wtd\Phi(u,v)=\Phi\big(\frac{u+v}2,\frac{u-v}2\big)
\end{align}
viewing $\mathring\Phi$ as a multivalued function on $\Sbb^1\times\Sbb^1$.
\end{cv}




Although the projective unitary representation $\mc U$ of $\WDS^2$ does not fix $\Omega$ up to $\Sbb^1$-multiplications, similar to \eqref{eq2}, there is a large class of $\Phi\in\mc Q$, called \textbf{primary fields}, satisfying the \textbf{conformal covariance property}: For each such $\Phi$, there exist $\delta,\wht\delta\in \Rbb_{\geq0}$ (called the \textbf{conformal weights} \index{00@Conformal weight} of $\Phi$) such that for all $(g,h)\in\WDS^2$ and 
\begin{align}\label{eq30}
\mc U(g,h)\mathring\Phi(e^{\im u},e^{\im v})\mc U(g,h)^{-1}=g'(e^{\im u})^\delta h'(e^{\im v})^{\wht\delta}\cdot\mathring\Phi(g(e^{\im u}),h(e^{\im v}))
\end{align}
in the sense of smeared operators.


\subsection{}

In the special case that $\varphi\in\mc V$, then \eqref{eq29} says that $\mathring\varphi$ is a single-valued function on $\Sbb^1$, and hence has a Fourier series expansion
\begin{align}
\mathring\varphi(z)=\sum_{n\in\Zbb}\mathring\varphi_nz^{-n-1}
\end{align}
So $\mathring\varphi_n=\Res_{z=0}\mathring\varphi(z)z^ndz$. The derivative $\mathring\varphi'(z)=\partial_z\mathring\varphi(z)$ is understood in the usual way, i.e., 
\begin{align*}
\mathring\varphi'(z)=\sum_n (-n-1)\mathring{\varphi}_nz^{-n-2}
\end{align*}
Now, writing $\mc U(g,1)$ as $\mc U(g)$, then for primary chiral $\varphi$, \eqref{eq30} becomes
\begin{align}\label{eq31}
\mc U(g)\mathring\varphi(z)\mc U(g)^{-1}=g'(z)^\delta\cdot\mathring \varphi(g(z))
\end{align}
We simply call $\delta$ the \textbf{conformal weight} of the chiral field $\varphi$. If \eqref{eq31} only holds for $g\in\UPSU$, we say that the chiral field $\varphi$ is \textbf{quasi-primary}.

\begin{rem}
For each primary (resp. quasi-primary) chiral $\varphi$, and for each $m\in\Zbb$ (resp. $m=0,\pm1$), we have
\begin{subequations}\label{eq32}
\begin{align}\label{eq32a}
[L_m,\mathring\varphi(z)]=z^{m+1}\mathring\varphi'(z)+\delta\cdot (m+1) z^m\mathring\varphi(z)
\end{align}
Equivalently, for each $n\in\Zbb$ we have
\begin{align}
[L_m,\mathring\varphi_n]=-(m+n+1)\mathring\varphi_{m+n}+\delta\cdot(m+1)\mathring\varphi_{m+n}
\end{align}
\end{subequations}
\end{rem}

\begin{proof}[Heuristic proof]
Let $t\mapsto g_t$ be the one-parameter group generated by $\xk=\sum_m a_ml_m$ (a finite sum) satisfying $\xk^\dagger=-\xk$, i.e., $\ovl{a_m}=-a_{-m}$. So $g_0(z)=z$ and $\partial_t g_t(z)\big|_{t=0}=\sum_m a_mz^{m+1}$. Set $X=\sum_m a_mL_m$. Then, informally, we have
\begin{gather*}
\frac d{dt} \mc U(g_t)\mathring\varphi(z)\mc U(g_t)^{-1}\big|_{t=0}=[X,\mathring\varphi(z)]
\end{gather*}
Also
\begin{align*}
\frac d{dt}\mathring\varphi(g_t(z))\big|_{t=0}=\mathring\varphi'(z)\cdot\partial_tg_t(z)\big|_{t=0}=\sum_m a_m z^{m+1}\mathring\varphi'(z)
\end{align*}
Since $\delta\cdot g_0'(z)^{\delta-1}=\delta\cdot  \big(\frac d{dz}(z)\big)^{\delta-1}=\delta$, we have
\begin{align*}
\frac d{dt}g_t'(z)^\delta\big|_{t=0}=\delta\cdot g'_0(z)^{\delta-1}\cdot \partial_t g_t'(z)\big|_{t=0}=\delta\sum_m (a_mz^{m+1})'=\delta\sum_m (m+1)a_m z^m
\end{align*}
Combining the above three results with \eqref{eq31}, we get \eqref{eq32a}.
\end{proof}


\newpage




\section{Local fields and chiral algebras}

In this section, we introduce a rigorous approach to the algebra $\mc V$ of chiral fields. We will give an axiomatic description of (the modes of) the chiral fields acting on $\Vbb$, the dense subspace of $\mc H_0$ with finite $L_0$-spectra. (So $\mc H_0$ is the Hilbert space completion of $\Vbb$.) Some of the proofs will be sketched or even omitted. But details can be found in \cite{Gui-V} (especially Sec. 7 and 8).

\subsection{}\label{lb28}



Unless otherwise stated, we fix a complex inner product space $\Vbb$ together with a diagonalizable operator $L_0\in\End(\Vbb)$ such that \uwave{the eigenvalues of $L_0$ belong to $\Nbb$}. Thus, we have \uwave{orthogonal} decomposition $\Vbb=\bigoplus_{n\in\Nbb}\Vbb(n)$ where $\Vbb(n)=\{v\in\Vbb:L_0v=nv\}$. If $v\in\Vbb$, we say that $v$ is \textbf{homogeneous} if $v\in\Vbb(n)$ for some $n$; in that case we write
\begin{align*}
\wt(v)=n
\end{align*}
The Hilbert space completion of $\Vbb$ is denoted by $\mc H_\Vbb$. \index{HV@$\mc H_\Vbb$, the Hilbert space completion of $\Vbb$} We assume that each $\Vbb(n)$ is finite-dimensional so that $\Vbb(n)^{**}=\Vbb(n)$. Define 
\begin{align*}
\Vbb^\ac=\prod_{n\in\Nbb}\Vbb(n)
\end{align*}
the \textbf{algebraic completion} \index{00@Algebraic completion} of $\Vbb$ \index{Vac@$\Vbb^{\ac}$}. Then clearly
\begin{align*}
\Vbb\subset\mc H_\Vbb\subset\Vbb^\ac
\end{align*}
Note that $L_0$ acts on $\Vbb^\ac$ in a canonical way by acting on each $\Vbb(n)$ as $n\cdot\id$. Similarly, for each $q\in\Cbb^\times$, $q^{L_0}$ acts on $\Vbb^\ac$. 

For each $n\in\Nbb$, we define the projection onto the $n$-th component \index{Pn@$P_n:\Vbb^\ac\rightarrow\Vbb(n)$}
\begin{align}
P_n:\Vbb^\ac\rightarrow\Vbb(n)
\end{align}
Then for any $\xi\in\Vbb^\ac$, it is clear that
\begin{align}
\xi\in\HV\qquad\Longleftrightarrow\qquad \sum_{n\in\Nbb}\Vert P_n\xi\Vert^2<+\infty
\end{align}
Note that $L_0$ and $q^{L_0}$ commute with $P_n$. We also let \index{Pn@$P_{\leq n}$}
\begin{align}
P_{\leq n}=\sum_{k\in\Nbb,k\leq n}P_n
\end{align}










\subsection{}


\begin{df}
An \textbf{(homogeneous) field} on $\Vbb$ is an element
\begin{align*}
A(z)=\sum_{n\in\Zbb}A_nz^{-n-1}\in\End(\Vbb)[[z^{\pm1}]]
\end{align*}
(where each $A_n$ is in $\End(\Vbb)$) satisfying
\begin{subequations}\label{eq33}
\begin{align}\label{eq33a}
[L_0,A(z)]=\wt(A)\cdot A(z)+z\partial_z A(z)
\end{align}
for some $\wt(A)\in\Nbb$ (called the \textbf{(conformal) weight}) \index{00@Conformal weight} of $A(z)$; equivalently,
\begin{align}\label{eq33b}
[L_0,A_n]=(\wt(A)-n-1)A_n
\end{align}
\end{subequations}
\end{df}



\begin{rem}\label{lb34}
Note that by \eqref{eq33b}, for each $d$, $A_n$ restricts to
\begin{align}\label{eq36}
A_n: \Vbb(d)\rightarrow\Vbb(d+\wt(A)-n-1)
\end{align}
Since no nonzero homogeneous vectors can have negative weights, we see that $A_nv=0$ when $n\gg0$, and that $\bk{A_n\cdot|v}=0$ when $n\ll 0$. Thus
\begin{align}\label{eq35}
A(z)v\in\Vbb((z))
\end{align}
for each homogeneous $v\in\Vbb$, and hence for all $v\in\Vbb$. This is called the \textbf{lower truncation property}. \index{00@Lower truncation property} 
\end{rem}

Note that $A_n$ can be extended to $A_n^{\tr\tr}:\Vbb^\ac\rightarrow\Vbb^\ac$. We abbreviate $A_n^{\tr\tr}$ to $A_n$ when no confusion arises.


\begin{eg}
The field $\idt(z)=\id_\Vbb$ is called the \textbf{vacuum field}. \index{1@$\idt$, the vacuum field} By \eqref{eq33}, we clearly have
\begin{align*}
\wt(\idt)=0
\end{align*}
\end{eg}



\subsection{}


Let $A(z)$ be a homogeneous field. By \eqref{eq36}, we have a well defined linear map $(A_n)^\dagger:\Vbb\rightarrow\Vbb$ being the formal adjoint of $A_n$, i.e.,
\begin{align*}
\bk{A_nu|v}=\bk{u|(A_n)^\dagger v}
\end{align*}
This is because the restriction $A_n:\Vbb(d)\rightarrow\Vbb(d+\wt(A)-n-1)$ has an adjoint due the the finite-dimensionality. Thus $(A_n)^\dagger$ restricts to
\begin{align}\label{eq38}
(A_n)^\dagger:\Vbb(d)\rightarrow\Vbb(d-\wt(A)+n+1)
\end{align}

If $z$ is a formal variable, we understand $\ovl z\equiv z^\dagger$ as the formal conjugate of $z$. So $z,\ovl z$ are mutually commuting formal variables.

\begin{df}\label{lb20}
Define the \textbf{quasi-primary contragredient} \pmb{$A^\theta(z)$} \index{A@$A^\theta(z)$, the quasi-primary contragredient of $A(z)$} of $A(z)$ to be
\begin{align}
A^\theta(z)=(-z^{-2})^{\wt(A)}A(\ovl{z^{-1}})^\dagger=(-z^{-2})^{\wt(A)}\cdot \sum_{n\in\Zbb} (A_n)^\dagger z^{n+1}
\end{align}
One shows easily that
\begin{align}\label{eq37}
A^\theta_n=(-1)^{\wt(A)}\cdot (A_{-n-2+2\wt(A)})^\dagger
\end{align}
Comparing \eqref{eq37} with \eqref{eq38}, we see that $A_n^\theta$ restricts to $\Vbb(d)\rightarrow\Vbb(d+\wt(A)-n-1)$. Hence $A^\theta$ is homogeneous with weight
\begin{align}\label{eq40}
\wt(A^\theta)=\wt(A)
\end{align}
One checks easily that $A^{\theta\theta}=A$.
\end{df}

The reason we need the extra term $(-z^{-2})^{\wt(A)}$ will be clear when studying PCT symmetry for chiral CFTs in the future (cf. Thm. \ref{lb90}). At present, we at least know that part of the reasons we need $z^{-2}$ and its power $\wt(A)$ is because we want \eqref{eq40} to be true.



\subsection{}

\begin{rem}\label{lb16}
The field $A^\theta(z)$ can also be understood in the following way: For each $u,v\in\Vbb$ we have
\begin{align}\label{eq39}
\bk{A^\theta(z)u|v}=(-z^{-2})^{\wt(A)}\bk{u|A(\ovl{z^{-1}})v}
\end{align}
as elements of $\Cbb[[z^{\pm1}]]$. By \eqref{eq35}, the LHS resp. RHS is in $\Cbb((z))$ resp. $\Cbb((z^{-1}))$, we conclude that \eqref{eq39} is in $\Cbb[z^{\pm1}]$. Similarly,
\begin{align*}
\bk{A(z)u|v}\in\Cbb[z^{\pm1}]
\end{align*}
Thus, $z\in\Cbb^\times\rightarrow\bk{A(z)u|v}\in\Cbb$ is a holomorphic function with finite poles at $0,\infty$, and \eqref{eq39} holds in $\scr O(\Cbb^\times)$. It follows that for each $m,n\in\Vbb$, 
\begin{align*}
z\in\Cbb^\times\mapsto P_m A(z)P_n
\end{align*}
is an $\Hom(\Vbb(n),\Vbb(m))$-valued holomorphic function.
\end{rem}



\begin{pp}\label{lb18}
Let $u,v\in\Vbb$. Let $A$ be a homogeneous field. Then for each $z,q\in\Cbb^\times$ we have
\begin{align}\label{eq41}
\bk{q^{L_0}A(z)q^{-L_0}u|v}=q^{\wt(A)}\cdot\bk{A(qz)u|v}
\end{align}
\end{pp}

In short, we have $q^{L_0}A(z)q^{-L_0}=q^{\wt(A)}A(qz)$ as linear maps $\Vbb\rightarrow\Vbb^\ac$. Compare this with Eq. \eqref{eq31}.

\begin{proof}
For each fixed $q\in\Cbb^\times$, by expanding both sides of \eqref{eq41} as Laurent series of $z$, we see that \eqref{eq41} is equivalent to
\begin{align}
\bk{q^{L_0}A_nq^{-L_0}u|v}=q^{\wt(A)-n-1}\bk{A_nu|v}
\end{align}
By linearity, it suffices to assume that $u,v$ are homogenous. In that case, this relation follows immediately from \eqref{eq36}.
\end{proof}




\subsection{}


\begin{df}
Let $A(z),B(z)$ be homogeneous fields on $\Vbb$. We say that $A(z),B(z)$ are mutually \textbf{local} \index{00@Local fields} if there exists $N\in\Nbb$ (depending on $A,B$) such that the following relation holds in $\End(\Vbb)[[z^{\pm1},w^{\pm1}]]$: 
\begin{align}\label{eq34}
(z-w)^N[A(z),B(w)]=0
\end{align}
We call $N$ an \textbf{order of pole between $A,B$}.
\end{df}


\begin{rem}
A field $A(z)$ is not necessarily local to itself. If $A(z)$ is local to $A(z)$, we say that $A(z)$ is \textbf{self-local}.\index{00@Self-local fields} A collection of fields $(A^i(z))_{i\in I}$ is called \textbf{mutually local} \index{00@Mutually local fields} if $A^i(z)$ is local to $A^j(z)$ whenever $i,j\in I$ and $i\neq j$.
\end{rem}


Eq. \eqref{eq34} needs explanation. Let $R$ be a $\Cbb$-algebra. Write $z_\blt=(z_1,\dots,z_k)$. Then $R[[z_\blt^{\pm1}]]$ is an $R[z_\blt]$-module. However, this module is not necessarily torsion-free:

\begin{eg}\label{lb14}
Fix $N\in\Nbb$. Let $\alpha,\beta$ be the expansions of the meromorphic function $(z_1-z_2)^{-N}$ in $|z_1|<|z_2|$ and $|z_1|>|z_2|$, i.e.
\begin{align*}
\alpha=\sum_{j\in\Nbb} {-N\choose j}z_1^j(-z_2)^{-N-j}\qquad \beta=\sum_{j\in\Nbb}{-N\choose j}z_1^{-N-j}(-z_2)^j
\end{align*}
Then $\alpha\in \Cbb[[z_2^{\pm1}]][z_1]$ and $\beta\in\Cbb[[z_1^{\pm 1}]][z_2]$, and both belong to $\Cbb[[z_\blt^{\pm1}]]$. So $\alpha\neq \beta$ as elements of $\Cbb[[z_\blt^{\pm1}]]$. However, $(z_1-z_2)^N\alpha=(z_1-z_2)^N\beta=1$. Thus $\alpha-\beta$ is an torsion element of the $\Cbb[z_\blt]$-module $\Cbb[[z_\blt^{\pm1}]]$.
\end{eg}





Then $R[[z_\blt^{\pm1}]]$ is not naturally a $\Cbb$-algebra. In particular, not every two elements of $R[[z_\blt^{\pm1}]]$ can be multiplied. For example, the square of $\sum_{n\in\Zbb}z^n$ does not make sense. Moreover, the associativity of products does not necessarily hold even if the elements involved can be multiplied, as shown by Exp. \ref{lb15}. %Indeed, for $f,g,h\in R[[z_\blt^{\pm1}]]$, we only have $(fg)h=f(gh)$ when two of $f,g,h$ are in $R((z_\blt))$. (This is because $R[[z_\blt^{\pm1}]]$ is an $R((z_\blt))$-module.)

\begin{eg}\label{lb15}
In Exp. \ref{lb14}, both  $(\alpha\cdot (z_1-z_2)^N)\cdot \beta$ and $\alpha\cdot ((z_1-z_2)^N\cdot\beta)$ can be defined. However,
\begin{gather*}
(\alpha\cdot (z_1-z_2)^N)\cdot \beta=\beta\qquad \alpha\cdot ((z_1-z_2)^N\cdot \beta)=\alpha
\end{gather*}
\end{eg}




\subsection{}\label{lb23}

Assume that $A,B$ are mutually local fields with order of pole $N$. Choose any $u,v\in\Vbb$. By Rem. \ref{lb16}, we have $\bk{A(z)B(w)u|v}=(-z^{-2})^{\wt(A)}\bk{B(w)|A^\theta(\ovl{z^{-1}})v}$, which belongs to $\Cbb((z^{-1},w))$ by the lower truncation property \eqref{eq35}. Thus
\begin{align*}
\bk{A(z)B(w)u|v}\in \Cbb((z^{-1},w))\qquad\bk{B(w)A(z)u|v}\in\Cbb((z,w^{-1}))
\end{align*}
Therefore, setting
\begin{align*}
g:=(z-w)^N\bk{A(z)B(w)u|v}=(z-w)^N\bk{B(w)A(z)u|v}
\end{align*}
we have that
\begin{align*}
g\in\Cbb((z^{-1},w))\cap\Cbb((z,w^{-1}))=\Cbb[z^{\pm1},w^{\pm1}]
\end{align*}
Since the $\alpha(z,w),\beta(z,w)$ in Exp. \ref{lb16} are respectively the inverses of $(z-w)^N$ in the $\Cbb$-algebras $\Cbb((z^{-1},w))$ and $\Cbb((z,w^{-1}))$, we see that
\begin{align*}
\bk{A(z)B(w)u|v}=\beta g\qquad \bk{B(w)A(z)u|v}=\alpha g
\end{align*}


Consequently, the series $\bk{A(z)B(w)u|v}$ of $z,w$ converges \textbf{absolutely and locally uniformly (a.l.u)} \index{00@a.l.u.=absolute and locally uniform} on the region $\{(z,w)\in\Cbb:0<|w|<|z|\}$ in the sense that it converges uniformly on any compact subset of that open set. This is because the series $\beta g$ converges a.l.u. on this domain. 

When $u,v$ are homogeneous, one sees easily that this a.l.u. convergence is equivalent to that of 
\begin{align*}
\sum_{n\in\Nbb}\bk{A(z)P_nB(w)u|v}
\end{align*}
viewed as a series of functions of $z,w$ on $\{0<|w|<|z|\}$. (This is because for each $n$, $\bk{A(z)P_nB(w)u|v}$ is a monomial of $z,w$.) Thus, by linearity, the a.l.u. convergence of this series of functions also holds for any $u,v\in\Vbb$. Similarly,
\begin{align*}
\sum_{n\in\Nbb}\bk{B(w)P_nA(z)u|v}
\end{align*}
converges a.l.u. on $\{0<|z|<|w|\}$. Moreover, the limit functions of these two series can be analytically extended to the same holomorphic function on $\Conf^2(\Cbb^\times)$, namely, the rational function $(z_1-z_2)^{-N}g(z_1,z_2)$.



\subsection{}


The results in the previous subsection can be generalized to the following theorem. The proof is similar, and hence will not be given here.  See \cite[Subsec. 8.2]{Gui-V} for details.


\begin{thm}\label{lb17}
Let $A^1,\dots,A^k$ be mutually local fields. Then for each $u,v\in\Vbb$ and each permutation $\sigma$ of $\{1,\dots,k\}$, the series of Laurent polynomials of $z_\blt$
\begin{align}\label{eq76}
\sum_{n_2,\dots,n_k\in\Nbb}\bk{A^{\sigma(1)}(z_{\sigma(1)})P_{n_2}A^{\sigma(2)}(z_{\sigma(2)})P_{n_3}\cdots P_{n_k} A^{\sigma(k)}(z_{\sigma(k)})u|v}
\end{align} 
converges a.l.u. on
\begin{align}
\{z_\blt\in\Cbb^k:0<|z_{\sigma(k)}|<\cdots<|z_{\sigma(1)}|\}
\end{align}
and can be extended to some $f_{u,v}\in\scr O(\Conf^k(\Cbb^\times))$ independent of $\sigma$. Indeed, $f_{u,v}$ is a rational function.
\end{thm}

\begin{rem}
We say that $u$ is \textbf{vacuum with respect to $A(z)$} \index{00@Vacuum with respect to a field} if $A(z)u\in\Vbb[[z]]$, i.e., if $A_nu=0$ if $n\geq0$. If $u$ is vacuum with respect to $A^1,\dots, A^k$, the same argument as in Subsec. \ref{lb23} shows that $f_{u,v}\in\scr O(\Conf^k(\Cbb))$. Thus \eqref{eq76} converges a.l.u. on
\begin{align*}
\{z_\blt\in\Cbb^k:|z_{\sigma(k)}|<\cdots<|z_{\sigma(1)}|\}
\end{align*}
\end{rem}



\begin{df}\label{lb46}
In the setting of Thm. \ref{lb17}, for each $u\in\Vbb$ and $z_\blt\in\Conf^k(\Cbb^\times)$, define
\begin{align}
A^1(z_1)\cdots A^k(z_k)u\in \Vbb^\ac
\end{align}
to be the one whose inner product with any $v\in\Vbb$ is $f_{u,v}(z)$. Thus $A^1(z_1)\cdots A^k(z_k)$ is a linear map $\Vbb\rightarrow\Vbb^\ac$, and for each $u,v\in\Vbb$ the function
\begin{align}\label{eq70}
z_\blt\in\Conf^k(\Cbb^\times)\mapsto \bk{A^1(z_1)\cdots A^k(z_k)u|v}\in\Cbb
\end{align}
is holomorphic. When $u$ is vacuum with respect to $A^1,\dots, A^k$, the same conclusion holds if we replace $\Cbb^\times$ with $\Cbb$.
\end{df}

It is clear that for each permutation $\sigma$ of $\{1,\dots,k\}$ we have
\begin{align}\label{eq77}
A^{\sigma(1)}(z_{\sigma(1)})\cdots A^{\sigma(k)}(z_{\sigma(k)})u=A^1(z_1)\cdots A^k(z_k)u
\end{align}
Some of the results about single operators can be generalized to products of operators:

\begin{pp}\label{lb19}
Let $A^1,\dots,A^k$ be mutually local fields. Then for each $z_\blt\in\Conf^k(\Cbb^\times)$ and $q\in\Cbb^\times$, we have in $\Hom(\Vbb,\Vbb^\ac)$ that
\begin{align}\label{eq42}
q^{L_0}A^1(z_1)\cdots A^k(z_k)=q^{\wt(A^1)+\cdots+\wt(A^k)}A^1(qz_1)\cdots A^k(qz_k)q^{L_0}
\end{align}
\end{pp}

\begin{proof}
Fix $q\in\Cbb^{\times}$ and $u,v\in\Vbb$. Let $f,g$ denote the LHS and the RHS of \eqref{eq42} inserted in $\bk{\cdot u|v}$. By Thm. \ref{lb17}, both $f$ and $g$ are holomorphic functions of $z_\blt\in\Conf^k(\Cbb^\times)$. Therefore, to prove $f=g$ on the connected region $\Conf^k(\Cbb^\times)$ it suffices to prove it on a nonempty open subset, say $\{0<|z_k|<\cdots<|z_1|\}$. In that case, the relation $f=g$ follows from the a.l.u. convergence in Thm. \ref{lb17} and the fact that for all $n_2,\dots,n_k\in\Nbb$ we have in $\Hom(\Vbb,\Vbb^\ac)$ that
\begin{align*}
&q^{L_0}A^1(z_1)P_{n_2}A^2(z_2)P_{n_3}\cdots P_{n_k}A^k(z_k)\\
=&q^{\wt(A^1)+\cdots+\wt(A^k)}A^1(qz_1)P_{n_2}A^2(qz_2)P_{n_3}\cdots P_{n_k}A^k(qz_k)q^{L_0}
\end{align*}
The latter is due to Prop. \ref{lb18} and the fact that $q^{L_0}$ commutes with each $P_{n_j}$.
\end{proof}


\begin{pp}\label{lb21}
Let $A^1,\dots,A^k$ be mutually local fields. Let $u,v\in\Vbb$. Then for each $z_\blt\in\Conf^k(\Cbb^\times)$ we have
\begin{align}\label{eq43}
\begin{aligned}
&\bk{A^1(z_1)\cdots A^k(z_k)u|v}\\
=&(-z_1^{-2})^{\wt(A^1)}\cdots (-z_k^{-2})^{\wt(A^k)}\bigbk{u\big|(A^k)^\theta\big(\ovl{z_k^{-1}}\big)\cdots(A^1)^\theta\big(\ovl{z_1^{-1}} \big)v}
\end{aligned}
\end{align}
\end{pp}


\begin{proof}
Similar to Prop. \ref{lb19}, it suffices to prove \eqref{eq43} when $0<|z_1|<\cdots<|z_k|$ (and hence $0<|\ovl{z_k^{-1}}|<\cdots<|\ovl{z_1^{-1}}|$). This special case follows from the a.l.u. convergence Thm. \ref{lb17} and Def. \ref{lb20}.
\end{proof}









\subsection{}


We now discuss a further generalization (or variant) of the convergence Thm. \ref{lb17}. Its proof gives another application of the trick of analytic continuation (as in the proof of Prop. \ref{lb19} and \ref{lb21}).

\begin{thm}\label{lb22}
Assume that $A^1,\dots,A^m$ and $B^1,\dots,B^k$ are mutually local fields. Let
\begin{align*}
O=\{(z_1,\dots,z_m,\zeta_1,\dots,\zeta_k)\in\Conf^{m+k}(\Cbb^{\times}):|z_i|>|\zeta_j|\text{ for all }i,j\}
\end{align*}
Then for each $u,v\in\Vbb$, the RHS of 
\begin{align}
\begin{aligned}
&\bk{A^1(z_1)\cdots A^m(z_m)B^1(\zeta_1)\cdots B^k(\zeta_k)u|v}\\
=&\sum_{n\in\Nbb}\bk{A^1(z_1)\cdots A^m(z_m)P_nB^1(\zeta_1)\cdots B^k(\zeta_k)u|v}
\end{aligned}
\end{align}
converges a.l.u. on $O$ to the LHS.
\end{thm}


\begin{proof}
It suffices to prove the a.l.u. on
\begin{align*}
O_r=\{(z_1,\dots,z_m,\zeta_1,\dots,\zeta_k)\in\Conf^{m+k}(\Cbb^{\times}):|z_i|>r|\zeta_j|\text{ for all }i,j\}
\end{align*}
for each $r>1$. In fact, we shall show that the series of functions
\begin{align*}
\sum_{n\in\Nbb} \bk{A^1(z_1)\cdots A^m(z_m)q^{L_0}P_nB^1(\zeta_1)\cdots B^k(\zeta_k)u|v}  \tag{a}\label{eq44}
\end{align*}
converges a.l.u. on $(z_\blt,\zeta_\star,q)\in O_r\times \Dbb_r^\times$ to
\begin{align*}
q^\delta\bk{A^1(z_1)\cdots A^m(z_m)B^1(q\zeta_1)\cdots B^k(q\zeta_k)q^{L_0}u|v}  \tag{b}\label{eq45}
\end{align*}
where $\delta=\wt(B^1)+\cdots+\wt(B^k)$. By Thm. \ref{lb17} and Prop. \ref{lb18}, on
\begin{align*}
O_r'=\{(z_\blt,\zeta_\star):0<r|\zeta_k|<\cdots<r|\zeta_1|<|z_m|<\cdots<|z_1|\}
\end{align*}
the series \eqref{eq44} is equivalent to
\begin{align*}
&\sum \bk{A^1(z_1)P_{\nu_2}\cdots P_{\nu_m}A^m(z_m)q^{L_0}P_nB^1(\zeta_1)P_{n_2}\cdots P_{n_k}B^k(\zeta_k)u|v}\\
=&\sum q^\delta\bk{A^1(z_1)P_{\nu_2}\cdots P_{\nu_m}A^m(z_m)P_nB^1(q\zeta_1)P_{n_2}\cdots P_{n_k}B^k(q\zeta_k)q^{L_0}u|v}
\end{align*}
and hence converges a.l.u. to \eqref{eq45}. Therefore, if we let $\sum_\nu f_\nu q^\nu$ be the Laurent series expansion of \eqref{eq45} (where $f_\nu\in\scr O(O_r)$), then this series converges a.l.u. on $O_r\times\Dbb_r^\times$, and this series equals the series \eqref{eq44} on $O_r'\times\Dbb_r^\times$. Thus $f_\nu$ equals the coefficient before $q^\nu$ of \eqref{eq44} on $O_r'$, and hence on $O_r$ by the holomorphicity of the coefficients (as functions on $O_r$). Thus \eqref{eq44} converges a.l.u. on $O_r$ to \eqref{eq45}.
\end{proof}

\begin{comment}
\begin{rem}
In Thm. \ref{lb22}, assume more over that $u$ is vacuum to $B^1,\dots,B^k$, and $v$ is vacuum to $(A^1)^\theta,\dots,(A^k)^\theta$. Let
\begin{align*}
\Gamma=\{(z_\blt,\zeta_\star)\in \Conf^{m+k}(\Pbb^1):z_i\neq0,\zeta_j\neq \infty\text{ for all }i,j\}
\end{align*}
Then by the proof of Thm. \ref{lb17}, one sees that the holomorphic function
\begin{gather*}
(z_\blt,\zeta_\star)\in\Conf^{m+k}(\Cbb^\times)\mapsto\\ (-z_1^2)^{\wt(A^1)}\cdots(-z_m^2)^{\wt(A^m)}\bk{A^1(z_1)\cdots A^m(z_m)B^1(\zeta_1)\cdots B^k(\zeta_k)u|v}
\end{gather*}
can be extended to a holomorphic function on $\Gamma$. 
\end{rem}
\end{comment}


The following theorem follows almost immediately from Thm. \ref{lb22}.

\begin{thm}\label{lb45}
Let $A^1,\dots,A^k$ be homogeneous fields such that any two distinct members of $A^1,\dots,A^k,(A^1)^\theta,\dots,(A^k)^\theta$ are mutually local. Let $v\in\Vbb$. Then we have a holomorhic function
\begin{gather}\label{eq93}
\begin{gathered}
\Conf^k(\Dbb^\times_1)\rightarrow \mc H_\Vbb\qquad z_\blt\mapsto A^1(z_1)\cdots A^k(z_k)v
\end{gathered}
\end{gather}
If $v$ is vacuum with respect to $A^1,\dots,A^k$, and if $\Conf^k(\Dbb^\times_1)$ is replaced by $\Conf^k(\Dbb_1)$, the function \eqref{eq93} is still holomorphic.
\end{thm}

\begin{proof}
Step 1. By Prop. \ref{lb21}, we have
\begin{align*}
&\sum_{n\in\Nbb}\Vert P_n  A^1(z_1)\cdots A^k(z_k)v\Vert^2\\
=&\sum_{n\in\Nbb}(-\ovl{z_1}^{-2})^{\wt(A^1)}\cdots(-\ovl{z_k}^{-2})^{\wt(A^k)}\\
&\cdot\bk{(A^k)^\theta(1/\ovl{z_k})\cdots (A^1)^\theta(1/\ovl{z_1})P_nA^1(z_1)\cdots A^k(z_k)v|v}
\end{align*}
By Thm. \ref{lb22}, this series converges a.l.u. on $\Conf^k(\Dbb_1^\times)$. Therefore, for each $z_\blt\in \Conf^k(\Dbb_1^\times)$ we have $A^1(z_1)\cdots A^k(z_k)v\in\mc H_\Vbb$. Moreover, the above a.l.u. convergence implies the a.l.u. convergence of the series of $\mc H_\Vbb$-valued functions
\begin{align*}
z_\blt\in\Conf^k(\Dbb_1^\times)\mapsto\sum_{n\in\Nbb}P_n A^1(z_1)\cdots A^k(z_k)v
\end{align*}
because the summands are mutually orthogonal for different $n$. Since the partial sums of this series are holomorphic, the limit of the above series (namely, $A^1(z_1)\cdots A^k(z_k)v$) is also holomorphic.\\[-1ex]

Step 2. We now address the case that $v$ is vacuum. We want to show that for each open disk $U\subset\Dbb_1^\times$ centered at $0$, if we let
\begin{align*}
\Gamma=\Conf^{k-1}(\Dbb_1\setminus U)
\end{align*} 
and define the holomorphic function $f:\Gamma\times U^\times\rightarrow\HV$ to be the restriction of \eqref{eq93} (where $U^\times=U\setminus\{0\}$), then $f$ can be extended to a holomorphic function on $\Gamma\times U$. The proof will be completed by replacing $z_k$ by any one of $z_1,\dots,z_k$. 

It suffices to prove that the Laurent series expansion $f=\sum_{n\in\Zbb}f_n(z_1,\dots,z_{k-1})z_k^n$ (where $f_n\in\scr O(\Gamma)$) satisfies $f_n=0$ for all $n<0$; then $\sum_{n\in\Zbb}f_nz_k^n$ converges a.l.u. on $\Gamma\times U$ to a holomorphic function extending $f$, finishing the proof. Since $\Gamma$ is connected, it suffices to prove $f_n=0$ on 
\begin{align*}
\{(z_1,\dots,z_{k-1})\in\Gamma:|z_1|>\cdots>|z_{k-1}|\}
\end{align*}
Choose any $(z_1,\dots,z_{k-1})$ in this set. Then for $z_k\in U$, and for each $w\in \Vbb$, we have
\begin{align*}
\bk{f(z_\blt)|w}=\sum_{n_2,\dots,n_k\in\Nbb} \bk{A^1(z_1)P_{n_2}\cdots P_{n_k}A^k(z_k)v|w}
\end{align*}
where  $\Res_{z_k=0}(\text{RHS})z_k^{-n-1}dz_k=0$ for $n<0$ (since $v$ is $A^k$-vacuum), noting that $\Res_{z_k=0}$ commutes with $\sum$ due to the a.l.u. convergence of the RHS above over $z_k\in U^\times$. Thus $f_n(z_1,\dots,z_{k-1})=0$ when $n<0$.
\end{proof}





\subsection{}



A linear combination of mutually local fields is clearly local to the original fields. It turns out that there is a non-associative ``product" $A_kB$ (where $k\in\Zbb$) that is local to any field $C$ whenever $A,B,C$ are mutually local. 

\begin{df}\label{lb24}
Let $A,B$ be mutually local fields. Let $k\in\Zbb$. For each $z\in\Cbb^\times$, define a linear map $(A_kB)(z):\Vbb\rightarrow\Vbb^\ac$ \index{AkB@$A_kB$ where $A,B$ are local fields} by
\begin{align}\label{eq46}
\bk{(A_kB)(z)u|v}=\oint_{\Gamma(z)}(\zeta-z)^k\bk{A(\zeta)B(z)u|v}\frac{d\zeta}{2\im\pi}
\end{align}
for each $u,v\in\Vbb$. Here, $\Gamma(z)$ is an anticlockwise circle around $z$. Clearly \eqref{eq46} is holomorphic over $z\in\Cbb^\times$. Let
\begin{align*}
(A_kB)_n:\Vbb\rightarrow\Vbb^\ac\qquad\bk{(A_kB)_nu|v}=\Res_{z=0}~z^n\bk{(A_kB)(z)u|v}dz
\end{align*}
So we have $(A_kB)(z)=\sum_{n\in\Zbb}(A_kB)_nz^{-n-1}$ in $\Hom(\Vbb,\Vbb^\ac)[[z]]$.
\end{df}


\subsection{}


Let $A,B$ be mutually local fields. 

\begin{thm}\label{lb25}
For each $n,k\in\Zbb$ we have
\begin{align}\label{eq47}
(A_kB)_n=\sum_{l\in\Nbb}(-1)^l{k\choose l}A_{k-l}B_{n+l}-\sum_{l\in\Nbb}(-1)^{k+l}{k\choose l}B_{k+n-l}A_l
\end{align}
\end{thm}

Note that by the lower truncation property \eqref{eq35}, the RHS of \eqref{eq47} is a finite sum when acting on each $v\in\Vbb$. 

\begin{proof}
Fix $w\in\Cbb^\times$. Then the function $f(z)=(z-w)^k\bk{A(z)B(w)u|v}$ is holomorphic on $z\in\Cbb^\times\setminus\{w\}$. Let $\Gamma_-,\Gamma_+$ be circles around $0$ with radii $<|w|$ and $>|w|$ respectively. Let $\Gamma(w)$ be a circle around $w$ and between $\Gamma_-$ and $\Gamma_+$. Then Cauchy's theorem implies that $\bk{(A_kB)(w)u|v}=\int_{\Gamma(w)}f(z)dz/2\im\pi$ equals $\int_{\Gamma_+-\Gamma_-}f(z)dz/2\im\pi$.


We compute that $\int_{\Gamma_+}f(z)\frac{dz}{2\im\pi}$ equals
\begin{align*}
\int_{\Gamma_+}(z-w)^k\bk{A(z)B(w)u|v}\frac{dz}{2\im\pi}=\int_{\Gamma_+}\sum_{\nu\in\Nbb} (z-w)^k\bk{A(z)P_\nu B(w)u|v}\frac{dz}{2\im\pi}
\end{align*}
By Thm. \ref{lb17}, the series in the integral converges uniformly on $z\in\Gamma_+$. Thus $\int_{\Gamma_+}$ and $\sum_\nu$ can be exchanged. Therefore
\begin{align*}
&\int_{\Gamma_+}f(z)\frac{dz}{2\im\pi}=\sum_{\nu\in\Nbb}\int_{\Gamma_+} (z-w)^k\bk{A(z)P_\nu B(w)u|v}\frac{dz}{2\im\pi}\\
=&\sum_{\nu\in\Nbb}\int_{\Gamma_+}\sum_{l\in\Nbb}{k\choose l}z^{k-l}(-w)^l \bk{A(z)P_\nu B(w)u|v}\frac{dz}{2\im\pi}\\
=&\sum_{\nu\in\Nbb}\sum_{l\in\Nbb}{k\choose l}(-w)^l\bk{A_{k-l}P_\nu B(w)u|v}=\sum_{l\in\Nbb}{k\choose l}(-w)^l\bk{A_{k-l} B(w)u|v}
\end{align*}
Similarly, since $(z-w)^k=\sum_{l\in\Nbb}{k\choose l}z^l(-w)^{k-l}$ when $z$ is on $\Gamma_-$, we have
\begin{align*}
\int_{\Gamma_-}f(z)\frac{dz}{2\im\pi}=\sum_{\nu\in\Nbb}\int_{\Gamma_-} (z-w)^k\bk{B(w)P_\nu A(z) u|v}\frac{dz}{2\im\pi}=\sum_{l\in\Nbb}{k\choose l}(-w)^{k-l}\bk{B(w)A_lu|v}
\end{align*}

To summarize, we have
\begin{align*}
\bk{(A_kB)(w)u|v}=\sum_{l\in\Nbb}{k\choose l}(-w)^l\bk{A_{k-l} B(w)u|v}-\sum_{l\in\Nbb}{k\choose l}(-w)^{k-l}\bk{B(w)A_lu|v}
\end{align*}
Applying $\Res_{w=0}w^n(\cdots)dw$ to both sides, we get \eqref{eq47}.
\end{proof}


\begin{co}\label{lb26}
Let $k\in\Zbb$. Then for each $n\in\Zbb$, the linear map $(A_kB)_n:\Vbb\rightarrow\Vbb^\ac$ has range in $\Vbb$. Moreover, $A_kB$ is a homogeneous field with weight
\begin{align}\label{eq48}
\wt(A_kB)=\wt(A)+\wt(B)-k-1
\end{align}
\end{co}

\begin{proof}
Eq. \eqref{eq47} shows that $(A_kB)_n$ sends each $\Vbb(d)$ to $\Vbb(d')$ where
\begin{align*}
&d'=d+(\wt(A)-k+l-1)+(\wt(B)-n-l-1)\\
=&d+(\wt(B)-k-n+l-1)+(\wt(A)-l-1)
\end{align*}
which equals $d+\wt(A_kB)-n-1$ if we let $\wt(A_kB)$ be the RHS of \eqref{eq48}.
\end{proof}



\subsection{}


With the help of $A_kB$, we obtain several equivalent descriptions of local fields:

\begin{thm}\label{lb27}
Let $A,B$ be homogeneous fields and $N\in\Nbb$. Then the following are equivalent.
\begin{enumerate}[label=(\arabic*)]
\item $A,B$ are mutually local with pole of order $N$.
\item For each $u,v\in\Vbb$, the series
\begin{gather}
\sum_{n\in\Nbb}\bk{A(z)P_nB(w)u|v}\qquad\text{and}\qquad \sum_{n\in\Nbb}\bk{B(w)P_nA(z)u|v}
\end{gather}
converge a.l.u. on
\begin{align}
\{(z,w)\in\Cbb^2:0<|w|<|z|\}\qquad\text{and}\qquad\{(z,w)\in\Cbb^2:0<|z|<|w|\}
\end{align}
respectively, and can be extended to a common function $f_{u,v}\in\scr O(\Conf^2(\Cbb^\times))$ such that $(z-w)^Nf_{u,v}$ is holomorphic on $(\Cbb^\times)^2$.
\item For each $j=0,1,\dots,N-1$ there exists a sequence $(C^j_n)_{n\in\Zbb}$ in $\End(\Vbb)$ such that for all $m,k\in\Zbb$ we have
\begin{align}\label{eq49}
[A_m,B_k]=\sum_{l=0}^{N-1}{m\choose l}C^l_{m+k-l}
\end{align}
\end{enumerate}
Moreover, if one of (1) and (2) is true, then $A_jB=0$ for all $j\geq N$, and (3) holds if for each $0\leq j\leq N$ we define $C^j(z)\equiv\sum_{n\in\Zbb}C^j_nz^{-n-1}$ to be
\begin{align*}
C^j(z)=(A_jB)(z)
\end{align*}
\end{thm}


\begin{proof}
(1)$\Rightarrow$(2) follows directly from Thm. \ref{lb17}.

(2)$\Rightarrow$(3):  Note that $A_jB$ can be defined and satisfies Cor. \ref{lb26} whenever (2) holds. Assume (2), and set $C^j(z)=(A_jB)(z)$. By Def. \ref{lb24}, if $j\geq N$ then 
\begin{align*}
\bk{(A_jB)(w)u|v}=\Res_{z=w}(z-w)^jf_{u,v}(z,w)dz=0
\end{align*}
because $z\mapsto (z-w)^jf_{u,v}(z,w)$ is holomorphic on a neighborhood at $w$. So $C^j=0$ for all $j\geq N$.

Fix $w\in\Cbb^\times$ and $g(z)=z^m\bk{A(z)B(w)u|v}$. Let $\Gamma_\pm,\Gamma(w)$ be as in the proof of Thm. \ref{lb25}. Then $\int_{\Gamma(w)}g(z)\frac{dz}{2\im\pi}=\int_{\Gamma_+-\Gamma_-}g(z)\frac{dz}{2\im\pi}$. Similar to the proof of Thm. \ref{lb25}, one computes that
\begin{gather*}
\int_{\Gamma_+}g(z)\frac{dz}{2\im\pi}=\bk{A_m B(w)u|v}\qquad \int_{\Gamma_-}g(z)\frac{dz}{2\im\pi}=\bk{B(w)A_mu|v}\\
\int_{\Gamma(w)}g(z)\frac{dz}{2\im\pi}=\sum_{l\in\Nbb}{m\choose l}w^{m-l}\bk{(A_lB)(w)u|v}
\end{gather*}
since $z^m=\sum_{l\in\Nbb}{m\choose l}(z-w)^lw^{m-l}$ when $z\in\Gamma(w)$. Thus, for all $w\in\Cbb^\times$ we have
\begin{align*}
\bk{[A_m,B(w)]u|v}=\sum_{l=0}^{N-1}{m\choose l}w^{m-l}\bk{C^l(w)u|v}
\end{align*}
Applying $\Res_{w=0} w^k(\cdots)dw$ to both sides, we get \eqref{eq49}.

(3)$\Rightarrow$(1): This is calculated by brutal force. Assume (3). Using $(z-w)^N=\sum_{j=0}^N{N\choose j}z^jw^{N-j}$, one computes that the coefficient before $z^{-m-1}w^{-n-1}$ of $(z-w)^N[A(z),B(w)]$ is
\begin{align*}
\Res_{z=0}\Res_{w=0}z^mw^n(z-w)^N[A(z),B(w)]dzdw=\sum_{l=0}^{N-1}\lambda_l C^l_{m+n+N-l}
\end{align*}
where $\lambda_l=\sum_{j=0}^N{N\choose j}(-1)^{N-j}{m+j\choose l}$ is a number depending on $N$ and $m\in\Zbb$. One shows that $p(z):=(1+z)^mz^N$ equals $\sum_{l\in\Nbb}\lambda_l z^l$ by first writing $p(z)$ as a polynomial of $(1+z)$, and then expanding each power of $1+z$. So $\lambda_l=0$ for $l< N$. This proves (1). See \cite[Subset. 7.8]{Gui-V} for details.
\end{proof}


\subsection{}

Thm. \ref{lb27} gives us useful methods of proving locality. In this subsection, we give applications of Thm. \ref{lb27}-(2). In the next subsection, we discuss applications of Thm. \ref{lb27}-(3). 

The following theorem is called \textbf{Dong's lemma} \index{00@Dong's lemma} or \textbf{Dong-Li's lemma}


\begin{thm}\label{lb29}
Let $A,B,C$ be mutually local fields. Then for each $k\in\Zbb$, $A_kB$ is local to $C$.
\end{thm}

\begin{proof}
Choose any $u,v\in\Vbb$. Define $g\in\scr O(\Conf^2(\Cbb^\times))$ by
\begin{align*}
g(z_2,z_3)=\Res_{z_1=z_2}(z_1-z_2)^n\bk{A(z_1)B(z_2)C(z_3)u|v}
\end{align*}
Using Def. \ref{lb24} and Thm. \ref{lb22}, one shows that
\begin{align*}
\sum_{n\in\Nbb}\bk{(A_kB)(z_2)P_nC(z_3)u|v}\qquad\text{resp.}\qquad \sum_{n\in\Nbb}\bk{C(z_3)P_n(A_kB)(z_2)u|v}
\end{align*}
converges a.l.u. on
\begin{align*}
\{(z_2,z_3)\in\Cbb^2:0<|z_3|<|z_2|\}\qquad\text{resp.}\qquad\{(z_2,z_3)\in\Cbb^2:0<|z_2|<|z_3|\}
\end{align*}
to $g(z_2,z_3)$. Moreover, since $\bk{A(z_1)B(z_2)C(z_3)u|v}$ is a rational function of $z_1,z_2,z_3$, one checks easily that $g$ has finite poles at $z_2-z_3=0$. Thus $A_kB$ is local to $C$ by Thm. \ref{lb27}-(2). See \cite[Subsec. 8.7]{Gui-V} for details.
\end{proof}


\begin{co}
Let $A,B$ be mutually local fields. Define $\partial A\equiv A'$ to be $\partial_zA(z)=\sum_{n\in\Zbb}(-n-1)A_nz^{-n-2}$, equivalently,
\begin{align}\label{eq51}
(\partial A)_n=-nA_{n-1}
\end{align}
Then $\partial A$ is homegeneous of weight
\begin{align}\label{eq52}
\wt(\partial A)=\wt(A)+1
\end{align}
Moreover, $\partial A$ is local to $B$.
\end{co}

\begin{proof}
Eq. \eqref{eq52} is clear from \eqref{eq51}. Using \eqref{eq47} and \eqref{eq51}, one checks that
\begin{align}
\partial A=(A_{-2}\idt)
\end{align}
So the corollary follows from Thm. \ref{lb29}.
\end{proof}

Note that $A^\theta$ is not necessarily local to $B$ even if $A$ is local to $B$.




\subsection{}


\begin{eg}\label{lb40}
Let $c\geq0$. A field $T(z)=\sum_n  L_nz^{-n-2}$ of weight $2$ is called a \textbf{unitary Virasoro field} \index{00@Unitary virasoro field $T(z)$} (or stress-energy field) of \textbf{central charge} $c$ \index{00@Central charge} if $L_0$ coincides with the one in Subsec. \ref{lb28}, and
\begin{align}\label{eq50}
[L_m,L_n]=(m-n)L_{m+n}+\frac c{12}(m^3-m)\delta_{m+n,0}\qquad L_n^\dagger=L_{-n}
\end{align}
for all $m,n\in\Zbb$. Note that $L_n^\dagger=L_{-n}$ means $\bk{L_nu|v}=\bk{u|L_{-n}v}$. Thus $T_n=L_{n-1}$. The \textbf{Virasoro relation} \eqref{eq50} shows that $T(z)$ is self-local.
\end{eg}


\begin{proof}[Proof of self-locality]
Eq. \eqref{eq50} is equivalent to
\begin{align*}
[T_m,T_n]=(m-n)T_{m+n-1}+\frac c2{m\choose 3}\delta_{m+n-2,0}
\end{align*}
So $[T_m,T_n]=\sum_{l=0}^3{m\choose l}C^l_{m+n-l}$ if we set
\begin{align*}
C^0_k=-kT_{k-1} \qquad  C^1_k=2T_k\qquad C^2_k=0\qquad C^3_k=\frac c2\delta_{k+1,0}
\end{align*}
In other words,
\begin{align*}
C^0(z)=\partial_zT(z)\qquad C^1(z)=2T(z)\qquad C^2(z)=0\qquad C^3(z)=\frac c2
\end{align*}
Thus, by Thm. \ref{lb27}-(3), $T(z)$ is self-local.
\end{proof}


\subsection{}



\begin{df}\label{lb30}
We say that $(\mc V,\Vbb)$ is a \textbf{(quasi-primary unitary) chiral algebra} \index{00@Chiral algebra} if $\Vbb$ is as in Subsec. \ref{lb28}, and $\mc V$ is a set of homogeneous fields satisfying the following conditions:
\begin{enumerate}[label=(\arabic*)]
\item Creation property: There is a distinguished vector $\Omega\in\Vbb(0)$ such that $A(z)\Omega\in\Vbb[[z]]$ (i.e., $A_n\Omega=0$ if $n\geq1$) for all $A\in\mc V$.
\item Locality: Any two fields of $\mc V$ are mutually local. In particular, every field of $\mc V$ is self-local.
\item Cyclicity: Vectors of the form $A^1_{n_1}\cdots A^k_{n_k}\Omega$ (where $k\in\Nbb$, $A^1,\dots,A^k\in\mc V$, and $n_1,\dots,n_k\in\Zbb$) span $\Vbb$
\item M\"obius covariance: The operator $L_0$ can be extended to $\{L_0,L_{\pm1}\}$ satisfying for all $A\in\mc V$ and $m\in\{0,1,-1\}$ that
\begin{subequations}\label{eq53}
\begin{gather}\label{eq53a}
[L_m,A(z)]=z^{m+1}\partial_z A(z)+\wt(A)\cdot(m+1)z^mA(z)
\end{gather}
in $\End(\Vbb)[[z^{\pm1}]]$. Equivalently, for all $n\in\Zbb$ we have
\begin{gather}\label{eq53b}
[L_m,A_n]=-(m+n+1)A_{m+n}+\wt(A)\cdot(m+1)A_{m+n}
\end{gather}
\end{subequations}
Moreover, we have
\begin{align}\label{eq67}
L_n\Omega=0\qquad\text{for all }n=0,\pm1
\end{align}
\item $\theta$-invariance: If $A\in\mc V$, then $A^\theta\in\mc V$.
\end{enumerate}
\end{df}

\begin{rem}
The adjective ``quasi-primary" means that \eqref{eq53} holds for $A\in\mc V$. However, for $A,B\in\mc V$ and $k\in\Zbb$, the fields $\partial A$ and $A_kB$ satisfy \eqref{eq53} only for $m=-1,0$, but not necessarily for $m=1$. In other words, $\partial A$ and $A_kB$ are not necessarily quasi-primary. Non quasi-primary fields satisfy a more complicated M\"obius covariance formula.
\end{rem}


\begin{df}
A chiral algebra $(\mc V,\Vbb)$ is called \textbf{conformal} \index{00@Conformal chiral algebra} if $L_0,L_{\pm1}$ can be extended to a sequence $(L_n)_{n\in\Zbb}$ in $\End(\Vbb)$ such that the Virasoro relation \eqref{eq50} holds for some central charge $c$, and that $T(z)=\sum_{n\in\Zbb}L_nz^{-n-2}$ satisfies $\wt(T)=2$ and belongs to $\mc V$.

If $(\mc V,\Vbb)$ is a conformal chiral algebra, we say that $A\in \mc V$ is a \textbf{primary field} \index{00@Primary field} if $A$ satisfies \eqref{eq53} for all $m\in\Zbb$. \hfill\qedsymbol
\end{df}

\begin{rem}
Note that when $m=0,\pm 1$, the Virasoro relation specializes to $[L_m,L_n]=(m-n)L_{m+n}$ (for all $n\in\Zbb$). Thus $T(z)$ automatically satisfies \eqref{eq53}. However, if $c\neq 0$ and $m\neq 0,\pm1$, then \eqref{eq53} does not hold for $T(z)$. Thus $T(z)$ is not primary.
\end{rem}


\begin{rem}
When $(\mc V,\Vbb)$ is a conformal chiral algebra, then \eqref{eq67} is redundant, since the creation proerty for $T(z)$ implies that
\begin{align*}
L_n\Omega=0\qquad\text{for all }n=-1,0,1,2,3,\dots
\end{align*}
\end{rem}


\subsection{}


\begin{df}
A \textbf{unitary Lie algebra} \index{00@Unitary Lie algebra}  is defined to be a complex Lie algebra $\gk$ together with an inner product (called \textbf{invariant inner product}) \index{00@Invariant inner product} on $\gk$ and an \textbf{involution} \index{00@Involution} $\dagger$ (i.e., an antilinear map $\dagger:\gk\rightarrow\gk$ satisfying $X^{\dagger\dagger}=X$ for all $X\in\gk$) satisfying the following properties for all $X,Y,Z\in\gk$:
\begin{enumerate}[label=(\arabic*)]
\item $\bk{[X,Y]|Z}=\bk{Y|[X^\dagger,Z]}$, i.e., the representation $X\mapsto [X,-]$ is unitary.
\item $[X,Y]^\dagger=[Y^\dagger,X^\dagger]$.
\item $\dagger:\gk\rightarrow\gk$ is antiunitary. 
\end{enumerate}
If $W$ is an inner product space, we say that $\pi:\gk\rightarrow\End(W)$ is a \textbf{unitary representation} if $\pi([X,Y])=[\pi(X),\pi(Y)]$ and $\pi(X)^\dagger=\pi(X^\dagger)$ (i.e. $\bk{\pi(X)u|v}=\pi(u|\pi(X^)\dagger v)$) for all $X,Y\in\gk$. 
\end{df}

\begin{rem}
One can show that a finitely dimensional complex Lie algebra $\gk$ is unitary iff it is isomorphic (but not necessarily unitarily isomorphic) to $\zk\oplus\gk_1\oplus\cdots\oplus\gk_n$ where $\zk$ is abelian (i.e. $\simeq \Cbb^k$ for some $k\in\Nbb$) and $\gk_1,\dots,\gk_n$ are complex simple Lie algebras. There are no canonical choices of the invariant inner products on $\zk$, since any two inner products are unitarily equivalent. However, $\gk_i$ has a canonical choice of invariant inner product: the one under which the longest root has length $\sqrt 2$. See \cite[Ch. II]{Was-10} for details. 
\end{rem}


\begin{eg}\label{lb41}
Let $\gk$ be a finite-dimensional unitary Lie algebra. Let $l\in\Rbb_{>0}$. Suppose that $\mc V$ is a set of fields $X(z)=\sum_{n\in\Zbb}X_nz^{-n-1}$ (where $X\in\gk$) such that 
\begin{align}\label{eq68}
[X_m,Y_n]=[X,Y]_{m+n}+l\cdot m\bk{X|Y^\dagger}\delta_{m+n,0}\qquad (X_n)^\dagger=(X^\dagger)_{-n}
\end{align}
for all $X,Y\in\gk$ and $m,n\in\Zbb$. Using Thm. \ref{lb27}-(3), one checks that any two fields of $\mc V$ are mutually local. We call $X(z)$ a \textbf{current field}. \index{00@Current field}


Now assume that the creation property and the cyclicity in Def. \ref{lb30} holds for $(\mc V,\Vbb)$. Assume that $\gk$ is abelian resp. simple. Let $h^\vee$ be $0$ resp. the dual Coxeter number of $\gk$. Define $T(z)=\sum_{n\in\Zbb}L_nz^{-n-2}$ via the \textbf{Sugawara construction}
\begin{align}
T(z)=\frac 1{2(l+h^\vee)}\sum_{i} \big((E_i^\dagger)_{-1}E_i\big)(z)
\end{align}
where $(E_i)$ is an orthonormal basis of $\gk$. Then $T^\theta(z)=T(z)$ and $X^\theta(z)=-X^\dagger(z)$ (where $X\in\gk$), and $\mc V\cup\{T(z)\}$ is a conformal chiral algebra, and all $X(z)$ are primary with
\begin{align*}
\wt(X)=1
\end{align*}
We call $\mc V$ the \textbf{current algebra} \index{00@Current algebra} of $\gk$ with \textbf{level} \index{00@Level of a current algebra} $l$. See \cite[Sec. 6]{Gui-V} for details.


If $\gk$ is abelian, all $l>0$ are possible. In fact, in this case, $(\mc V,\Vbb)$ can be constructed from Bosonic Fock spaces. See \cite[Subsec. 6.13]{Gui-V}. If $\gk$ is simple and the invariant inner product is the canonical one (i.e., the one under which the longest root has length $\sqrt 2$), one can show that all possible $l>0$ form $\Zbb_+$. See \cite[Ch. III]{Was-10} for details.   \hfill\qedsymbol 
\end{eg}




\newpage


\section{Energy-bounded fields and their smeared fields}



We assume the setting of Subsec. \ref{lb28}. Recall that $\Vbb=\bigoplus_{n\in\Nbb}\Vbb(n)$ is graded by $L_0$. Therefore, $L_0$ is a symmetric operator on $\mc H$, and hence closable. Recall that $\oplus_n v_n\in\Vbb^\ac$ belongs to $\HV$ iff $\sum_n\Vert v_n\Vert^2<+\infty$. For each $n\in\Zbb$, define
\begin{align}
\ek_n:\Sbb^1\rightarrow\Cbb\qquad \ek_n(z)=z^n
\end{align}
For each $f\in C^\infty(\Sbb^1)$ and $t\in\Rbb$, define the $t$-th order Sobolev norm \pmb{$|f|_t$} to be
\begin{align}
|f|_t=\sum_{n\in\Zbb} (1+|n|)^t|\wht f(n)|
\end{align}
where $f=\sum_{n\in\Nbb}\wht f(n)\cdot \ek_n$ is the Fourier series. Note that $|f|_t<+\infty$ because $n\in\Zbb\mapsto \wht f(n)$ is rapidly decreasing.









\subsection{}

The following proposition implies, for example, that if $q\in\Cbb^\times$ then the action of $q^{L_0}$ on $\Vbb^\ac$ described in Subsec. \ref{lb28} is compatible with the Borel functional calculus $q^{\ovl{L_0}}$ on $\HV$. 

\begin{pp}\label{lb33}
The closure $\ovl {L_0}$ is a (closed self-adjoint) positive operator. Moreover, for each Borel function $f:\Rbb_{\geq0}\rightarrow\Cbb$ we have
\begin{subequations}\label{eq54}
\begin{align}\label{eq54a}
\Dom(f(\ovl {L_0}))=\Big\{\oplus_n v_n\in\mc H_\Vbb:\sum_n |f(n)|^2\Vert v_n\Vert^2<+\infty\Big\}
\end{align}
Moreover, $\Vbb$ is a core for $f(\ovl{L_0})$. For each $\oplus_n v_n\in\HV$ we have
\begin{align}\label{eq54b}
f(\ovl{L_0})(\oplus_n v_n)=\oplus_n f(n)v_n
\end{align}
\end{subequations}
\end{pp}


In particular, if $r\in\Rbb$, we understand $(1+\ovl{L_0})^r$ as $f(\ovl{L_0})$ where $f(x)=(1+x)^r$. 


\begin{proof}




%Clearly $L_0$ is symmetric. Since $\Rng(\im+L_0)$ and $\Rng(\im-L_0)$ both contain $\Vbb$, they are dense in $\HV$. Thus, by Prop. \ref{lb31}, $\ovl{L_0}$ is self-adjoint and has core $\Vbb$. Since $\bk{\ovl{L_0}v|v}\geq0$ for each $v\in\Vbb$, by Rem. \ref{lb32}, we have $\ovl{L_0}\geq0$.

By choosing an orthonormal basis for each $\Vbb(n)$, we can view $\HV$ as a direct sum $\bigoplus_{j\in J}L^2(\Nbb,\mu_j)$ where $J$ is a countable set and $\mu_j$ is a Dirac measure on $\Nbb$. Then clearly
\begin{align*}
\Vbb=\bigcup_{n\in\Nbb} M_{\chi_{[0,n]}}\HV
\end{align*}
This shows that $\Vbb$ is a core for the multiplication operator $M_f$ (since $(M_{\chi_{[0,n]}})_{n\in\Nbb}$ is a sequence of bounding projections for $M_f$, cf. \cite[Sec. 8]{Gui-S}).

Now let $x:\Nbb\rightarrow\Cbb,n\mapsto n$. Then $M_x|_\Vbb=L_0$. Thus $\ovl{L_0}$ is the positive operator $M_x$. Hence $f(\ovl{L_0})=f(M_x)=M_f$. Thus \eqref{eq54} follows from the fact that $\Dom(M_f)$ equals the RHS of \eqref{eq54a}, and that the action of $M_f$ on $\Dom(M_f)$ is described by the RHS of \eqref{eq54b}.
\end{proof}

\begin{co}\label{lb69}
Restrict each $P_n$ to a projection on $\HV$. Then the von Neumann algebras $\{\ovl{L_0}\}''$ and $\{P_n:n\in\Nbb\}''$ are equal.
\end{co}

\begin{proof}
By Prop. \ref{lb33}, we have $P_n=\chi_{\{n\}}(\ovl{L_0})$, and hence $P_n\in\{\ovl{L_0}\}''$. Thus $\{\ovl{L_0}\}''\supset\{P_n:n\in\Nbb\}''$. If $U$ is a unitary operator on $\HV$ commuting with each $P_n$, then $U$ preserves each $\Vbb(n)$. This implies $L_0U=L_0U$, and hence $U\in\{\ovl{L_0}\}'$. Thus $\{\ovl{L_0}\}''\subset\{P_n:n\in\Nbb\}''$.
\end{proof}



\subsection{}

\begin{df}
Let $r\in\Rbb$. The \textbf{\pmb{$r$}-th order Sobolev norm} \index{zz@$\Vert\cdot\Vert_r$, the $r$-th Sobolev norm} is defined to be
\begin{gather*}
\Vert\cdot\Vert_r:\Vbb^\ac\rightarrow[0,+\infty]\qquad \big\Vert\xi\big\Vert_r^2=\sum_{n\in\Nbb}(1+n)^{2r}\Vert P_n\xi\Vert^2
\end{gather*}
Moreover, if $r\geq0$, define the \textbf{\pmb{$r$}-th order Sobolev space} to be \index{HV@$\mc H_\Vbb^r,\HV^\infty$} 
\begin{align*}
\mc H_\Vbb^r:=\Dom((1+\ovl {L_0})^r)\xlongequal{\eqref{eq54a}} \{\xi\in\Vbb^\ac:\Vert\xi\Vert_r<+\infty\}
\end{align*}
Then, on $\HV^r$, the norm $\Vert\cdot\Vert_r$ is induced by the \textbf{\pmb{$r$}-th order Sobolev inner product} \index{zz@$\Vert\cdot\Vert_r$, the $r$-th Sobolev inner product}
\begin{align*}
\bk{\xi|\eta}_r=\sum_{n\in\Nbb}(1+n)^{2r}\bk{P_n\xi|P_n\eta}
\end{align*}
Clearly for $r\leq r'$ we have for all $\xi\in\Vbb^\ac$ that
\begin{align*}
\Vert\xi\Vert_r\leq\Vert\xi\Vert_{r'} \qquad\text{and hence} \HV^r\supset\HV^{r'}
\end{align*}
We set
\begin{align*}
\HV^\infty=\bigcap_{r\geq0}\HV^r
\end{align*}
Vectors in \pmb{$\HV^\infty$} are called \textbf{smooth vectors}. \index{00@Smooth vectors}
\end{df}

\begin{rem}\label{lb37}
Note that for each $r\in\Rbb$, the projection $P_n:(\HV^r,\bk{\cdot,\cdot}_r)\rightarrow (\HV^r,\bk{\cdot,\cdot}_r)$ onto $\Vbb(n)$ is a projection in the sense that $P_n^2=P_n$ and $\bk{P_n\xi|\eta}_r=\bk{\xi|P_n\eta}_r$ for all $\xi,\eta\in\HV^r$. Thus $P_n$ has operator norm $\leq 1$ under $\bk{\cdot,\cdot}_r$. The same can be said about $P_{\leq n}$.
\end{rem}


\begin{rem}\label{lb51}
If $T$ is a positive operator on a Hilbert space $\mc H$ satisfying $T\geq a$ for some $a>0$, then $\Dom(T)$ is complete under the inner product $\bk{\xi|\eta}_T:=\bk{T\xi|T\eta}$. It follows that for each $r\geq0$, $\HV^r$ is complete under $\bk{\cdot|\cdot}_r$.
\end{rem}


\begin{proof}
By (e.g.) spectral theory, $T^2-a^2$ is positive. Thus $\Vert T\xi\Vert^2\geq a^2\Vert\xi\Vert^2$. Thus, if $(\xi_n)$ is a Cauchy sequence in $\Dom(T)$ under $\bk{\cdot|\cdot}_T$, then $T\xi_n$ and $\xi_n$ both converge in $\mc H$. Let $\xi=\lim_n\xi_n$. Since $T$ is closed, we see that $(\xi_n,T\xi_n)$ converges in $\mc H\times\mc H$ to some $(\xi,T\xi)$ in the graph of $T$. So $\lim_n\Vert \xi-\xi_n\Vert_T=0$.
\end{proof}



\begin{df}
Let $r\geq0$. We say that a field $A(z)$ satisfies \textbf{\pmb{$r$}-th order energy bounds} \index{00@Energy bounds and energy bounded} if there exist $M,t\geq0$ such that for any $n\in\Zbb$ and $v\in\Vbb$ we have
\begin{align}\label{eq55}
\Vert A_nv\Vert\leq M(1+|n|)^t \Vert v\Vert_r
\end{align}
In other words, the following linear map
\begin{align*}
A_n:(\Vbb,\Vert\cdot\Vert_r)\rightarrow (\Vbb,\Vert\cdot\Vert)
\end{align*}
is bounded with operator norm $\leq M(1+|n|)^t$.

First order energy bounds are called \textbf{linear energy bounds}. \index{00@Linear energy bounds} A field satisfying $r$-th energy bounds for some $r\geq 0$ is called \textbf{(polynomial) energy-bounded}. \hfill\qedsymbol
\end{df}



\subsection{}


\begin{pp}\label{lb35}
Let $A(z)$ be a field satisfying \eqref{eq55}. Then for any $p\in\Rbb$, there exists $M_p\geq0$ such that for any $n\in\Zbb$ and $v\in\Vbb$ we have
\begin{align*}
\Vert A_nv\Vert_p\leq M_p(1+|n|)^{|p|+t}\Vert v\Vert_{p+r}
\end{align*} 
\end{pp}


\begin{proof}
We want to prove
\begin{align*}
\Vert A_nv\Vert_p^2\leq M_p^2(1+|n|)^{2(|p|+t)}\Vert v\Vert_{p+r}^2  \tag{a}\label{eq56}
\end{align*}
For different $m\in\Nbb$, $P_mv$ are mutually orthonormal under the $(p+r)$-th order Sobolev inner product. By Rem. \ref{lb34}, for different $m$, $A_nP_mv=P_{m+\wt(A)-n-1}A_nv$ are mutually orthonormal under the $p$-th order Sobolev inner product. Therefore, by replacing $v$ with $P_mv$, it suffices to prove \eqref{eq56} under the assumption that $v$ is homogeneous. We also assume that $A_nv\neq 0$. Then by Rem. \ref{lb34}, we have $\wt(A)+\wt(v)-1-n\geq0$.

By \eqref{eq55}, we have
\begin{align*}
\Vert A_nv\Vert^2\leq M^2(1+|n|)^{2t}(1+\wt(v))^{2r}\Vert v\Vert^2
\end{align*}
Thus, using Rem. \ref{lb34}, we get
\begin{align*}
&\Vert A_nv\Vert^2_p=(\wt(A)+\wt(v)-n)^{2p}\Vert A_nv\Vert^2\\
\leq& (\wt(A)+\wt(v)-n)^{2p}M^2(1+|n|)^{2t}(1+\wt(v))^{2r}\Vert v\Vert^2\\
=&\Big(\frac{\wt(A)-n+\wt(v)}{1+\wt(v)}\Big)^{2p}M^2(1+|n|)^{2t}(1+\wt(v))^{2(p+r)}\Vert v\Vert^2\\
=&\Big(\frac{\wt(A)-n+\wt(v)}{1+\wt(v)}\Big)^{2p}M^2(1+|n|)^{2t}\Vert v\Vert^2_{p+r}
\end{align*}
If $p\geq0$, then we can choose $M_p=(1+\wt(A))^pM$, since
\begin{align*}
&\Big(\frac{\wt(A)-n+\wt(v)}{1+\wt(v)}\Big)^{2p}\leq \Big(\frac{1+\wt(A)+|n|+\wt(v)}{1+\wt(v)}\Big)^{2p}\\
\leq&(1+\wt(A)+|n|)^{2p}\leq (1+\wt(A))^{2p}(1+|n|)^{2p}
\end{align*}
Now assume $p<0$. If $1\leq \wt(A)-n$, then
\begin{align*}
\Big(\frac{\wt(A)-n+\wt(v)}{1+\wt(v)}\Big)^{2p}=\Big(\frac{1+\wt(v)} {\wt(A)-n+\wt(v)}\Big)^{2|p|}\leq 1\leq (1+|2n|)^{2|p|}
\end{align*}
If $1\geq \wt(A)-n$, then since $\wt(A)-n+\wt(v)\geq1$ (cf. the first paragraph),
\begin{align*}
&\Big(\frac{1+\wt(v)} {\wt(A)-n+\wt(v)}\Big)^{2|p|}=\Big(1+\frac{1+n-\wt(A)} {\wt(A)-n+\wt(v)}\Big)^{2|p|}\\
\leq&(2+n-\wt(A))^{2|p|}\leq (2+2n+2\wt(A))^{2|p|}\leq 2^{2|p|}(1+\wt(A))^{2|p|}(1+|n|)^{2|p|}
\end{align*}
Therefore, if $p<0$, we can choose $M_p=2^{|p|}(1+\wt(A))^{2|p|}M$.
\end{proof}

\begin{co}
Assume that $A(z)$ satisfies $r$-th order energy bounds where $r\geq0$. Then the quasi-primary contragredient $A^\theta(z)$ (cf. Def. \ref{lb20}) also satisfies $r$-th order energy bounds.
\end{co}


\begin{proof}
Assume that $A(z)$ satisfies \eqref{eq55}. Then for each $u,v\in\Vbb$, we use \eqref{eq37} to compute that
\begin{align*}
&\big|\bk{A^\theta_nu|v}\big|=\big|\bk{u|A_{-n-2+\wt(A)}v}\big|=\big|\bk{(1+L_0)^ru|(1+L_0)^{-r}A_{-n-2+\wt(A)}v}   \big|\\
\leq& \Vert u\Vert_r\cdot \Vert A_{-n-2+\wt(A)}v\Vert_{-r}
\end{align*}
By Prop. \ref{lb35}, we have
\begin{align*}
\Vert A_{-n-2+\wt(A)}v\Vert_{-r}\leq M_{-r}(1+|n+2+\wt(A)|)^{r+t}\Vert v\Vert\leq C (1+|n|)^{r+t}\Vert v\Vert
\end{align*}
where $C=M_{-r}(2+\wt(A))^{r+t}$. Thus, for any $u\in\Vbb$, we have
\begin{align*}
\big|\bk{A^\theta_nu|v}\big|\leq C(1+|n|)^{r+t} \Vert u\Vert_r\cdot \Vert v\Vert
\end{align*}
for all $v\in\Vbb$, and hence $\Vert A^\theta_n u\Vert\leq C(1+|n|)^{r+t}\Vert u\Vert_r$.
\end{proof}







\subsection{}\label{lb44}

To prepare for the study of smeared fields we need:

\begin{lm}\label{lb36}
Let $F:\Sbb^1\rightarrow \Hom(\Vbb,\Vbb^\ac)$ satisfying the following properties:
\begin{enumerate}[label=(\alph*)]
\item For each $u,v\in\Vbb$, the function $z\in\Sbb^1\mapsto \bk{F(z)u|v}\in\Cbb$ is continuous.
\item For each $z\in\Sbb^1$, there is a (necessarily unique) $F(z)^\dagger\in\Hom(\Vbb,\Vbb^\ac)$ such that
\begin{align*}
\bk{F(z)u|v}=\bk{u|F(z)^\dagger v}
\end{align*}
for all $u,v\in\Vbb$.
\end{enumerate}
Then as elements of $\Hom(\Vbb,\Vbb^\ac)$ we have
\begin{align}\label{eq58}
\bigg(\oint\nolimits_{\Sbb^1}F(z)\frac{dz}{2\im\pi}\bigg)^\dagger=\oint\nolimits_{\Sbb^1}z^{-2}\cdot F(z)^\dagger \frac{dz}{2\im\pi}
\end{align}
Namely, defining $S,T\in\Hom(\Vbb,\Vbb^\ac)$ by
\begin{gather*}
\bk{Su|v}=\oint_{\Sbb^1}\nolimits\bk{F(z)u|v}\frac{dz}{2\im\pi}\qquad \bk{Tu|v}=\oint_{\Sbb^1}\nolimits\bk{z^{-2}F(z)^\dagger u|v}\frac{dz}{2\im\pi}
\end{gather*}
(for all $u,v\in\Vbb$), then $\bk{Su|v}=\bk{u|Tv}$.
\end{lm}

For example, by Rem. \ref{lb16}, if $A(z)$ is a homogeneous field, then $A$ satisfies the above properties, because for $z\in\Sbb^1$ we have
\begin{align}\label{eq61}
A(z)^\dagger=(-z^2)^{\wt(A)}A^\theta(z)
\end{align}

\begin{proof}
We compute that
\begin{align*}
&\bk{Su|v}=\int_{-\pi}^\pi \bk{F(e^{\im\theta})u|v}\cdot \frac{e^{\im\theta}d\theta}{2\pi}=\ovl{\int_{-\pi}^\pi \bk{v|F(e^{\im\theta})u}\cdot \frac{e^{-\im\theta}d\theta}{2\pi}}\\
=&\ovl{\int_{-\pi}^\pi \bk{F(e^{\im\theta})^\dagger v|u}\cdot \frac{e^{-\im\theta}d\theta}{2\pi}}=\ovl{\int_{-\pi}^\pi \bk{e^{-2\im\theta}F(e^{\im\theta})^\dagger v|u}\cdot \frac{e^{\im\theta}d\theta}{2\pi}}=\ovl{\bk{Tv|u}}
\end{align*}
\end{proof}


\begin{rem}
There is a non-rigorous but heuristic way to prove \eqref{eq58} (and to help memorize \eqref{eq58}). Note that $\ovl z=z^{-1}$ on $\Sbb^1$. Therefore
\begin{align*}
\bigg(\oint\nolimits_{\Sbb^1}F(z)\frac{dz}{2\im\pi}\bigg)^\dagger=\oint\nolimits_{\Sbb^1} F(z)^\dagger \frac{d{z^{-1}}}{-2\im\pi}=\oint\nolimits_{\Sbb^1}z^{-2} F(z)^\dagger\frac{dz}{2\im\pi}
\end{align*}
\end{rem}


\subsection{}

We now define smeared fields. Let $A(z)$ be a field satisfying $r$-th order energy bounds.

\begin{df}\label{lb38}
For each $f\in C^\infty(\Sbb^1)$, define the \textbf{smeared field} \index{Af@$A(f)$, the smeared field of $A(z)$}
\begin{gather*}
A(f):\Vbb\rightarrow\Vbb^\ac\\
\bk{A(f)u|v}=\oint\limits_{\Sbb^1} \bk{A(z)u|v}f(z)\frac{dz}{2\im\pi}\equiv \int_{-\pi}^\pi \bk{A(e^{\im \theta})u|v}f(e^{\im \theta})\cdot \frac{e^{\im\theta}d\theta}{2\pi}
\end{gather*} 
for all $u,v\in\Vbb$. (The domain of $A(f)$ will be slightly extended, see Conv. \ref{lb48}.)
\end{df}








\begin{thm}\label{lb39}
Let $f\in C^\infty(\Sbb^1)$. The following are true.
\begin{enumerate}[label=(\alph*)]
\item We have $A(f)\Vbb\subset\HV$, and hence $A(f)$ is an unbounded operator on $\HV$ with dense domain $\Vbb$. Moreover, writing $f(e^{\im\theta})=\sum_{n\in\Zbb}\wht f(n)e^{\im n\theta}$, then for each $u\in\Vbb$, the RHS below converges in $\HV$ to the LHS:
\begin{align}\label{eq59}
A(f)u=\sum_{n\in\Zbb}\wht f(n) A_nu
\end{align}
\item We have
\begin{align}\label{eq62}
(-1)^{\wt(A)}A^\theta\big(\ovl{\ek_{2-2\wt(A)}f}\big)\subset A(f)^*
\end{align}
Consequently, $A(f)$ is closable (because $\Dom(A(f)^*)$ is dense).
\item $\HV^\infty$ is an \textbf{invariant core} \index{00@Invariant core} for $\ovl{A(f)}$. (Namely, $\HV^\infty\subset \Dom(\ovl{A(f)})$ is a core for $\ovl{A(f)}$, and $\ovl{A(f)}\HV^\infty\subset\HV^\infty$.) Moreover, assume that $A$ satisfies \eqref{eq55}. Let $p\geq0$, and let $M_p$ be as in Prop. \ref{lb35}. Then for each $\xi\in\HV^\infty$  we have
\begin{align}\label{eq63}
\Vert \ovl{A(f)}\xi\Vert_p\leq M_p|f|_{t+|p|}\cdot \Vert \xi\Vert_{p+r}
\end{align}
\end{enumerate}
\end{thm}

Note that \eqref{eq63} means that the linear map
\begin{align}
\ovl{A(f)}\big|_{\HV^\infty}:(\HV^\infty,\Vert\cdot\Vert_{p+r})\rightarrow (\HV^\infty,\Vert\cdot\Vert_p)
\end{align}
is bounded with operator norm $\leq M_p\cdot |f|_{t+|p|}$.


\begin{proof}
(a): Let $u\in\Vbb$ be homogemeous. Then for each $v\in\Vbb$, since $\bk{A(z)u|v}=\sum_{n\in\Zbb}\bk{A_nu|v}z^{-n-1}$ (where the RHS is a finite sum), we have
\begin{align*}
&\bk{A(f)u|v}=\int_{-\pi}^\pi \bk{A(e^{\im \theta})u|v}f(e^{\im \theta})\cdot \frac{e^{\im\theta}d\theta}{2\pi}=\int_{-\pi}^\pi \sum_{n\in\Zbb}\bk{A_nu|v}e^{-\im n\theta}\cdot f(e^{\im \theta})\cdot \frac{d\theta}{2\pi}\\
=& \sum_{n\in\Zbb}\int_{-\pi}^\pi\bk{A_nu|v}e^{-\im n\theta}\cdot f(e^{\im \theta})\cdot \frac{d\theta}{2\pi}=\sum_{n\in\Zbb} \bk{A_nu|v}\wht f(n)
\end{align*}
where all sums $\sum_{n\in\Zbb}$ are indeed finite. So \eqref{eq59} holds when both sides are multiplied by $P_m$ (for all $m\in\Nbb$). In other words, \eqref{eq59} holds in $\Vbb^\ac$. 

Now assume that $A$ satisfies \eqref{eq55}. Then $A_nu$ are mutually orthogonal for different $n$ (due to Rem. \ref{lb34}), and hence
\begin{align*}
\sum_{n\in\Zbb} \Vert \wht f(n)A_nu\Vert^2\leq \sum_{n\in\Zbb} |\wht f(n)|^2 \cdot M^2(1+|n|)^{2t}\cdot\Vert u\Vert_r^2
\end{align*}
is finite because $n\mapsto\wht f(n)$ is rapidly decreasing. Thus, the RHS of \eqref{eq59} converges in $\HV$. So \eqref{eq59} holds in $\HV$.\\[-1ex]

(b): Using Lem. \ref{lb36} and \eqref{eq61}, we compute
\begin{align*}
&A(f)^\dagger=\Big(\ointn_{\Sbb^1}f(z)A(z)\frac{dz}{2\im\pi}\Big)^\dagger=\ointn_{\Sbb^1} \ovl{z^2f(z)}A(z)^\dagger\frac{dz}{2\im\pi}\\
=&\ointn_{\Sbb^1} \ovl{z^2f(z)}\cdot (-z^2)^{\wt(A)}A^\theta(z)\frac{dz}{2\im\pi}=\ointn_{\Sbb^1}(-1)^{\wt(A)}\ovl{z^{2-2\wt(A)}f(z)}A^\theta(z)\frac{dz}{2\im\pi}
\end{align*}
which equals $(-1)^{\wt(A)}A^\theta\big(\ovl{\ek_{2-2\wt(A)}f}\big)$. This proves \eqref{eq62}.\\[-1ex]

(c): Let $u\in\Vbb$. For each $m\in\Nbb$, by \eqref{eq59} and Prop. \ref{lb35}, we have
\begin{align*}
&\Vert P_{\leq m}A(f)u\Vert_p\leq\sum_{n\in\Nbb}|\wht f(n)|\cdot \Vert P_{\leq m}A_nu\Vert_p\leq \sum_{n\in\Nbb}|\wht f(n)|\cdot \Vert A_nu\Vert_p\\
\leq& \sum_{n\in\Nbb} |\wht f(n)|\cdot M_p(1+|n|)^{|p|+t}\Vert u\Vert_{p+t}=|f|_{|p|+t}M_p\Vert u\Vert_{p+t}
\end{align*}
noting that the second sum is finite, and also noting Rem. \ref{lb37}. Since $m$ is arbitrary, we have thus proved \eqref{eq63} for $\xi=u\in\Vbb$. In particular, $A(f)u\in\HV^p$ (for all $p\geq0$).


Now we consider an arbitrary $\xi\in\HV^\infty$. Applying \eqref{eq63} to $(P_{\leq n}-P_{\leq m})\xi$ and $p=0$, we get 
\begin{align*}
\Vert  \ovl{A(f)}P_{\leq n}\xi-\ovl{A(f)}P_{\leq m}\xi\Vert^2\leq M_0|f|_t\cdot\Vert (P_{\leq n}-P_{\leq m})\xi\Vert_r
\end{align*}
where the RHS converges to $0$ as $m,n\rightarrow\infty$. Thus $\lim_n \ovl{A(f)}P_{\leq n}\xi$ converges in $\HV$. Since $\lim_n P_{\leq n}\xi$ converges, by the closedness of $\ovl{A(f)}$, we see that $\xi\in\Dom(\ovl{A(f)})$ and
\begin{align*}
\lim_{n\rightarrow\infty} \ovl{A(f)}P_{\leq n}\xi=\ovl{A(f)}\xi \tag{$\triangle$} \label{eq64}
\end{align*}
In particular, we have proved that $\HV^\infty$ is contained in $\Dom(\ovl{A(f)})$, and hence is a core for $\ovl{A(f)}$ (since it contains $\Vbb$). 

Moreover,  for each $m\in\Nbb$, since $ (1+{\ovl L_0})^pP_{\leq m}$ is bounded, we have
\begin{align*}
&\Vert P_{\leq m}\ovl{A(f)}\xi\Vert_p=\Vert (1+{\ovl L_0})^pP_{\leq m}\ovl{A(f)}\xi\Vert\xlongequal{\eqref{eq64}}\lim_{n\rightarrow\infty}\Vert (1+{\ovl L_0})^pP_{\leq m}\ovl{A(f)}P_{\leq n}\xi\Vert\\
=&\lim_{n\rightarrow\infty}\Vert P_{\leq m}\ovl{A(f)}P_{\leq n}\xi\Vert_p\leq \limsup_{n\rightarrow\infty}\Vert \ovl{A(f)}P_{\leq n}\xi\Vert_p\\
\leq& \limsup_{n\rightarrow\infty}M_p|f|_{t+|p|}\cdot \Vert P_{\leq n}\xi\Vert_{p+r}= M_p|f|_{t+|p|}\cdot \Vert \xi\Vert_{p+r}
\end{align*}
Since $m$ is arbitrary, we conclude that $\ovl{A(f)}\xi\in \HV^p$, and that \eqref{eq63} holds. Since $p$ can be arbitrary, we conclude that $\ovl{A(f)}\xi\in\HV^\infty$ (for all $\xi\in\HV^\infty$).
\end{proof}


\subsection{}


\begin{cv}\label{lb48}
From now on, if $A$ is an energy-bounded field and $f\in C^\infty(\Sbb^1)$, we let
\begin{align*}
A(f):\HV^\infty\rightarrow\HV^\infty
\end{align*}
be the unique closable operator on $\HV$ with dense domain $\Dom(A(f))=\HV^\infty$ such that $\Vbb$ is a core for $A(f)$, and that $A(f)|_\Vbb$ is the original map defined in Def. \ref{lb38}. Such $A(f)$ exists by Thm. \ref{lb39}. \index{00@Smeared field (as linear operator on $\HV^\infty$)} 

By \eqref{eq62}, we have $\HV^\infty\subset\Dom(A(f)^*)$ and $A(f)^*\HV^\infty\subset\HV^\infty$. Thus we define \index{Af@$A(f)^\dagger$}
\begin{align*}
\pmb{A(f)^\dagger}:= A(f)^*|_{\HV^\infty}:\HV^\infty\rightarrow\HV^\infty
\end{align*}
viewed as an unbounded operator on $\HV$. Thus,
\begin{align}\label{eq131}
A(f)^\dagger=(-1)^{\wt(A)}A^\theta(\ovl{e_{2-2\wt(A)}f})
\end{align}
In general, if $T$ is a closable operator on $\HV$ with invariant domain $\HV^\infty$, and if $T^*\HV^\infty\subset \HV^\infty$, we let \index{T@$T^\dagger$, the formal adjoint} 
\begin{align*}
\pmb{T^\dagger}:=T^*|_{\HV^\infty}:\HV^\infty\rightarrow\HV^\infty
\end{align*}
called the \textbf{formal adjoint} of $T$. \qedsymbol
\end{cv}


\begin{pp}\label{lb49}
Let $A^1,\dots,A^N$ be energy-bounded fields. Let $Q(z_1,\dots,z_N)$ be a polynomial with non-commuting variables $z_1,\dots,z_N$. Then there exists $r\geq0$ such that for any $p\geq0$ and $f_1,\dots,f_N\in C^\infty(\Sbb^1)$, there exists $C_p\geq0$ such that
\begin{align}\label{eq75}
\Vert Q(A^1(f_1),\dots,A^N(f_N))\xi\Vert_p\leq C_p \cdot \Vert\xi\Vert_{p+r}
\end{align}
for all $\xi\in\HV^\infty$. Moreover, $Q(A^1(f_1),\dots,A^N(f_N))$ is closable with core $\Vbb$.
\end{pp}

Note that by Conv. \ref{lb48}, $Q(A^1(f_1),\dots,A^N(f_N))$ is an unbounded operator on $\HV^\infty$ with invariant domain $\HV^\infty$.


\begin{proof}
We prove the proposition for the case that $Q(z_\blt)=z_1\cdots z_N$; the general case is similar. Eq. \eqref{eq75} follows easily from \eqref{eq74}. Note that
\begin{align}\label{eq128}
(A^1(f_1)\cdots A^N(f_N))^*\supset A^N(f_N)^*\cdots A^1(f_1)^*\supset A^N(f_N)^\dagger\cdots A^1(f_1)^\dagger
\end{align}
where the RHS has dense invariant domain $\HV^\infty$. So $A^1(f_1)\cdots A^N(f_N)$ is closable. If $\xi\in\HV^\infty$, then $\lim_n P_{\leq n}\xi=\xi$, and (similar to the proof of Thm. \ref{lb39}-(c)) Formula \eqref{eq75} implies that $(A^1(f_1)\cdots A^N(f_N)P_{\leq n}\xi)_{n\in\Nbb}$ is a Cauchy sequence, which must converge to $\ovl{A^1(f_1)\cdots A^N(f_N)}\xi$. So $\Vbb$ is a core for $A^1(f_1)\cdots A^N(f_N)$.
\end{proof}



\subsection{}


So far, we haven't given any nontrivial example of energy-bounded field. Here we introduce one. 

\begin{df}
For each field $A$ we write \index{An@$A_{(n)}$}
\begin{align}
A(z)=\sum_{n\in\Zbb} A_{(n)}z^{-n-\wt(A)}
\end{align}
Thus, by Rem. \ref{lb34}, for each $d\in\Nbb$ we have
\begin{align}
[L_0,A_{(n)}]=-nA_{(n)}
\end{align}
Namely, $A_{(n)}$ increases the weight by $-n$.
\end{df}

For example, for the Virasoro field $T(z)$ in Exp. \ref{lb40} we have $T_{(n)}=L_n$. For the current field $X(z)$ in Exp. \ref{lb41} we have $X_{(n)}=X_n$.

\begin{thm}\label{lb42}
Let $A(z)$ be a field. Assume that there exist $C,t,r\geq0$ such that for each $n\in\Zbb$ and $v\in\Vbb$ we have
\begin{align}\label{eq66}
\big|\bk{[A_{(n)},A_{(n)}^\dagger]v|v}\big|\leq C^2(1+|n|)^{2t}\Vert v\Vert_r^2
\end{align}
Then there exists $M\geq0$ such that for each $v\in\Vbb$ and each \uwave{nonzero} $n\in\Zbb$, we have
\begin{align}\label{eq65}
\Vert A_{(n)}v\Vert \leq M(1+|n|)^t\Vert v\Vert_{r+\frac 12}
\end{align}
\end{thm}



\begin{proof}
We first assume that $n>0$ and prove the estimate for $A_{(-n)}$. For each $d\in\Nbb$, let $K_d$ be the operator norm of the restriction $A_{(-n)}:\Vbb(d)\rightarrow\Vbb(d+n)$, which is also the operator norm of $A_{(-n)}^\dagger:\Vbb(d+n)\rightarrow\Vbb(d)$. Choose any $v\in\Vbb(d)$ with $\Vert v\Vert\leq 1$. Then
\begin{align*}
&\Vert A_{(-n)}v\Vert^2=\Vert A_{(-n)}^\dagger v\Vert^2+\bk{[A_{(-n)}^\dagger,A_{(-n)}]v|v}\leq K_{d-n}^2\Vert v\Vert^2+C^2(1+|n|)^{2t}\Vert v\Vert_r^2\\
\leq& K_{d-n}^2+C^2(1+|n|)^{2t}(1+d)^{2r}
\end{align*}
Therefore
\begin{align*}
&K_d^2\leq K_{d-n}^2+C^2(1+|n|)^{2t}(1+d)^{2r}\\
\leq& K_{d-2n}^2+C^2(1+|n|)^{2t}\big((1+d)^{2r}+(1+d-n)^{2r}\big)\leq\cdots\\
\leq& C^2(1+|n|)^{2t}\int_1^{2+2d}x^{2r}dx\leq \frac{2^{2r+1}C^2}{2r+1}(1+|n|)^{2t}(1+d)^{2r+1}
\end{align*}
Take $M=\sqrt{\frac{2^{2r+1}}{2r+1}}C$. Then since $K_d=\Vert A_{(-n)}|_{\Vbb(d)}\Vert=\Vert A_{(-n)}^\dagger|_{\Vbb(d+n)}\Vert$, we have
\begin{align*}
\Vert A_{(-n)}v\Vert \leq M(1+|n|)^t\Vert v\Vert_{r+\frac 12}\qquad \Vert A_{(-n)}^\dagger v\Vert \leq M(1+|n|)^t\Vert v\Vert_{r+\frac 12}
\end{align*}
for each homogeneous $v$, and hence for all $v$ (by the same reasoning as in the proof of Prop. \ref{lb35}).

Let $B(z)=\sum_{n\in\Zbb}B_{(n)}z^{-n-\wt A}$ where $B_{(n)}=A_{(-n)}^\dagger$. Then $[B_{(n)},B_{(n)}^\dagger]=-[A_{(-n)},A_{(-n)}^\dagger]$ satisfies a similar inequality to \eqref{eq66}. Therefore, by the above paragraph, if $n>0$ then $B_{(-n)}^\dagger=A_{(n)}$ satisfies a similar estimate to \eqref{eq65}.
\end{proof}




\subsection{}

Let $(\mc V,\Vbb)$ be a chiral algebra. 

\begin{co}\label{lb43}
The field $\sum_{n\in 0,\pm1}L_nz^{-n-2}$ is linearly energy-bounded. If $(\mc V,\Vbb)$ is a conformal chiral algebra, then $T(z)=\sum_{n\in\Zbb}L_nz^{-n-2}$ is linearly energy-bounded.
\end{co}


\begin{proof}
By the Virasoro relation \eqref{eq50}, there is $C\geq0$ such that $\bk{[L_n,L_n^\dagger]v|v}\leq C(1+|n|)^3\Vert v\Vert_{\frac 12}^2$ for all $v\in\Vbb$. Therefore, by Thm. \ref{lb42}, $\sum_{n\neq 0}L_nz^{-n-2}$ is linearly energy-bounded. So $\sum_n L_nz^{-n-2}$ is also linearly energy-bounded.
\end{proof}


\begin{co}
Let $A\in\mc V$ and $\wt(A)>0$. Assume that the field $\sum_{n\in\Zbb}[A_{(n)},A_{(n)}^\dagger]z^{-n-1}$ satisfies $r$-th order energy bounds. Then $A(z)$ satisfies $(r+\frac 32)$-th order energy bounds.
\end{co}

\begin{proof}
Assume that there exist $M,t\geq0$ such that $\Vert[A_{(n)},A_{(n)}^\dagger]v\Vert\leq M(1+|n|)^t\Vert v\Vert_r$ for all $n$. Then
\begin{align*}
|\bk{[A_{(n)},A_{(n)}^\dagger]v|v}|\leq \Vert[A_{(n)},A_{(n)}^\dagger]v\Vert\cdot \Vert v\Vert\leq M(1+|n|)^t\Vert v\Vert_r^2
\end{align*}
Thus, by Thm. \ref{lb42}, $\sum_{n\neq 0}A_{(n)}z^{-n-1}$ satisfies $(r+\frac 12)$-th order energy bounds. Since $A_{(n)}=A_{n+\wt(A)-1}$, by \eqref{eq53b} we have $[L_{-1},A_{(1)}]=-\wt(A)\cdot A_{(0)}$. By Cor. \ref{lb43}, $L_{-1}$ satisfies $1$-st order energy bounds. Since $A_1$ satisfies $(r+\frac 12)$-th order energy bounds, by using Prop. \ref{lb35} we see that $A_0=-\wt(A)^{-1}[L_{-1},A_1]$ satisfies $(r+\frac 32)$-th order energy bounds.
\end{proof}


\begin{co}\label{lb109}
Assume that $(\mc V,\Vbb)$ is the current algebra associated to a finite-dimensional unitary Lie algebra $\gk$. Then for each $X\in\gk$, the field $X(z)=\sum_{n\in\Zbb}X_nz^{-n-1}$ is linearly energy-bounded.
\end{co}

\begin{proof}
It suffices to consider the case that $X^\dagger=X$. Then by \eqref{eq68} we have
\begin{align*}
[X_n,X_n^\dagger]=[X_n,X_{-n}]=l\cdot n \Vert X\Vert^2\cdot \idt
\end{align*}
Therefore, by Thm. \ref{lb42}, $\sum_{n\neq 0}X_nz^{-n-1}$ satisfies $\frac 12$-th order energy bounds. 

It remains to study $X_0$. We have orthogonal decomposition $\gk=\gk_1\oplus\gk_2$ where $\gk_1$ is abelian and $\gk_2$ is a direct sum of simple Lie algebras. Then $[\gk_1,\gk]=0$ and $[\gk,\gk]=\gk_2$. It suffices to assume that either $X\in \gk_1$ or $X\in\gk_2$. In the former case, by \eqref{eq68}, we see that $X_0$ commutes with $Y_n$ for every $Y\in\gk$ and $n\in\Zbb$. By the creating property in Def. \ref{lb30}, we have $X_0\Omega=0$. By the cyclicity in Def. \ref{lb30}, we know that $X_0v=0$ for all $v\in\Vbb$. Then $X_0$ is clearly linearly energy bounded.

In the later case, since $[\gk,\gk]=\gk_2$, it suffices to assume that $X=[Y,Z]$ for some $Y,Z\in\gk$. Then by \eqref{eq68} we have $X_0=[Y_1,Z_{-1}]+l\bk{Y|Z^*}\cdot\idt$. Since we have proved that $Y_1$ and $Z_{-1}$ satisfy $\frac 12$-th order energy bounds, by Prop. \ref{lb35}, $[Y_1,Z_{-1}]$ is linearly energy-bounded.
\end{proof}



\subsection{}

\begin{pp}\label{lb105}
Let $A,B$ be mutually local fields. Assume that $A,B$ are both energy-bounded. Then for each $k\in\Zbb$, $A_kB$ is energy-bounded.
\end{pp}


\begin{proof}
As in the proof of Prop. \ref{lb35}, it suffices to establish the inequality for homogeneous vectors. Recall \eqref{eq47}, i.e.
\begin{align*}
(A_kB)_n=\sum_{l\in\Nbb}(-1)^l{k\choose l}A_{k-l}B_{n+l}-\sum_{l\in\Nbb}(-1)^{k+l}{k\choose l}B_{k+n-l}A_l \tag{$\star$}\label{eq69}
\end{align*}
Note that $\limsup_{l\rightarrow\infty}|{k\choose l}|l^{-|k|}<+\infty$. Choose any homogeneous $v\in\Vbb$. Then by the energy-boundedness of $A,B$, and by Prop. \ref{lb35}, there exist constants $C_\blt,r_\blt,t_\blt$ independent of $v$ and $n$ (but possibly depending on $k$) such that
\begin{align*}
&\Big\Vert \sum_{l\in\Nbb}(-1)^{k+l}{k\choose l}B_{k+n-l}A_lv  \Big\Vert\leq \sum_{l\geq0}C_1l^{|k|}\big\Vert B_{k+n-l}A_lv\big\Vert\\
\leq&\sum_{l\geq0}C_2l^{|k|}(1+|k+n-l|)^{t_1}\big\Vert A_lv\Vert_{r_1}\\
\leq &\sum_{0\leq l\leq \wt(A)+\wt(v)-1}C_3l^{|k|}(1+|n|)^{t_1}(1+l)^{t_1} \cdot (1+l)^{t_2}\Vert v\Vert_{r_2}\\
\leq&\sum_{0\leq l\leq \wt(A)+\wt(v)-1}C_3(1+|n|)^{t_1}(1+l)^{|k|+t_1+t_2} \cdot \Vert v\Vert_{r_2}\\
\leq& C_4(1+|n|)^{t_1}(1+\wt(v))^{1+|k|+t_1+t_2} \cdot \Vert v\Vert_{r_2}=C_4(1+|n|)^{t_1}\Vert v\Vert_{1+|k|+t_1+t_2+r_2}
\end{align*}
By a similar calculation, we obtain the desired bound for the first summand on the RHS of \eqref{eq69} acting on $v$.
\end{proof}



\subsection{}


Our last topic of this section is a relationship between products of smeared fields and ``products" of pointed fields. We first expand the setting in Subsec. \ref{lb44}.


\begin{df}\label{lb52}
Let $I_1,\dots,I_N$ be real oriented smooth $1$-submanifolds of $\Cbb$. Let \index{I@$I_\blt=I_1\times\cdots\times I_N$} 
\begin{align*}
I_\blt=I_1\times\cdots\times I_N
\end{align*}
Suppose that $F:I_\blt\rightarrow\Hom(\Vbb,\Vbb^\ac)$ is continuous in the sense that for each $u,v\in\Vbb$, the function $z_\blt\in I_\blt\rightarrow\bk{F(z_\blt)u|v}\in\Cbb$ is continuous. Define \index{zz@$\int_{I_\blt}F(z_\blt)\sd z_\blt$}
\begin{gather*}
\int_{I_\blt}F(z_\blt)\sd z_\blt:\Vbb\rightarrow\Vbb^\ac
\end{gather*}
such that for each $u,v\in\Vbb$,
\begin{align}
\begin{aligned}
&\Bigbk{\int_{I_\blt}F(z_\blt)\sd z_\blt\cdot u\Big| v}=\int_{I_N}\cdots\int_{I_1}\bk{F(z_\blt)u|v}\frac{dz_1}{2\im\pi}\cdots \frac{dz_N}{2\im\pi}\\
\equiv&\int_{a_N}^{b_N}\cdots \int_{a_1}^{b_1}F(\gamma_1(\theta_1),\dots,\gamma_N(\theta_N))\gamma_1'(\theta_1)\cdots\gamma_N'(\theta_N)\frac{d\theta_1}{2\im\pi}\cdots \frac{d\theta_N}{2\im\pi}
\end{aligned}
\end{align}
where each $\gamma_i:(a_i,b_i)\rightarrow I_i$ is a positive parametrization of $I_i$. (In Conv. \ref{lb53}, we will expand the domain of $\int_{I_\blt}F(z_\blt)\sd z_\blt$ to $\HV^\infty$ under reasonable conditions.)

A similar definition applies to any $F:I_\blt\rightarrow\Vbb^\ac$ whose evaluation with any $v\in\Vbb$ is continuous over $I_\blt$.  \hfill\qedsymbol
\end{df}

Thus, in short we have \index{dz@$\sd z_\blt$}
\begin{align*}
\sd z_\blt=\frac{dz_1}{2\im\pi}\cdots \frac{dz_N}{2\im\pi}
\end{align*}


\subsection{}


\begin{thm}\label{lb50}
Let $A^1,\dots,A^N$ be mutually commuting energy-bounded fields. Let $I_1,\dots,I_N$ be mutually disjoint intervals of $\Sbb^1$. Choose $f_i\in C_c^\infty(I_i)$ for each $i$. Then as linear maps $\Vbb\rightarrow\Vbb^\ac$ we have
\begin{align}\label{eq71}
A^1(f_1)\cdots A^N(f_N)\big|_{\Vbb}=\int_{I_\blt} f_1(z_1)\cdots f_N(z_N)A^1(z_1)\cdots A^N(z_N)\sd z_\blt
\end{align}
\end{thm}

Recall Def. \ref{lb46} for the continuity of \eqref{eq70}, the function $z_\blt\mapsto \bk{A^1(z_1)\cdots A^N(z_N)u|v}$. %The following proof is an expansion of the brief argument of \cite[p.115]{BS90} Step 1.


\begin{proof}
Let $u,v\in\Vbb$ be homogeneous. Let $\Delta$ be the set of all $(r_1,\dots,r_N)$ satisfying $r_1,\dots,r_N\in[1/2,1)$ and $r_{i+1}/r_i\in[1/2,1)$. For each $r_\blt\in\Delta$, Let
\begin{align*}
F(r_\blt)=\int_{I_\blt}f(z_1)\cdots f_N(z_N)\bk{A^1(r_1z_1)\cdots A^N(r_Nz_N)u|v}\sd z
\end{align*}
Let $\Rbf$ be the map on the RHS of \eqref{eq71}. Then by the continuity of \eqref{eq70} we have
\begin{align*}
\lim_{r_1,\dots,r_N\nearrow 1}F(r_\blt)=\bk{\Rbf u|v}
\end{align*}



For each fixed $r_\blt\in\Delta$, since $|r_1z_1|>\cdots>|r_Nz_N|>0$ for all $z_\blt\in I_\blt$, by Prop. \ref{lb18},
\begin{align*}
&\bk{A^1(r_1z_1)\cdots A^N(r_Nz_N)u|v}=\sum_{n_\blt\in\Nbb^N}\bk{P_{n_1}A^1(r_1z_1)P_{n_2}A^2(r_2z_2)\cdots P_{n_N}A^N(r_Nz_N)u|v}\\
=&\sum_{n_\blt\in\Nbb^N}\bk{P_{n_1}A^1(z_1)P_{n_2}A^2(z_2)\cdots P_{n_N}A^N(z_N)u|v}\cdot \Delta(r_\blt)
\end{align*}
which converges a.l.u. for $z_\blt\in I_\blt$, where
\begin{align*}
\Delta(r_\blt)=r_1^{n_1} \Big(\frac{r_2}{r_1}\Big)^{n_2}\cdots  \Big(\frac{r_N}{r_{N-1}}\Big)^{n_N}\cdot r_1^{-\wt(A^1)}\cdots r_N^{-\wt(A^N)}
\end{align*}
Hence the sum commutes with the integrals, yielding
\begin{align*}
&F(r_\blt)=\sum_{n_\blt\in\Nbb^N}\int_{I_\blt}f_1(z_1)\cdots f_N(z_N)\bk{P_{n_1}A^1(z_1)P_{n_2}A^2(z_2)\cdots P_{n_N}A^N(z_N)u|v}\cdot \Delta(r_\blt)\\
=&\sum_{n_\blt\in\Nbb^N}\bk{P_{n_1}A^1(f_1)P_{n_2}A^2(f_2)\cdots P_{n_N}A^N(f_N)u|v}\cdot \Delta(r_\blt)\tag{$\lozenge$}\label{eq72}
\end{align*}



By \eqref{eq59}, we have
\begin{align*}
\begin{aligned}\label{eq73}
&\sum_{n_\blt\in\Nbb^N}\big\Vert P_{n_1}A^1(f_1)P_{n_2}A^2(f_n)\cdots P_{n_N}A^N(f_N)u\big\Vert\\
=&\sum_{k_\blt\in\Nbb^N}\big\Vert A^1_{k_1}A^2_{k_2}\cdots A^N_{k_N}u\big\Vert\cdot |\wht f_1(k_1)\cdots \wht f_N(k_N)|
\end{aligned}\tag{$\star$}
\end{align*}
which is finite by Prop. \ref{lb35} (and the fact that $k_i\mapsto\wht f_i(k_i)$ is rapidly decreasing). Hence, the function
\begin{align*}
\sup_{r_\blt\in\Delta}\Big|\bigbk{P_{n_1}A^1(f_1)P_{n_2}A^2(f_n)\cdots P_{n_N}A^N(f_N)u|v } \Big|\Delta(r_\blt)<+\infty
\end{align*}
Therefore, applying the dominated convergence theorem to \eqref{eq72}, we get
\begin{align*}
&\lim_{r_1,\dots,r_N\nearrow1}F(r_\blt)=\sum_{n_\blt\in\Nbb^N}\bk{P_{n_1}A^1(f_1)P_{n_2}A^2(f_2)\cdots P_{n_N}A^N(f_N)u|v}\\
=& \bk{A^1(f_1)\cdots A^N(f_N)u|v}
\end{align*}
where the last equality is due to Lem. \ref{lb47}. This proves \eqref{eq71}.
\end{proof}

\begin{lm}\label{lb47}
Let $T_1,\dots,T_N$ be closable operators on $\HV$ such that for each $1\leq i\leq N$, $\HV^\infty$ is contained in $\Dom(T_i)\cap\Dom(T_i^*)$ and invariant under both $T_i$ and $T_i^*$. Then for each $\xi,\eta\in\HV^\infty$ we have
\begin{align}\label{eq74}
\sum_{n_N\in\Nbb}\cdots \sum_{n_1\in\Nbb}\bk{P_{n_1}T_1P_{n_2}T_2\cdots P_{n_N}T_N\xi|\eta}=\bk{T_1\cdots T_N\xi|\eta}
\end{align}
\end{lm}


\begin{proof}
The case $N=0$ is obvious. Assume case $N-1$ is true. Then for case $N$, the LHS of \eqref{eq74} equals
\begin{align*}
&\sum_{n_N}\cdots \sum_{n_2}\sum_{n_1}\bk{T_1P_{n_2}T_2\cdots P_{n_N}T_N\xi|P_{n_1}\eta}=\sum_{n_N}\cdots \sum_{n_2}\bk{T_1P_{n_2}T_2\cdots P_{n_N}T_N\xi|\eta}\\
=&\sum_{n_N}\cdots \sum_{n_2}\bk{P_{n_2}T_2\cdots P_{n_N}T_N\xi|T_1^*\eta}=\bk{T_2\cdots T_N\xi|T_1^*\eta}=\bk{T_1T_2\cdots T_N\xi|\eta}
\end{align*}
\end{proof}



\subsection{}


\begin{cv}\label{lb53}
Let $F:I_\blt\rightarrow\Hom(\Vbb,\Vbb^\ac)$ be as in Def. \ref{lb52}. Suppose that the linear map $\int_{I_\blt}F(z_\blt)\sd z_\blt:\Vbb\rightarrow\Vbb^\ac$ has range in $\HV$ and is closable. Suppose also that $\HV^\infty$ is contained in the domain of the closure of $\int_{I_\blt}F(z_\blt)\sd z_\blt$. We then let \index{zz@$\int_{I_\blt}F(z_\blt)\sd z_\blt$} 
\begin{align*}
\int_{I_\blt}F(z_\blt)\sd z_\blt:\HV^\infty\rightarrow\HV
\end{align*}
denote the restriction to $\HV^\infty$ of the closure of the original $\int_{I_\blt}F(z_\blt)\sd z_\blt$. 
\end{cv}


\begin{eg}\label{lb54}
Assume the setting of Thm. \ref{lb50}, and let $F(z_\blt)=f_1(z_1)\cdots f(z_N)A^1(z_1)\cdots A^N(z_N)$ be the integrand on the RHS of \eqref{eq71}. Then by Thm. \ref{lb50}, $\int_{I_\blt}F(z_\blt)\sd z_\blt$ equals $A^\blt(f_\blt)|_\Vbb$. By Prop. \ref{lb49}, $\Vbb$ is a core for $A^\blt(f_\blt):\HV^\infty\rightarrow\HV^\infty$, and hence $\Dom(\ovl{A^\blt(f_\blt)|_\Vbb})\supset\HV^\infty$. Therefore Conv. \ref{lb53} applies to $F(z_\blt)$, and hence
\begin{align}
A^1(f_1)\cdots A^N(f_N)=\int_{I_\blt} f_1(z_1)\cdots f_N(z_N)A^1(z_1)\cdots A^N(z_N)\sd z_\blt
\end{align}
holds as closable operators on $\HV$ with common domain $\HV^\infty$.
\end{eg}


\begin{co}\label{lb66}
Let $A,B$ be energy-bounded and mutually local fields. Let $I,J$ be mutually disjoint intervals of $\Sbb^1$. Then for each $f\in C_c^\infty(I)$ and $g\in C_c^\infty(J)$ we have (on $\HV^\infty$) that
\begin{align*}
A(f)B(g)=B(g)A(f)
\end{align*}
\end{co}


\begin{proof}
By Exp. \ref{lb54}, as unbounded operators with domain $\HV^\infty$ we have
\begin{align*}
A(f)B(g)=\int_{I\times J}f(z_1)g(z_2)A(z_1)B(z_2)\sd z_\blt=B(g)A(f)
\end{align*}
noting \eqref{eq77} that the $A(z_1)$ and $B(z_2)$ in the integral are interchangeable.
\end{proof}







\newpage




\section{Methods of unbounded operators}


In this section, we collect some useful properties about unbounded operators that will be used later.

Let $\mc H$ be a Hilbert space. Recall \cite[Sec. 4]{Gui-S} for the meaning of unbounded positive operators. Assume that $T$ is self-adjoint. By spectral theorem, $T\geq0$ iff $\bk{T\xi|\xi}\geq0$ for all $\xi\in\Dom(T)$. Thus, if $\Dom_0\subset\Dom(T)$ is a core for $T$, then $T\geq0$ iff $\bk{T\xi|\xi}\geq0$ for all $\xi\in\Dom_0$.


In this section, we fix a positive operator $H$ on $\mc H$ satisfying $H-a\geq0$ for some real number $a>0$. Recall Rem. \ref{lb51} that for each $p\geq0$, $\Dom(H^p)$ is complete under the inner product
\begin{align}\label{eq57}
\bk{\xi|\eta}_p:=\bk{H^p\xi|H^p\eta}
\end{align}
We always let $\Vert\cdot\Vert_p$ be the norm associated to this inner product. 

\begin{rem}
Note that for each $r\geq0$,
\begin{align*}
\mc H^\infty:=\bigcap_{p\geq0} \Dom(H^p)
\end{align*}
is a core for $H^r$. This is because $\mc H^\infty$ contains $\bigcup_{\lambda\geq0}\Rng(\chi_{[0,\lambda]}(H))$ where the latter is a core for $f(H)$.
\end{rem}


\subsection{}

Let $O\subset\Cbb$ be open. Recall that a \textbf{holomorphic function} \index{00@Holomorphic $\mc H$-valued functions} $f:O\rightarrow\mc H$ is a  function whose derivative $f'(z)$ exists everywhere on $\Omega$. More generally, if $O\subset\Cbb^n$ is open, a \textbf{holomorphic function} $f:O\rightarrow\mc H$ is defined to be a continuous function which is holomorphic in each variable (when the other variables are fixed).


\begin{pp}
Let $V$ be a dense linear subspace of $\mc H$. Let $O\subset \Cbb$ be open and $f:O\rightarrow\mc H$ be a function. Then the following are equivalent:
\begin{enumerate}[label=(\arabic*)]
\item $f$ is holomorphic.
\item $f$ is continuous, and $z\in O\mapsto\bk{f(z)|v}$ is holomorhic for all $v\in V$.
\item On each closed disk $D\subset O$ with center $z_0$ and radius $r$, $f(z)$ is the limit of a series $\sum_{n\in\Nbb}\xi_n(z-z_0)^n$ where $\xi_n\in\mc H$ and $\sum_{n\in\Nbb}\Vert\xi_n\Vert r^n<+\infty$.
\end{enumerate}
\end{pp}

\begin{proof}
(1)$\Rightarrow$(2) and (3)$\Rightarrow$(2) are obvious.

(2)$\Rightarrow$(1,3): Assume (2). Let $C$ be an anticlockwise circle in $O$ centered at $z_0$ with radius $\rho$ slightly larger than $r$. Since $f$ is continuous, for each $|z-z_0|<\rho$ we can define the Riemann integral
\begin{align*}
g(z)=\ointn_C \frac{f(\zeta)}{\zeta-z}\sd\zeta
\end{align*}
By the residue theorem, for each $v\in V$ and each $z\in D$ we have $\bk{f(z)|v}=\bk{g(z)|v}$. So $f=g$ on $D$. Thus, whenever $|z-z_0|<r$ we have
\begin{align*}
\frac{f(z)-f(z_0)}{z-z_0}=\ointn_C\frac{f(\zeta)}{(\zeta-z)(\zeta-z_0)}\sd\zeta
\end{align*}
whose limit under $z\rightarrow z_0$ exists because the integrand on the RHS converges uniformly (over $\zeta\in C$) to $f(\zeta)(\zeta-z_0)^2$. This proves (1). 

Since the RHS below converges uniformly on $(\xi,z)\in C\times D$ to the LHS
\begin{align*}
(\zeta-z)^{-1}=\sum_{n\in\Nbb}(\zeta-z_0)^{-n-1}(z-z_0)^n
\end{align*}
we conclude that $f(z)=\sum_{n\in\Nbb}\xi_n(z-z_0)^n$ where
\begin{align*}
\xi_n=\ointn_C(\zeta-z_0)^{-n-1}f(\zeta)\sd\zeta
\end{align*} 
Since $\Vert\xi_n\Vert\leq \rho^{-n-1}M$ where $M=\sup_{\zeta\in C}\Vert f(\zeta)\Vert$, we have $\sum_n\Vert\xi_n\Vert r^n<+\infty$. This proves (3).
\end{proof}









\subsection{}

We know that if $A$ is self-adjoint, then $\Dom(A)$ is precisely the set of all $\xi$ such that $t\in\Rbb\mapsto e^{\im tA}\xi$ is differentiable at $0$. In a similar spirit, the following theorem describes the domain of $A^r$.


\begin{thm}\label{lb56}
Let $A$ be a positive operator on $\mc H$ satisfying $\chi_{\{0\}}(A)=0$ (equivalently, $A$ is injective). Let $r>0$ and 
\begin{align*}
\fk I=\{z\in\Cbb:-r\leq\Imag z\leq0\}
\end{align*}
Let $\xi\in\mc H$. Then the following are equivalent.
\begin{enumerate}[label=(\arabic*)]
\item $\xi\in\Dom(A^r)$.
\item The function $t\in\Rbb\mapsto A^{\im t}\in\mc H$ can be extended to a (necessarily unique) continuous function $F:\fk I\rightarrow\mc H$ and holomorphic on $\Int\fk I$.
\item There exist $\psi\in\mc H$ and a core $\Dom_0$ for $A^r$ such that for each $\eta\in\Dom_0$, there is a function $f_\eta$ continuous on $\fk I$, holomorphic on $\Int\fk I$, such that $f_\eta(t)=\bk{A^{\im t}\xi|\eta}$ for each $t\in\Rbb$, and that $f_\eta(-\im r)=\bk{\psi|\eta}$.
\end{enumerate}
Moreover, if the above statements are true, the function in (2) takes the form $F(z)=A^{\im z}\xi$. In particular, $F(-\im r)=A^r\xi$.
\end{thm}


\begin{proof}
(1)$\Rightarrow$(2): The uniqueness follows from Lem. \ref{lb55}. As for the existence, one defines $F:z\in\fk I\mapsto A^{\im z}\xi$, noting that condition (1) ensures that $\xi\in\Dom(A^{\im z})$ for all $z\in\fk I$. The continuity of $(\tau,\mu)\in\fk I\times\fk I\mapsto \bk{A^{\im\tau}\xi|A^{\im\mu}\xi}$ (which follows from the spectral theory and the dominated convergence theorem) shows that $F$ is continuous; the holomorphicity of $F|_{\Int\fk I}$ follows e.g. from the spectral theory and Morera's theorem.

(2)$\Rightarrow$(3): Let $f_\eta(z)=\bk{F(z)|\eta}$ and $\psi=F(-\im r)$.

(3)$\Rightarrow$(1): Since $\eta\in\Dom_0$, the map $z\in\fk I\mapsto A^{\im z}\eta\in\mc H$ is continuous on $\fk I$ and holomorphic on $\Int\fk I$. Therefore, we have a continuous function $\wtd f_\eta:\fk I\rightarrow\Cbb$, holomorphic on $\Int\fk I$, such that
\begin{align*}
\wtd f_\eta(z)=\bk{\xi|A^{-\im\ovl z}\eta}
\end{align*}
Extend $f_\eta$ to a continuous function $\fk I$ holomorphic on $\Int\fk I$. For each $t\in\Rbb$ we have
\begin{align*}
\wtd f_\eta(t)=\bk{\xi|A^{-\im t}\eta}=\bk{A^{\im t}\xi|\eta}=f_\eta(t)
\end{align*}
Thus, by the following Lem. \ref{lb55}, we have $\wtd f_\eta=f_\eta$ on $\fk I$. Therefore
\begin{align*}
\bk{\psi|\eta}=f_\eta(-\im r)=\wtd f_\eta(-\im r)=\bk{\xi|A^r\eta}
\end{align*}
Since all such $\eta$ form a core $\Dom_0$ for $A^r$, we see that $\xi\in\Dom((A^r|_{\Dom_0})^*)=\Dom(A^r)$ and $A^r\xi=\psi$.
\end{proof}





\begin{lm}\label{lb55}
Let $\Omega$ be an open subset of $\{z\in\Cbb:\Imag z\geq0\}$ whose interior $\Int_\Cbb\Omega=\{z\in\Omega:\Imag z>0\}$ is connected. Suppose that $I$ is a non-empty open interval contained in $\Rbb\cap\Omega$. Let $F:\Omega\rightarrow\mc H$ be a continuous function holomorphic on $\Int_\Cbb\Omega$. If $F|_I=0$, then $F=0$.
\end{lm}

Note that by applying a M\"obious transform, the lemma also applies to the case that $\Omega$ is an open subset of the unit closed disk $\ovl\Dbb_1$ with connected interior, and $I$ is an interval of $\Sbb^1$.

\begin{proof}
By considering the function $z\in\Omega\mapsto\bk{F(z)|\eta}$ for all $\eta\in\mc H$, it suffices to assume that $\mc H=\Cbb$. Let $\eps>0$ and $D=\{z\in\Cbb:\Real z\in I,0\leq \Imag z<\eps\}$ such that $D\subset \Omega$. If we can show that $F|_{\Int D}=0$, then $F|_{\Int\Omega}=0$ because $\Int\Omega$ is connected, and hence $F=0$ by the continuity. 

By the Schwarz reflection principle, $F|_D$ can be extended to a holomorphic function on $\wtd D=\{z\in\Cbb:\Real z\in I,-\eps\leq \Imag z<\eps\}$. (In fact, one defines $F(z)=\ovl{F(\ovl z)}$ if $\ovl z\in D$, and uses Morera's theorem to show that $F|_{\wtd D}$ is holomorphic.) Now $\wtd D$ is an open subset of $\Cbb$. It is well-known that any holomorphic function on $\wtd D$ vanishing on an interval must be vanishing everywhere. Thus $F|_I=0$ implies $F|_{\wtd D}=0$.
\end{proof}



\begin{co}\label{lb57}
Let $A$ be a positive operator on $\mc H$. Let $\fk I=\{z\in\Cbb:\Imag z\geq0\}$. Then for each $\xi\in\mc H$, the function $F:z\in\fk I\mapsto e^{\im zA}\xi\in\mc H$ is continuous, and $F|_{\Int\fk I}$ is holomorphic.
\end{co}

\begin{proof}
Apply Thm. \ref{lb56} to the operator $B=e^{-A}$, which is bounded (and hence $\mc H=\Dom(B^r)$ for all $r\geq0$) because $A\geq0$. Note that the fact $B^{\im z}=e^{-\im zA}$ is due to the composition rule for Borel functional calculus, i.e., if $f,g:\Cbb\rightarrow\Cbb$ are Borel functions then $(f\circ g)(A)=f(g(A))$, c.f. \cite[Sec. 9]{Gui-S}.
\end{proof}









\subsection{}




\begin{pp}\label{lb31}
Let $T$ be a symmetric unbounded operator on $\mc H$. The following are true.
\begin{enumerate}[label=(\arabic*)]
\item $\ovl T$ is self-adjoint iff $\Rng(\im+T)$ and $\Rng(\im-T)$ are dense in $\mc H$.
\item Assume that $\ovl T$ is self-adjoint, and let $\Dom_0$ be a linear subspace of $\Dom(T)$. Then $\Dom_0$ is a core for $T$ (equivalently, for $\ovl T$) iff $(\im+T)\Dom_0$ is dense in $\mc H$.
\end{enumerate}
\end{pp}

\begin{proof}
Let $\scr G(\ovl T)$ be the graph of $T$. Then
\begin{align*}
\Phi_\pm: \scr G(\ovl T)\rightarrow \Rng(\im\pm\ovl T)\qquad (\xi,\ovl T\xi)\mapsto \im\xi\pm\ovl T\xi
\end{align*}
is unitary (because $\ovl T$ is symmetric), restricting to a unitary map $\scr G(T|_{\Dom_0})\rightarrow (\im\pm T)\Dom_0$. Thus $\Dom_0$ is a core for $\ovl T$ iff $\scr G(T|_{\Dom_0})$ is dense in $\scr G(\ovl T)$, iff $(\im+ T)\Dom_0$ is dense in $\Rng(\im+T)$, iff $(\im- T)\Dom_0$ is dense in $\Rng(\im-T)$. Therefore (2) follows from (1).

Since $\Dom(T)$ is a core for $\ovl T$, by setting $\Dom_0=\Dom(T)$, we see that $\Rng(\im\pm T)$ is dense in $\Rng(\im\pm\ovl T)$. Since $\ovl T$ is self-adjoint iff $\Rng(\im+\ovl T)=\Rng(\im-\ovl T)=\mc H$ (cf. \cite[Sec. 10]{Gui-S}), (1) is proved.
\end{proof}





\subsection{}

We use Prop. \ref{lb31} to give two criteria for cores.


\begin{thm}\label{lb58}
Let $A,B$ be self-adjoint closed operators on $\mc H$, and assume that $B$ is affiliated with $\{A\}''$ (e.g. $B=f(A)$ for some Borel $f:\Rbb\rightarrow\Rbb$). Choose a dense linear subspace $\Dom_0\subset\Dom(B)$ such that $e^{\im tA}\Dom_0\subset\Dom_0$ for any $t\in\Rbb$. Then $\Dom_0$ is a core for $B$.
\end{thm}


\begin{proof}
By Prop. \ref{lb31}, it suffices to prove that $(\im+B)\Dom_0$ is dense in $\mc H$. Choose any $\eta\in\mc H$ orthogonal to $(\im+B)\Dom_0$. Then for each $\xi\in\Dom_0$ and $t\in\Rbb$, since $e^{\im tA}$ belongs to $\mc M:=\{H\}''$ and hence commutes strongly with $\im+B$, we have $e^{-\im tA}(\im+B)\subset(\im+B)e^{-\im tA}$. Therefore, since $e^{-\im tA}\xi\in\Dom_0$, we have
\begin{align*}
\bk{(\im+B)\xi|e^{\im tA}\eta}=\bk{(\im+B)e^{-\im tA}\xi|\eta}=0
\end{align*}

Let $\scr A=\Span_\Cbb\{e^{\im tA}:t\in\Rbb\}$, which is a unital $*$-algebra generating $\mc M$. Then clearly
\begin{align*}
\bk{(\im+B)\xi|x\eta}=0
\end{align*}
for all $x\in\scr A$, and hence (by passing to the strong operator closure) for all $x\in\mc M$. Since $\{B\}''\subset\mc M$, the projection $E_\lambda=\chi_{[-\lambda,\lambda]}(B)$ belongs to $\mc M$ for all $\lambda\geq0$. Thus, noting that $(-\im+B)E_\lambda$ is bounded (cf. \cite[Prop. 8.1]{Gui-S}) and hence $E_\lambda\eta\in\Dom(-\im+B)$, we have
\begin{align*}
\bk{\xi|(-\im+B)E_\lambda\eta}=\bk{(\im+B)\xi|E_\lambda\eta}=0
\end{align*}
Since $\xi\in\Dom_0$ is arbitrary and $\Dom_0$ is dense, we have $(-\im+B)E_\lambda\eta=0$. Thus $E_\lambda\eta=0$ because $-\im+B$ is injective. Letting $\lambda\nearrow+\infty$, we get $\eta=0$.
\end{proof}


\subsection{}

Case (1) of the following theorem is \cite[Lem.7.2]{CKLW18}. Although Case (2) is motivated by the treatment of Case (1) in \cite{CKLW18}, and although Case (1) (as well as its proof) is interesting in its own right, we will not use Case (1) in this course. Thus, the proof of Case (1) below can be safely skipped.

\begin{thm}\label{lb76}
Let $A,B$ be self-adjoint closed operators on $\mc H$, and assume that $B$ is affiliated with $\{A\}''$.  Let $\Dom_0$ be a dense subspace of $\Dom(B)$. Suppose that there exist $\delta>0$ and a dense subspace $\Dom_\delta\subset\Dom_0$ such that $e^{\im tA}\Dom_\delta\subset\Dom_0$ for all $-\delta<t<\delta$. Then $\Dom_0$ is a core for $B$ provided that one of the following is true.
\begin{enumerate}[label=(\arabic*)]
\item $B=A^k$ for some $k\in\Nbb$.
\item $A\geq0$.
\end{enumerate} 
\end{thm}



\begin{proof}
By Prop. \ref{lb31}, to show that $\Dom_0$ is a core for $A$, it suffices to choose any $\eta\in\mc H$ orthogonal to $(\im+B)\Dom_0$ and show that $\eta=0$. Since $e^{-\im tA}$ commutes strongly with $B$, we have $e^{-\im tA}(\im+B)\subset(\im+B)e^{-\im tA}$. Therefore, for each $\xi\in\Dom_\delta$ and $|t|<\delta$, since $e^{-\im tA}\xi\in\Dom_0$, we have
\begin{align*}
\bk{(\im+B)\xi|e^{\im tA}\eta}=\bk{(\im+B)e^{-\im tA}\xi|\eta}=0
\end{align*}

Case (1). Let $B=A^k$. Choose $h\in C_c^\infty(-\delta,\delta)$ with $\int_\Rbb h=1$, and let $\wht h$ be its Fourier transform. Recall that $\wht h(A)=\int_\Rbb h(t)e^{-\im tA}dt$, i.e., for each $\psi\in\mc H$ we have $\wht h(A)\psi=\int_\Rbb h(t)e^{-\im tA}\psi dt$ where the RHS is understood as the improper $\mc H$-valued Riemann integral. (See the end of \cite[Sec. 10]{Gui-S}.) Therefore
\begin{align*}
\bk{(\im+A^k)\xi|\wht h(A)\eta}=0
\end{align*}
By the basic properties of Borel functional calculus (cf. \cite[Sec. 9]{Gui-S}), if $\alpha,\beta:\Rbb\rightarrow\Cbb$ are Borel functions, then $(\alpha\beta)(A)=\ovl{\alpha(A)\beta(A)}$ where the latter equals $\alpha(A)\beta(B)$ when $\beta$ is bounded (cf. \cite[Prop. 8.1]{Gui-S}). Thus $A^k\wht h(A)$ equals $(x^k\wht h)(A)=(-\im)^k\wht{h^{(k)}}(A)$, which is a bounded operator because $\wht{h^{(k)}}$ is bounded. In particular $\wht h(A)\eta\in\Dom(-\im+A^k)$. So
\begin{align*}
\bk{\xi|(-\im+A^k)\wht h(A)\eta}=\bk{(\im+A^k)\xi|\wht h(A)\eta}=0
\end{align*}
Since $\xi\in\Dom_\delta$ is arbitrary, we get $(-\im+A^k)\wht h(A)\eta=0$, and hence $\wht h(A)\eta=0$. 

Replacing $h$ with $h_n(x)=nh(nx)$, we have $\wht h_n(A)\eta=0$. Since 
\begin{align*}
\lim_{n\rightarrow\infty}\wht h_n(A)\eta=\lim_{n\rightarrow\infty} \int_\Rbb nh(nx)e^{-\im tA}\eta dx=\eta
\end{align*}
we get $\eta=0$.\\[-1ex]

Case (2). Let $\fk I=\{z\in\Cbb:\Imag z\geq0\}$ and $\xi\in\Dom_\delta$. By Cor. \ref{lb57}, the function 
\begin{align*}
\fk I\mapsto\mc H\qquad z\mapsto \bk{(\im+B)\xi|e^{-\im \ovl zA}\eta}
\end{align*}
is continuous, and is holomorphic on $\Int\fk I$. Since we have shown that this function is zero on an interval of $\Rbb$, by Lem. \ref{lb55}, it is zero for all $z\in\fk I$ and, in particular, all $z=t\in\Rbb$. Thus, as in the proof of Thm. \ref{lb58}, for each $\lambda\geq0$ and $E_\lambda=\chi_{[-\lambda,\lambda]}(B)$ we have
\begin{align*}
\bk{\xi|(-\im+B)E_\lambda\eta}=\bk{(\im+B)\xi|E_\lambda\eta}=0
\end{align*}
for all $\xi$ in the dense space $\Dom_\delta$. Hence $(-\im+B)E_\lambda\eta=0$, and hence $E_\lambda\eta=0$ (for all $\lambda$), and hence $\eta=0$.
\end{proof}




\subsection{}

As our last application of Prop. \ref{lb31}, we give an important criterion for self-adjointness originally due to A. Jaffe (cf. \cite[Thm. 19.4.3]{GJ}). Our presentation follows \cite{FL74}. Recall our assumption on $H$ at the beginning of this section.


\begin{rem}\label{lb60}
Suppose that $A$ is a self-adjoint operator on $\mc H$. Then any symmetric extension of $A$ equals $A$, i.e., if $A\subset B$ where $B$ is a symmetric operator, then $A=B$. Indeed, we have $A\subset B\subset B^*\subset A^*$ and $A=A^*$. Thus $A=B$.
\end{rem}


\begin{thm}\label{lb59}
Let $T$ be a symmetric (i.e. $T\subset T^*$) closed operator on $\mc H$. Assume that $\Dom_0\subset\Dom(H)\cap\Dom(T)$ is a core for $H$, and there exists $C\geq0$ such that for every $\xi,\eta\in\Dom_0$ we have
\begin{subequations}\label{eq78}
\begin{gather}
\Vert T\xi\Vert\leq C\Vert H\xi\Vert\label{eq78a}\\
\big|\bk{T\xi|H\eta}-\bk{H\xi|T\eta} \big|\leq C\Vert H\xi\Vert\cdot\Vert\eta\Vert  \label{eq78b}
\end{gather}
\end{subequations}
Then $\Dom_0$ is a core for $T$, and $T=T^*$.
\end{thm}

We are mainly interested in the case that $\Dom_0$ is invariant under $T$ and $H$. Then \eqref{eq78b} reads that for all $\xi,\eta\in\Dom_0$,
\begin{align*}
\big|\bk{[H,T]\xi|\eta}\big|\leq C\Vert H\xi\Vert\cdot\Vert\eta\Vert 
\end{align*}
equivalently, for all $\xi\in\Dom_0$ we have
\begin{align*}
\big\Vert [H,T]\xi\big\Vert\leq C\Vert H\xi\Vert
\end{align*}
Thus \eqref{eq78} holds, e.g., when $\Dom_0=\HV^\infty$, $H=\ovl{L_0}$, and $T$ is the closure of a linearly energy-bounded field that is symmetric.



\begin{proof}
Step 1. By Rem. \ref{lb60}, if we can prove that $\ovl{T|_{\Dom_0}}$ is self-adjoint, then $\ovl{T|_{\Dom_0}}=T$. Thus $T=T^*$ and $T$ has core $\Dom_0$. Therefore, by replacing $T$ with $\ovl{T|_{\Dom_0}}$, we may assume at the beginning that $\Dom_0$ is a core for $T$, and we only need to prove $T=T^*$. 

We first show that $\Dom(H)\subset\Dom(T)$, and that \eqref{eq78} holds for all $\xi,\eta\in\Dom(H)$. Choose $\xi\in\Dom(H)$. Since $\Dom_0$ is an $H$-core, there exists a sequence $(\xi_n)$ in $\Dom_0$ converging to $\xi$ such that $H\xi_n$  converges to $H\xi$. By \eqref{eq78a}, $(T\xi_n)$ is a Cauchy sequence. Since $T$ is closed, we see that $\xi\in \Dom(T)$ and $T\xi=\lim_n T\xi_n$. 

Let $\xi,\eta\in\Dom(H)$. By the above paragraph, we have sequences $(\xi_n),(\eta_\blt)$ converging to $\xi,\eta$ such that $\lim H\xi_n=H\xi$, $\lim T\xi_n=T\xi$, $\lim H\eta_n=H\eta$, $\lim T\eta_n=T\eta$. Since \eqref{eq78} holds when $\xi,\eta$ are replaced by $\xi_n,\eta_n$,  we see that \eqref{eq78} holds for $\xi,\eta$.\\[-1ex] 

Step 2. Since $H\geq a>0$, by spectral theorem (or by the relation between $H$ and its inverse mentioned in \cite[Sec. 4]{Gui-S}), we have $\Rng(H^{-1})=\Dom(H)$ and $HH^{-1}=\idt_{\mc H}$. So
\begin{align*}
\Rng(H^{-1})=\Dom(H)\subset\Dom(T)
\end{align*}
Let us prove for each $\xi,\eta\in\mc H$ that
\begin{subequations}\label{eq79}
\begin{gather}
\big|\bk{TH^{-1}\xi|\eta}-\bk{\xi|TH^{-1}\eta} \big|\leq C\Vert\xi\Vert\cdot \Vert H^{-1}\eta\Vert\label{eq79a}\\
\big|\bk{TH^{-1}\xi|\eta}-\bk{\xi|TH^{-1}\eta} \big|\leq C\Vert H^{-1}\xi\Vert\cdot \Vert \eta\Vert\label{eq79b}
\end{gather}
\end{subequations}
By \eqref{eq78b} (with $\xi,\eta$ replaced by $H^{-1}\xi,H^{-1}\eta$, both are in $\Dom(H)$), we get \eqref{eq79a}. Since $\bk{TH^{-1}\xi|\eta}-\bk{\xi|TH^{-1}\eta}$ has complex conjugate $-\bk{TH^{-1}\eta|\xi}+\bk{\eta|TH^{-1}\xi}$, \eqref{eq79b} follows by exchanging the $\xi$ and $\eta$ in \eqref{eq79a}.\\[-1ex]

Step 3. Let us prove for each $\xi,\eta\in\mc H$ that
\begin{align}\label{eq80}
|\bk{TH^{-2}\xi|\eta}-\bk{\xi|TH^{-2}\eta}|\leq 2C\Vert H^{-1}\xi\Vert\cdot\Vert H^{-1}\eta\Vert
\end{align}
By \eqref{eq79a} (replacing $\xi$ with $H^{-1}\xi$), we have
\begin{gather*}
\big|\bk{TH^{-2}\xi|\eta}-\bk{H^{-1}\xi|TH^{-1}\eta} \big|\leq C\Vert H^{-1}\xi\Vert\cdot\Vert H^{-1}\eta\Vert
\end{gather*}
By \eqref{eq79b} (replacing $\eta$ with $H^{-1}\eta$), we have
\begin{align*}
\big|\bk{TH^{-1}\xi|H^{-1}\eta}-\bk{\xi|TH^{-2}\eta} \big|\leq C\Vert H^{-1}\xi\Vert\cdot \Vert H^{-1}\eta\Vert
\end{align*}
This proves \eqref{eq80}, since $\bk{H^{-1}\xi|TH^{-1}\eta}=\bk{TH^{-1}\xi|H^{-1}\eta}$.\\[-1ex]








Step 4. For each $\xi\in\mc H$, we have
\begin{align*}
2\im\cdot\Imag\bk{TH^{-2}\xi|\xi}=\bk{TH^{-2}\xi|\xi}-\bk{\xi|TH^{-2}\xi}
\end{align*}
Thus \eqref{eq80} implies
\begin{align}\label{eq81}
|2\Imag\bk{TH^{-2}\xi|\xi}|\leq 2C\Vert H^{-1}\xi\Vert^2
\end{align}

Now, to prove that $T=T^*$, by Prop. \ref{lb31}, it suffices to find some $\lambda\in\Rbb\setminus\{0\}$ such that $T+\lambda\im$ and $T-\lambda\im$ have dense ranges. Choose $\lambda\in\Rbb\setminus\{0\}$ whose value will be determined shortly, let $\xi\in\mc H$ be orthogonal to the range of $T+\lambda\im$. Then
\begin{align*}
\Imag\bk{(T+\lambda\im)H^{-2}\xi|\xi}=0
\end{align*}
namely,
\begin{align*}
\lambda \bk{H^{-2}\xi|\xi}=-\Imag\bk{TH^{-2}\xi|\xi}
\end{align*}
Thus, by \eqref{eq81}, we have $|\lambda|\cdot |\bk{H^{-2}\xi|\xi}|\leq C\Vert H^{-1}\xi\Vert^2$, i.e.
\begin{align*}
|\lambda|\cdot \Vert H^{-1}\xi\Vert^2\leq C\Vert H^{-1}\xi\Vert^2
\end{align*}
So $\xi=0$ whenever $|\lambda|>C$. Thus, taking $\lambda=C+1$, we see that $T+\lambda\im$ and $T-\lambda\im$ have dense ranges.
\end{proof}



\subsection{}


Our next goal is to prove Thm. \ref{lb63}, an extremely important criterion for strong commutativity. It will imply that two energy-bounded fields smeared over disjointly supported smooth functions are strongly commuting if one of them is linearly energy-bounded and its smeared field is symmetric.



Let $T$ be a closable operator on $\mc H$ with \textbf{invariant domain} $\mc H^\infty$ (i.e. $\Dom(T)=\mc H^\infty$ and $T\mc H^\infty\subset\mc H^\infty$). Note that $H^n\Dom(H^m)\subset\Dom(H^{n+m})$ shows that $\mc H^\infty$ is $H^n$-invariant for every $n\in\Zbb$. 

\begin{df}
We say that $T$ satisfies \textbf{\pmb{$H$}-bounds of order $r$}\index{00@$H$-bounds of order $r$} (where $r\geq 0$) if for each $n\in\Nbb$ there exists a constant $|T|_n\geq0$ (the \textbf{\pmb{$n$}-th bounding constant}) such that for every $\xi\in\mc H^\infty$ we have
\begin{align}
\lVert H^nT\xi\lVert\leq |T|_n\cdot \lVert H^{n+r}\xi\lVert.\label{eqa7}	
\end{align}
In other words, if for each $s\in\Rbb$ we define
\begin{align*}
\bk{\xi|\eta}_s=\bk{H^s\xi|H^s\eta}\qquad\text{for all }\xi,\eta\in\Dom(H^s)
\end{align*}
then $T:(\mc H^\infty,\Vert\cdot\Vert_{n+r})\rightarrow (\mc H^\infty,\Vert\cdot\Vert_n)$ is a bounded linear map. $H$-bounds of order $1$ are called \textbf{linear \pmb{$H$}-bounds}.
\end{df}

Clearly, if $0\leq r_1\leq r_2$, then $H$-bounds of order $r_1$ implies $H$-bounds of order $r_2$.


\begin{comment}
\begin{lm}
Choose $n\in\Nbb$. Assume \eqref{eqa7} holds for all $\xi\in\mc H^\infty$. Then
\begin{align*}
\Dom(H^{n+r})\subset\Dom(\ovl T)\qquad	 \ovl T\Dom(H^{n+r})\subset \Dom(H^n)
\end{align*}
and \eqref{eqa7} holds for all $\xi\in\Dom(H^{n+r})$ if $T$ is replaced by $\ovl T$.
\end{lm}


\begin{proof}
Let $\xi\in\Dom(H^{n+r})\subset\Dom(H^r)$. Choose $\xi_k=\chi_{(-k,k)}(H)\xi$, which is in $\mc H^\infty$. Since $H^r\xi_k$ converges to $H^r\xi$, by \eqref{eqa7} (where $n=0$), $T\xi_k$ is a Cauchy sequence. So, since $\xi_k\rightarrow \xi$, we conclude $\xi\in\Dom(\ovl T)$ and $T\xi_k\rightarrow\ovl T\xi$. 

Similarly, since $H^{n+r}\xi_k\rightarrow H^{n+r}\xi$, by \eqref{eqa7}, we conclude that $H^nT\xi_k$ converges to a vector whose norm is bounded by $|T|_n\lVert H^{n+r}\xi\lVert$. Thus, because $T\xi_k\rightarrow \ovl T\xi$, we have $\ovl T\xi\in\Dom(H^n)$ and $H^n\ovl T\xi$ has norm bounded by $|T|_n\lVert H^{n+r}\xi\lVert$.
\end{proof}
\end{comment}



\begin{lm}\label{lba7}
Assume $T$ is closable with invariant  domain $\mc H^\infty$. Assume also that $T$ satisfies linear $H$-bounds. Then for every $n\in\Zbb$ there exists a bounding constant $|T|_n\geq 0$ such that for every $\xi\in\mc H^\infty$ we have
\begin{align*}
\lVert H^nT\xi\lVert\leq |T|_n\cdot \lVert H^{n+1}\xi\lVert	
\end{align*}
\end{lm}

\begin{proof}
We know this is true when $n\geq 0$. Now assume $n<0$ and let $m=-n$. Then for every $\xi,\eta\in\mc H^\infty$,
\begin{align*}
&\big|\bk{H^{-m}T\xi|H^m\eta} \big|=\big|\bk{\xi|T\eta} \big|	=\big|\bk{H^{-m+1}\xi|H^{m-1}T\eta}  \big|\\
\leq&	\lVert H^{-m+1}\xi\lVert\cdot |T|_{m-1}\lVert H^m\eta\lVert.	
\end{align*}
Since $H^m\mc H^\infty$ is dense (because $\mc H^\infty$ is a core for $H^m$, and $\Rng(H^m)$ is dense), we conclude that $\lVert H^{-m}T\xi\lVert\leq |T|_{m-1}\lVert H^{-m+1}\xi\lVert$. 
\end{proof}



\subsection{}


The following theorem is \cite[Prop. 2.1]{Tol99}.

\begin{thm}\label{lba9}
Assume $T$ is a symmetric operator on $\mc H$ with invariant domain $\mc H^\infty$. Assume that both $T$ and $[H,T]$ satisfy linear $H$-bounds. Then $\ovl T$ is self-adjoint. Moreover,  for any $n\in\Nbb$, $\Dom(H^n)$ is $e^{\im t\ovl T}$-invariant, and there exists a constant $C_n\geq 0$ such that for every  $\xi\in\Dom(H^n)$ and $t\in\Rbb$ we have
\begin{align}
\lVert H^n e^{\im t\ovl T}\xi\lVert\leq e^{C_n|t|}\cdot \lVert H^n\xi\lVert	\label{eqa11}
\end{align}
\end{thm}

In short, our conclusion is that $\mc H^\infty$ is $e^{\im t\ovl T}$-invariant, and $e^{\im t\ovl T}:(\mc H^\infty,\bk{\cdot|\cdot}_n)\rightarrow(\mc H^\infty,\bk{\cdot|\cdot}_n)$ is bounded with operator norm $\leq e^{C_n|t|}$. 

\begin{proof}[Idea of the proof]
We know from Thm. \ref{lb59} that $\ovl T$ is self-adjoint. Take $N=H^{2n}$. Using the fact that $[H,T]$ satisfies linear $H$-bounds, it is not hard to check that $[N,T]$ satisfies $H$-bounds of order $2n$, and hence satisfies linear $N$-bounds. From this, one shows that $\im[N,T]\leq cN$ for $c>0$, namely, $\im\bk{[N,T]\xi|\xi}\leq c\bk{N\xi|\xi}$ where $\xi\in\mc H^\infty$.

We ``integrate" the inequality $\im[N,T]\leq cN$ for $t\geq 0$, which gives $e^{-\im t\ovl T}Ne^{\im t\ovl T}\leq e^{ct}N$ evaluate in $\bk{\cdot \xi|\xi}$ for every ``nice" vector $\xi$. This shows $\lVert H^ne^{\im t\ovl T}\xi\lVert^2\leq e^{ct}\lVert H^n\xi\lVert^2$. To be more precise about ``integrating" the inequality, we consider the function $f_\xi(t)=e^{-ct}\bk{e^{-\im t\ovl T}Ne^{\im t\ovl T}\xi|\xi}$. Then $e^{ct}f_\xi'(t)=\im\bk{e^{-\im t\ovl T}[N,T]e^{\im t\ovl T}\xi|\xi}- c\bk{Ne^{\im t\ovl T}\xi|e^{\im t\ovl T}\xi}\leq 0$. So $f_\xi(t)\leq f_\xi(0)$, which shows \eqref{eqa11} when $t\geq 0$. In the case $t\leq 0$, we replace $t$ by $-t$ and $T$ by $-T$, to obtain again \eqref{eqa11}.

The problem with this argument is that we don't know if every $\xi\in\Dom(\mc H^n)$ is good or not. To overcome this difficulty, we replace $N$ by the bounded operator $N_\epsilon=N(1+\epsilon N)^{-1}$ so that all the expressions in the above paragraph can be defined. One shows again that $\im\bk{[N_\epsilon,T]\xi|\xi}\leq c\bk{N_\epsilon\xi|\xi}$ where $\xi\in\mc H^\infty$, and by approximation, $\im\bk{N_\epsilon \ovl T\xi|\xi}-\im\bk{N_\epsilon\xi|\ovl T\xi}\leq c\bk{N_\epsilon\xi|\xi}$ when $\xi\in\Dom(\ovl T)$. Using the argument in the previous paragraph, we obtain (for $t\geq 0$) $e^{-\im t\ovl T}N_\epsilon e^{\im t\ovl T}\leq e^{ct}N_\epsilon$  evaluated in $\bk{\cdot \xi|\xi}$ whenever $\xi\in\Dom(\ovl T)$, and hence whenever $\xi\in\mc H$. Assume $\xi\in\Dom(H^n)$.  Then the limit of this inequality when $\epsilon\searrow0$ yields the desired result.
\end{proof}




\begin{proof}
Step 1. By symmetry, it suffices to prove the claim for $t\geq 0$. By Thm. \ref{lb59}, $\ovl T$ is self-adjoint. Set $N=T^{2n}$. 

For each $k\in\Zbb$, let $|T|_k$ be a $k$-th bounding constant for both $T$ and $[H,T]$. (Cf. Lemma \ref{lba7}.) Then, when acting on $\mc H^\infty$, 
\begin{align*}
	[N,T]=\sum_{j=0}^{2n-1}H^j[H,T]H^{2n-1-j}
\end{align*}
Set $c=\sum_{j=0}^{2n-1}|T|_{j-n}$. For each $\xi\in\mc H^\infty$,  
\begin{align*}
	&\im \bk{[N,T]\xi|\xi}=\im \bk{H^{-n}[N,T]\xi|H^n\xi}\leq \lVert H^n\xi \lVert\cdot \lVert H^{-n}[N,T]\xi \lVert\\
\leq&\sum_{j=0}^{2n-1}	\lVert H^n\xi \lVert\cdot\lVert H^{j-n}[H,T]H^{2n-1-j}\xi \lVert\leq \sum_{j=0}^{2n-1}	|T|_{j-n}	\lVert H^n\xi \lVert^2 = c\bk{N\xi|\xi}\tag{a}\label{eqa12}
\end{align*}


Step 2. In general,  $[S^{-1},T]=-S^{-1}[S,T]S^{-1}$ if a linear map $S:\mc H^\infty\rightarrow\mc H^\infty$  has inverse $S^{-1}:\mc H^\infty\rightarrow\mc H^\infty$. Choose $\epsilon>0$. Then, when acting on $\mc H^\infty$, we have
\begin{align*}
	[(1+\epsilon N)^{-1},T]=-\epsilon (1+\epsilon N)^{-1}[N,T](1+\epsilon N)^{-1}
\end{align*}
and hence
\begin{align*}
	&[N(1+\epsilon N)^{-1},T]=-N\cdot \epsilon (1+\epsilon N)^{-1}[N,T](1+\epsilon N)^{-1}+[N,T](1+\epsilon N)^{-1}\\
	=&	(1+\epsilon N)^{-1}[N,T](1+\epsilon N)^{-1}
\end{align*}
Note that $\mc H^\infty$ is invariant under $T,N$. Also $\mc H^\infty=\bigcap_{p\geq0}\Dom(H^p)$ is invariant under $f(H)$ for any bounded Borel function $f$, because $f(H)H^p\subset H^pf(H)$. Thus $\mc H^\infty$ is invariant under $(1+\epsilon N)^{-1}$. It follows that for every $\xi\in\mc H^\infty$, we have (by \eqref{eqa12})
\begin{align*}
&\im\bk{[N(1+\epsilon N)^{-1},T]\xi|\xi}=\im\bk{[N,T](1+\epsilon N)^{-1}\xi|(1+\epsilon N)^{-1}\xi}\\
\leq &c\bk{(1+\epsilon N)^{-1}N(1+\epsilon N)^{-1}\xi|\xi}
\end{align*}
By Borel functional calculus, the bounded operator $N(1+\epsilon N)^{-1}-N(1+\epsilon N)^{-2}$ is positive (since it equals $g(N)$ where $g(x)=x(1+\epsilon x)^{-1}-x(1+\epsilon x)^{-2}$ is positive). Therefore
\begin{align*}
\im\bk{[N(1+\epsilon N)^{-1},T]\xi|\xi}\leq 	c\bk{N(1+\epsilon N)^{-1}\xi|\xi}\tag{b}\label{eqa13}
\end{align*}
for all $\xi\in\mc H^\infty$.

Now suppose $\xi\in\Dom(\ovl T)$. Since $\mc H^\infty=\Dom(T)$ is a core for $\ovl T$, we may choose a sequence $\xi_n\in\mc H^\infty$ converging to $\xi$ such that $T\xi_n$ converges to $T\xi$. Note that $N(1+\epsilon N)^{-1}$ is bounded. Therefore, since each $\xi_n$ satisfies \eqref{eqa13}, we obtain
\begin{align*}
\im\bk{N(1+\epsilon N)^{-1}\ovl T\xi|\xi}-\im\bk{N(1+\epsilon N)^{-1}\xi|\ovl T\xi}	\leq 	c\bk{N(1+\epsilon N)^{-1}\xi|\xi}\tag{c}\label{eqa14}
\end{align*}


Step 3. Note that $N^{\frac 12}(1+\epsilon N)^{-\frac 12}$ and $N(1+\epsilon N)^{-1}$ are  bounded. For any $\xi\in\mc H$, we set
\begin{gather*}
\xi_t=e^{\im t\ovl T}\xi\\
f_{\epsilon,\xi}(t)=	e^{-ct}\big\lVert{N^{\frac 12}(1+\epsilon N)^{-\frac 12}\xi_t}\big\lVert^2=e^{-ct}\bk{N(1+\epsilon N)^{-1}\xi_t|\xi_t}
\end{gather*}


Assume $\xi\in\Dom(\ovl T)$. Then $\xi_t\in\Dom(\ovl T)$ and $\frac d{dt}\xi_t=\im\ovl T\xi_t$. So the derivative $f'_{\epsilon,\xi}(t)=\frac d{dt}f_{\epsilon,\xi}(t)$ exists for all $t\in\Rbb$.  We compute
\begin{align*}
f'_{\epsilon,\xi}(t)=-cf_{\epsilon,\xi}(t)+\im e^{-ct}\bk{N(1+\epsilon N)^{-1}\ovl T\xi_t|\xi_t}-\im e^{-ct}\bk{N(1+\epsilon N)^{-1}\xi_t|\ovl T\xi_t}\tag{d}\label{eqa9}
\end{align*}
By \eqref{eqa14},
\begin{align*}
f'_{\epsilon,\xi}(t)\leq-cf_{\epsilon,\xi}(t)+ce^{-ct}\bk{N(1+\epsilon N)^{-1}\xi_t|\xi_t}=0
\end{align*}
Therefore, when $t\geq 0$, we have $f_{\epsilon,\xi}(t)\leq f_{\epsilon,\xi}(0)$, namely,
\begin{align}
e^{-ct}\big\lVert{N^{\frac 12}(1+\epsilon N)^{-\frac 12}\xi_t}\big\lVert^2\leq \big\lVert{N^{\frac 12}(1+\epsilon N)^{-\frac 12}\xi}\big\lVert^2\tag{e}\label{eqa10}
\end{align}
By approximation, \eqref{eqa10} holds for all $\xi\in\mc H$.

By spectral theory, the RHS of \eqref{eqa10} increases as $\epsilon$ decreases. Moreover, by the monotone convergence theorem, we see that
\begin{align*}
\xi\in\Dom(N^{\frac12})=\Dom(H^n)\qquad\Longleftrightarrow\qquad\limsup_{\epsilon\searrow0}\big\lVert{N^{\frac 12}(1+\epsilon N)^{-\frac 12}\xi}\big\lVert^2<+\infty
\end{align*}
Therefore, if we now choose $\xi\in\Dom(H^n)=\Dom(N^{\frac 12})$, then by \eqref{eqa10} and the monotone convergence theorem, we have $\xi_t\in\Dom(H^n)$ and 
\begin{align*}
e^{-ct}\Vert H^n\xi_t\Vert^2\leq \Vert H^n\xi\Vert^2
\end{align*}
The proof is completed by setting $C_n=c/2$.
\end{proof}




\subsection{}

\begin{df}
For each $n\in\Nbb$, we let $o_H(h^n)$ \index{oh@$o_H(h^n)$} be the set of $\mc H^\infty$-valued functions $\psi=\psi(h)$ where each $\psi$ is  defined on a neighborhood of $0\in\Rbb$ and satisfies  for all $m\in\Nbb$ that
\begin{align}
	\lim_{h\rightarrow 0}\frac{~\lVert H^m \psi(h)\lVert~}{h^n}=0.\label{eqa16} 
\end{align}
Clearly, if \eqref{eq16} holds for all $m\in\Nbb$, then it holds for all $m\in\Rbb_{\geq0}$.
\end{df}



\begin{rem}\label{lba11}
If $S$ is closable on $\mc H$ with invariant domain $\mc H^\infty$, and if $S$ satisfies $H$-bounds of some order $r$, then for all $n\in\Nbb$ it is clear that
\begin{align*}
S\cdot o_H(h^n)\subset o_H(h^n).	
\end{align*}
Moreover, by Thm. \ref{lba9}, we also have for all $t\in\Rbb$ that
\begin{align*}
e^{\im t\ovl S}o_H(h^n)\subset o_H(h^n),\qquad e^{\im (t+h)\ovl S}o_H(h^n)\subset o_H(h^n).	
\end{align*}
\end{rem}

By Taylor series expansion, if $f$ is a smooth function on $(a,b)\subset\Rbb$, then for any $t,t+h\in(a,b)$ and $n\in\Nbb$, we have (cf. \cite[Thm. 9.29]{Apo})
\begin{align}
	f(t+h)=\sum_{k=0}^n\frac{f^{(k)}(t)}{k!}h^k+\frac 1{n!}\int_t^{t+h}(t-s)^nf^{(n+1)}(s)ds.	\label{eqa15}
\end{align}
We are now ready to prove the Taylor theorem for $e^{\im t\ovl T}\xi$ due to \cite[Cor. 2.2]{Tol99}.



\begin{thm}\label{lba10}
Let $T$ be as in Thm. \ref{lba9}. Then for every $\xi\in\mc H^\infty,n\in\Nbb,t\in\Rbb$, we have
\begin{align}
e^{\im(t+h)\ovl T}\xi=\sum_{k=0}^n\frac{(\im T)^k}{k!}e^{\im t\ovl T}\xi+R(h)\xi	
\end{align}
where each summand is in $\mc H^\infty$, and  $R_t(h)\xi\in o_H(h^n)$.
\end{thm}


\begin{proof}
For every $\xi\in\mc H^\infty$, since $e^{\im t\ovl T}\xi$ belongs to $\mc H^\infty$ (Thm. \ref{lba9}), we have $e^{\im t\ovl T}\xi\in\Dom(\ovl T^m)$ for all $m\in\Nbb$. Thus the function $t\in\Rbb\mapsto e^{\im t\ovl T}\xi$ has derivative $\im Te^{\im t\ovl T}\xi$ everywhere. Applying \eqref{eqa15} to $f(t)=\bk{e^{\im(t+h)\ovl T}\xi|\eta}$ (for each $\eta\in\mc H$),  we obtain
\begin{align*}
R_t(h)\xi=\frac 1{n!}\int_t^{t+h}(t-s)^nT^{n+1}e^{\im s\ovl T}\xi ds.
\end{align*}
Thus, for any $\eta\in\mc H^\infty$, by \eqref{eqa7} and Thm. \ref{lba9}, there exist $\lambda,C>0$ such that
\begin{align*}
&|\bk{H^mR_t(h)\xi|\eta}|=|\bk{R_t(h)\xi|H^m\eta}|\leq \frac {h^{n+1}}{(n+1)!}\sup_{t\leq s\leq t+h}\big|\bk{T^{n+1}e^{\im s\ovl T}\xi|H^m\eta}\big|\\
=&\frac {h^{n+1}}{(n+1)!}\sup_{t\leq s\leq t+h}\big|\bk{H^mT^{n+1}e^{\im s\ovl T}\xi|\eta}\big|\leq\frac {h^{n+1}}{(n+1)!}\lambda\cdot\sup_{t\leq s\leq t+h}\lVert H^{m+n+1}e^{\im s\ovl T}\xi\lVert\cdot\lVert\eta\lVert\\
	\leq&\frac {h^{n+1}}{(n+1)!}\lambda e^{C|h|}\cdot\lVert H^{m+n+1}e^{\im t\ovl T}\xi\lVert\cdot\lVert\eta\lVert
\end{align*}
This proves $R_t(h)\xi\in o_H(h^n)$.
\end{proof}
	


\subsection{}


In this course, instead of using Thm. \ref{lba10} directly, we will use Cor. \ref{lb61}.

\begin{df}
Let $I$ be an interval in $\Rbb$. Let $U,V$ be inner product spaces, and let $T:t\in\Rbb\mapsto T_t\in\mc L(U,V)$. We say that $T$ has \textbf{(strong operator) derivative} at $t_0\in I$ if for each $\xi\in U$, the limit $\dps\lim_{t\rightarrow t_0}\frac{T_t\xi-T_{t_0}\xi}{t-t_0}$ converges to an element of $V$. In that case, we denote that limit by $\partial_tT\xi|_{t=t_0}$.
\end{df}


\begin{pp}\label{lb62}
Let $I\subset\Rbb$ be an interval. Let $U,V,W$ be inner product spaces. Let $T:t\in I\mapsto T_t\in\mc L(U,V)$ and $S:t\in I\mapsto S_t\in\mc L(V,W)$. Let $t_0\in I$. Assume that both $S,T$ have derivatives at $t_0$, and that
\begin{align}\label{eq84}
\sup_{t\in I}\Vert S_t\Vert<+\infty
\end{align}
Then we have the \textbf{Leibniz rule}
\begin{align*}
\partial_t(S_t\circ T_t)\big|_{t=t_0}=\partial_tS_t\big|_{t=t_0}\circ T_{t_0}+S_{t_0}\circ \partial_tT_t\big|_{t=t_0}
\end{align*}
\end{pp}

\begin{proof}
Let $\xi\in U$. Assume for simplicity that $t_0=0$. Then for each $\xi\in U$,
\begin{align*}
(S_t T_t-S_0T_0)\xi=S_t(T_t-T_0)\xi+(S_t-S_0)T_0\xi
\end{align*}
Let $S'=\partial_t S_t|_{t=0}$ and $T'=\partial_t T_t|_{t=0}$. Then
\begin{align*}
\lim_{t\rightarrow0}t^{-1}(S_t-S_{t_0})T_0\xi=S'T_0\xi
\end{align*}
Moreover, we can write $t^{-1}(T_t-T_0)\xi=T'\xi+\eta(t)$ where $\lim_{t\rightarrow0}\eta(t)=0$. Since $t\mapsto S_tT'\xi$ is differentiable at $0$, it is continuous at $0$, and hence
\begin{align*}
\lim_{t\rightarrow 0}S_tT'\xi=S_0T'\xi
\end{align*}
Let $C$ be the finite constant in \eqref{eq84}. Then $\Vert S_t\eta(t)\Vert\leq C\Vert\eta(t)\Vert$, and hence $\lim_{t\rightarrow 0}S_t\eta(t)=0$. Thus $\lim_{t\rightarrow0}t^{-1}(S_t T_t-S_0T_0)\xi$ converges to $S_0T'\xi+S'T_0\xi$.
\end{proof}








\begin{co}\label{lb61}
Let $T$ be as in Thm. \ref{lba9}. Then for any $m\in\Nbb$, the family $t\in\Rbb\mapsto e^{\im t\ovl T}$ where
\begin{align}\label{eq82}
e^{\im t\ovl T}:(\mc H^\infty,\bk{\cdot|\cdot}_m)\rightarrow (\mc H^\infty,\bk{\cdot|\cdot}_m)
\end{align}
has derivative $Te^{\im t\ovl T}\big|_{\mc H^\infty}=e^{\im t\ovl T}T\big|_{\mc H^\infty}$ at each $t\in\Rbb$.
\end{co}

\begin{proof}
This is obvious from Thm. \ref{lba10}. Note that by Thm. \ref{lba9}, the linear map \eqref{eq82} is bounded.
\end{proof}


	
\begin{thm}\label{lb63}
Let $S,T$ be closable operators on $\mc H$ with common ($S$- and $T$-)invariant domain $\Dom(S)=\Dom(T)=\mc H^\infty$. Assume $T$ is symmetric, $T$ and $[H,T]$ satisfy linear $H$-bounds,  $S$ satisfies $H$-bounds of some order $r\geq 0$, and $[S,T]=0$ on $\mc H^\infty$. Then $\ovl S$ commutes strongly with $\ovl T$.
\end{thm}	
	



\begin{proof}
By Thm. \ref{lba9}, $\ovl T$ is self-adjoint, and $e^{\im t\ovl T}$ leaves $\mc H^\infty$ invariant. Since $\{\ovl T\}''=\{e^{\im t\ovl T}:t\in\Rbb\}$, we need to show $e^{\im t\ovl T}\ovl Se^{-\im t\ovl T}=\ovl S$ for all $t$, which follows if we can show $e^{\im t\ovl T}Se^{-\im t\ovl T}=S$ (on $\mc H^\infty$).

By enlarging $r$, we assume that $r\in\Nbb$. Since $S$ satisfies $r$-th order $H$-bounds, the linear map $S:(\mc H^\infty,\bk{\cdot|\cdot}_r)\rightarrow\mc H^\infty$ is bounded. By Cor. \ref{lb61}, the maps $t\mapsto e^{-\im t\ovl T}\in\fk L(\mc H^\infty,\bk{\cdot|\cdot}_r)$ and $t\mapsto e^{\im t\ovl T}\in\fk L(\mc H^\infty)$ have derivatives $-Te^{\im t\ovl T}|_{\mc H^\infty}$ and $Te^{\im t\ovl T}|_{\mc H^\infty}$ everywhere. Thus, by the Leibniz rule Prop. \ref{lb62} 
%(noting that  \eqref{eq84} is ensured by Thm. \ref{lba9}) 
and the fact that $[S,T]=0$, we see that $t\in\Rbb\mapsto e^{\im t\ovl T}Se^{-\im t\ovl T}$ has derivative $0$ everywhere. Thus for each $\xi,\eta\in\mc H^\infty$, the function $t\mapsto \bk{e^{\im t\ovl T}Se^{-\im t\ovl T}\xi|\eta}$ has derivative $0$, and hence must be constant, and hence is always $\bk{S\xi|\eta}$.
\end{proof}


\subsection{}

We close this section with another type of criteria for strong commutativity. %In this section, instead of the assumption $H\geq a$, we assume the weaker one that $H$ is self-adjoint.


Let $A_1,\dots,A_m$ and $B_1,\dots,B_n$ be closable operators on $\mc H$. Let $P(z_1,\dots,z_m)$ and $Q(\zeta_1,\dots,\zeta_n)$ be polynomials of non-commuting formal variables $z_1,\dots,z_m$ and $\zeta_1,\dots,\zeta_n$. Then we have (not necessarily densely-defined) unbounded operators $P(A_\blt)=P(A_1,\dots,A_m)$ and $Q(B_\blt)=Q(B_1,\dots,B_n)$, and their respective extensions $P(\ovl {A_\blt})=P(\ovl {A_1},\dots,\ovl {A_n})$ and $Q(\ovl{B_\blt})=B(\ovl{B_1},\dots,\ovl{B_n})$.


\begin{pp}
Assume that $P(\ovl{A_\blt})$ and $Q(\ovl{B_\blt})$ are densely-defined and closable, and each $\ovl{A_i}$ commutes strongly with each $\ovl{B_j}$. Then $\ovl{P(\ovl{A_\blt})}$ commutes strongly with $\ovl{Q(\ovl{B_\blt})}$.
\end{pp}

%From the following proof, it is clear that for this proposition, the assumption that $\mc H^\infty$ is an invariant domain of each $A_i$ and $B_j$ can be replaced by the weaker assumption that $P(\ovl{A_\blt})$ and $Q(\ovl{B_\blt})$ are densely-defined.

\begin{proof}
Let $\mc M=\{\ovl{A_1},\dots,\ovl{A_m}\}''$ and $\mc N=\{\ovl{B_1},\dots,\ovl{B_n}\}''$. By assumption, we have $[\mc M,\mc N]=0$. One shows easily that each element of $\mc M'=\{\ovl{A_1},\dots,\ovl{A_m}\}'$ commutes strongly with $P(\ovl{A_\blt})$, and hence (by taking closure) commutes strongly with $\ovl{P(\ovl{A_\blt})}$. This proves
\begin{align}
\{\ovl{P(\ovl{A_\blt})}\}''\subset\{\ovl{A_1},\dots,\ovl{A_m}\}''
\end{align}
and similarly $\{\ovl{Q(B_\blt)}\}''\subset\mc N$.
\end{proof}


In general, it is not clear whether $P(\ovl{A_\blt})$ and $Q(\ovl{B_\blt})$ is closable. Therefore, it is often more convenient to consider $\ovl{P(A_\blt)}$ and $\ovl{Q(B_\blt)}$ instead of $\ovl{P(\ovl{A_\blt})}$ and $\ovl{Q(\ovl{B_\blt})}$. To prove a similar result for $\ovl{P(A_\blt)}$ and $\ovl{Q(B_\blt)}$, we first need the following lemma.



\begin{lm}\label{lb64}
Let $x\in\fk L(\mc H)$. Let $h\in C_c^\infty(\Rbb)$. Define
\begin{align}
x(h)=\int_\Rbb h(t)e^{\im tH}xe^{-\im tH}dt
\end{align}
sending each $\xi\in\mc H$ to $\int_\Rbb h(t)e^{\im tH}xe^{-\im tH}\xi dt$. The following are true.
\begin{enumerate}
\item We have $x(h)\mc H^\infty\subset\mc H^\infty$ and $x(h)^*\mc H^\infty\subset\mc H^\infty$.
\item Assume  $\int_\Rbb h=1$. Let $h_r(t)=r h(r t)$. Then $\dps\lim_{r\rightarrow+\infty}x(h_r)$ converges strongly* to $x$.
\end{enumerate}
\end{lm}


\begin{proof}
For each $s\in\Rbb$, let $\tau_sh(t)=h(t-s)$. Then one checks easily
\begin{align}
e^{\im sH}x(h)=x(\tau_sh)e^{\im sH}
\end{align} 
One shows easily that $\partial_s x(\tau_sh)$ converges strongly (at every $s$) to $-x(\tau_sh')$. Therefore, by Prop. \ref{lb62}, for each $\xi\in\Dom(H)$ we have
\begin{align}\label{eq85}
\partial_se^{\im sH}x(h)\xi=-x(\tau_sh')e^{\im sH}\xi+\im x(\tau_sh)e^{\im sH}H\xi
\end{align}
In particular, since the derivative on the LHS exists at $s=0$, we see that $x(h)\xi\in\Dom(H)$. Now for each $\xi\in\mc H^\infty$, Eq. \eqref{eq85} shows that $\partial_s^ne^{\im sH}\xi$ exists for any $n\in\Nbb$. This proves $x(h)\mc H^\infty\subset\mc H^\infty$. Since
\begin{align*}
x(h)^*=x^*(\ovl h)
\end{align*}
we get $x(h)^*\mc H^\infty\subset\mc H^\infty$.

Now assume $\int_\Rbb h=1$. One shows easily that $x(h_r)$ converges strongly to $x$, and similarly, $x(h_r)^*=x^*(\ovl{h_r})$ converges strongly to $x^*$.
\end{proof}



\begin{pp}\label{lb68}
Let $\delta>0$. For each $1\leq i\leq m$ and $1\leq j\leq n$, assume that $\mc H^\infty$ is an invariant domain of $A_i$ and $B_j$, and that $e^{\im tH}\ovl{A_i}e^{-\im tH}$ commutes strongly with $\ovl{B_j}$ for each $-\delta<t<\delta$. Assume that $P(A_\blt)$ and $Q(B_\blt)$ (with common domain $\mc H^\infty$) are closable. Then $\ovl{P(A_\blt)}$ commutes strongly with $\ovl{Q(B_\blt)}$. 
\end{pp}




\begin{proof}
We first prove the special case that $m=1$ and $P(z_1)=z_1$. Let $\mc M=\{\ovl A_1\}''$. For each $x\in\mc M$, we have that $e^{\im tH}xe^{-\im tH}$ commutes strongly with each $\ovl{B_j}$ for each $t\in(-\delta,\delta)$. Therefore, if we let $h\in C_c^\infty(-\delta,\delta)$ such that $\int h=1$, then $x(h)$ commutes strongly with each $\ovl{B_j}$. 

Noting Lem. \ref{lb64}-1, we have $x(h)B_j=B_jx(h)$ and $x(h)^*B_j=B_jx(h)^*$ as unbounded operators with common domain $\mc H^\infty$. Thus $x(h)Q(B_\blt)=Q(B_\blt)x(h)$ and $x(h)^*Q(B_\blt)=Q(B_\blt)x(h)$, i.e., $x(h)$ commutes strongly with $Q(B_\blt)$, and hence commutes strongly with $\ovl{Q(B_\blt)}$. By Lem. \ref{lb64}-2, we see that $x$ commutes strongly with $\ovl{Q(B_\blt)}$. This proves that $\ovl A_1$ commutes strongly with $\ovl{Q(B_\blt)}$. 

Note that for each $t\in(-\delta,\delta)$, $e^{\im tH}A_1e^{-\im tH}$ and $B_\blt$ satisfy a similar property. Thus $e^{\im tH}\ovl {A_1}e^{-\im tH}$ commutes strongly with $\ovl{Q(B_\blt)}$.

Now we consider general $m$ and $P$. Since $\ovl{A_i}$ commutes strongly with $e^{-\im tH}\ovl{B_j}e^{\im tH}$ for all $t\in(-\delta,\delta)$, the above paragraph shows that $\ovl{P(A_\blt)}$ commutes strongly with $e^{-\im tH}\ovl{B_j}e^{\im tH}$ for all $t\in(-\delta,\delta)$. Thus, by the above paragraph again, $\ovl{P(A_\blt)}$ commutes strongly with $\ovl{Q(B_\blt)}$.
\end{proof}



\newpage


\section{M\"obius covariance of smeared chiral fields}


In this section, we fix a chiral algebra $(\mc V,\Vbb)$.

\begin{df}
We say that $\Vbb$ is \textbf{energy-bounded}, \index{00@Energy-bounded chiral algebra}if every $A(z)\in\mc V$ is energy-bounded.
\end{df} 

\subsection{}


Choose $c_{-1},c_0,c_1\in\Cbb$, and define a holomorphic vector field
\begin{align}\label{eq83}
X=c_{-1}l_{-1}+c_0l_0+c_1l_1=(c_{-1}+c_0z+c_1z^2)\partial_z
\end{align}
on $\Cbb$. (Recall $l_n=z^{n+1}\partial_z$.) The \textbf{Virasoro operator associated to \pmb{$X$}} is the element $L_X\in\End(\Vbb)$ defined by \index{LX@$L_X$, the Virasoro operator associated to the vector field $X$}
\begin{align}
L_X=c_{-1}L_{-1}+c_0L_0+c_1L_1
\end{align}
Then $L_X$ can be viewed as a smeared operator for the Virasoro field $T(z)=\sum_{n=0,\pm1}L_nz^{-n-2}$. Since $T(z)$ is energy-bounded (cf. Cor. \ref{lb43}), by Conv. \ref{lb48}, $L_X$ can be viewed as a closable operator on $\HV$ with invariant domain $\HV^\infty$.



$X$ extends to a global holomorphic vector field on $\Pbb^1=\Cbb\cup\{\infty\}$. In fact, setting $\varpi=1/z$, we have on $\Pbb^1\setminus\{0\}$ that
\begin{align}\label{eqb7}
X=-(c_1+c_0\varpi+c_{-1}\varpi^2)\partial_\varpi
\end{align}
Therefore, it generates a holomorphic flow \index{exp@$\exp^X_\tau$}
\begin{align}
(\tau,z)\in\Cbb\times\Pbb^1\mapsto \exp^X_\tau(z)\in\Pbb^1
\end{align}
Namely, it is a two-variable holomorphic function satisfying
\begin{align}\label{eq100}
\exp^X_{\tau_1}\circ\exp^X_{\tau_2}(z)=\exp^X_{\tau_1+\tau_2}(z)
\end{align}
and it is determined by the differential equation with initial condition
\begin{subequations}\label{eq87}
\begin{gather}
\partial_\tau\exp^X_\tau(z)=c_{-1}+c_0\exp^X_\tau(z)+c_1(\exp^X_\tau(z))^2\label{eq87a}\\
\exp_0^X(z)=z
\end{gather}
\end{subequations}
for each $\tau\in\Cbb,z\in\Pbb^1$. Since $\exp_\tau^X:\Pbb^1\rightarrow\Pbb^1$ is a biholomorphism, it is a linear fractional transform. 


From the uniqueness of solutions of \eqref{eq87}, it is clear that for each $\lambda\in\Cbb$ we have
\begin{align}
\exp^{\lambda X}_\tau=\exp^X_{\lambda\tau}
\end{align}

\subsection{}
For $X=\eqref{eq8}$, let
\begin{subequations}\label{eqb6}
\begin{gather}
X^\tr=c_{-1}l_1+c_0l_0+c_1l_{-1}\label{eqb6a}\\
X^\Co=\ovl {c_{-1}}l_{-1}+\ovl{c_0}l_0+\ovl {c_1} l_1\\
 X^\dagger=(X^\Co)^\tr=(X^\tr)^\Co=\ovl {c_{-1}}l_1+\ovl {c_0}l_0+\ovl {c_1} l_{-1}
\end{gather}
\end{subequations}
Then clearly
\begin{align*}
(L_X)^\dagger=L_{X^\dagger}\qquad\text{on $\HV^\infty$}
\end{align*}
i.e., $\bk{L_X\xi|\eta}=\bk{\xi|L_{X^\dagger}\eta}$ for all $\xi,\eta\in\HV^\infty$. Therefore, if $X=X^\dagger$ then $L_X$ is a symmetric operator.

\begin{rem}\label{lb75}
By Cor. \ref{lb43}, $T(z)$ is linearly energy-bounded. Therefore, if $X=X^\dagger$, then by Thm. \ref{lb59}, the closure $\ovl{L_X}$ is self-adjoint. By Thm. \ref{lba9}, for all $t\in\Rbb$ we have $e^{\im t\ovl{L_X}}\HV^\infty\subset\HV^\infty$.
\end{rem}


The flows of  $X,X^\tr,X^\Co$ are related by the following formulas:

\begin{lm}\label{lb73}
For each $\tau\in\Cbb$ and $z\in\Pbb^1$ we have
\begin{align}\label{eqb9}
1/\exp_\tau^X(1/z)=\exp_{-\tau}^{X^\tr}(z)
\end{align}
\end{lm}
\begin{proof}
Comparing \eqref{eqb7} with \eqref{eqb6a}, we see that the biholomorphism $z\in\Pbb^1\mapsto z^{-1}\in\Pbb^1$ pulls $X$ back to $-X^\tr$.
\end{proof}





\begin{lm}\label{lbb5}
For each $\tau\in\Cbb$ and $z\in\Pbb^1$ we have
\begin{align}
\ovl{\exp^X_\tau(z)}=\exp^{X^\Co}_{\ovl\tau}(\ovl z)
\end{align}
\end{lm}
\begin{proof}
By \eqref{eq87a} we have $\partial_{\ovl\tau}\exp^{X^\Co}_{\ovl\tau}(\ovl z)=\ovl{c_{-1}}+\ovl{c_0}\exp^{X^\Co}_{\ovl\tau}(\ovl z)+\ovl{c_1}\exp^{X^\Co}_{\ovl\tau}(\ovl z)^2$. Therefore, 
\begin{align*}
\partial_\tau \ovl{\exp^{X^\Co}_{\ovl\tau}(\ovl z)}=\ovl{\partial_{\ovl\tau}\exp^{X^\Co}_{\ovl\tau}(\ovl z)}=c_{-1}+c_0\ovl{\exp^{X^\Co}_{\ovl\tau}(\ovl z)}+c_1\ovl{\exp^{X^\Co}_{\ovl\tau}(\ovl z)}^2
\end{align*}
Clearly $\ovl{\exp^{X^\Co}_{\ovl\tau}(\ovl z)}\Big|_{\tau=0}=z$. So $\ovl{\exp^{X^\Co}_{\ovl\tau}(\ovl z)}$ and $\exp^X_\tau(z)$ satisfy the same differential equation (over $\tau$) and the same initial condition. So they are equal.
\end{proof}


\begin{rem}\label{lb74}
Suppose that $X=X^\dagger$. Then by Lem. \ref{lb73} and \ref{lbb5}, we have
\begin{align*}
\ovl{\exp^X_{\im t}(1/z)}=1/\exp^X_{\im t}(\ovl z)
\end{align*}
for each $t\in\Rbb$. Therefore, if $z\in\Sbb^1$ (i.e. $\ovl z=1/z$), then $\exp_{\im t}^X(z)\in\Sbb^1$. Thus $\exp_{\im t}^X=\exp^{\im X}_t$ restricts to a diffeomorphism of $\Sbb^1$, i.e., it belongs to $\PSU$. Thus, it also restricts to a diffeomorphism of $\Dbb^1$.
\end{rem}



\subsection{}

We now give the \textbf{M\"obius covariance theorem} for smeared fields.

\begin{thm}\label{lb65}
Let $A(z)\in\mc V$ be energy-bounded. Let $f\in C^\infty(\Sbb^1)$. Let $X=\eqref{eq83}$ satisfy $X=X^\dagger$. Then for each $t\in\Rbb$, the following relation holds as unbounded operators with common invariant domain $\HV^\infty$:
\begin{align}\label{eq86}
e^{\im t\ovl{L_X}} A(f)e^{-\im t\ovl{L_X}}=\ointn_{\Sbb^1} \big(\partial_z\exp^{\im X}_t(z)\big)^{\wt(A)}f(z)\cdot A\big(\exp^{\im X}_t(z)\big)\sd z
\end{align}
\end{thm}


Note that by Def. \ref{lb52}, the RHS of \eqref{eq86} is a priori a linear map $\Vbb\rightarrow\Vbb^\ac$. By the change of coordinate $\zeta=\exp^{\im X}_t(z)$, the RHS of \eqref{eq86} can be expressed as a smeared operator. Hence by Conv. \ref{lb53}, the  RHS of \eqref{eq86} can be viewed as a closable operator with invariant domain $\HV^\infty$.

Note also that if $f\in C_c^\infty(I)$ where $I$ is an open interval of $\Sbb^1$, then the RHS of \eqref{eq86} is smeared over a function in $C_c^\infty(\exp_t^{\im X}(I))$. 

\begin{proof}
We denote the RHS of \eqref{eq86} by $R_t$, which has invariant domain $\HV^\infty$ by the basic property of smeared operators (cf. Conv. \ref{lb48}). It suffices to show that the function $t\in\Rbb\rightarrow e^{-\im t\ovl{L_X}} R_te^{\im t\ovl{L_X}}$ has constant derivative $0$ when evaluated with any $\xi,\eta\in\HV^\infty$.\\[-1ex]




Step 1. Suppose that $A$ satisfies $r$-th order energy bounds. By Thm. \ref{lb39}-(c), there exist $M,q\geq0$ such that for each $f\in C^\infty(\Sbb^1)$, the linear map $A(f):(\HV^\infty,\bk{\cdot|\cdot}_r)\rightarrow\HV^\infty$ is bounded with operator norm $\leq M|f|_q$. Therefore, if we from now on equip $C^\infty(S^1)$ with the norm $|\cdot|_q$, then
\begin{gather}\label{eq89}
C^\infty(\Sbb^1)\rightarrow\fk L\big((\HV^\infty,\bk{\cdot|\cdot}_r),\HV^\infty\big)\qquad f\mapsto A(f)
\end{gather}
is a bounded linear map with operator norm $\leq M$.

Now fix $f\in C^\infty(\Sbb^1)$, and write $R_t=A(g_t)$ where $g_t\in C^\infty(\Sbb^1)$ can be explicitly computed by the change of variable formula. From the computation, one sees that the vector valued function
\begin{align*}
t\in\Rbb\mapsto g_t\in C^\infty(\Sbb^1)
\end{align*}
is differentiable (where $C^\infty(\Sbb^1)$ is given the $|\cdot|_q$-norm). Therefore, by the boundedness of \eqref{eq89}, the function
\begin{align*}
\Rbb\rightarrow\fk L\big((\HV^\infty,\bk{\cdot|\cdot}_r),\HV^\infty\big) \tag{a}\qquad t\mapsto R_t=A(g_t)\label{eq88}
\end{align*}
is differentiable, and its derivative $\partial_tR_t$ equals the smeared operator $A(\partial_tg_t)$.\\[-1ex]



Step 2. Let us compute $\bk{A(\partial_tg_t)u|v}=\bk{\partial_tR_tu|v}$ for each $u,v\in\Vbb$. By \eqref{eq87a}, 
\begin{align*}
&\partial_t\partial_z\exp^{\im X}_t(z)=\partial_z\partial_t\exp^{\im X}_t(z)=\im\partial_z\sum_{n=0,\pm1} c_n \big(\exp^{\im X}_t(z)\big)^{n+1}\\
=&\im\sum_{n=0,\pm1} c_n (n+1)\big(\exp^{\im X}_t(z)\big)^n\cdot\partial_z\exp^{\im X}_t(z)
\end{align*}
Thus $\partial_t\big(\partial_z\exp^{\im X}_t(z)\big)^{\wt(A)}$ equals
\begin{align*}
\wt(A)\big(\partial_z\exp^{\im X}_t(z)\big)^{\wt(A)-1}\cdot \im\sum_{n=0,\pm1} c_n (n+1)\big(\exp^{\im X}_t(z)\big)^n\cdot\partial_z\exp^{\im X}_t(z)
\end{align*}
Therefore, the RHS of \eqref{eq86}, when evaluated in $\bk{\cdot u|v}$, has derivative
\begin{align*}
&\im \sum_{n=0,\pm1}\ointn_{\Sbb^1}\wt(A)\big(\partial_z\exp^{\im X}_t(z)\big)^{\wt(A)}\cdot c_n (n+1)\big(\exp^{\im X}_t(z)\big)^n\\
&\qquad\qquad\cdot f(z)\bigbk{A\big(\exp^{\im X}_t(z)\big)u|v}\sd z\\
&+\ointn_{\Sbb^1} \big(\partial_z\exp^{\im X}_t(z)\big)^{\wt(A)}f(z)\cdot \partial_t\exp^{\im X}_t(z) \bigbk{A'\big(\exp^{\im X}_t(z)\big)u|v}\sd z
\end{align*}
which is the sum of an $A$-smeared field and an $A'$-smeared field put in $\bk{\cdot u|v}$. (Note that $A'(z)=\partial_zA(z)$ is also energy-bounded.) Thus, when inserted in $\bk{\cdot u|v}$, we have
\begin{align*}
\partial_tR_t=&\im \sum_{n=0,\pm1}\ointn_{\Sbb^1}\big(\partial_z\exp^{\im X}_t(z)\big)^{\wt(A)}f(z)\\
&\qquad\qquad\quad \cdot\wt(A) c_n (n+1)\big(\exp^{\im X}_t(z)\big)^n A\big(\exp^{\im X}_t(z)\big)\sd z\\
&+\im \sum_{n=0,\pm1}\ointn_{\Sbb^1} \big(\partial_z\exp^{\im X}_t(z)\big)^{\wt(A)}f(z)\cdot c_n \exp^{\im X}_t(z)^{n+1} A'\big(\exp^{\im X}_t(z)\big)\sd z  \tag{b}\label{eq133}
\end{align*} 
where \eqref{eq87a} has been used again to express $\partial_t\exp_t^{\im X}(z)$.\\[-1ex]

Step 3. By \eqref{eq53a}, when inserted in $\bk{\cdot u|v}$, we have for all $z\in\Cbb^\times$ that
\begin{align*}
L_XA(z)-A(z)L_X=\sum_{n=0,\pm1}c_nz^{n+1}A'(z)+\sum_{n=0,\pm1}\wt(A)\cdot c_n(n+1)z^nA(z)
\end{align*}
(Note that the first $L_X$ acts on $\Vbb^\ac$.) Using this formula, one checks easily that
\begin{align*}
\im[L_X,R_t]=\im\ointn_{\Sbb^1} \big(\partial_z\exp^{\im X}_t(z)\big)^{\wt(A)}f(z)\cdot \big[L_X,A\big(\exp^{\im X}_t(z)\big)\big]\sd z
\end{align*}
equals the expression of $\partial_tR_t=\eqref{eq133}$ when inserted in $\bk{\cdot u|v}$. So
\begin{align*}
\partial_tR_t=\im [L_X,R_t]
\end{align*}
holds when inserted in $\bk{\cdot u|v}$. So it holds as closable operators with common domain $\HV^\infty$, because both sides have core $\Vbb$. (To see this, apply Prop. \ref{lb49} to the unbounded operators $A(\partial_tg_t)=\partial_t R_t$ and $[L_X,R_t]$ with domain $\HV^\infty$.)\\[-1ex]

Step 4. We view $e^{-\im t\ovl{L_X}} R_te^{\im t\ovl{L_X}}$ as the composition of the bounded linear operators $e^{\im t\ovl{L_X}}\in\fk L(\HV^\infty,\bk{\cdot|\cdot}_r)$ (where the boundedness is due to Thm. \ref{lba9}), \eqref{eq88}, and $e^{-\im t\ovl{L_X}}\in\fk L(\HV^\infty)$. Note that Cor. \ref{lb61} shows the derivative of the first and the third one. Therefore by the Leibniz rule Prop. \ref{lb62}, the function
\begin{align*}
\Rbb\rightarrow\fk L\big((\HV^\infty,\bk{\cdot|\cdot}_r),\HV^\infty\big) \qquad t\mapsto e^{-\im t\ovl{L_X}}R_te^{\im t\ovl{L_X}}
\end{align*}
is differentiable, and has derivative
\begin{align*}
e^{-\im t\ovl{L_X}}\big(\partial_tR_t-\im [L_X,R_t] \big)e^{\im t\ovl{L_X}}
\end{align*}
By the last step, this derivative is zero. 
\end{proof}




\subsection{}


We shall give several important applications of the following \textbf{rotation covariance} property. Recall that $\ek_k(z)=z^k$.

\begin{co}\label{lb67}
Let $A(z)\in\mc V$ be energy-bounded. Let $f\in C^\infty(\Sbb^1)$. For each $t\in\Rbb$, define $\tau_tf\in C^\infty(\Sbb^1)$ by
\begin{align*}
\tau_tf(e^{\im\theta})=f(e^{\im(\theta-t)})
\end{align*}
Then, as unbounded operators with domain $\HV^\infty$ we have
\begin{align}
e^{\im t\ovl{L_0}}A(f)e^{-\im t\ovl{L_0}}=e^{\im t(\wt(A)-1)}A(\tau_tf)
\end{align}
\end{co}

\begin{proof}
This is clear from Thm. \ref{lb65}, noting that $\exp^{\im l_0}_t(z)=e^{\im t}z$.
\end{proof}


Let us recall from Thm. \ref{lb39}-(b) that as unbounded operators with invariant domain $\HV^\infty$, 
\begin{align}\label{eq90}
(-1)^{\wt(A)}A^\theta\big(\ovl{\ek_{2-2\wt(A)}f}\big)=A(f)^\dagger
\end{align}
where Conv. \ref{lb48} is applied here.

\begin{thm}\label{lb108}
Let $A,B\in\mc V$ such that $A$ is linearly energy-bounded, and $B$ is energy-bounded. Let $I,J$ be disjoint intervals of $\Sbb^1$. Let $f\in C_c^\infty(I),g\in C_c^\infty(J)$. Then $\ovl{A(f)}$ commutes strongly with $\ovl{B(g)}$. 
\end{thm}


\begin{proof}
By Cor. \ref{lb66}, we have $[A(f),B(g)]=0$ and (by \eqref{eq90}) $[A(f)^\dagger,B(g)]=0$. More generally, there is $\delta>0$ such that for all $-\delta<t<\delta$ we have $\tau_tg\in C_c^\infty(J)$. Then by Cor. \ref{lb67}, $e^{\im t\ovl{L_0}}Be^{-\im t\ovl{L_0}}$ is a $B$-smeared operator supported in $J$, and hence
\begin{align*}
[A(f),e^{\im t\ovl{L_0}}B(g)e^{-\im t\ovl{L_0}}]=[A(f)^\dagger,e^{\im t\ovl{L_0}}B(g)e^{-\im t\ovl{L_0}}]=0
\end{align*} 
Let $S_+=(A(f)+A(f)^\dagger)/2$ and $S_-=(A(f)-A(f)^\dagger)/2\im$. Then $S_\pm^\dagger=S_\pm$, and $[S,e^{\im t\ovl{L_0}}Be^{-\im t\ovl{L_0}}]=0$ for all $|t|<\delta$. By Thm. \ref{lb63}, both $\ovl S_+$ and $\ovl S_-$ commute strongly with $e^{\im t\ovl{L_0}}\ovl{B(g)}e^{-\im t\ovl{L_0}}$. Therefore, by Prop. \ref{lb68}, $\ovl{A(f)}=\ovl{S_++\im S_-}$ commutes strongly with $\ovl{B(g)}$.
\end{proof}



\subsection{}


\begin{thm}[\textbf{Reeh-Schlieder}]\index{00@Reeh-Schlieder theorem}\label{lb70}
Assume that $\mc V$ is energy-bounded. Choose open intervals $I_1,I_2,\dots$ of $\Sbb^1$. Then vectors of the form
\begin{align}\label{eq91}
A^1(f_1)\cdots A^N(f_N)\Omega
\end{align}
(where $N\in\Zbb_+$, and $A^i\in\mc V,f_i\in C_c^\infty(I_i)$ for each $1\leq i\leq N$) span a dense subspace of $\HV$.
\end{thm}

\begin{proof}
Step 1. Let $\mc W$ denote the subspace spanned by \eqref{eq91}. Let us prove that $\mc W^\perp$ is $e^{\im t\ovl{L_0}}$-invariant for each $t\in\Rbb$. To see this, choose any $\xi\in\mc W^\perp$. Choose any $\psi$ of the form \eqref{eq91}. Then there exists $\delta>0$ such that $\tau_tf_i\in C_c^\infty(I_i)$ for each $i$ and $-\delta<t<\delta$. Since $L_0\Omega=0$ (cf. Def. \ref{lb30}), by the rotation covariance Cor. \ref{lb67}, we see for all $t\in\Rbb$.
\begin{align*}
e^{\im t\ovl{L_0}}A^1(f_1)\cdots A^N(f_N)\Omega=e^{\im t(\wt(A_1)+\cdots+\wt(A_N)-N)}A^1(\tau_tf_1)\cdots A^N(\tau_tf_N)\Omega
\end{align*}
Thus $\bk{e^{\im t\ovl{L_0}}\xi|\psi}=0$ for all $-\delta<t<\delta$. Since $\ovl{L_0}\geq0$, by Cor. \ref{lb57}, the function $\tau\mapsto\bk{e^{\im\tau\ovl{L_0}}\xi|\psi}$ is continuous on $\fk I=\{\tau\in\Cbb:\Imag\tau\geq0\}$ and holomorphic on $\Int\fk I$. Therefore, by Lem. \ref{lb55}, we have $\bk{e^{\im t\ovl{L_0}}\xi|\psi}=0$ for all $t\in\Rbb$. This proves $e^{\im t\ovl{L_0}}\xi\in\mc W^\perp$. 


We have proved that $\mc W^\perp$ is $e^{\im t\ovl{L_0}}$-invariant. Let $E$ be the projection onto $\mc W^\perp$. Then $E\in\{e^{\im t\ovl{L_0}}:t\in\Rbb\}'=\{\ovl{L_0}\}'=\{P_n:n\in\Nbb\}'$ (cf. Cor. \ref{lb69}). Thus $[E,P_n]=0$, and hence $EP_n$ is the projection onto $\mc W^\perp\cap\Vbb(n)$. Therefore $\mc W^\perp=0$ iff $\mc W^\perp\cap\Vbb(n)=0$ for all $n\in\Nbb$, iff $\mc W^\perp\cap\Vbb=0$.\\[-1ex]

Step 2. By Step 1, to show $\mc W^\perp=0$, it suffices to choose any $v\in\Vbb$ orthogonal to $\mc W$, and show that $v=0$. In fact, we shall only need that $v\in\HV^\infty$.

Fix $N\in\Zbb_+$. Let us prove, by induction on $k=N+1,N,\dots,1$, that $v$ is orthogonal to 
\begin{align*}
A^1(f_1)\cdots A^{k-1}(f_{k-1})A^k(\tau_{t_k}f_k)\cdots A^N(\tau_{t_N}f_N)\Omega
\end{align*}
for any $A^i\in\mc V,f_i\in C_c^\infty(I_i)$ and $t_k,t_{k+1},\dots,t_N\in\Rbb$. The case $k=N+1$ (i.e., $v$ is orthogonal to all \eqref{eq91}) is trivial. Assume that case $k+1$ is true. Then, noting that each $A^i(f_i)^\dagger$ is a smeared operator (cf. \eqref{eq90}) and hence has invariant domain $\HV^\infty$, we have
\begin{align*}
A^{k-1}(f_{k-1})^\dagger\cdots A^1(f_1)^\dagger v\quad\perp\quad A^k(f_k) A^{k+1}(\tau_{t_{k+1}}f_{k+1})\cdots A^N(\tau_{t_N}f_N)\Omega
\end{align*}
for all $A^i\in\mc V,f_i\in C_c^\infty(I_i)$ and $t_{k+1},\dots,t_N\in\Rbb$. Then, similar to the argument in Step 1, we have for all $A^i\in\mc V,f_i\in C_c^\infty(I_i)$ and $t_k,t_{k+1},\dots,t_N\in\Rbb$ that
\begin{align*}
A^{k-1}(f_{k-1})^\dagger\cdots A^1(f_1)^\dagger v\quad\perp\quad e^{\im t_k\ovl{L_0}}A^k(f_k) A^{k+1}(\tau_{t_{k+1}}f_{k+1})\cdots A^N(\tau_{t_N}f_N)\Omega
\end{align*}
and hence (by rotation covariance)
\begin{align*}
A^{k-1}(f_{k-1})^\dagger\cdots A^1(f_1)^\dagger v\quad\perp\quad A^k(\tau_{t_k}f_k) A^{k+1}(\tau_{t_{k+1}}f_{k+1})\cdots A^N(\tau_{t_N}f_N)\Omega
\end{align*}
for all $A^i\in\mc V,f_i\in C_c^\infty(I_i)$ and $t_k,t_{k+1},\dots,t_N\in\Rbb$. This finishes the inductive proof.

Now, the case $k=1$ shows that $v$ is orthogonal to all
\begin{align*}
A^1(\tau_{t_1}f_1)\cdots A^k(\tau_{t_N}f_N)\Omega
\end{align*}
where $A^i\in\mc V,f_i\in C_c^\infty(I_i)$ and $t_1,\dots,t_N\in\Rbb$. Since any $A^1_{n_1}\cdots A^N_{n_N}\Omega$ (where $n_i\in\Zbb$) is a linear combination of the vectors of the above form, and since all $A^1_{n_1}\cdots A^N_{n_N}\Omega$ (where $N$ goes through all natural numbers) span the dense subspace $\Vbb$ (cf. Def. \ref{lb30}), we conclude that $v$ is orthogonal to $\Vbb$, and hence $v=0$.
\end{proof}


\subsection{}


\begin{co}\label{lb88}
Assume that $\mc V$ is energy-bounded. Choose open intervals $I_1,I_2,\dots$ of $\Sbb^1$ such that $\inf_{i\in\Zbb_+}|I_i|>0$. Let $\mc W_{I_\blt}$ be the subspace spanned by vectors of the form \eqref{eq91} as in the Reeh-Schlieder Thm. \ref{lb70}. Let $k\in\Zbb_+$. Let $B^1,\dots,B^k\in\mc V$ and $g_1,\dots,g_k\in C^\infty(\Sbb^1)$. Then $\mc W_{I_\blt}$ is a core for any polynomial $\wtd B$ of $B^1(g_1),\dots,B^k(g_k)$ (with domain $\HV^\infty$).
\end{co}


\begin{proof}
Since $\inf_i|I_i|>0$, there exists $\delta>0$ such that for each $i$, there is an interval $J_i\subset I_i$ such that $\exp^{\im l_0}_t(J_i)\subset I_i$ for each $-\delta<t<\delta$. By the Reeh-Schlieder Thm. \ref{lb70}, $\mc W_{J_\blt}$ is dense in $\HV$. By the rotation covariance Cor. \ref{lb67}, we have $e^{\im t\ovl{L_0}}\mc W_{J_\blt}\subset\mc W_{I_\blt}$ for all $-\delta<t<\delta$. Therefore, by Thm. \ref{lb76}, $\mc W_{I_\blt}$ is a core for $(1+\ovl{L_0})^r$ for each $r\geq0$. By Prop. \ref{lb49}, there exist $M,r\geq0$ such that $\Vert \wtd B\xi\Vert\leq M\Vert (1+\ovl{L_0})^r\xi\Vert$ for all $\xi\in\HV^\infty$. From this one easily sees that  $\mc W_{I_\blt}$ is a core for $\wtd B$.
\end{proof}






\subsection{}




We take this opportunity to prove a density property for pointed fields similar in spirit to the Reeh-Schlieder Thm. \ref{lb70}.


Recall that if $X$ is a topological space with subset $E$, an \textbf{accumulation point of \pmb{$E$} in \pmb{$X$}} \index{00@Accumulation point} is an element $x\in X$ belonging to the closure of $E\setminus\{x\}$. 

\begin{rem}\label{lb71}
Recall that if $E$ is a subset of an open connected $O\subset\Cbb$, if $E$ has at least one accumulation point $z_0\in O$, and if $f\in\scr O(O)$ vanishes on $E$, then by complex analysis we have $f=0$. (Proof: Write $f(z)=\sum_{n\in\Nbb}a_n(z-z_0)^n$, and show $a_n=0$ by induction on $n$.)
\end{rem}


\begin{thm}\label{lb72}
Let $E_1,E_2,\dots$ be a sequence of mutually disjoint subsets of $\Dbb_1=\{z\in\Cbb:|z|<1\}$. Assume that each $E_i$ has an accumulation point $\zeta_i\in \Dbb_1$. Then vectors of the form
\begin{align}\label{eq92}
A^1(z_1)\cdots A^N(z_N)\Omega
\end{align}
(where $N\in\Zbb_+$, and $A^i\in\mc V,z_i\in E_i$ for each $1\leq i\leq N$) span a dense subspace of $\HV$.
\end{thm}


Recall from Thm. \ref{lb45} that \eqref{eq92} belongs to $\HV$, and that \eqref{eq92} is holomorphic over $z_\blt\in\Conf^k(\Dbb_1)$. Also, note that energy-boundedness is not assumed in Thm. \ref{lb72}. (Thus, its proof does not rely on the M\"obius/rotation covariance of smeared fields.)

\begin{proof}
Let $\mc W$ be the subspace spanned by all vectors of the form \eqref{eq92}. Let us prove that $\mc W^\perp$ is rotation invariant. Choose $\xi\in\mc W^\perp$. Then for each fixed $N$ and $A^\blt$, the function $z_\blt\in \Conf^k(\Dbb_1)\mapsto \bk{A^1(z_1)\cdots A^N(z_N)\Omega|\xi}$ is holomorphic, and is vanishing on $E_\blt$. Therefore, it is everywhere zero by Rem. \ref{lb71}. By Prop. \ref{lb19} (noting also $q^{L_0}=q^{\ovl{L_0}}$ on $\HV$ by Prop. \ref{lb33}) and the fact that $L_0\Omega=0$, we see that
\begin{align*}
\bk{A^1(z_1)\cdots A^N(z_N)\Omega|e^{-\im t\ovl{L_0}}\xi}=e^{\im t\sum_i\wt(A^i)}\bk{A^1(e^{\im t}z_1)\cdots A^N(e^{\im t}z_N)\Omega|\xi}
\end{align*}
is zero for all $t\in\Rbb$. This proves $e^{\im t\ovl{L_0}}\mc W^\perp\subset\mc W^\perp$.

Therefore, similar to the proof of Thm. \ref{lb70}, it suffices choose any $v\in\Vbb$ orthogonal to $\mc W$, and show that $v=0$. By the first paragraph, for each $N$ and $A^i$ and $z_\blt\in\Conf^k(\Dbb_1)$, the expression
\begin{align*}
f(z_\blt)=\bk{A^1(z_1)\cdots A^N(z_N)\Omega|v}
\end{align*}
equals $0$. Choose $0<r_1<\cdots<r_N<1$ and $\Gamma_i=r_i\Sbb^1$. Then for each $n_1,\dots,n_N$,
\begin{align*}
\bk{A^1_{n_1}\cdots A^N_{n_N}\Omega|v}=\ointn_{\Gamma_N}\cdots\ointn_{\Gamma_1}f(z_\blt)z_1^{n_1}\cdots z_N^{n_N}\sd z_\blt
\end{align*}
equals zero. Thus, by the cyclicity in Def. \ref{lb30}, we have $v=0$.
\end{proof}


\subsection{}

Our last application of Thm. \ref{lb65} is a M\"obius covariance property of pointed fields which generalized Prop. \ref{lb19}. It will be used in the proof of the PT theorem (cf. the proof of Prop. \ref{lb83}). Recall Rem. \ref{lb74}. 




\begin{thm}\label{lb80}
Let $X=\eqref{eq83}$ such that $X=X^\dagger$. Let $k\in\Zbb_+$. Let $A^1,\dots,A^k\in\mc V$ be energy-bounded. Then for each  $z_\blt\in\Conf^k(\Dbb_1)$ and $t\in\Rbb$, we have in $\HV$ that
\begin{align}\label{eq94}
\begin{aligned}
&e^{\im t\ovl{L_X}}A^1(z_1)\cdots A^k(z_k)\Omega\\
=&\big(\partial_{z_1}\exp^{\im X}_t(z_1)\big)^{\wt(A^1)}\cdots \big(\partial_{z_k}\exp^{\im X}_t(z_k)\big)^{\wt(A^k)}\\
&\cdot A^1\big(\exp^{\im X}_t(z_1)\big)\cdots A^k\big(\exp^{\im X}_t(z_k)\big)\Omega
\end{aligned}
\end{align}
\end{thm}


Recall $L_0\Omega=L_{\pm 1}\Omega=0$ (cf. \eqref{eq67}), and hence $e^{\im t\ovl{L_X}}\Omega=\Omega$.

\begin{proof}
Step 1. Fix $t\in\Rbb$ throughout the proof. We write \eqref{eq94} as
\begin{align*}
e^{\im t\ovl{L_X}}A^\blt(z_\blt)\Omega=h_t(z_\blt)\cdot A^\blt(\exp_t^{\im X}(z_\blt))\Omega
\end{align*}
Since $\ovl{L_0}\geq0$, by Cor. \ref{lb57}, for each $\xi\in\mc H$ the function $\sigma\mapsto e^{\im\sigma\ovl{L_0}}\xi\in\HV$ is continuous on $\{\sigma\in\Cbb:\Imag\sigma\geq0\}$ and holomorphic on its interior. Therefore $q\in\ovl\Dbb_1\mapsto q^{\ovl{L_0}}\xi$ is continuous on $\ovl\Dbb_1$ and holomorphic on $\Dbb_1$.

We now choose mutually disjoint intervals $I_1,\dots,I_k$ of $\Sbb^1$. Let $f_i\in C_c^\infty(I_i)$ and
\begin{align*}
\xi=\int_{I_\blt}f_\blt(z_\blt)A^\blt(z_\blt)\Omega\sd z_\blt\qquad(\text{where }f_\blt(z_\blt)=f_1(z_1)\cdots f_k(z_n))
\end{align*}
which is an element of $\HV$ (cf. Exp. \ref{lb54}). In other words,
\begin{align*}
\xi=A^\blt(f_\blt)\Omega=A^1(f_1)\cdots A^k(f_k)\Omega
\end{align*}
Let $\fk I=\{q\in\Cbb:0<|q|\leq 1\}$. Then we have a continuous function
\begin{align*}
F:\fk I\rightarrow\HV\qquad q\mapsto e^{\im t\ovl{L_X}}q^{\ovl{L_0}}\xi
\end{align*}
holomorphic on $\Int\fk I$. \\[-1ex]

Step 2. We now compute the function $F$ in two ways. Let $\Delta=\wt(A^1)+\cdots+\wt(A^k)$. Note that $\xi$ is a product of smeared operators acting on $\Omega$. Therefore, we can use Thm. \ref{lb65} to show that for each $q\in\Sbb^1$,
\begin{align}
q^{\ovl{L_0}}\xi=\int_{I_\blt} q^\Delta f_\blt(z_\blt)A^\blt(qz_\blt)\Omega\sd z_\blt\label{eq96}
\end{align}
where $A^\blt(qz_\blt)=A^1(qz_1)\cdots A^k(qz_k)$. By Def. \ref{lb46}, for each $v\in\Vbb$, the function
\begin{align*}
q\in\Cbb^\times\mapsto \int_{I_\blt} q^\Delta f_\blt(z_\blt)\bk{A^\blt(qz_\blt)\Omega|v}\sd z_\blt\tag{$\star$}\label{eq99}
\end{align*}
is holomorphic, and hence is continuous on $\fk I$ and holomorphic on $\Int\fk I$. This function agrees with $\bk{q^{\ovl{L_0}}\xi|v}$ when $q\in\Sbb^1$. Therefore, they agree on $\fk I$ due to Lem. \ref{lb55} and the fact that $q\mapsto q^{\ovl{L_0}}\xi$ is continuous on $\fk I$ and holomorphic on $\Int\fk I$. 

Fix $q\in\fk I$.  Then the RHS of \eqref{eq99} equals the RHS of \eqref{eq96} (as a vector of $\Vbb^\ac$) evaluated with $v$; the integral on the RHS of \eqref{eq96} is understood in the sense of Def. \ref{lb52}.  Therefore  \eqref{eq96} holds in $\Vbb^\ac$ (and hence in $\HV$), and hence
$F(q)=e^{\im t\ovl{L_X}}\cdot \text{the RHS of }\eqref{eq96}$.\\[-1ex]

Step 3. The second way to compute $F(q)$ is as follows. If $q\in\Sbb^1$, then the RHS of \eqref{eq96} is a product of smeared operators acting on $\Omega$, and hence we can use Thm. \ref{lb65} to show that
\begin{gather}
F(q)=\int_{I_\blt}h_t(qz_\blt)\cdot  q^\Delta f_\blt(z_\blt)A^\blt\big(\exp_t^{\im X}(qz_\blt)\big)\Omega\sd z_\blt\label{eq95}\\
\text{where}\qquad A^\blt\big(\exp_t^{\im X}(q z_\blt)\big)=A^1\big(\exp_t^{\im X}(qz_1)\big)\cdots A^k\big(\exp_t^{\im X}(qz_k)\big)\nonumber
\end{gather}
By Def. \ref{lb46}, for each $v\in\Vbb$, the function
\begin{align*}
q\in\Cbb^\times\mapsto \int_{I_\blt}h_t(qz_\blt)\cdot  q^\Delta f_\blt(z_\blt)\bigbk{A^\blt\big(\exp_t^{\im X}(qz_\blt)\big)\Omega\big|v}\sd z_\blt
\end{align*}
is holomorphic. Therefore, by Step 1 and Lem. \ref{lb55}, the above function equals $\bk{F(q)|v}$ on $q\in\fk I$. Hence, as in Step 2, we conclude that for each $q\in\fk I$, Eq. \eqref{eq95} holds in $\HV$ (where the integral on the RHS is understood as in Def. \ref{lb52}).\\[-1ex]

Step 4. By \eqref{eq96} and \eqref{eq95}, for all $q\in\fk I$ we have in $\HV$ that
\begin{align}\label{eq97}
e^{\im t\ovl{L_X}}\int_{I_\blt} q^\Delta f_\blt(z_\blt)A^\blt(qz_\blt)\Omega\sd z_\blt=\int_{I_\blt}h_t(qz_\blt)\cdot  q^\Delta f_\blt(z_\blt)A^\blt\big(\exp_t^{\im X}(qz_\blt)\big)\Omega\sd z_\blt
\end{align}
where the two integrals are understood in the sense of Def. \ref{lb52}. 

Now we assume $0<|q|<1$. By Thm. \ref{lb45}, the two integrands in \eqref{eq97} are $\HV$-valued continuous functions over $z_\blt\in I_\blt$. Hence the two integrals are Riemann integrals of continuous $\HV$-valued functions. Therefore, if we fix $\zeta_i\in I_i$, and let $f_1,\dots,f_n$ satisfy $\int_{I_i}f_i\sd z_i=1$ and converge respectively to the $\delta$-functions at $\zeta_1,\dots,\zeta_n$ (as in Lem. \ref{lb64}-2), then the term after $e^{\im t\ovl{L_X}}$ on the LHS of \eqref{eq97} (resp. the term on the RHS of \eqref{eq97}) converges in $\HV$ to
\begin{align*}
q^\Delta A^\blt(q\zeta_\blt)\Omega\qquad\text{resp.}\qquad h_t(q\zeta_\blt)\cdot  q^\Delta A^\blt\big(\exp_t^{\im X}(q\zeta_\blt)\big)\Omega
\end{align*}
Letting $z_i=q\zeta_i$, we conclude that for each $0<r<1$ and each $z_1\in rI_1,\dots,z_k\in rI_k$, we have in $\HV$ that
\begin{align}\label{eq98}
e^{\im t\ovl{L_X}} \cdot A^\blt(z_\blt)\Omega=h_t(z_\blt)\cdot   A^\blt\big(\exp_t^{\im X}(z_\blt)\big)\Omega
\end{align} 
By Thm. \ref{lb45}, both sides of \eqref{eq98} are $\HV$-valued holomorphic functions over $z_\blt\in\Conf^k(\Dbb_1)$. Therefore, due to Rem. \ref{lb71}, we have \eqref{eq98} for all $z_\blt\in\Conf^k(\Dbb_1)$.
\end{proof}



\newpage



\section{The PT and PCT theorems for energy-bounded chiral algebras}



In this section, we fix an energy-bounded chiral algebra $(\mc V,\Vbb)$. Recall $L_0\Omega=L_{\pm 1}\Omega=0$ (cf. \eqref{eq67}), and hence $e^{\im t\ovl{L_X}}\Omega=\Omega$ for each $X=\eqref{eq83}$ satisfying $X=X^\dagger$.


Define the \textbf{dilation generator} \index{00@Dilation generator} 
\begin{align}\label{eq127}
K=\frac{l_1-l_{-1}}{2\im}
\end{align}
Then $K=K^\dagger$. Thus $\ovl{L_K}$ is a self-adjoint operator on $\HV$, and $e^{\im t\ovl{L_K}}\HV^\infty\subset\HV^\infty$, cf. Rem. \ref{lb75}. We also let \index{S@$\Sbb^1_+,\Sbb^1_-$}
\begin{align*}
\Sbb^1_+=\{z\in\Sbb^1:\Imag z>0\}\qquad \Sbb^1_-=\{z\in\Sbb^1:\Imag z<0\}
\end{align*}





\subsection{}



By solving Eq. \eqref{eq87}, one sees that $\delta_\tau:=\exp^{\im K}_\tau$ (called a \textbf{dilation}) equals \index{zz@$\delta_\tau$, the dilation}
\begin{align}\label{eqb21}
\delta_\tau(z):=\exp^{\im K}_\tau=\exp^K_{\im\tau}(z)=\frac{z\cosh(\tau/2)-\sinh(\tau/2)}{-z\sinh(\tau/2)+\cosh(\tau/2)}
\end{align}
for each $\tau\in\Cbb$ and $z\in\Pbb^1$. We compute that $\delta'_\tau(z)=\partial_z\delta_\tau(z)$ equals
\begin{align}\label{eqb25}
\delta_\tau'(z)=\frac 1{(-z\sinh(\tau/2)+\cosh(\tau/2))^2}
\end{align}

\begin{rem}
It is clear that if  $\tau=t$ or $\tau=t+\im\pi$ (where $t\in\Rbb$), and if $z\in\Sbb^1$, then $\delta_\tau'(z)\neq \infty$ (since $|\coth(t/2)|>1$ and $|z|=|z^{-1}|=1$). It is also clear that $\delta'_\tau(z)\neq0$.
\end{rem}



\begin{rem}\label{lbb15}
Consider the Cayley transform $\Phi:\Pbb^1\rightarrow\Pbb^1$ defined by $\dps\Phi(z)=-\frac{z-\im}{z+\im}$ with inverse $\dps\Phi^{-1}(z)=-\im \frac{z-1}{z+1}$. Then $\Phi(\Rbb\cup\{\infty\})=\Sbb^1$, and $\Phi$ sends the open upper (resp. lower) half plane to the inside (resp. outside) of $\Sbb^1$. $\Phi$ sends $\Rbb_{>0}$, $\Rbb_{<0}$, $\im\Rbb_{}>0$ respectively $\Sbb^1_+$, $\Sbb^1_-$, and the line interval $(-1,1)$.

\begin{figure}[h]
	\centering
	\includegraphics[height=2cm]{fig3.png}
	\caption{. The  orbits of $t\mapsto\delta_t$ (for $0\leq t<+\infty$) and $t\mapsto\delta_{\im t}$ (for $0\leq t\leq\pi$)}\label{lb101}
\end{figure}

The pullback of $\delta_\tau$ by $\Phi$ is
\begin{align}
\Phi^*\delta_\tau(z):=\Phi^{-1}\circ\delta_\tau\circ \Phi(z)=e^\tau z
\end{align} 
By looking at the picture of $\Phi^*\delta_\tau$, it is easy to see:
\begin{itemize}
\item[(a)] If $0<\Imag \tau<\pi$ (resp. $-\pi<\Imag\tau<0$), $\delta_\tau$ sends $\Sbb^1_+$ (resp. $\Sbb^1_-$) to the inside of $\Sbb^1$.
\item[(b)] If $-\pi/2<\Imag\tau<\pi/2$, then $\delta_\tau$ sends $\{x\in\Rbb:-1<x<1\}$ to the inside of $\Sbb^1$.
\item[(c)] If $t\in\Rbb$, then $\delta_t$ restricts to a diffeomorphisms of $\Sbb^1_+$, $\Sbb^1_-$, and $\{x\in\Rbb:-1<x<1\}$.
\item[(d)] $\delta_{\im\pi}:\Sbb^1_+\rightarrow\Sbb^1_-,z\mapsto z^{-1}$ is a diffeomorphism.
\end{itemize}
Fig. \ref{lb101} shows the picture of $\delta_t$ and $\delta_{\im t}$. So one can imagine $\delta_{s+\im t}=\delta_{\im t}\circ\delta_s$.   \hfill\qedsymbol
\end{rem}





\subsection{}


\begin{df}
Let $f\in C^\infty(\Sbb^1)$ and $A\in\mc V$. For each $t\in\Rbb$, $\delta_t,\delta_{t+\im}:\Sbb^1\rightarrow\Sbb^1$ are diffeomorphisms. Define $\delta_t^A(f)\in C^\infty(\Sbb^1)$ by
\begin{subequations}\label{eq102}
\begin{gather}\label{eq102a}
\delta_t^A(f)(z)=\delta'_{-t}(z)^{1-\wt A}\cdot (f\circ\delta_{-t}(z))
\end{gather}
Note that if $f\in C_c^\infty(\Sbb^1_\pm)$ then $\delta^A_t(f)\in C_c^\infty(\Sbb^1_\pm)$. We also define $\delta_{t+\im\pi}^A(f)\in C^\infty_c(\Sbb^1)$ by
\begin{align}\label{eq102b}
\delta_{t+\im\pi}^A(f)(z)=-\delta'_{-t-\im\pi}(z)^{1-\wt A}\cdot (f\circ\delta_{-t-\im\pi}(z))
\end{align}
\end{subequations}
\end{df}

Specializing to $t=0$, one checks that 
\begin{align}\label{eq103}
\delta_{\pm\im\pi}(z)=z^{-1}\qquad \delta'_{\pm\im\pi}(z)=-z^{-2}
\end{align}
This is where the factor $(-z^{-2})$ in the definition of $A^\theta$ comes from. So we have
\begin{gather}
\delta^A_{\im\pi}(f)(z)=-(-z^{-2})^{1-\wt A}\cdot f(z^{-1})=(-1)^{\wt A}z^{2\wt A-2}f(z^{-1})\label{eq114}\\[1ex]
\delta_t^A \big(C^\infty_c(\Sbb^1_\pm)\big)=C^\infty_c(\Sbb^1_\pm)\qquad \delta_{t+\im\pi}^A \big(C^\infty_c(\Sbb^1_\pm)\big)=C^\infty_c(\Sbb^1_\mp)\nonumber
\end{gather}

\begin{lm}\label{lbb22}
We have $\delta^A_{t+\im\pi}(f)=\delta^A_t(\delta^A_{\im\pi}(f))$.
\end{lm}

\begin{proof}
For each $z\in\Sbb^1$, we have
\begin{align*}
&\delta^A_t(\delta^A_{\im\pi}(f))(z)=\delta'_{-t}(z)^{1-\wt A}\cdot (\delta^A_{\im\pi}(f)\circ\delta_{-t}(z))\\
=&-\delta'_{-t}(z)^{1-\wt A}\cdot(-\delta_{-t}(z)^{-2})^{1-\wt A}f(1/\delta_{-t}(z))\\
=&-\big((-\delta_{-t}(z)^{-2})\cdot \delta_{-t}'(z) \big)^{1-\wt A}f(\delta_{-\im\pi}\circ\delta_{-t}(z))\\
=&-\big(\delta_{-\im\pi}'(\delta_{-t}(z))\cdot \delta_{-t}'(z) \big)^{1-\wt A}f(\delta_{-\im\pi-t}(z))\\
=&-\big(\delta_{-\im\pi-t}'(z) \big)^{1-\wt A}f(\delta_{-\im\pi-t}(z))=\delta_{t+\im\pi}^A(f)(z)
\end{align*}
where, in the second last equality, we have applied the chain rule to $\delta_{-\im\pi}\circ\delta_{-t}=\delta_{-\im\pi-t}$.
\end{proof}



\begin{lm}\label{lb84}
We have $\delta^A_{\im\pi}(\delta^A_{\im\pi}(f))=f$.
\end{lm}


\begin{proof}
This can be calculated easily using \eqref{eq114}.
\end{proof}





\subsection{}




\begin{rem}\label{lb77}
Let $\Gamma$ be a real oriented connected $C^1$-submanifold (with boundary) of $\Cbb$. Recall that for each complex $C^1$-function $f$ on $\Gamma$ we define
\begin{align*}
\int_\Gamma f(z)dz:=\int_a^b (f\circ\gamma(t))\cdot \gamma'(t)dt
\end{align*}  
where $\gamma:[a,b]\rightarrow\Gamma$ is any orientation-preserving $C^1$-diffeomorphism. Now let $\varphi:\Gamma\rightarrow \Cbb$ be a embedding of $C^1$-manifolds. Let $\varphi':\Gamma\rightarrow\Cbb$ be 
\begin{align}
\varphi'(z)\equiv\partial_z\varphi(z)=(\gamma'(t))^{-1}(\varphi\circ\gamma)'(t)\big|_{t=\gamma^{-1}(z)}
\end{align}
which agrees with the usual holomorphic derivative when $\varphi$ is a holomorphic function on a neighborhood of $\Gamma$. (Here, $\gamma$ is a $C^1$-parametrization of $\Gamma$ that is not necessarily orientation-preserving. And one checks easily that the definition of $\varphi'$ is independent of $\gamma$.) Let $\varphi(\Gamma)$ be the pushforward \uwave{oriented} curve of $\Gamma$ under $\varphi$. Then for each $C^1$-function $g:\varphi(\Gamma)\rightarrow\Cbb$ one easily checks that 
\begin{align*}
\int_{\varphi(\Gamma)} f(\zeta)d\zeta=\int_\Gamma (f\circ\varphi)(z)\cdot \varphi'(z)dz
\end{align*}
\end{rem}



\begin{pp}\label{lb78}
Let $A\in\mc V$ and $f\in C^\infty(\Sbb^1)$. Then as unbounded operators with domain $\HV^\infty$ we have
\begin{subequations}\label{eq101}
\begin{gather}
A(\delta_t^A(f))=\ointn_{\Sbb^1} \delta'_t(z)^{\wt A}f(z)A(\delta_t(z))\sd z\label{eq101a}\\
A(\delta^A_{t+\im\pi}(f))=\ointn_{\Sbb^1} \delta'_{t+\im\pi}(z)^{\wt A}f(z)A(\delta_{t+\im\pi}(z))\sd z\label{eq101b}
\end{gather}
\end{subequations}
\end{pp}


\begin{proof}
Let us prove \eqref{eq101b}; the other one is similar. Set  $\zeta=\delta_{t+\im\pi}(z)$. Since $\delta_{t+\im\pi}:\Sbb^1\rightarrow\Sbb^1$ is orientation reversing, we have $\delta_{t+\im\pi}(\Sbb^1)=-\Sbb^1$. Hence, by Rem. \ref{lb77}, the RHS of \eqref{eq101b} equals
\begin{align*}
\ointn_{\Sbb^1}-(\delta'_{t+\im\pi}\circ\delta_{-t-\im\pi}(\zeta))^{\wt A-1}\cdot (f\circ\delta_{-t-\im\pi}(\zeta))A(\zeta)\sd\zeta
\end{align*}
For each $\tau\in\Cbb$, since $\delta_\tau\circ\delta_{-\tau}(z)=z$, the chain rule implies
\begin{align*}
(\delta'_\tau\circ\delta_{-\tau}(z))\cdot\delta'_{-\tau}(z)=1
\end{align*}
Hence $(\delta'_{t+\im\pi}\circ\delta_{-t-\im\pi}(\zeta))^{\wt A-1}=-\delta_{-t-\im\pi}'(\zeta)^{1-\wt A}$, finishing the proof.
\end{proof}









\begin{co}\label{lbb17}
Let $A\in\MV$ and $f\in C^\infty(\Sbb^1)$. Let $t\in\Rbb$. Then as unbounded operators with invariant domain $\mc H^\infty_\Vbb$ we have
\begin{align}
e^{\im t\ovl{L_K}}A(f)e^{-\im t{\ovl{L_K}}}=A(\delta^A_t(f))
\end{align}
\end{co}

\begin{proof}
This is clear from Thm. \ref{lb65} and Prop. \ref{lb78}.
\end{proof}










\subsection{}

\begin{rem}\label{lb81}
Using the same proof as in Prop. \ref{lb78} (and noting Exp. \ref{lb54}), we can easily prove the following fact: Let $A^1,\dots,A^n\in\MV$. Let $I_1,\dots,I_n$ be mutually disjoint intervals of $\Sbb^1$. Let $f_i\in C_c^\infty(I_i)$. Write
\begin{gather*}
f_\blt(z_\blt)=f_1(z_1)\cdots f_n(z_n)\qquad
\delta_\tau'(z_\blt)^{\wt A^\blt}=\delta'_\tau(z_1)^{\wt A^1}\cdots \delta'_\tau(z_n)^{\wt A^n}\\
A^\blt(f_\blt)=A^1(f_1)\cdots A^n(f_n)\qquad A^\blt(z_\blt)=A^1(z_1)\cdots A^n(z_n)
\end{gather*}
where the order of product can be freely exchanged. Then for each $v\in\Vbb$ we have 
\begin{subequations}\label{eq104}
\begin{gather}
A^\blt(\delta_t^{A^\blt}(f_\blt))v=\int_{I_\blt} \delta'_t(z_\blt)^{\wt A^\blt}f_\blt(z_\blt)A^\blt(\delta_t(z_\blt))v\sd z_\blt\label{eq104a}\\
A^\blt(\delta_{t+\im\pi}^{A^\blt}(f_\blt))v=\int_{I_\blt} \delta'_{t+\im\pi}(z_\blt)^{\wt A^\blt}f_\blt(z_\blt)A^\blt(\delta_{t+\im\pi}(z_\blt))v\sd z_\blt\label{eq104b}
\end{gather}
\end{subequations}
where the RHS's are Riemann integrals of continuous functions from $\Sbb^1$ to $\HV$.
\end{rem}




\subsection{}

In order to prove the PT theorem we need some preliminary results. 





\begin{lm}\label{lb79}
Let $A^1,\dots,A^n,B^1,\dots,B^k\in\MV$. Then
\begin{subequations}\label{eq105}
\begin{align}\label{eq105a}
\sum_{\nu\in\Nbb}\bk{A^\blt(z_\blt)\Omega|P_\nu\cdot B^\star(\ovl{w_\star})\Omega}
\end{align}
converges a.l.u. over $(z_\blt,w_\star)\in \Conf^n(\ovl\Dbb_1)\times \Conf^k(\Dbb_1)$, whose limit is denoted by
\begin{align}\label{eq105b}
\bk{A^\blt(z_\blt)\Omega|B^\star(\ovl{w_\star})\Omega}
\end{align}
\end{subequations}
Moreover, \eqref{eq105b} is continuous on $(z_\blt,w_\star)\in \Conf^k(\ovl\Dbb_1)\times \Conf^k(\Dbb_1)$ and holomorphic on $\Conf^k(\Dbb_1)\times\Conf^k(\Dbb_1)$.
\end{lm}

\begin{proof}
By Prop. \ref{lb21}, the series \eqref{eq105a} is equivalent to
\begin{align*}
\sum_{\nu\in\Nbb}(-z_\blt^{-2})^{\wt B^\star}\bk{B^{\star,\theta}(1/w_\star)P_nA^\blt(z_\blt)\Omega|\Omega}
\end{align*}
and hence converges a.l.u. by Thm. \ref{lb22}. The continuity and the holomorphicity of the limit follows from that of each summand.
\end{proof}








\begin{co}\label{lb82}
Let $A^1,\dots,A^n,B^1,\dots,B^k\in\MV$. Let $I_1,\dots,I_n\subset\Sbb^1$ be mutually disjoint intervals. Then  for each $f_i\in C_c^\infty(I_i)$ and $w_\star\in\Conf^k(\Dbb_1)$ we have
\begin{align}
\bk{A^\blt(f_\blt)\Omega|B^\star(w_\star)\Omega}=\int_{I_\blt} f_\blt(z_\blt)\bk{A^\blt(z_\blt)\Omega|B^\star(w_\star)\Omega}\sd z_\blt
\end{align}
\end{co}

Note that the LHS is the evaluation between two vectors of $\HV$. The expression $\bk{A^\blt(z_\blt)\Omega|B^\star(w_\star)\Omega}$ on the RHS of the integral is understood by \eqref{eq105}.

\begin{proof}
By Thm. \ref{lb50}, for each $\nu\in\Nbb$ we have
\begin{align*}
\bk{A^\blt(f_\blt)\Omega|P_\nu B^\star(w_\star)\Omega}=\int_{I_\blt} f_\blt(z_\blt)\bk{A^\blt(z_\blt)\Omega|P_\nu B^\star(w_\star)\Omega}\sd z_\blt
\end{align*}
The corollary follows immediately from the a.l.u. convergence in Lem. \ref{lb79}.
\end{proof}


\subsection{}




\begin{pp}\label{lb83}
Let $J$ be the line interval $(-1,1)$. Choose any $0\leq r<\pi/2$. Then vectors of the form
\begin{align}\label{eq106}
B^\star(w_\star)\Omega=B^1(w_1)\cdots B^k(w_k)\Omega
\end{align}
(where $k\in\Nbb$, $B^1,\dots,B^k\in\MV$, and $w_\star\in\Conf^k(J)$) span a core for $e^{r{\ovl{L_K}}}$ and $e^{-r{\ovl{L_K}}}$. Moreover, for each $\tau$ in $\fk I=\{\tau\in\Cbb: -\pi/2<\Imag\tau < \pi/2 \}$ we have
\begin{align}\label{eq107}
e^{\im \tau\ovl{L_K}}B^\star(w_\star)\Omega=\delta'_\tau(w_\star)^{\wt(B^\star)}\cdot B^\star(\delta_\tau(w_\star))\Omega
\end{align}
\end{pp}


\begin{proof}
We first note that if $\tau\in\fk I$ then $\delta_\tau(w_i)\in\Dbb_1$. Consider the function
\begin{align*}
F:\fk I\rightarrow\HV\qquad F(\tau)=\delta'_\tau(w_\star)^{\wt(B^\star)}\cdot B^\star(\delta_\tau(w_\star))\Omega
\end{align*}
which is holomorphic by Thm. \ref{lb45}. Let $\xi=B^\star(w_\star)\Omega$. Then by Thm. \ref{lb80}, for each $t\in\Rbb$ we have $F(t)=e^{\im t\ovl{L_K}}\xi$. Therefore, by Thm. \ref{lb56}, $\xi$ belongs to the domains of $e^{\pm r{\ovl{L_K}}}$, and $F(\tau)=e^{\im \tau\ovl{L_K}}\xi$ whenever $-r\leq\Imag\tau\leq r$. Since this holds for all $0\leq r<\pi/2$, we obtain \eqref{eq107}.

We have proved that the subspace $\mc W$ spanned by all \eqref{eq106} is a subspace of the domains of $e^{\pm r\ovl{L_K}}$. By Thm. \ref{lb72}, $\mc W$ is dense in $\HV$. By \eqref{eq107}, we have $e^{\im t{\ovl L_K}}\mc W\subset \mc W$ for all $t\in\Rbb$. Therefore, by Thm. \ref{lb58}, $\mc W$ is a core for $e^{\pm r\ovl{L_K}}$.
\end{proof}



\subsection{}


\begin{thm}[\textbf{PT theorem}]\label{lb85}\index{00@PT theorem for chiral algebras}
Let $I_1,\dots,I_n\subset\Sbb^1_+$ be mutually disjoint intervals. Let $f_i\in C_c^\infty(I_i)$. Let $A^i\in\MV$. Then $A^\blt(f_\blt)\Omega$ is in $\Dom(e^{-\pi\ovl{L_K}})$, and
\begin{align}
e^{-\pi\ovl{L_K}}A^\blt(f_\blt)\Omega=A^\blt(\delta_{\im\pi}^{A^\blt}(f_\blt))\Omega
\end{align}
\end{thm}


\begin{proof}
Step 1. We let
\begin{align*}
\xi=A^\blt(f_\blt)\Omega\qquad \psi=A^\blt(\delta_{\im\pi}^{A^\blt}(f_\blt))\Omega
\end{align*}
Let $\fk I=\{\tau\in\Cbb:0\leq\Imag\tau\leq\pi\}$. Define $F:\fk I\rightarrow\HV$ by
\begin{gather}\label{eq108}
F(\tau)=\int_{I_\blt}\delta'_\tau(z_\blt)^{\wt A^\blt}f_\blt(z_\blt)A^\blt(\delta_\tau(z_\blt))\Omega\sd z_\blt
\end{gather}
Note that if $\tau\in\Int\fk I$ then $\delta_\tau(z_i)\in\Dbb_1$, and hence \eqref{eq108} is the Riemann integral of a holomorphic $\HV$-valued function (cf. Thm. \ref{lb45}). If $\tau\in\partial\fk I$, the integral is understood as in Def. \ref{lb52}. Therefore $F$ is holomorphic on $\Int\fk I$. Moreover, by Rem. \ref{lb81}, for each $t\in\Rbb$ we have
\begin{gather}\label{eq110}
F(t)=A^\blt(\delta_t^{A^\blt}(f_\blt))\Omega\qquad F(t+\im\pi)=A^\blt(\delta_{t+\im\pi}^{A^\blt}(f_\blt))\Omega
\end{gather}
By the M\"obius covariance Cor. \ref{lbb17} (and that $K\Omega=0$), we have
\begin{align}\label{eq111}
F(t)=e^{\im t\ovl{L_K}}\xi\qquad F(t+\im\pi)=e^{\im t\ovl{L_K}}\psi
\end{align}
where Lem. \ref{lbb22} is also used in the second equation.

Our goal is to prove that $F$ is continuous on $\fk I$. Suppose this can be proved. Then Thm. \ref{lb56} immediately concludes the proof of the whole theorem.\\[-1ex]

Step 2. The difficulty lies in proving that $F$ is continuous near $\partial\fk I$. In this step, we prove a weaker continuity: It is not hard to show that $F$ is \textbf{pointed-continuous}, which means that for each $B^1,\dots,B^k\in\MV$ and $w_\star\in\Conf^k(\Dbb_1)$, the following function is continuous:
\begin{align*}
\tau\in\fk I\rightarrow \bk{F(\tau)|B^\star(w_\star)\Omega}
\end{align*}
To see this, note that both $F(\tau)$ and
\begin{align}\label{eq112}
\eta:=B^\star(w_\blt)\Omega
\end{align}
are in $\HV$. Clearly
\begin{align}\label{eq109}
\bk{F(\tau)|\eta}=\int_{I_\blt}\delta'_\tau(z_\blt)^{\wt A^\blt}f_\blt(z_\blt)\bk{A^\blt(\delta_\tau(z_\blt))\Omega|\eta}\sd z_\blt
\end{align}
holds when $\tau\in\Int\fk I$. By Cor. \ref{lb82} and \eqref{eq110}, Eq. \eqref{eq109} also holds when $\tau\in\partial\fk I$. By Lem. \ref{lb79}, the integrand on the RHS of \eqref{eq109} is continuous over $(\tau,z_\blt)\in\fk I\times I_\blt$. Thus the RHS of \eqref{eq109} is continuous over $\tau\in\fk I$. Hence $\bk{F(\tau)|\eta}$ is continuous over $\tau\in\fk I$.\\[-1ex]


Step 3. We claim that 
\begin{enumerate}
\item[(a)] $\xi$ belongs to the domain of  $e^{-\pi\ovl{L_K}/4}$.
\item[(b)] $\psi$ belongs to the domain of $e^{\pi\ovl{L_K}/4}$.
\end{enumerate}
Suppose that Claim (a) is true. Then in view of \eqref{eq111}, and by Thm. \ref{lb56}, the function $t\in\Rbb\mapsto F(t)\in\HV$ can be extended to a continuous function $F_+:\fk I_+\rightarrow\HV$, holomorphic on $\Int\fk I_+$, where
\begin{align*}
\fk I_+=\{\tau\in\Cbb:0\leq\Imag\tau\leq\pi/4\}
\end{align*}
Since both $F_+$ and $F$ are pointed-continuous on $\fk I_+$, for each $\eta=\eqref{eq112}$, the function $f_\eta:\tau\in\fk I_+\mapsto\bk{F(\tau)-F_+(\tau)|\eta}$ is continuous. Moreover, $f_\eta$ is holomorphic on $\Int\fk I$, and $f_\eta|_\Rbb=0$. Therefore, by Lem. \ref{lb55}, we have $f_\eta=0$ (on $\fk I_+$). Since all such $\eta$ form a dense subspace (cf. Prop. \ref{lb83}), we get $F=F_+$ on $\fk I_+$. Hence $F$ is continuous on $\fk I_+$.

Noting the second half of \eqref{eq111}, and applying a similar argument to $\psi$, we conclude that $F$ is continuous on $\fk I_-=\{\tau\in\Cbb:3\pi/4<\Imag\tau\leq\pi\}$. Therefore $F$ is continuous on $\fk I$. This finishes the proof.\\[-1ex]

Step 4. Let us prove Claim (a). Let $\mc W$ be the subspace spanned by all $\eta$ of the form \eqref{eq112}. Since $F$ is pointed-continuous on $\fk I$ and holomorphic on $\Int\fk I$, the function
\begin{align*}
t\in\Rbb\mapsto\bk{e^{\im t\ovl{L_K}}\xi|\eta}=\bk{F(t)\xi|\eta}
\end{align*}
can be extended to a continuous function on the strip $\fk I_+$ (defined in Step 3) and holomorphic on $\Int\fk I_+$. Moreover, $\mc W$ is a core for $e^{-\pi\ovl{L_K}/4}$ due to Prop. \ref{lb83}. Therefore Claim (a) follows from Thm. \ref{lb56}. A similar argument (together with the fact that $e^{\im t\ovl{L_K}}\psi=F(t+\im\pi)$) proves Claim (b).
\end{proof}


\subsection{}


Recall from Conv. \ref{lb48} that for each $A\in\MV$, $\HV^\infty$ is the invariant domain of
\begin{align}\label{eq119}
A(f)^\dagger=(-1)^{\wt A}A^\theta(\ovl{\ek_{2-2\wt A}f})
\end{align}

\begin{thm}[\textbf{PCT theorem}]\label{lb86}\index{00@PCT theorem for chiral algebras}
There is an anti-unitary operator $\Theta:\HV\rightarrow\HV$ (called the \textbf{PCT operator}) \index{zz@$\Theta$, the PCT operator} such that for each $n\in\Nbb$, $A^1,\dots,A^n\in\MV$, mutually disjoint intervals $I_1,\dots,I_n\subset\Sbb^1_+$, and $f_i\in C^\infty_c(I_i)$, we have
\begin{align}\label{eq115}
\Theta A^\blt(f_\blt)\Omega=A^\blt(\delta^{A^\blt}_{\im\pi}(f_\blt))^\dagger\Omega
\end{align}
\end{thm}

In particular, when $n=0$, Eq. \eqref{eq115} reads $\Theta\Omega=\Omega$.

\begin{proof}
Step 1. It suffices to prove that there is an antiunitary isometry $\Theta$ satisfying the desired property. Then $\Theta$ is automatically surjective, since the vectors in the form of the RHS \eqref{eq115} span a dense subspace of $\HV$ due to the Reeh-Schlieder Thm. \ref{lb70}.

To prove that such an antiunitary isometry exists, it suffices to prove that for each $A^1,\dots,A^n\in\MV$, $B^1,\dots,B^k\in\MV$, mutually disjoint $I_1,\dots,I_n\subset\Sbb^1_+$, mutually disjoint $J_1,\dots,J_k\subset\Sbb^1_+$, and $f_i\in C_c^\infty(I_i),g_j\in C_c^\infty(J_j)$, we have
\begin{align}\label{eq116}
\bk{A^\blt(f_\blt)\Omega|B^\star(g_\star)\Omega}=\bk{B^\star(\delta^{B^\star}_{\im\pi}(g_\star))^\dagger\Omega|A^\blt(\delta_{\im\pi}^{A^\blt}(f_\blt))^\dagger\Omega}
\end{align}

We claim that \eqref{eq116} holds if, instead of assuming $J_1,\dots,J_k\subset\Sbb^1_+$, we assume that $J_1,\dots,J_k\subset\Sbb^1_-$. Suppose this claim is true. Then for each $g_\star$, choosing $\delta>0$ such that $\tau_sg_j\in C_c^\infty(J_j)$ for each $j$, we have
\begin{align*}
\bk{A^\blt(f_\blt)\Omega|B^\star(\tau_sg_\star)\Omega}=\bk{B^\star(\delta^{B^\star}_{\im\pi}(\tau_sg_\star))^\dagger\Omega|A^\blt(\delta_{\im\pi}^{A^\blt}(f_\blt))^\dagger\Omega} \tag{$\triangle$} \label{eq118}
\end{align*}
for each $-\delta<s<\delta$. By the rotation covariance Cor. \ref{lb67} and \eqref{eq123}, there exists $\lambda\in\Rbb$ relying only on $\wt B^1,\dots,\wt B^k$ such that
\begin{align*}
\bk{A^\blt(f_\blt)\Omega|e^{\im s\ovl{L_0}}B^\star(g_\star)\Omega}=e^{\im \lambda s }\bk{e^{\im s\ovl{L_0}}B^\star(\delta^{B^\star}_{\im\pi}(g_\star))^\dagger\Omega|A^\blt(\delta_{\im\pi}^{A^\blt}(f_\blt))^\dagger\Omega}\tag{$\star$}\label{eq117}
\end{align*}
for all $-\delta<s<\delta$. Since $\ovl{L_0}\geq0$, by Lem. \ref{lb55} and Cor. \ref{lb57}, we conclude that \eqref{eq117} holds for all $s\in\Rbb$. By the rotation covariance again, \eqref{eq118} holds for all $s\in\Rbb$. Since we can rotate mutually disjoint intervals of $\Sbb^1_-$ simultaneously to those in $\Sbb^1_+$, we conclude that \eqref{eq116} holds whenever $J_1,\dots,J_k\subset\Sbb^1_+$, finishing the proof.\\[-1ex]


Step 2. Let us prove the claim. Now $I_1,\dots,I_n,J_1,\dots J_k$ are mutually disjoint. Thus the smeared operators supported on them are mutually commuting. So \eqref{eq116} becomes
\begin{align*}
\bk{A^\blt(f_\blt)B^\star(g_\star)^\dagger\Omega|\Omega}=\bk{A^\blt(\delta_{\im\pi}^{A^\blt}(f_\blt))B^\star(\delta^{B^\star}_{\im\pi}(g_\star))^\dagger\Omega|\Omega}
\end{align*}
which is equivalent (by \eqref{eq119}) to
\begin{align*}
\bk{A^\blt(f_\blt)B^{\star,\theta}(\ek_{2\wt B^\star-2} \ovl{g_\star})\Omega|\Omega}=\bk{A^\blt(\delta_{\im\pi}^{A^\blt}(f_\blt))B^{\star,\theta}(\ek_{2\wt B^\star-2}\ovl{\delta^{B^\star}_{\im\pi}(g_\star)})\Omega|\Omega}
\end{align*}
Using \eqref{eq114} one checks easily that
\begin{align*}
\delta^{B^j}_{\im\pi}\big(\ek_{2\wt B^j-2} \ovl{g_j}\big)(z)=(-1)^{\wt B^j}\ovl {g_j(z^{-1})}=\big(\ek_{2\wt B^j-2}\ovl{\delta^{B^j}_{\im\pi}(g_j)}\big)(z)
\end{align*}
Therefore, setting $h_j=\ek_{2\wt B^j-2} \ovl{g_j}$, what we needs to prove becomes
\begin{align*}
\bk{A^\blt(f_\blt)B^{\star,\theta}(h_\star)\Omega|\Omega}=\bk{A^\blt(\delta_{\im\pi}^{A^\blt}(f_\blt))B^{\star,\theta}(\delta_{\im\pi}^{B^\star}(h_\star))\Omega|\Omega}   \tag{$\heartsuit$}\label{eq120}
\end{align*}
To prove this, note that since $K\Omega=0$ (and hence $e^{-\pi \ovl{L_K}}\Omega=\Omega$), the LHS of \eqref{eq120} is
\begin{align*}
\bk{A^\blt(f_\blt)B^{\star,\theta}(h_\star)\Omega|e^{-\pi \ovl{L_K}}\Omega}=\bk{e^{-\pi \ovl{L_K}}A^\blt(f_\blt)B^{\star,\theta}(h_\star)\Omega|\Omega}
\end{align*}
By the PT Thm. \ref{lb85}, the RHS above equals the RHS of \eqref{eq120}. The proof is finished.
\end{proof}


\subsection{}


\begin{pp}\label{lb87}
For each $t\in\Rbb$ we have $\Theta e^{\im t\ovl{L_0}}=e^{-\im t\ovl{L_0}}\Theta$. Consequently, $\Theta\Vbb(n)=\Vbb(n)$ for each $n\in\Nbb$. So we have $\Theta\Vbb=\Vbb$ and $\Theta\HV^\infty=\HV^\infty$.
\end{pp}

Thus $\Theta$ can be extended canonically to an antilinear map $\Theta:\Vbb^\ac\rightarrow\Vbb^\ac$.

\begin{proof}
For each $\xi=A^\blt(f_\blt)\Omega$ as in Thm. \ref{lb86}, we choose $\delta>0$ such that $\tau_tf_i\in C_c^\infty(I_i)$ for all $-\delta<t<\delta$.  Then using the rotation covariance Cor. \ref{lb67}, and setting $\Delta=\wt(A^1)+\cdots+\wt(A^n)-n$, we get for $-\delta<t<\delta$ that
\begin{align*}
\Theta e^{\im t\ovl{L_0}}\xi=e^{-\im t\Delta}\Theta A^\blt(\tau_tf_\blt)\Omega=e^{-\im t\Delta}A^\blt(\delta^{A^\blt}_{\im\pi}\tau_t(f_\blt))^\dagger\Omega
\end{align*}
Using \eqref{eq114} we compute that for all $t\in\Rbb$,
\begin{align}\label{eq123}
\delta_{\im\pi}^A \tau_t(f)=e^{\im t(2-2\wt A)}\cdot \tau_{-t}\delta_{\im\pi}^A(f)
\end{align}
Thus, for $-\delta<t<\delta$ we have
\begin{align*}
&\Theta e^{\im t\ovl{L_0}}\xi=e^{\im t\Delta}A^\blt(\tau_{-t}\delta^{A^\blt}_{\im\pi}(f))^\dagger\Omega=(e^{-\im t\ovl{L_0}}A^\blt(\delta^{A^\blt}_{\im\pi}(f_\blt))e^{\im t\ovl{L_0}})^\dagger\Omega\\
=&e^{-\im t\ovl{L_0}}A^\blt(\delta^{A^\blt}_{\im\pi}(f_\blt))^\dagger\Omega=e^{-\im t\ovl{L_0}}\Theta\xi
\end{align*}
Since $\ovl{L_0}\geq0$, by Lem. \ref{lb55} and Cor. \ref{lb57}, we see that $\Theta e^{\im t\ovl{L_0}}\xi=e^{-\im t\ovl{L_0}}\Theta\xi$ for all $t\in\Rbb$. This proves $\Theta e^{\im t\ovl{L_0}}=e^{-\im t\ovl{L_0}}\Theta$ on $\HV$ for all $t\in\Rbb$. The other relations follow directly from this one.
\end{proof}



\subsection{}

We now give two more characterizations of $\Theta$.


\begin{thm}[\textbf{PCT theorem}]\label{lb89}
$\Theta|_{\HV^\infty}:\HV^\infty\rightarrow\HV^\infty$ is the unique continuous antilinear map satisfying $\Theta\Omega=\Omega$ and
\begin{align}\label{eq122}
\Theta A(f)=A(\delta^A_{\im\pi}(f))^\dagger\Theta \qquad\text{on }\HV^\infty
\end{align}
for all $A\in\MV$, $f\in C^\infty(\Sbb^1)$, and $\xi\in\HV^\infty$. 
\end{thm}

It follows that for all $A^1,\dots,A^n\in\MV$ and $f_1,\dots,f_n\in C^\infty(\Sbb^1)$ we have
\begin{subequations}\label{eq121}
\begin{align}
\Theta A^1(f_1)\cdots A^n(f_n)\Omega=A^1(\delta^{A^1}_{\im\pi}(f_1))^\dagger\cdots A^n(\delta^{A^n}_{\im\pi}(f_n))^\dagger\Omega
\end{align}
which generalizes Thm. \ref{lb86}. In view of Lem. \ref{lb84}, we also have
\begin{align}\label{eq121b}
\Theta A^1(\delta^{A^1}_{\im\pi}(f_1))\cdots A^n(\delta^{A^n}_{\im\pi}(f_n))\Omega=A^1(f_1)^\dagger\cdots A^n(f_n)^\dagger\Omega
\end{align}
\end{subequations}

\begin{proof}
The uniqueness is clear from \eqref{eq121} and the Reeh-Schlieder Thm. \ref{lb70}. Let us show that the PCT operator $\Theta$ satisfies \eqref{eq122}. By linearity, it suffices to prove \eqref{eq122} when $f\in C_c^\infty(I)$ for some interval $I\subset\Sbb^1$ whose length is $<\pi$. 

We first consider the case that $I\subset\Sbb^1_+$. Let $T$ be the product of several smeared operators supported in mutually disjoint intervals of $\Sbb^1_+$ that are also disjoint from $I$. Then by Thm. \ref{lb86}, we have $\Theta A(f)T\Omega=A(\delta^A_{\im\pi}(f))^\dagger\Theta T\Omega$. Let $\mc W$ be the subspace spanned by vectors of the form $T\Omega$. Then $\Theta A(f)=A(f)^\dagger\Theta$ on $\mc W$. By Cor. \ref{lb88}, $\mc W$ is a core for $A(f)$. Therefore $\Theta A(f)=A(\delta^A_{\im\pi}(f))^\dagger\Theta$ on $\HV^\infty$.



By Prop. \ref{lb87}, for each $t\in\Rbb$ we have $\Theta e^{\im t\ovl{L_0}}A(f)e^{-\im t\ovl{L_0}}=e^{-\im t\ovl{L_0}}A(\delta^A_{\im\pi}(f))^\dagger e^{\im t\ovl{L_0}}\Theta$ on $\HV^\infty$. The rotation covariance Cor. \ref{lb67} then implies that 
\begin{align*}
e^{-\im t(\wt(A)-1)}\Theta A(\tau_t f)=e^{\im t(\wt(A)-1)}A(\tau_{-t}\delta_{\im\pi}^A(f))^\dagger\Theta
\end{align*}
on $\HV^\infty$. 
Therefore, in view of \eqref{eq123}, we obtain $\Theta A(\tau_tf)=A(\delta^A_{\im\pi}\tau_t(f))^\dagger\Theta$ on $\HV^\infty$. This finishes the proof, since any smooth function on $\Sbb^1$ supported in an interval with length $<\pi$ is of the form $\tau_tf$.
\end{proof}


\subsection{}


\begin{thm}\label{lb90}
$\Theta|_\Vbb$ is the unique antilinear map $\Vbb\rightarrow\Vbb$ satisfying $\Theta\Omega=\Omega$ and
\begin{align}
\Theta A_n=A^\theta_n\Theta
\end{align}
(on $\Vbb$) for all $A\in\MV$ and $n\in\Zbb$. The latter is equivalent to that for each $A\in\MV$ and $z\in\Cbb^\times$, as antilinear maps $\Vbb\rightarrow\Vbb^\ac$ we have
\begin{align}\label{eq124}
\Theta A(z)=A^\theta(\ovl z)\Theta
\end{align}
\end{thm}

Eq. \eqref{eq124} shows that $\Theta$ implements the PCT symmetry, i.e., the combination of the PT symmetry $z\mapsto \ovl z=z^{-1}$ (on $\Sbb^1$) and the charge symmetry $A\mapsto A^\theta$.



\begin{proof}
Recall that $A_n=A(\ek_n)$. Then Thn. \ref{lb89} shows that $\Theta A_n=A(\delta^A_{\im\pi}(\ek_n))^\dagger\Theta$. By \eqref{eq114} one computes that $\delta^A_{\im\pi}(\ek_n)=(-1)^{\wt A}\ek_{2\wt A-2-n}$. Hence, by \eqref{eq37}, we get
\begin{align*}
\Theta A_n=A(\delta^A_{\im\pi}(\ek_n))^\dagger\Theta=(-1)^{\wt A}(A_{2\wt A-2-n})^\dagger\Theta=A^\theta_n\Theta
\end{align*}
The uniqueness follows from the cyclicity in Def. \ref{lb30}.
\end{proof}



\begin{co}
We have $\Theta^2=\idt$.
\end{co}

\begin{proof}
This is clear from Thm. \ref{lb90} and the fact that $A^{\theta\theta}=A$.
\end{proof}

%% Proofread

\newpage



\section{Tomita-Takesaki theory for unbounded operators}


 

\begin{ass}\label{lb95}
In this section, unless otherwise stated, we assume that $\mc M\curvearrowright\mc H$ is a von Neumann algebra with a fixed cyclic separating vector (cf. Def. \ref{lb91}) $\Omega$.
\end{ass}


\subsection{}


\begin{df}\label{lb91}
We say that a vector $\Omega\in\mc H$ is \textbf{cyclic} \index{00@Cyclic vector} (with respect to $\mc M$) if $\mc M\Omega$ is dense in $\mc H$.  We say that $\Omega$ is \textbf{separating} (with respect to $\mc M$) \index{00@Separating vector} if one of the following equivalent conditions holds:
\begin{enumerate}
\item[(1)] $\Omega$ is cyclic with respect to the commutant $\mc M'$.
\item[(2)] Any element $x\in\mc M$ satisfying $x\Omega=0$ must be zero.
\end{enumerate}
\end{df}





\begin{proof}[Proof of equivalence]
Assume (1). If $x\in\mc M$ and $x\Omega=0$, then $xy\Omega=yx\Omega=0$ for all $y\in\mc M'$, and hence $x=0$ (because all $y\Omega$ form a dense space). This proves (2).

Conversely, assume (2). Let $e$ be the projection onto $\ovl{\mc M'\Omega}$. Then $ye=eye$ for all $y\in\mc M'$; similarly $y^*e=ey^*e$. So $ey=eye=ye$, and hence $e\in\mc M$. Since $\Omega\in e\mc H$, we have $e^\perp\Omega=0$. Therefore $e^\perp=0$, and hence $\Omega$ is $\mc M'$-cyclic.
\end{proof}


\subsection{}


Suppose that $T:\mc H_1\rightarrow\mc H_2$ is a (densely-defined) unbounded antilinear map. Let $C_1$ and $C_2$ be antiunitary operators on $\mc H_1$ and $\mc H_2$. We say that $T$ is \textbf{closed} resp. \textbf{closable} if $TC_1$ is so, equivalently, if $C_2T$ is so. The \textbf{adjoint} is defined to be $T^*=(C_2T)^*C_2$, equivalently, $T^*=C_1(TC_1)^*$. These notions are independent of the choice of $C_1,C_2$. 

It is clear that a vector $\eta\in\mc H_2$ belongs to the domain $\Dom(T^*)$ iff there exists $\psi\in\mc H_1$ such that $\bk{T\xi|\eta}=\bk{\psi|\xi}$. Then we must have $\psi=T^*\eta$, and hence
\begin{align}
\bk{T\xi|\eta}=\bk{T^*\eta|\xi}
\end{align}
Notice the change of order on the RHS. We still have that $T$ is closable iff $\Dom(T^*)$ is dense in $\mc H$.  \hfill\qedsymbol


\begin{rem}\label{lba20}
Let $T$ be an antilinear closed operator on a Hilbert space $\mc H$. Let $C$ be any antiunitary operator on $\mc H$. Then the polar decomposition theorem for $CT$ implies that $T$ can uniquely be written as $T=UH$ where $H$ is self-adjoint, and $U$ is an antilinear partial isometry satisfying $U^*U=\ovl{\Rng(H)}$. 

Moreover, if $x$ is a bounded operator on $H$, then the following are equivalent:
\begin{enumerate}[label=(\alph*)]
\item $xT\subset Tx$ and $x^*T\subset Tx^*$.
\item We have $xU=Ux$, $x^*U=Ux^*$, $xH\subset Hx$. (So we automatically have $x^*H\subset Hx^*$.)
\end{enumerate}
\end{rem}

A similar result holds for the right polar decomposition.

\begin{proof}[Proof of the equivalence]
``(b)$\Rightarrow$(a)" is obvious. Tssume (a). Then $CxC^{-1}\cdot CT\subset CT\cdot x$ and $Cx^*C^{-1}T\subset CT\cdot x^*$. Consider the diagram
\begin{equation*}
\begin{tikzcd}
\mc H \arrow[d, "CT"'] \arrow[r, "x"] & \mc H \arrow[d, "CT"] \\
\mc H \arrow[r, "CxC^{-1}"]           & \mc H                
\end{tikzcd}
\end{equation*}
Define $\mc K=\mc H\oplus\mc H\oplus\mc H\oplus\mc H$ where the four components are the northwest, northeast, southwest, southeast $\mc H$ of the above diagram.

Define $X$ on $\mc K$ sending $(\alpha,\beta,\gamma,\delta)$ to $(0,x\alpha,0,CxC^{-1}\gamma)$. Define $Y$ to be a closed operator on $\mc K$ with domain $\Dom(T)\oplus\mc H\oplus\Dom(T)\oplus\mc H$ sending  $(\alpha,\beta,\gamma,\delta)$ to $(0,0,CT\alpha,CT\beta)$. Then $XY\subset YX$ and $X^*Y\subset YX^*$. Namely, $X$ commutes strongly with $Y$. Therefore, if we let $Y=VK$ be the polar decomposition of $Y$, then $[X,V]=[V^*,X]=0$ and $XK\subset KX$. Clearly $K$ and $V$ send $(\alpha,\beta,\gamma,\delta)$ respectively to $(H\alpha,H\beta,0,0)$ and $(0,0,CU\alpha,CU\beta)$. This proves (b).
\end{proof}













\subsection{}


\begin{df}
Define (densely-defined) antilinear unbounded operators $S_{\mc M},F_{\mc M}$ on $\mc H$ with domains $\mc M\Omega,\mc M'\Omega$ respectively, such that
\begin{gather*}
S_{\mc M}:x\Omega\in\mc M\Omega\mapsto x^*\Omega,\qquad F_{\mc M}:y\Omega\in\mc M'\Omega\mapsto y^*\Omega.	
\end{gather*}
$S_{\mc M},F_{\mc M}$ are called the \textbf{modular \pmb{$S,F$} operators} \index{00@Modular $S,F$ operators} of $\mc M$ with respect to $\Omega$. Clearly $S_{\mc M}=F_{\mc M'}$ and $S_{\mc M'}=F_{\mc M}$. We abbreviate $S_{\mc M},F_{\mc M}$ to $S,F$ when no confusion arises.
\end{df}




\begin{lm}
$S\subset F^*$ and $F\subset S^*$.
\end{lm}	




\begin{proof}
For each $x\in\mc M,y\in\mc M'$,
\begin{align*}
\bk{Sx\Omega|y\Omega}=\bk{x^*\Omega|y\Omega}=\bk{\Omega|yx\Omega}=\bk{y^*\Omega|x\Omega}=\bk{Fy\Omega|x\Omega}.	
\end{align*}
This proves $S\subset F^*$. The other relation can be proved in the same way.
\end{proof}

Consequently, $S$ and $F$ are closable (since their adjoints are densely defined). From now on, \uwave{$S,F$ denote the closures of the original ones}.

\begin{df}\label{lb92}
For each $\xi\in\mc H$, define unbounded operators $\scr L(\xi)$ and $\scr R(\xi)$ with domains $\mc M'\Omega,\mc M\Omega$ respectively, such that for each $x\in\mc M,y\in\mc M'$,
\begin{align*}
\scr L(\xi)y\Omega=y\xi,\qquad \scr R(\xi)x\Omega=x\xi.	
\end{align*}
If $\scr L(\xi)$ or $\scr R(\xi)$ is closable, then $\scr L(\xi),\scr R(\xi)$ denote the closure of the original one. If $\scr L(\xi)$ resp. $\scr R(\xi)$ are bounded, we write it as $L(\xi)$ resp. $R(\xi)$. \index{LR@$\scr L(\xi),\scr R(\xi),L(\xi),R(\xi)$}
\end{df}


\begin{pp}\label{lba13}
If $\scr L(\xi)$ is closable, then $\scr L(\xi)$ is affiliated with $\mc M$. If $\scr R(\xi)$ is closable, then $\scr R(\xi)$ is affiliated with $\mc M'$.
\end{pp}

\begin{proof}
We show that $\scr L(\xi)$ commutes strongly with any $y\in\mc M'$. For each $b\in\mc M'$, $\scr L(\xi)yb\Omega=yb\xi=y\scr L(\xi)b$. So $y\scr L(\xi)|_{\mc M'}\subset \scr L(\xi)|_{\mc M'}\cdot y$. Passing to the closure, we obtain $y\scr L(\xi)\subset \scr L(\xi)y$. Similarly, $y^*\scr L(\xi)\subset \scr L(\xi)y^*$.
\end{proof}



\subsection{}

The following theorem is one of the most important criteria for the closability of $\scr L(\xi),\scr R(\xi)$.

\begin{thm}\label{lba12}
Let $\xi\in\mc H$. The following two are equivalent.
\begin{enumerate}[label=(\alph*)]
	\item $\Omega\in\Dom(\scr L(\xi)^*)$.
	\item $\xi\in\Dom(F^*)$.
\end{enumerate}
If either of them is true, then $\scr L(F^*\xi)\subset\scr L(\xi)^*$, and hence $\scr L(\xi)$ is closable and $F^*\xi=\scr L(\xi)^*\Omega$.

Similarly, let $\eta\in\mc H$, then the following two are equivalent.
\begin{enumerate}
	\item[(a')] $\Omega\in\Dom(\scr R(\eta)^*)$.
	\item[(b')] $\eta\in\Dom(S^*)$.
\end{enumerate}
If either of them is true, then $\scr R(S^*\eta)\subset\scr R(\eta)^*$, and hence $\scr R(\eta)$ is closable and $S^*\eta=\scr R(\eta)^*\Omega$.
\end{thm}





\begin{proof}
First, assume (a) is true. Then for any $y\in\mc M'$,
\begin{align*}
\bk{Fy\Omega|\xi}=\bk{y^*\Omega|\xi}=\bk{\Omega|y\xi}=\bk{\Omega|\scr L(\xi)y\Omega}=\bk{\scr L(\xi)^*\Omega|y\Omega}.
\end{align*}
This proves (b) (and that $F^*\xi=\scr L(\xi)^*\Omega$).

Assume (b). Choose any $y_1,y_2\in\mc M'$. Then
\begin{align*}
&\bk{\scr L(F^*\xi)y_1\Omega|y_2\Omega}=\bk{y_1F^*\xi|y_2\Omega}=\bk{F^*\xi|y_1^*y_2\Omega}=\bk{y_2^*y_1\Omega|\xi}\\
=&\bk{y_1\Omega|y_2\xi}=\bk{y_1\Omega|\scr L(\xi)y_2\Omega}.	
\end{align*}
So $\scr L(F^*\xi)\subset \scr L(\xi)^*$. Since $\scr L(F^*\xi)$ has dense domain, so does $\scr L(\xi)^*$. So $\scr L(\xi)$ is closable. Since $\Omega\in \Dom(\scr L(F^*\xi))$ and $\scr L(F^*\xi)\Omega=F^*\xi$, we have $\Omega\in\Dom(\scr L(\xi)^*)$ (i.e., (a) is true) and $\scr L(\xi)^*\Omega=F^*\xi$.
\end{proof}



\begin{thm}\label{lb93}
	$S=F^*$.
\end{thm}

Thus, conditions (b) and (b') in Prop. \ref{lb12} read $\xi\in\Dom(S)$ and $\eta\in\Dom(F)$.

\begin{proof}
	As $S\subset F^*$, it suffices to prove that $\mathcal M\Omega$, which is a core of $S$, is also a core of $F^*$. Choose any $\xi\in\Dom(F^*)$. Let $A=\scr L(\xi)$. By Thm. \ref{lba12}, we have that $\Omega\in\Dom(A)\cap\Dom(A^*)$, and that $\scr L(\xi)$ is closable (and hence closed, cf. Def. \ref{lb92}). By. Prop. \ref{lba13}, $A$ is affiliated with $\mc M$.

Let $A=HU$ be the right polar decomposition of $A$ with $H$ positive and $U$ unitary. For each $\lambda>0$, let $E_\lambda=\chi_{[0,\lambda]}(H)$. Then $\overline{E_\lambda H}U\in\mathcal M$, and hence $E_\lambda HU\Omega\in\mathcal M\Omega$. Clearly $\lim_{\lambda\rightarrow+\infty}E_\lambda HU\Omega=HU\Omega=A\Omega=\xi$. Since $S\subset F^*$, we have
	\begin{align*}
		F^*E_\lambda HU\Omega=S\overline{E_\lambda H} U\Omega=U^*\cdot HE_\lambda\Omega.
	\end{align*}
Since $\Omega\in\Dom(A^*)=\Dom(H)$, when $\lambda\rightarrow+\infty$, $HE_\lambda\Omega$ converges and equals $H\Omega$. So $F^*E_\lambda HU\Omega$ converges to a vector, which must be $F^*\xi$. Thus our proof is done.
\end{proof}


Thm. \ref{lba12} is a powerful tool for proving strong commutativity: Let $A,B$ be closed operators on $\mc H$. Suppose that $\Omega\in\Dom(A)\cap\Dom(B)$, and let $\xi=A\Omega,\eta=B\Omega$. Suppose that we can prove that $\xi\in\Dom(S_{\mc M})$ and $\eta\in\Dom(F_{\mc M})$. Then $\scr L(\xi)$ and $\scr R(\eta)$ are affiliated with $\mc M$ and $\mc M'$ respectively, and hence they are strongly commuting. Suppose we can also prove that $A=\scr L(\xi)$ and $B=\scr R(\eta)$, then we can conclude that $A$ commutes strongly with $B$.




\subsection{}



A converse of Thm. \ref{lba12} is:

\begin{pp}\label{lba14}
The following are true.
\begin{enumerate}
\item[(1)] Let $A$ be a closed operator on $\mc H$ affiliated with $\mathcal M$. Assume $\Omega\in\Dom(A)\cap\Dom(A^*)$. Then $A\Omega\in\Dom(S)$, $SA\Omega=A^*\Omega$, and $\scr L(A\Omega)\subset A\subset \scr L(A^*\Omega)^*$.
\item[(2)] Let $B$ be a closed operator on $\mc H$ affiliated with $\mathcal M'$. Assume $\Omega\in\Dom(B)\cap\Dom(B^*)$. Then $B\Omega\in\Dom(F)$, $FB\Omega=B^*\Omega$, and $\scr R(B\Omega)\subset B\subset \scr R(B^*\Omega)^*$.
\end{enumerate}
\end{pp}


\begin{proof}
In (1), since $A$ commutes strongly with each $y\in\mc M'$, we have $yA\subset Ay$. In particular, $y\Dom(A)\subset \Dom(A)$ and hence $y\Omega\in\Dom(A)$. So $\mc M'\Omega\subset\Dom(A)$. Moreover, $\scr L(A\Omega)y\Omega=yA\Omega=Ay\Omega$. So $\scr L(A\Omega)\subset A$. Similarly, $\scr L(A^*\Omega)\subset A^*$, whose adjoint implies $A\subset(\scr L(A^*\Omega))^*$. 


Similarly, we have $A^*\subset(\scr L(A\Omega))^*$. So $\Omega\in\Dom(A^*)\subset\Dom(\scr L(A\Omega)^*)$. Therefore, by Thm. \ref{lba12} (and Thm. \ref{lb93}), we have $A\Omega\in\Dom(S)$ and $SA\Omega=\scr L(A\Omega)^*\Omega=A^*\Omega$.
\end{proof}


\subsection{}


\begin{thm}\label{lba15}
Let $\mc N\subset\mc M$ be von Neumann algebras on $\mc H$ with a common cyclic separating vector $\Omega$. Clearly $S_{\mc N}\subset S_{\mc M}$. Then $\mc N=\mc M$ if and only if $S_{\mc N}=S_{\mc M}$.
\end{thm}



\begin{proof}
Assume $S_{\mc N}=S_{\mc M}$. Choose any $A\in\mc M$. We shall show $A\in\mc N$. 

Clearly $A\Omega\in\Dom(S_{\mc M})$. So $A\Omega\in\Dom(S_{\mc N})$. By Thm. \ref{lba12},  $\scr L_{\mc N}(A\Omega)$ is  closed. Therefore, by Prop. \ref{lba13}, $\scr L_{\mc N}(A\Omega)$ is affiliated with $\mc N$. For each $y\in\mc M'$, we have $\scr L_{\mc N}(A\Omega)y\Omega=yA\Omega=Ay\Omega$. Thus $\scr L_{\mc N}(A\Omega)|_{\mc M'\Omega}=A|_{\mc M'\Omega}$. Taking the closure (and noting that $A$ is bounded), we get $\scr L_{\mc N}(A\Omega)\supset A$. Since $\Dom(A)=\mc H$, we must have $\scr L_{\mc N}(A\Omega)=A$. So $A$ is affiliated with $\mc N$, and hence $A\in\mc N$.
\end{proof}











The following corollary is more or less equivalent to Thm. \ref{lba15}.

\begin{co}\label{lba16}
Let $\mc A$ be a unital $*$-subalgebra of $\mc M$. Then $\mc A''=\mc M$ if and only if $\mc A\Omega$ is a core for $S_{\mc M}$.
\end{co}

\begin{proof}
Suppose $\mc A''=\mc M$. Then for each $x\in\mc M$, there is a net $x_\alpha\in\mc A$ converging strongly* to $x$. So $x_\alpha\Omega\rightarrow x\Omega$ and $Sx_\alpha\Omega=x_\alpha^*\Omega\rightarrow x^*\Omega=Sx\Omega$. So $\mc A\Omega$ is a core.

Suppose that $\mc A\Omega$ is a core for $S=S_{\mc M}$. Let $\mc N=\mc A''$. Then $\Omega$ is clearly separating and (as $\mc A\Omega$ is dense in $\mc H$) cyclic for $\mc N$. By the first paragraph, $\mc A\Omega$ is a core for $S_{\mc N}$. So $S_{\mc N}=S_{\mc M}$. Therefore, by Thm. \ref{lba15}, $\mc N=\mc M$.
\end{proof}

We give a more straightforward proof of ``$\Leftarrow$". The remaining part of this subsection can be safely skipped.

\begin{proof}[Second proof of ``$\Leftarrow$"]
Assume that $\mc A\Omega$ is a core for $S=S_{\mc M}$. Then for each $x\in\mc M$, there exists a net $x_\alpha$ in $\mc A$ such that $x_\alpha\Omega\rightarrow x\Omega$ and $x_\alpha^*\Omega=Sx_\alpha\Omega\rightarrow Sx\Omega=x^*\Omega$. Thus, for each $y\in\mc M'$ we have $x_\alpha y\Omega=yx_\alpha\Omega\rightarrow yx\Omega=xy\Omega$ and similarly $x_\alpha^*y\Omega\rightarrow x^*y\Omega$. In other words, the net $(x_\alpha^*)$ ``converges strongly* on $\mc M'\Omega$ to $x$". By the next lemma, we conclude that $x\in\mc A''$.
\end{proof}


\begin{lm}
Let $\mc N$ be a von Neumann algebra on a Hilbert space $\mc H$, and let $A_\alpha$ be a net of closed operators on $\mc H$ affiliated with $\mc N$. Let $A$ be a closed operator on $\mc H$ satisfying the following conditions:
\begin{enumerate}[label=(\alph*)]
\item There exists a core $\Dom_1$ for $A$ contained in $\Dom(A_\alpha)$ for all $\alpha$, such that $\lim_\alpha A_\alpha\xi=A\xi$ for all $\xi\in\Dom_1$.
\item There exists a core $\Dom_2$ for $A^*$ contained in $\Dom(A_\alpha^*)$ for all $\alpha$, such that $\lim_\alpha A_\alpha^*\eta=A^*\eta$ for all $\eta\in\Dom_1$.
\end{enumerate}
Then $A$ is affiliated with $\mc N$.
\end{lm}

This lemma might be useful in its own right. But we will not use it in this course.

\begin{proof}
Choose any $y\in\mc N'$ and $\xi\in\Dom_1$. Since $A_\alpha$ is affiliated with $\mc N$, we have $yA_\alpha\subset A_\alpha y$ and hence $y\Dom_1\subset y\Dom(A_\alpha)\subset \Dom(A_\alpha)$. Thus, for all $\eta\in\Dom_1$ we have
\begin{align*}
\bk{y\xi|A^*\eta}=\lim_\alpha \bk{y\xi|A_\alpha^*\eta}=\lim_\alpha \bk{A_\alpha y\xi|\eta}=\lim_\alpha \bk{yA_\alpha\xi|\eta}=\bk{yA\xi|\eta}
\end{align*}
So $y\xi\in\Dom((A^*|_{\Dom_2})^*)$ and $(A^*|_{\Dom_2})^*y\xi=yA\xi$. Since $\Dom_2$ is a core for $A^*$, we have $y\xi\in\Dom(A)$ and $Ay\xi=yA\xi$. This proves $yA|_{\Dom_1}\subset Ay$. Since $\Dom_1$ is a core for $A$, we get $yA\subset Ay$. Similarly, $y^*A\subset Ay^*$. So $\mc N'\subset\{A\}'$, and hence $A$ is affiliated with $\mc N$.
\end{proof}




\subsection{}


\begin{df}
We let $S_{\mc M}=\fk J_{\mc M}\Delta_{\mc M}^{\frac 12}$ (or simply $S=\fk J\Delta$) be the left polar decomposition for $S$, where $\Delta=S^*S$. We call $\Delta_{\mc M}$ and $\fk J_{\mc M}$ the \textbf{modular $\Delta$ operator} \index{00@Modular $\Delta$ operator} and the \textbf{modular conjugation} \index{00@Modular conjugation} of $\mc M$ with respect to $\Omega$. 
\end{df}


\begin{rem}
Note that by spectral theory, if $T$ is a normal closed operator on $\mc H$, then $\chi_{\{0\}}(T)$ is the projection onto $\Ker(T)$. (Proof: View $T$ as a multiplication operator.) So $\chi_{\{0\}}(T)=0$ iff $T$ is injective. Therefore, such such $T$, if $T\geq0$, then we can define $\log T$ and $T^\alpha$ for all $\alpha\in\Cbb$.
\end{rem}


\begin{lm}\label{lb94}
Let $T$ be a closed linear (resp. antilinear) operator on $\mc H$ with left polar decomposition $T=UH$ where $U$ is a linear (resp. antilinear) partial isometry. Assume that $T$ is injective with dense range (so that $S^{-1}$ is a densely defined unbounded operator). Then $H$ is injective, $U$ is (anti)unitary, and $T^{-1}=H^{-1}U^{-1}$.
\end{lm}

In particular, $T^{-1}$ is a closed operator with right polar decomposition $T^{-1}=H^{-1}U^{-1}$.


\begin{proof}
We address the linear case; the antilinear case can be obtained by applying the result to $CT$ where $C$ is an arbitrary antiunitary operator on $\mc H$.  

Since $T$ is injective, clearly $H$ is injective. Since $U^*U$ is the projection onto $\Ker(T)^\perp$, we have $U^*U=1$. Since $UU^*$ is the projection onto the closure of $T$, we have $UU^*=1$. So $U$ is unitary.  Then we clearly have $T^{-1}=H^{-1}U^{-1}$. 
\end{proof}



\begin{pp}
We have $\fk J^2=1$ and $\chi_{\{0\}}(\Delta^{\frac 12})=0$. The left and right polar decompositions for $S,F$ are
\begin{gather}\label{eq125}
\begin{gathered}
S=\fk J\Delta^{\frac 12}=\Delta^{-\frac 12}\fk J\\
F=\fk J\Delta^{-\frac 12}=\Delta^{\frac 12}\fk J
\end{gathered}
\end{gather}
Moreover, we have $\fk J\Omega=\Delta\Omega=\Omega$.
\end{pp}

Note that by spectral theory, $\Delta\Omega=\Omega$ implies $f(\Delta)\Omega=f(1)\cdot\Omega$ for any Borel function $f:\Rbb_{>0}\rightarrow\Cbb$.




\begin{proof}
We first show that $S^2$ equals $1$ on $\Dom(S)=\Dom(\Delta^{\frac 12})$. Choose any $\xi\in\Dom(S)$. Then by Thm. \ref{lba12}, $\scr L(\xi)$ is closable, $\Omega\in\Dom(\scr L(\xi)^*)$, and $S\xi=\scr L(\xi)^*\Omega$. Let $A=\scr L(\xi)^*$, which is affiliated to $\mc M$ by Prop. \ref{lba13}.	Then by Prop. \ref{lba14}, $A\Omega\in\Dom(S)$, and $SA\Omega=A^*\Omega$. Namely, $SS\xi=\scr L(\xi)\Omega=\xi$. It follows that $S$ is a bijection from $\Dom(\Delta^{\frac 12})$ onto $\Dom(\Delta^{\frac 12})$. In particular, $S$ is injective, and hence $\Delta^{\frac 12}$ is injective.


We have proved $S=S^{-1}$. Thus, by Lem. \ref{lb94}, $S$ has right polar decomposition $\Delta^{-\half}\fk J^{-1}$, and hence left polar decomposition $\fk J\cdot \fk J\Delta^{-\frac 12}\fk J^{-1}$. By the uniqueness of polar decompositions, we have $\fk J=\fk J^{-1}$ and $\Delta^{\frac 12}=\fk J\Delta^{-\frac 12}\fk J^{-1}$. This proves $\fk J^2=1$ and \eqref{eq125}.

Finally, since $S^*S\Omega=\Omega$, we have $\Delta\Omega=\Omega$. Thus $\fk J\Omega=S\Delta^{-\half}\Omega=\Omega$.
\end{proof}










\subsection{}




\begin{thm}
We have $\Delta^{\im t}\mc M\Delta^{-\im t}=\mc M$ (and hence $\Delta^{\im t}\mc M'\Delta^{-\im t}=\mc M'$) for all $t\in\Rbb$. Moreover, we have $\fk J\mc M\fk J=\mc M'$.
\end{thm}

The above theorem is the celebrated \textbf{Tomita-Takesaki theorem}. It's proof is very involved and hard to understand. Thus we will not use this theorem in this course. 

Indeed, in the context of QFT, $\Delta^{\im t}\mc M\Delta^{-\im t}=\mc M$ can be proved by the Lorentz-boost/dilation covariance (cf. Cor. \ref{lbb17}), and $\fk J\mc M\fk J\subset\mc M'$ can be proved by the PCT symmetry (cf. Thm. \ref{lb89}). Then we will have $\fk J\mc M\fk J=\mc M'$:

\begin{thm}\label{lb100}
If $\fk J\mc M\fk J\subset\mc M'$, then $\fk J\mc M\fk J=\mc M'$.
\end{thm}

\begin{proof}
The modular $S$ operator of $\fk J\mc M\fk J$ is $\fk J S_{\mc M}\fk J$, which is $\fk J\fk J\Delta^\half\fk J=\Delta^\half\fk J=F_{\mc M}$, the modular $S$ operator of $\mc M'$. 
Therefore, by Thm. \ref{lba15} or Cor. \ref{lba16}, we have $\fk J\mc M\fk J=\mc M'$.
\end{proof}

\subsection{}



Our final goal is to prove a uniqueness theorem for modular operators that will aid the proof of the Bisognano-Wichmann theorem. We first need a lemma.


\begin{lm}\label{lbb25}
Let $H,K$ be positive closed operators on  $\mc H$. Suppose that there is a subspace $\Dom_0\subset\Dom(H)\cap\Dom(K)$ such that $\Dom_0$ is a core for $H$, and that for each $\xi,\eta\in\Dom_0$ we have
\begin{align}\label{eqb38}
\bk{H\xi|H\eta}=\bk{K\xi|K\eta}
\end{align}
Suppose also that $\Dom(H)$ contains a core for $K$. Then $H=K$.
\end{lm}



\begin{proof}
Eq. \eqref{eqb38} shows that there is a unique partial isometry $U$ on $\mc H$ such that $UH\xi=K\xi$ for all $\xi\in\Dom_0$, and that $U^*U$ is the projection onto the closure of $H\Dom_0$. Thus $UH|_{\mc \Dom_0}\subset K$. Since $\Dom_0$ is a core for $H$, we get $UH\subset K$. Since $\Dom(UH)=\Dom(H)$ contains a core for $K$, we have $UH=K$. So $K^2=HU^*UH=H^2$, and hence $K=H$.
\end{proof}


\subsection{}



%In the following theorem, we do not assume Asmp. \ref{lb95}.


\begin{thm}[\textbf{Modular uniqueness theorem}]\label{lbb26}
Let $D$ be an injective positive closed operator on a Hilbert space $\mc H$. Let $\scr A$ be a set of closed operators on $\mc H$. Let $\mc M=\scr A''$, and assume that $\Omega\in\mc H$ is a cyclic separating vector of $\mc M$. Assume that the following conditions are true:
\begin{enumerate}[label=(\alph*)]
\item $D^{\im t}\Omega=\Omega$ for all $t\in\Rbb$.
\item $D^{\im t}\scr AD^{-\im t}\subset\scr A$ for all $t\in\Rbb$. 
\item For any $A\in\scr A$, we have $\Omega\in\Dom(A)\cap\Dom(A^*)$ and $A\Omega\in\Dom(D^{\frac 12})$. For any $A,B\in\scr A$, we have
\begin{align}\label{eqb40}
\bk{D^{\frac 12}A\Omega|D^{\frac 12}B\Omega}=\bk{B^*\Omega|A^*\Omega}
\end{align}
\item $\scr A\Omega$ spans a dense subspace of $\mc H$.
\end{enumerate}
Then $\scr A\Omega$ spans a core for $D^{\frac 12}$, and $D=\Delta_{\mc M}$. Moreover, there is a unique antiunitary operator $J$ on $\mc H$ satisfying
\begin{align}\label{eqb37}
JD^{\frac 12}A\Omega=A^*\Omega
\end{align}
for all $A\in\scr A$. We have $J=\fk J_{\mc M}$.
\end{thm}





\begin{proof}
Step 1. We first note that by  Prop. \ref{lba14}, if $A\in\scr A$, then $A\Omega\in\Dom(\Delta_{\mc M}^{\frac 12})$, and $\fk J_{\mc M}\Delta_{\mc M}^{\frac 12}A\Omega=A^*\Omega$. The same can be said about $B\in\scr A$. Therefore \eqref{eqb40} is equivalent to
\begin{align}\label{eqb39}
\bk{D^{\frac 12}A\Omega|D^{\frac 12}B\Omega}=\bk{\Delta_{\mc M}^{\frac 12}A\Omega|\Delta_{\mc M}^{\frac 12}B\Omega}
\end{align}

Let $\Dom_0$ be the dense subspace spanned by $\scr A\Omega$. By conditions (a) and (b), for each $t\in\Rbb$ we have $D^{\im t}\Dom_0\subset\Dom_0$. Thus, by Thm. \ref{lb58}, $\Dom_0$ is a core for $D^{\frac 12}$. If we can prove that $\Dom(D^{\frac 12})$ contains a core for $\Delta_{\mc M}^{\frac 12}$, then by \eqref{eqb39} and Lem. \ref{lbb25}, we must have $D^{\frac 12}=\Delta_{\mc M}^{\frac 12}$.\\[-1ex]

Step 2. Let us prove that $\Dom(D^{\frac 12})$ contain a core for $S_{\mc M}$ (and hence for $\Delta_{\mc M}^{\frac 12}$). For any $r>0$, define a (clearly bounded linear) operator $x_r$ on $\mc H$ sending each $\xi\in\mc H$ to the improper Riemann integral
\begin{align}
x_r\xi=\sqrt{\frac r\pi} \int_\Rbb e^{-rs^2}D^{\im s}xD^{-\im s}\xi ds
\end{align}
By condition (b), we have $D^{\im s}\mc MD^{-\im s}\subset\mc M$. Therefore $x_r\in\mc M$.

By (a), for each $t\in\Rbb$, we have
\begin{align*}
D^{\im t}x_r\Omega=\sqrt{\frac r\pi} \int_\Rbb e^{-r(s-t)^2}D^{\im s}x\Omega ds
\end{align*}
This function over $t\in\Rbb$ can be extended to an entire function
\begin{align*}
\tau\in\Cbb\mapsto \sqrt{\frac r\pi} \int_\Rbb e^{-r(s-\tau)^2}D^{\im s}x\Omega ds\in\mc H
\end{align*}
Therefore, by Thm. \ref{lb56}, we have $x_r\Omega\in\Dom(D^a)$ for each $a\in\Rbb_{\geq0}$. In particular, $x_r\Omega\in\Dom(D^{\frac 12})$. 

As $r\rightarrow+\infty$, we clearly have $x_r\rightarrow x$ strongly, and hence $x_r\Omega\rightarrow x\Omega$. Since $(x_r)^*=(x^*)_r$, we have $x_r^*\Omega\rightarrow x^*\Omega$. Therefore $S_{\mc M}x_r\Omega=x_r^*\Omega\rightarrow x^*\Omega$. Since $x_r\Omega\in\Dom(D^{\frac 12})$, we conclude that $\Dom(D^{\frac 12})$ contains a core for $S_{\mc M}$, namely, $\Span\{x_r\Omega:r>0,x\in\mc M\}$.\\[-1ex]

Step 3. By \eqref{eqb39}, there is a unique antilinear partial isometry $J$ on $\mc H$ satisfying \eqref{eqb37} for all $A\in\scr A$, and satisfies that $J^*J$ is the projection onto the closure of $D^{\frac 12}\Dom_0$. Since $D^{\frac 12}$ has dense range (because $\chi_{\{0\}}(D^{\frac 12})=0$), and since we have proved that $\Dom_0$ is a core for $D^{\frac 12}$,  $D^{\frac 12}\Dom_0$ must be dense in $\mc H$. So $J^*J=1$. By Prop. \ref{lba14}, for each $A\in\scr A$ we have
\begin{align*}
JD^{\frac 12}A\Omega=A^*\Omega= S_{\mc M}A\Omega=\fk J_{\mc M}\Delta_{\mc M}^{\frac 12}A\Omega=\fk J_{\mc M}D_{\mc M}^{\frac 12}A\Omega
\end{align*}
Therefore, $J$ and $\fk J_{\mc M}$ agree on the dense space $D^{\frac 12}\Dom_0$. Hence $J=\fk J_{\mc M}$. In particular, $J$ is antiunitary. We have proved the existence of $J$ satisfying \eqref{eqb37}. The uniqueness also follows from the density of $D^{\frac 12}\Dom_0$.
\end{proof}





\subsection{}

Assume the setting of Thm. \ref{lbb26}. By Thm. \ref{lb56}, for each $A,B\in\scr A$, the function $\Rbb\rightarrow\mc H$ defined by $t\mapsto D^{-\im t/2}A\Omega$ resp. $t\mapsto D^{\im t/2}B\Omega$ can be extended to a continuous function on
\begin{align*}
\fk I=\{\tau\in\Cbb:0\leq\Imag \tau\leq 1\}
\end{align*}
and holomorphic resp. antiholomorphic on $\Int\fk I$, namely, it can be extended to $\tau\mapsto D^{-\im\tau/2}A\Omega$ resp. $\tau\mapsto D^{\im\ovl\tau/2}B\Omega$. Therefore, noting that $\bk{D^{-\im t/2}A\Omega|D^{\im t/2}B\Omega}$ equals $\bk{A\Omega|D^{\im t}B\Omega}$, we see that  condition \eqref{eqb40} is equivalent to the well-known \textbf{KMS condition}: \index{00@KMS condition}

For each $t\in\Rbb$, let $\sigma_t:\scr A\rightarrow\scr A$ be defined by $\sigma_t(X)=D^{\im t}XD^{-\im t}$. Let $A,B\in\scr A$. Then the function
\begin{align*}
t\in\Rbb\mapsto \bk{A\Omega|\sigma_t(B)\Omega}
\end{align*}
can be extended to a continuous function $f:\fk I\rightarrow\Cbb$ and holomorphic on $\fk I$ such that  $f(\im)=\bk{B^*\Omega|A^*\Omega}$.  




\newpage



\section{Bisognano-Wichmann theorem for strongly local chiral algebras}




We fix an energy-bounded chiral algebra $(\mc V,\Vbb)$. Let $\mc J$ be the set of all intervals of $\Sbb^1$. For each $I\in\mc J$, we let\index{I@$I'=\Int(\Sbb^1\setminus I)$}
\begin{align*}
I'=\Int(\Sbb^1\setminus I)
\end{align*}
Define a von Neumann algebra on $\HV$ to be
\begin{align*}
\fk A(I)=\{\ovl{A(f)}:A\in\mc V,I\in\mc J\}''
\end{align*}
We call $\fk A=(\fk A(I))_{I\in\mc J}$ the \textbf{net of von Neumann algebras associated to $(\mc V,\Vbb)$}.






\subsection{}


\begin{df}
Assume that $\mc M\curvearrowright\HV$ is a von Neumann algebra. Define
\begin{align*}
\mc M_\infty=\{x\in\mc M:x\HV^\infty\subset\HV^\infty,x^*\HV^\infty\subset\HV^\infty\}
\end{align*}
which is clearly a unital $*$-subalgebra of $\mc M$. Elements of $\mc M_\infty$ are called \textbf{smooth}.
\end{df}


\begin{rem}\label{lb96}
From the definition of $\fk A(I)$, we have
\begin{align*}
\fk A(I)=\bigg(\bigcup_{J\Subset I}\fk A(J)\bigg)''
\end{align*}
where $J\Subset I$ means $J\in\mc J$ and $\ovl J\subset I$. This is because any $A(f)$ (where $A\in\mc V,f\in C_c^\infty(I)$) is affiliated with $\fk A(J)$ for any $J\Subset I$ containing the support of $f$.
\end{rem}


\begin{pp}\label{lb97}
For each $I\in\mc J$, then $(\fk A(I)_\infty)''=\fk A(I)$.
\end{pp}


\begin{proof}
By Rem. \ref{lb96}, any element of $\fk A(I)$ is approximated strongly by elements of $\bigcup_{J\Subset I}\fk A(I)$. Therefore, it suffices to prove that for each $J\Subset I$, any $x\in\fk A(J)$ can be approximated strongly by elements of $\fk A(I)$. 


We first note that by the rotation covariance Cor. \ref{lb67}, we have
\begin{align}\label{eq126}
e^{\im t\ovl{L_0}}\fk A(J)e^{-\im t\ovl{L_0}}=\fk A(\exp^{\im l_0}_tJ)
\end{align}
Therefore, we have $e^{\im t\ovl{L_0}}xe^{-\im t\ovl{L_0}}\in\fk A(I)$ for sufficiently small $|t|$. Now choose any $h\in C_c^\infty(\Rbb)$ such that $\int_\Rbb h=1$. Let $h_r(x)=rh(rx)$. Then for sufficiently large $r$, the element $x(h_r)$ (cf. Lem. \ref{lb64}) converges belongs to $\fk A(I)$. Moreover, by Lem. \ref{lb64}, each $x(h_r)$ is smooth, and $\lim_{r\rightarrow+\infty} x(h_r)$ converges strongly to $x$.
\end{proof}




\subsection{}




\begin{pp}\label{lb98}
Let $I\in\mc J$. Let $A^1,\dots,A^n\in\mc V$ and $f_1,\dots,f_n\in C_c^\infty(I)$. Let $B$ be a polynomial of $A^1(f_1),\dots,A^n(f_n)$ (with dense invariant domain $\HV^\infty$). Then the closure $\ovl B$ is affiliated with $\fk A(I)$.
\end{pp}


\begin{proof}
Choose any $y\in \fk A(I)'$. We want to show that $y$ commutes strongly with $\ovl B$. Choose $J\Subset I$ containing the supports of $f_1,\dots,f_n$. By \eqref{eq126}, for sufficiently small $|t|$ we have $[\fk A(J),e^{\im t\ovl{L_0}}ye^{-\im t\ovl{L_0}}]=0$. Therefore, if we let $h_r$ be as in the proof of Prop. \ref{lb97}, then one can find $\delta>0$ and $r_0\geq0$ such that for all $-\delta<t<\delta$ and $r\geq r_0$ we have
\begin{align*}
[e^{\im t\ovl{L_0}}\fk A(J)e^{-\im t\ovl{L_0}},y(h_r)]=0
\end{align*}
and hence $e^{\im t\ovl{L_0}}A^i(f_i)e^{-\im t\ovl{L_0}}$ commutes strongly with $y(h_r)$ for all $1\leq i\leq n$. By Lem. \ref{lb64}, $x(h_r)$ is smooth. Therefore, by Prop. \ref{lb68}, $\ovl B$ commutes strongly with $y(h_r)$ for all $r\geq r_0$. Letting $r\rightarrow+\infty$, we see that $\ovl B$ commutes strongly with $y$.
\end{proof}



\subsection{}




\begin{df}
We say that $\mc V$ is \textbf{strongly local} \index{00@Strongly local chiral algebras} if $[\fk A(I),\fk A(J)]=0$ whenever $I\cap J=\emptyset$, equivalently, if $\ovl{A(f)}$ commutes strongly with $\ovl B(g)$ whenever $I,J\in\mc J$ are disjoint and $f\in C_c^\infty(I),g\in C_c^\infty(J)$.
\end{df}


\begin{eg}
By Thm. \ref{lb108}, if every field of $\mc V$ is linearly energy-bounded, then $\MV$ is strongly local. For example, by Cor. \ref{lb43} and Cor. \ref{lb109}, if $\MV$ has only the Virasoro field, or if $\MV$ is a current algebra, then $\MV$ is linearly energy bounded, and hence strongly local.
\end{eg}



\begin{thm}\label{lb99}
Assume that $\MV$ is strongly local. Then $\fk A$ satisfies the following properties for all $I,J\in\mc J$.
\begin{enumerate}[label=(\alph*)]
\item Isotony: If $I\subset J$ then $\fk A(I)\subset\fk A(J)$.
\item Locality: If $I\cap J$ then $[\fk A(I),\fk A(J)]=0$.
\item M\"obius covariance: For each $X=\eqref{eq83}$ satisfying $X^\dagger=X$, and for each $t\in\Rbb$, we have $e^{\im t\ovl{L_X}}\Omega=\Omega$ and
\begin{align*}
e^{\im t\ovl{L_X}}\fk A(I)e^{-\im t\ovl{L_X}}=\fk A(\exp^{\im X}_t I)
\end{align*}
\item Positive energy: We have $\ovl{L_0}\geq0$.
\item Reeh-Schlieder: The vacuum vector $\Omega$ is a cylic vector of $\fk A(I)$.
\end{enumerate}
\end{thm}

It follows that $\Omega$ is also separating under $\fk A(I)$, because $\fk A(I)'\supset \fk A(I')$ (by locality) and $\Omega$ is $\fk A(I')$-cyclic. Therefore, the Tomita-Takesaki theory can be applied to the pair $(\fk A(I),\Omega)$.

\begin{proof}
The M\"obius covariance follows from Thm. \ref{lb65}. It remains to prove the Reeh-Schlieder property, since the other are obvious. By Prop. \ref{lb98}, the closure of any polynomial $B$ of smeared operators supported in $I$ is affiliated with $\fk A(I)$. By Thm. \ref{lb70}, vectors of the form $B\Omega$ form a dense subspace of $\HV$. Let $\ovl B=HU$ be the right polar decomposition of $B$, and let $E_\lambda=\chi_{[0,\lambda]}(H)$. Then $E_\lambda HU\in\fk A(I)$, and $\lim_{\lambda\rightarrow+\infty}E_\lambda HU\Omega=B\Omega$. Thus $B\Omega$ can be approximated by elements of $\fk A(I)\Omega$. Thus $\fk A(I)\Omega$ is dense.
\end{proof}


\begin{co}
$\MV$ is strongly local iff $[\fk A(\Sbb^1_+),\fk A(\Sbb^1_-)]=0$.
\end{co}

\begin{proof}
``$\Rightarrow$" is obvious. Conversely, choose any $I\in\mc J$. Let $K$ be the dilation generator. Then there exists $s,t\in\Rbb$ such that $\exp^{\im l_0}s\exp^{\im tK}_tI=\Sbb^1_+$. Therefore, the M\"obius covariance transforms $[\fk A(\Sbb^1_+),\fk A(\Sbb^1_-)]=0$ to $[\mc A(I),\mc A(I')]=0$. This proves ``$\Leftarrow$".
\end{proof}



\subsection{}


The five properties in Thm. \ref{lb99} is very close to the definition of a \textbf{M\"obius covariant net}. The latter means a collection of von Neumann algebras $(\mc A(I))_{I\in\mc J}$ on a common Hilbert space $\mc H$, together with a distinguished unit vector $\Omega\in\mc H$ satisfying (a)-(e) of Thm. \ref{lb99}, except that (c) and (e) are replaced by the following:
\begin{enumerate}
\item[(c')] M\"obius covariance: A strongly continuous unitary representation $\mc U$ of $\PSU$ on $\mc H$ is equipped such that $\Omega$ is fixed by $\PSU$, and that $\mc U(g)\mc A(I)\mc U(g)*=\mc A(gI)$ for all $g\in\PSU$ and $I\in\mc I$.
\item[(e')] $\bigvee_{I\in\mc J}\mc A(I)\Omega$ is dense in $\mc H$. 
\end{enumerate}
In fact, in the context of Thm. \ref{lb99}, one use the results in \cite{Tol99} to show that there is a strongly continuous unitary representation $\mc U$ of $\PSU$ on $\mc H$ such that $\mc U(e^{\im X}_t)=e^{\im tX}$ for all $t\in\Rbb$ and $X$ satisfying $X=X^\dagger$. Then (c') follows from (c). On the other hand, (e) clearly implies (e'). But one can also use the method in the proof of Thm. \ref{lb70} to prove that (e') implies (e).




\subsection{}

From now on, we let $K$ be the dilation generator \eqref{eq127}.


\begin{thm}[\textbf{Bisognano-Wichmann}]\label{lb102}\index{00@Bisognano Wichmann theorem for chiral algebras}
Assume that $\MV$ is strongly local. Let $S=\fk J\Delta^{\frac 12}$ be the modular operators of $\mc M:=\fk A(\Sbb^1_+)$ with respect to $\Omega$. Then
\begin{align}\label{eq129}
\fk J=\Theta\qquad \Delta^{\frac 12}=e^{-\pi \ovl{L_K}}
\end{align}
Moreover, for all $t\in\Rbb$ we have $\Delta^{\im t}\mc M\Delta^{-\im t}=\mc M$ and $\fk J\mc M\fk J=\mc M'$.
\end{thm}

Note that the assumption on strong locality cannot be removed. Otherwise, one cannot show that $\Omega$ is $\mc M$-separating.


\begin{proof}
Let $\scr A$ be the set of all $\ovl B$ where $B=A^1(f_1),\dots,A^k(f_k)$ for some $k\in\Nbb$, $A^1,\dots,A^k\in\MV$, and $f_1,\dots,f_k$ are supported in \textit{mutually disjoint} intervals of $\Sbb^1_+$. By Thm. \ref{lb98}, we have $\mc M=\scr A''$. Let us check that the conditions in the modular uniqueness Thm. \ref{lbb26} are satisfied for $D^{\frac 12}=e^{-\pi\ovl{L_K}}$ and $\Theta$. 

Clearly $L_K\Omega=0$. For each $t\in\Rbb$, by the M\"obius covariance Thm. \ref{lb65} (and noting that the dilation group fixes $\Sbb^1_+$),  $e^{\im t\ovl{L_X}} Be^{-\im t\ovl{L_X}}$ is a polynomial of smeared fields supported in mutually disjoint intervals of $\Sbb^1_+$. Therefore, since $U\ovl B U^*=\ovl{UBU^*}$ for any unitary $U$, we get $e^{\im t\ovl{L_X}}\scr Ae^{-\im t\ovl{L_X}}\subset\scr A$. Clearly $\Omega\in\Dom(A)\cap\Dom(A^*)$ for any $A\in\scr A$ (cf. \eqref{eq128}). By the Reeh-Schlieder Thm. \ref{lb70}, $\Span_\Cbb\scr A\Omega$ is dense. By the PT Thm. \ref{lb85} and the PCT Thm. \ref{lb89} (or more precisely, its consequence \eqref{eq121b}), for each $A,B\in\scr A$ we have
\begin{align*}
\bk{\Theta e^{-\pi\ovl{L_K}}A\Omega|\Theta e^{-\pi\ovl{L_K}}B\Omega}=\bk{A^*\Omega|B^*\Omega}
\end{align*}
noting that we can switch the order of products of smeared fields supported in mutually disjoint intervals (Cor. \ref{lb66}). 

Now we can use \ref{lbb26}, which tells us that \eqref{eq129} is true. Since we have proved $e^{\im t\ovl{L_X}}\scr Ae^{-\im t\ovl{L_X}}\subset\scr A$, we get $e^{\im t\ovl{L_X}}\fk A(\Sbb^1_+)e^{-\im t\ovl{L_X}}\subset\fk A(\Sbb^1_+)$ for all $t\in\Rbb$. Replacing $t$ with $-t$, we see that $e^{\im t\ovl{L_X}}\fk A(\Sbb^1_+)e^{-\im t\ovl{L_X}}=\fk A(\Sbb^1_+)$. Finally, the PCT Thm. \ref{lb89} shows that $\Theta\scr A\Theta^{-1}$ is the closure of a product of smeared fields supported in $\Sbb^1_-$ (since $\delta_{\im\pi}\Sbb^1_+=\Sbb^1_-$). Therefore 
\begin{align}\label{eq130}
\Theta\fk A(\Sbb^1_+)\Theta^{-1}\subset\fk A(\Sbb^1_-)\subset\fk A(\Sbb^1_+)'
\end{align}
where the second inclusion is due to the strong locality of $\MV$. Since $\Theta=\fk J$, by Thm. \ref{lb100}, we see that $\Theta\fk A(\Sbb^1_+)\Theta^{-1}=\fk A(\Sbb^1_+)'$. Thus $\fk J\mc M\fk J=\mc M'$.
\end{proof}



\begin{co}[\textbf{Haag duality}]\label{lb103}
Assume that $\MV$ is strongly local. Then for each $I\in\mc J$ we have $\fk A(I)'=\fk A(I')$.
\end{co}

\begin{proof}
The proof of Thm. \ref{lb102} shows that the three terms of \eqref{eq130} are equal. Therefore $\fk A(I)'=\fk A(I')$ holds when $I=\Sbb^1_+$. By the M\"obius covariance in Thm. \ref{lb99} we have $\fk A(I)'=\fk A(I')$ for all $I$.
\end{proof}




\subsection{}


The Haag duality is the most fundamental property required in the construction of unitary braided tensor category $\Rep(\fk A)$ of (normal) representations of $\fk A$. Therefore, the Haag duality is the cornerstone of the AQFT approach to quantum symmetries, low dimensional topologies, etc.

The construction of $\Rep(\fk A)$ is originally due to \cite{FRS89,FRS92}. These works use the language of superselection theory and rely essentially on the fact that each $\fk A(I)$ is a type III factor. An alternative construction of $\Rep(\fk A)$ using Connes' relative tensor products is given \cite{Gui21} and does not rely essentially on the fact that $\fk A(I)$ is type III or that $\fk A(I)$ is a factor. (The equivalence of the two approaches is also proved in Ch. 6 of \cite{Gui21}.)

Since this topic is beyond the scope of this course, we will not say more about it. Instead, we give another interesting application of the Bisognano-Wichmann theorem due to \cite{CKLW18}, namely Cor. \ref{lb107}. We first need a preparation: The following theorems says that strong commutativity implies a version of ordinary commutativity.


\begin{thm}\label{lb104}
Let $A,B$ be strongly commuting closed operators on $\mc H$. Assume $\xi\in\Dom(BA)\cap\Dom(B)$. Then $\xi\in\Dom(AB)$ and $AB\xi=BA\xi$.
\end{thm}


\begin{proof}
Let $B=UH$ be the left polar decomposition of $B$. Let $E_r=\chi_{[0,r]}(H)$ and $F_r=UF_rU^*$. Then $F_rB\subset BE_r$, and they are bounded operators in $\{B\}''$ and hence commute strongly with $A$. So $BE_rA\subset ABE_r$. In particular, $BE_r\Dom(A)\subset\Dom(A)$. 

Choose $\xi\in\Dom(BA)\cap\Dom(B)$. In particular $\xi\in\Dom(A)$. Hence $BE_r\xi\in\Dom(A)$. As $r\rightarrow+\infty$, we have $BE_r\xi=F_rB\xi\rightarrow B\xi$ (since $\xi\in\Dom(B)$) and $ABE_r\xi=BE_rA\xi=F_rBA\xi\rightarrow BA\xi$ (since $\xi\in\Dom(BA)$). We see that $(BE_r\xi,ABE_r\xi)$ approaches $(B\xi,BA\xi)$. Since $A$ is closed, $(B\xi,BA\xi)$ must be on the graph of $A$. So $B\xi\in\Dom(A)$ and $AB\xi=BS\xi$.
\end{proof}




\subsection{}



\begin{thm}\label{lb106}
Assume that $\MV$ is strongly local. Let $I\in\mc J$. Let $T$ be an unbounded operator on $\HV$ with invariant domain $\HV^\infty$ satisfying the following conditions:
\begin{enumerate}[label=(\arabic*)]
\item For each $B\in\MV$ and $g\in C_c^\infty(I')$ we have $[T,B(g)]=0$ on $\HV^\infty$.
\item There exist $M,r\geq0$ such that $\Vert T\xi\Vert\leq M\Vert(1+\ovl{L_0})^r\xi\Vert$ for all $\xi\in\HV^\infty$. 
\item $\Dom(T^*)$ contains $\Omega$.
\end{enumerate}
Then $T$ is closable, and $\ovl T$ is affiliated with $\fk A(I)$.
\end{thm}

\begin{proof}
Step 1. By the M\"obius covariance Thm. \ref{lb65} it suffices to assume that $I=\Sbb^1_+$. Let $S=\fk J\Delta^{\frac 12}$ be the modular operators of $(\fk A(\Sbb^1_+),\Omega)$. In this step, we prove that $T\Omega\in\Dom(S)$. 

Let $B$ be a product of smeared fields supported in $\Sbb^1_-$. By \eqref{eq128}, we have $B^*\supset B^\dagger$ and hence $\Omega\in\Dom(B^*)$. By Prop. \ref{lb98}, $\ovl B$ is affiliated with $\fk A(\Sbb^1_-)$, and hence affiliate with $\fk A(\Sbb^1_+)'$. Therefore, by Prop. \ref{lba14} and $F=S^*$, we have $S^*B\Omega=B^*\Omega=B^\dagger\Omega$ where $B^\dagger=B^*|_{\HV^\infty}$. By (1), we have $[T,B^\dagger]=0$ because $B^\dagger$ is also a product of smeared field supported in $\Sbb^1_-$. Thus
\begin{align*}
\bk{T\Omega|S^*B\Omega}=\bk{T\Omega|B^\dagger\Omega}=\bk{BT\Omega|\Omega}=\bk{TB\Omega|\Omega}=\bk{B\Omega|T^*\Omega}
\end{align*}
Let $\mc W$ be the subspace spanned by all vectors of the form $B\Omega$. Then $\mc W$ is dense by the Reeh-Schlieder Thm. \ref{lb70}. We claim that $\mc W$ is a core for $S$. Suppose this is true, then the above computation shows that $T\Omega\in\Dom(S)$ (and $ST\Omega=T^*\Omega$).


By the Bisognano-Wichmann Thm. \ref{lb102}, we have $\Delta^{\frac 12}=e^{-\pi\ovl{L_K}}$. By the M\"obius covariance Thm. \ref{lb65}, for all $t\in\Rbb$ we have $e^{\im t\ovl{L_K}}\mc W\subset\mc W$, and hence $\Delta^{\im t}\mc W\subset\mc W$. Therefore, by Thm. \ref{lb58}, $\mc W$ is a core for $\Delta^{\frac 12}$ and hence for $S$.\\[-1ex]

Step 2. Let $\xi=T\Omega$. Since $\xi\in\Dom(S)$, by Thm. \ref{lba12}, $\scr L(\xi)=\scr L_{\fk A(\Sbb^+)}(\xi)$ is closed. By Thm. \ref{lba13}, $\scr L(\xi)$ is affiliated with $\fk A(\Sbb^1_+)$. It remains to prove that $T$ is closable and $\ovl T=\scr L(\xi)$.

For each $B$ as in Step 1, by (1) we have $[T,B]=0$, and hence
\begin{align*}
TB\Omega=BT\Omega=B\xi
\end{align*}
One the other hand, since (by Prop. \ref{lb98}) $\ovl B$ is affiliated with $\fk A(\Sbb^1_-)$, it commutes strongly with $\scr L(\xi)$. Since $\xi=\scr L(\xi)\Omega\in\HV^\infty=\Dom(B)$ we have $\Omega\in\Dom(\ovl B\scr L(\xi))$. Since $\Omega\in\Dom(\ovl B)$, by Thm. \ref{lb104} we have $\Omega\in\Dom(\scr L(\xi)\ovl B)$ and $\scr L(\xi)\ovl B\Omega=\ovl B\scr L(\xi)\Omega$, and hence 
\begin{align*}
\scr L(\xi)\ovl B\Omega=\ovl B\xi
\end{align*}
Since $\mc W$ is spanned by all such $B\Omega$, we conclude that $\mc W\subset\Dom(\scr L(\xi))$ and
\begin{align*}
T|_{\mc W}=\scr L(\xi)|_{\mc W}
\end{align*}
By Cor. \ref{lb88} (or by its proof), $\mc W$ is a core for $(1+\ovl{L_0})^r$ for each $r\geq0$. Therefore, by condition (2), $\mc W$ is a core for $T$. We conclude that $T\subset \scr L(\xi)$. Therefore $T$ is closable (because $\scr L(\xi)$ is closed), and $\ovl T\subset\scr L(\xi)$.




Note that $\fk A(\Sbb^1_-)_\infty\Omega$ is contained in $\HV^\infty=\Dom(T)$. Suppose we can prove that $\fk A(\Sbb^1_-)_\infty\Omega$ is a core for $\scr L(\xi)$, then by $\ovl T\subset\scr L(\xi)$, we must have $\ovl T=\scr L(\xi)$, finishing the proof.

Note that $\scr L(\xi)$, by definition, has core $\fk A(\Sbb^1_+)'\Omega$. By the Haag duality Cor. \ref{lb103}, $\scr L(\xi)$ has core $\fk A(\Sbb^1_-)\Omega$. Therefore, it suffices to prove that $\fk A(\Sbb^1_-)_\infty\Omega$ is a core for $\scr L(\xi)\big|_{\fk A(\Sbb^1_-)\Omega}$. Choose any $y\in\fk A(\Sbb^1_-)$. By Prop. \ref{lb97}, there is a net $(y_\alpha)$ in $\fk A(\Sbb^1_-)_\infty$ converging strongly to $y$. Therefore, we have $\lim_\alpha y_\alpha\Omega=\Omega$ and
\begin{align*}
\lim_\alpha\scr L(\xi)y_\alpha\Omega=\lim_\alpha y_\alpha \scr L(\xi)\Omega=\lim_\alpha y_\alpha\xi=y\xi=\scr L(\xi)y\Omega
\end{align*}
This shows that $\fk A(\Sbb^1_-)_\infty\Omega$ is a core for $\scr L(\xi)|_{\fk A(\Sbb^1_-)\Omega}$.
\end{proof}


\subsection{}

The following Corollary is \cite[Thm. 8.1]{CKLW18}.

\begin{co}\label{lb107}
Assume that $\MV$ is strongly local. Let $k\in\Nbb$, $A^1,\dots,A^{k+1}\in\MV$, and $n_1,\dots,n_k\in\Zbb$. Let
\begin{align}\label{eq132}
A=A^1_{n_1}\cdots A^k_{n_k}A^{k+1}
\end{align}
where the order of product is from right to left. Recall that $A$ is energy-bounded by Thm. \ref{lb29} (Dong-Li's lemma) and Prop. \ref{lb105}. Then for each $I\in\mc J$ and $f\in C_c^\infty(I)$, the operator $\ovl{A(f)}$ is affiliated with $\fk A(I)$.
\end{co}

The meaning of ``from right to left" is, for example, $A=A^1_{n_1}(A^2_{n_2}(A^3_{n_3}A^4))$ when $k=3$. 


\begin{proof}
Let use check that $T=A(f)$ satisfies the three conditions of Thm. \ref{lb106}. (3) is obvious, since $T^*\supset A(f)^\dagger=\eqref{eq131}$. By Thm. \ref{lb29}, $A$ is local to any field $B\in\MV$. Therefore (1) follows from Cor. \ref{lb66}. Finally, (2) follows from Thm. \ref{lb39}.
\end{proof}




Cor. \ref{lb107} provides a useful tool for constructing new strongly local chiral algebras from old ones: Suppose we know that $\MV$ is strongly local. Then we use some quasiprimary fields of the form \eqref{eq132} to define a new chiral algebra $\mc U$. Then by Cor. \ref{lb107}, $\mc U$ is also (energy-bounded and) strongly local.



































%\hypertarget{beforeindex}{}	




\newpage



\printindex


\begin{thebibliography}{999999}
\footnotesize	

\bibitem[Apo]{Apo}
Apostol, Tom (1974), Mathematical analysis, second edition, Addison–Wesley.


\bibitem[BS90]{BS90}
Buchholz, D., \& Schulz-Mirbach, H. (1990). Haag duality in conformal quantum field theory. Reviews in Mathematical Physics, 2(01), 105-125.


\bibitem[CKLW18]{CKLW18}
Carpi, S., Kawahigashi, Y., Longo, R., \& Weiner, M. (2018). From vertex operator algebras to conformal nets and back (Vol. 254, No. 1213). American Mathematical Society.


\bibitem[FL74]{FL74}
Faris, W. G., \& Lavine, R. B. (1974). Commutators and self-adjointness of Hamiltonian operators. Communications in Mathematical Physics, 35, 39-48.

\bibitem[FRS89]{FRS89}
Fredenhagen, K., Rehren, K.H. and Schroer, B., 1989. Superselection sectors with braid group statistics and exchange algebras. Communications in Mathematical Physics, 125(2), pp.201-226.


\bibitem[FRS92]{FRS92}
Fredenhagen, K., Rehren, K.H. and Schroer, B., 1992. Superselection sectors with braid group statistics and exchange algebras II: Geometric aspects and conformal covariance. Reviews in Mathematical Physics, 4(spec01), pp.113-157.


\bibitem[GJ]{GJ}
Glimm, J., \& Jaffe, A. (1981). Quantum physics: a functional integral point of view. Springer Science \& Business Media.


\bibitem[Gui-S]{Gui-S}
Gui, B. (2021). Spectral Theory for Strongly Commuting Normal Closed Operators. See \href{https://binguimath.github.io/}{https://binguimath.github.io/}

\bibitem[Gui-V]{Gui-V}
Gui, B. (2022). Lectures on Vertex Operator Algebras and Conformal Blocks. See \href{https://binguimath.github.io/}{https://binguimath.github.io/}

\bibitem[Gui21]{Gui21}
Gui, B. (2021). Categorical extensions of conformal nets. Communications in Mathematical Physics, 383, 763-839.


\bibitem[Haag]{Haag}
Haag, G. Local quantum physics. Fields, particles, algebras. 2nd., rev. and enlarged ed. Berlin: Springer-Verlag (1996)

\bibitem[Tol99]{Tol99}
Toledano-Laredo, V. (1999). Integrating unitary representations of infinite-dimensional Lie groups. Journal of functional analysis, 161(2), 478-508.

\bibitem[Was-10]{Was-10}
Wassermann, A. (2010). Kac-Moody and Virasoro algebras. arXiv preprint arXiv:1004.1287.







\end{thebibliography}
\noindent {\small \sc Yau Mathematical Sciences Center, Tsinghua University, Beijing, China.}

\noindent {\textit{E-mail}}: binguimath@gmail.com







\end{document}




