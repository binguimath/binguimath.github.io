% !TeX spellcheck = en_US
% !TEX program = pdflatex
\documentclass[12pt,b5paper,notitlepage]{article}
\usepackage[b5paper, margin={0.5in,0.65in}]{geometry}
%\usepackage{fullpage}
\usepackage{amsmath,amscd,amssymb,amsthm,mathrsfs,amsfonts,layout,indentfirst,graphicx,caption,mathabx, stmaryrd,appendix,calc,imakeidx,upgreek} % mathabx for \wtidecheck
%\usepackage{ulem} %wave underline
\usepackage[dvipsnames]{xcolor}
\usepackage{palatino}  %template

\usepackage{slashed} % Dirac operator
\usepackage{mathrsfs} % Enable using \mathscr
%\usepackage{eufrak}  another template/font
\usepackage{extarrows} % long equal sign, \xlongequal{blablabla}
\usepackage{enumitem} % enumerate label change e.g. [label=(\alph*)]  shows (a) (b) 


%%%%%%%%%%%%%%%%%%%%%%%%%%%%%%

%\usepackage{fontspec}
%\setmainfont{Palatino Linotype}
%\usepackage{emoji}


% emoji, use lualatex  remove \usepackage{palatino}

%%%%%%%%%%%%%


\usepackage{CJK}   % Chinese package





\usepackage{csquotes} % \begin{displayquote}   \begin{displaycquote}  for quotation
\usepackage{epigraph}   %\epigraph{}{}  for quotation
%\pmb  mandatory math bold 

\usepackage{fancyhdr} % date in footer

%\usepackage{soul}  %\ul underline break line automatically

\usepackage{ulem}  % \uline  underline break line   also    \uwave

\usepackage{relsize} % use \mathlarger \larger \text{\larger[2]$...$} to enlarge the size of math symbols

\usepackage{verbatim}  % comment environment


\usepackage{halloweenmath} % Interesting halloween math symbols

%%%%%%%%%%%%%%%%%%%%%%%%%%%%%%
\usepackage{tcolorbox}
\tcbuselibrary{theorems}
% box around equations   \tcboxmath
%%%%%%%%%%%%%%%%%%%%%%%%%%%%%%%%%%





%%%%%%%%%%%%%%%%%%%%%%%%%%%%%
% circled colon and thick colon \hcolondel and \colondel

\usepackage{pdfrender}

\newcommand*{\hollowcolon}{%
	\textpdfrender{
		TextRenderingMode=Stroke,
		LineWidth=.1bp,
	}{:}%
}

\newcommand{\hcolondel}[1]{%
	\mathopen{\hollowcolon}#1\mathclose{\hollowcolon}%
}
\newcommand{\colondel}[1]{%
	\mathopen{:}#1\mathclose{:}%
}

%%%%%%%%%%%%%%%%%%%%%%%%%%%%%%%%






\usepackage{tikz}
\usetikzlibrary{fadings}
\usetikzlibrary{patterns}
\usetikzlibrary{shadows.blur}
\usetikzlibrary{shapes}

\usepackage{tikz-cd}
\usepackage[nottoc]{tocbibind}   % Add  reference to ToC


\makeindex


% The following set up the line spaces between items in \thebibliography
\usepackage{lipsum}  
\let\OLDthebibliography\thebibliography
\renewcommand\thebibliography[1]{
	\OLDthebibliography{#1}
	\setlength{\parskip}{0pt}
	\setlength{\itemsep}{2pt} 
}


%\hyperref{page.10}{...}

\allowdisplaybreaks  %allow aligns to break between pages
\usepackage{latexsym}
\usepackage{chngcntr}
\usepackage[colorlinks,linkcolor=blue,anchorcolor=blue, linktocpage,
%pagebackref
]{hyperref}
\hypersetup{ urlcolor=cyan,
	citecolor=[rgb]{0,0.5,0}}


\setcounter{tocdepth}{1}	 %hide subsections in the content


\counterwithin{figure}{section}

%\counterwithin*{footnote}{section}   %% Footnote numbering is recounted from the beginning of each subsection



\pagestyle{plain}

\captionsetup[figure]
{
	labelsep=none	
}













\theoremstyle{definition}
\newtheorem{df}{Definition}[section]
\newtheorem{eg}[df]{Example}
\newtheorem{exe}[df]{Exercise}
\newtheorem{rem}[df]{Remark}
\newtheorem{obs}[df]{Observation}
\newtheorem{ass}[df]{Assumption}
\newtheorem{cv}[df]{Convention}
\newtheorem{prin}[df]{Principle}
\newtheorem{nota}[df]{Notation}
\newtheorem*{axiom}{Axiom}
\newtheorem{coa}[df]{Theorem}
\newtheorem{srem}[df]{$\star$ Remark}
\newtheorem{seg}[df]{$\star$ Example}
\newtheorem{sexe}[df]{$\star$ Exercise}
\newtheorem{sdf}[df]{$\star$ Definition}
\newtheorem{concl}[df]{Conclusion}



\newtheorem{prob}{\color{red}Problem}[section]
%\renewcommand*{\theprob}{{\color{red}\arabic{section}.\arabic{prob}}}
\newtheorem{sprob}[prob]{\color{red}$\star$ Problem}
%\renewcommand*{\thesprob}{{\color{red}\arabic{section}.\arabic{sprob}}}
% \newtheorem{ssprob}[prob]{$\star\star$ Problem}



\theoremstyle{plain}
\newtheorem{thm}[df]{Theorem}
\newtheorem{ccl}[df]{Conclusion}
\newtheorem{thd}[df]{Theorem-Definition}
\newtheorem{pp}[df]{Proposition}
\newtheorem{co}[df]{Corollary}
\newtheorem{lm}[df]{Lemma}
\newtheorem{sthm}[df]{$\star$ Theorem}
\newtheorem{slm}[df]{$\star$ Lemma}
\newtheorem{claim}[df]{Claim}
\newtheorem{spp}[df]{$\star$ Proposition}
\newtheorem{scorollary}[df]{$\star$ Corollary}


\newtheorem{cond}{Condition}
\newtheorem{Mthm}{Main Theorem}
\renewcommand{\thecond}{\Alph{cond}} % "letter-numbered" theorems
\renewcommand{\theMthm}{\Alph{Mthm}} % "letter-numbered" theorems


%\substack   multiple lines under sum
%\underset{b}{a}   b is under a


% Remind: \overline{L_0}



\usepackage{calligra}
\DeclareMathOperator{\shom}{\mathscr{H}\text{\kern -3pt {\calligra\large om}}\,}
\DeclareMathOperator{\sext}{\mathscr{E}\text{\kern -3pt {\calligra\large xt}}\,}
\DeclareMathOperator{\Rel}{\mathscr{R}\text{\kern -3pt {\calligra\large el}~}\,}
\DeclareMathOperator{\sann}{\mathscr{A}\text{\kern -3pt {\calligra\large nn}}\,}
\DeclareMathOperator{\send}{\mathscr{E}\text{\kern -3pt {\calligra\large nd}}\,}
\DeclareMathOperator{\stor}{\mathscr{T}\text{\kern -3pt {\calligra\large or}}\,}


\usepackage{aurical}
\DeclareMathOperator{\VVir}{\text{\Fontlukas V}\text{\kern -0pt {\Fontlukas\large ir}}\,}

\newcommand{\vol}{\text{\Fontlukas V}}
\newcommand{\dvol}{d~\text{\Fontlukas V}}

\usepackage{aurical}
\usepackage[T1]{fontenc}








\newcommand{\fk}{\mathfrak}
\newcommand{\mc}{\mathcal}
\newcommand{\wtd}{\widetilde}
\newcommand{\wht}{\widehat}
\newcommand{\wch}{\widecheck}
\newcommand{\ovl}{\overline}
\newcommand{\udl}{\underline}
\newcommand{\tr}{\mathrm{t}} %transpose
\newcommand{\Tr}{\mathrm{Tr}}
\newcommand{\End}{\mathrm{End}} %endomorphism
\newcommand{\idt}{\mathbf{1}}
\newcommand{\id}{\mathrm{id}}
\newcommand{\Hom}{\mathrm{Hom}}
\newcommand{\Conf}{\mathrm{Conf}}
\newcommand{\Res}{\mathrm{Res}}
\newcommand{\res}{\mathrm{res}}
\newcommand{\KZ}{\mathrm{KZ}}
\newcommand{\ev}{\mathrm{ev}}
\newcommand{\coev}{\mathrm{coev}}
\newcommand{\opp}{\mathrm{opp}}
\newcommand{\Rep}{\mathrm{Rep}}
\newcommand{\diag}{\mathrm{diag}}
\newcommand{\Dom}{\scr{D}}
\newcommand{\loc}{\mathrm{loc}}
\newcommand{\con}{\mathrm{c}}
\newcommand{\uni}{\mathrm{u}}
\newcommand{\ssp}{\mathrm{ss}}
\newcommand{\di}{\slashed d}
\newcommand{\Diffp}{\mathrm{Diff}^+}
\newcommand{\Diff}{\mathrm{Diff}}
\newcommand{\PSU}{\mathrm{PSU}(1,1)}
\newcommand{\Vir}{\mathrm{Vir}}
\newcommand{\Witt}{\mathscr W}
\newcommand{\Span}{\mathrm{Span}}
\newcommand{\pri}{\mathrm{p}}
\newcommand{\ER}{E^1(V)_{\mathbb R}}
\newcommand{\prth}[1]{( {#1})}
\newcommand{\bk}[1]{\langle {#1}\rangle}
\newcommand{\bigbk}[1]{\big\langle {#1}\big\rangle}
\newcommand{\Bigbk}[1]{\Big\langle {#1}\Big\rangle}
\newcommand{\biggbk}[1]{\bigg\langle {#1}\bigg\rangle}
\newcommand{\Biggbk}[1]{\Bigg\langle {#1}\Bigg\rangle}
\newcommand{\GA}{\mathscr G_{\mathcal A}}
\newcommand{\vs}{\varsigma}
\newcommand{\Vect}{\mathrm{Vec}}
\newcommand{\Vectc}{\mathrm{Vec}^{\mathbb C}}
\newcommand{\scr}{\mathscr}
\newcommand{\sjs}{\subset\joinrel\subset}
\newcommand{\Jtd}{\widetilde{\mathcal J}}
\newcommand{\gk}{\mathfrak g}
\newcommand{\hk}{\mathfrak h}
\newcommand{\xk}{\mathfrak x}
\newcommand{\yk}{\mathfrak y}
\newcommand{\zk}{\mathfrak z}
\newcommand{\pk}{\mathfrak p}
\newcommand{\hr}{\mathfrak h_{\mathbb R}}
\newcommand{\Ad}{\mathrm{Ad}}
\newcommand{\DHR}{\mathrm{DHR}_{I_0}}
\newcommand{\Repi}{\mathrm{Rep}_{\wtd I_0}}
\newcommand{\im}{\mathbf{i}}
\newcommand{\Co}{\complement}
%\newcommand{\Cu}{\mathcal C^{\mathrm u}}
\newcommand{\RepV}{\mathrm{Rep}^\uni(V)}
\newcommand{\RepA}{\mathrm{Rep}(\mathcal A)}
\newcommand{\RepN}{\mathrm{Rep}(\mathcal N)}
\newcommand{\RepfA}{\mathrm{Rep}^{\mathrm f}(\mathcal A)}
\newcommand{\RepAU}{\mathrm{Rep}^\uni(A_U)}
\newcommand{\RepU}{\mathrm{Rep}^\uni(U)}
\newcommand{\RepL}{\mathrm{Rep}^{\mathrm{L}}}
\newcommand{\HomL}{\mathrm{Hom}^{\mathrm{L}}}
\newcommand{\EndL}{\mathrm{End}^{\mathrm{L}}}
\newcommand{\Bim}{\mathrm{Bim}}
\newcommand{\BimA}{\mathrm{Bim}^\uni(A)}
%\newcommand{\shom}{\scr Hom}
\newcommand{\divi}{\mathrm{div}}
\newcommand{\sgm}{\varsigma}
\newcommand{\SX}{{S_{\fk X}}}
\newcommand{\DX}{D_{\fk X}}
\newcommand{\mbb}{\mathbb}
\newcommand{\mbf}{\mathbf}
\newcommand{\bsb}{\boldsymbol}
\newcommand{\blt}{\bullet}
\newcommand{\Vbb}{\mathbb V}
\newcommand{\Ubb}{\mathbb U}
\newcommand{\Xbb}{\mathbb X}
\newcommand{\Kbb}{\mathbb K}
\newcommand{\Abb}{\mathbb A}
\newcommand{\Wbb}{\mathbb W}
\newcommand{\Mbb}{\mathbb M}
\newcommand{\Gbb}{\mathbb G}
\newcommand{\Cbb}{\mathbb C}
\newcommand{\Nbb}{\mathbb N}
\newcommand{\Zbb}{\mathbb Z}
\newcommand{\Qbb}{\mathbb Q}
\newcommand{\Pbb}{\mathbb P}
\newcommand{\Rbb}{\mathbb R}
\newcommand{\Ebb}{\mathbb E}
\newcommand{\Dbb}{\mathbb D}
\newcommand{\Hbb}{\mathbb H}
\newcommand{\cbf}{\mathbf c}
\newcommand{\Rbf}{\mathbf R}
\newcommand{\wt}{\mathrm{wt}}
\newcommand{\Lie}{\mathrm{Lie}}
\newcommand{\btl}{\blacktriangleleft}
\newcommand{\btr}{\blacktriangleright}
\newcommand{\svir}{\mathcal V\!\mathit{ir}}
\newcommand{\Ker}{\mathrm{Ker}}
\newcommand{\Cok}{\mathrm{Coker}}
\newcommand{\Sbf}{\mathbf{S}}
\newcommand{\low}{\mathrm{low}}
\newcommand{\Sp}{\mathrm{Sp}}
\newcommand{\Rng}{\mathrm{Rng}}
\newcommand{\vN}{\mathrm{vN}}
\newcommand{\Ebf}{\mathbf E}
\newcommand{\Nbf}{\mathbf N}
\newcommand{\Stb}{\mathrm {Stb}}
\newcommand{\SXb}{{S_{\fk X_b}}}
\newcommand{\pr}{\mathrm {pr}}
\newcommand{\SXtd}{S_{\wtd{\fk X}}}
\newcommand{\univ}{\mathrm {univ}}
\newcommand{\vbf}{\mathbf v}
\newcommand{\ubf}{\mathbf u}
\newcommand{\wbf}{\mathbf w}
\newcommand{\CB}{\mathrm{CB}}
\newcommand{\Perm}{\mathrm{Perm}}
\newcommand{\Orb}{\mathrm{Orb}}
\newcommand{\Lss}{{L_{0,\mathrm{s}}}}
\newcommand{\Lni}{{L_{0,\mathrm{n}}}}
\newcommand{\UPSU}{\widetilde{\mathrm{PSU}}(1,1)}
\newcommand{\Sbb}{{\mathbb S}}
\newcommand{\Gc}{\mathscr G_c}
\newcommand{\Obj}{\mathrm{Obj}}
\newcommand{\bpr}{{}^\backprime}
\newcommand{\fin}{\mathrm{fin}}
\newcommand{\Ann}{\mathrm{Ann}}
\newcommand{\Real}{\mathrm{Re}}
\newcommand{\Imag}{\mathrm{Im}}
%\newcommand{\cl}{\mathrm{cl}}
\newcommand{\Ind}{\mathrm{Ind}}
\newcommand{\Supp}{\mathrm{Supp}}
\newcommand{\Specan}{\mathrm{Specan}}
\newcommand{\red}{\mathrm{red}}
\newcommand{\uph}{\upharpoonright}
\newcommand{\Mor}{\mathrm{Mor}}
\newcommand{\pre}{\mathrm{pre}}
\newcommand{\rank}{\mathrm{rank}}
\newcommand{\Jac}{\mathrm{Jac}}
\newcommand{\emb}{\mathrm{emb}}
\newcommand{\Sg}{\mathrm{Sg}}
\newcommand{\Nzd}{\mathrm{Nzd}}
\newcommand{\Owht}{\widehat{\scr O}}
\newcommand{\Ext}{\mathrm{Ext}}
\newcommand{\Tor}{\mathrm{Tor}}
\newcommand{\Com}{\mathrm{Com}}
\newcommand{\Mod}{\mathrm{Mod}}
\newcommand{\nk}{\mathfrak n}
\newcommand{\mk}{\mathfrak m}
\newcommand{\Ass}{\mathrm{Ass}}
\newcommand{\depth}{\mathrm{depth}}
\newcommand{\Gode}{\mathrm{Gode}}
\newcommand{\Fbb}{\mathbb F}
\newcommand{\sgn}{\mathrm{sgn}}
\newcommand{\Aut}{\mathrm{Aut}}
\newcommand{\Modf}{\mathrm{Mod}^{\mathrm f}}
\newcommand{\codim}{\mathrm{codim}}
\newcommand{\card}{\mathrm{card}}
\newcommand{\dps}{\displaystyle}
\newcommand{\Int}{\mathrm{Int}}
\newcommand{\Nbh}{\mathrm{Nbh}}
\newcommand{\Pnbh}{\mathrm{PNbh}}
\newcommand{\Cl}{\mathrm{Cl}}
\newcommand{\diam}{\mathrm{diam}}
\newcommand{\eps}{\varepsilon}
\newcommand{\Vol}{\mathrm{Vol}}
\newcommand{\LSC}{\mathrm{LSC}}
\newcommand{\USC}{\mathrm{USC}}
\newcommand{\Ess}{\mathrm{Rng}^{\mathrm{ess}}}
\newcommand{\Jbf}{\mathbf{J}}
\newcommand{\SL}{\mathrm{SL}}
\newcommand{\GL}{\mathrm{GL}}
\newcommand{\Lin}{\mathrm{Lin}}
\newcommand{\ALin}{\mathrm{ALin}}
\newcommand{\bwn}{\bigwedge\nolimits}
\newcommand{\nbf}{\mathbf n}
\newcommand{\dive}{\mathrm{div}}
\newcommand{\QC}{\mathrm{QCoh}_{\mathrm L}}
\newcommand{\Coh}{\mathrm{Coh}_{\mathrm L}}
\newcommand{\rad}{\mathrm{rad}}
\newcommand{\SLF}{\mathrm{SLF}}
\newcommand{\op}{\mathrm{op}}
\newcommand{\trc}{\mathrm{tr}}
\newcommand{\Mob}{\textrm{Möb}}
\newcommand{\Poid}{{\mathrm P}^+(1,d)}
\newcommand{\xbf}{\mathbf x}
\newcommand{\ybf}{\mathbf y}
\newcommand{\ebf}{\mathbf e}
\newcommand{\SO}{\mathrm{SO}}








\usepackage{tipa} % wierd symboles e.g. \textturnh
\newcommand{\tipar}{\text{\textrtailr}}
\newcommand{\tipaz}{\text{\textctyogh}}
\newcommand{\tipaomega}{\text{\textcloseomega}}
\newcommand{\tipae}{\text{\textrhookschwa}}
\newcommand{\tipaee}{\text{\textreve}}
\newcommand{\tipak}{\text{\texthtk}}
\newcommand{\mol}{\upmu}
\newcommand{\dmol}{d\upmu}




\usepackage{tipx}
\newcommand{\tipxgamma}{\text{\textfrtailgamma}}
\newcommand{\tipxcc}{\text{\textctstretchc}}
\newcommand{\tipxphi}{\text{\textqplig}}















\numberwithin{equation}{section}




\title{Topics in Operator Algebras: Algebraic Conformal Field Theory}
\author{\sc{ Bin Gui}
	%\\
	%{\small \sc Yau Mathematical Sciences Center, Tsinghua University.}\\
	%{\small binguimath@gmail.com\qquad bingui@tsinghua.edu.cn}
}
\date{}

%\definecolor{mycolor}{RGB}{227,237,205} \pagecolor{mycolor}

\begin{document}\sloppy % avoid stretch into margins
	\pagenumbering{arabic}
	%\pagenumbering{gobble}
	\setcounter{page}{1}
%	\setcounter{section}{-1}
	%\setcounter{equation}{6}



	






	%%%%%%%%%%%%%%%%%%%%%%%%%%%%%%%%%%%%%%%%%%%%%%%%%%%%%%%%%



	
	\maketitle
%\small   \hyperlink{page.7}{Last page of TOC}


%\hyperlink{beforeindex}{Last page before index}~~~~~~  
%\hypertarget{beforeindex}{}



%\noindent Sections on history include but are not limited to: 
%\ref{lb55} (point-set topology),  \ref{lb550} (integral theory, Fourier series), \ref{lb543} (Banach-Alaoglu, Hahn-Banach),  \ref{lb548} (quotient Banach spaces, Hahn-Banach), \ref{lb671} and most part of Ch. \ref{lb672} (Hilbert spaces, integral equations), \ref{lb733} (measurable sets), \ref{mc89} (Riesz-Fischer theorem, $L^p$-$L^q$ duality), \ref{lb896} (functional calculus, spectral theory)
%\normalsize
%\thispagestyle{empty}	 %remove page number of this page


%Contents hyperlinks: \hyperlink{page.2}{Page 2}, \hyperlink{page.3}{Page 3}

%%%%%%%%%%%%%%%%%%%%%%%%%%%%%
%\vspace{-0.5cm}

%\makeatletter
%\newcommand*{\toccontents}{\@starttoc{toc}}
%\makeatother
%\toccontents



	
% title and table of contents same page, no content title

%%%%%%%%%%%%%%%%%%%%%%%%%%%%%

\normalsize

\hyperlink{beforeindex}{Current page of writing}~~~~~~ 

\tableofcontents



\newpage


\section*{Notations}

$\Nbb=\{0,1,2,\dots\}$, $\Zbb_+=\{1,2,3,\dots\}$.

Unless otherwise stated, an \textbf{unbounded operator} $T:\mc H\rightarrow\mc K$ (where $\mc H,\mc K$ are Hilbert spaces) denotes a linear map from a dense linear subspace $\Dom(T)\subset \mc H$ to $\mc H$. $\Dom(T)$ is called the \textbf{domain} of $T$. We let $T^*$ be the adjoint of $T$. In practice, we are also interested in $T^*$ defined on a dense subspace of its domain $\Dom(T^*)$. We call its restriction a \textbf{formal adjoint} of $T$ and denote it by $T^\dagger$.


Given a Hilbert space $\mc H$, its inner product is denoted by $(\xi,\eta)\in\mc H^2\mapsto\bk{\xi|\eta}$. We assume that it is linear on the first variable and antilinear on the second one. (Namely, we are following mathematician's convention.) 


Whenever we write $\bk{\xi,\eta}$, we understand that it is linear on both variables. E.g. $\bk{\cdot,\cdot}$ denotes the pairing between a vector space and its dual space.


If $\mc H,\mc K$ are Hilbert spaces, we let
\begin{align}
\Hom (\mc H,\mc K)=\{\text{Bounded linear maps }\mc H\rightarrow\mc H\}\qquad\End(\mc H)=\Hom(\mc H,\mc H)
\end{align}

If $X$ is a set, the $n$-fold \textbf{configuration space} \index{00@Configuration space} $\Conf^n(X)$ \index{Conf@$\Conf^n(X)$} is
\begin{align}
\Conf^n(X)=\{(x_1,\dots,x_n)\in X:x_i\neq x_j\text{ if }i\neq j\}
\end{align}

\begin{df}
A map of complex vector spaces $T:V\rightarrow V'$ is called \textbf{antilinear} \index{antilinear} or \textbf{conjugate linear} if $T(a\xi+b\eta)=\ovl aT\xi+\ovl bT\eta$ for all $\xi,\eta\in V$ and $a,b\in\Cbb$. If $V$ and $V'$ are (complex) inner product spaces, we say that $T$ is \textbf{antiunitary} \index{00@Antiunitary} if it is am antiliear surjective and satisfies $\Vert T\xi\Vert=\Vert\xi\Vert$ for all $\xi\in V$, equivalently,
\begin{align}\label{eq14}
\bk{T\xi|T\eta}=\ovl{\bk{\xi|\eta}}\equiv\bk{\eta|\xi}
\end{align}
for all $\xi,\eta\in V$.
\end{df}


\newpage



\section{Introduction: PCT symmetry, Bisognano-Wichmann, Tomita-Takesaki}


Algebraic quantum field theory (AQFT) is a mathematically rigorous approach to QFT using the language of functional analysis and operator algebras. The main subject of this course is 2d \textbf{algebraic conformal field theory (ACFT)}, namely, 2d CFT in the framework of AQFT.


\subsection{}




Let $d\in\Zbb_+$. We first sketch the general picture of an $(1+d)$ dimensional Poincar\'e invariant QFT in the spirit of \textbf{Wightman axioms}. We consider Bosonic theory for simplicity. 

We let $\Rbb^{1,d}$ be the $(1+d)$-dimensional \textbf{Minkowski space}. So it is $\Rbb^{1+d}$ but with metric tensor
\begin{align}
ds^2=(dx^0)^2-(dx^1)^2-\cdots-(dx^d)^2
\end{align}
Here $x^0$ denotes the time coordinate, and $x^1,\dots,x^d$ denote the spatial coordinates.  The (restricted) \textbf{Poincar\'e group} \index{00@Poincar\'e group $\Poid$, restricted} is 
\begin{align*}
\Poid=\Rbb^{1,d}\rtimes\mathrm{SO}^+(1,d)
\end{align*}
Here, $\Rbb^{1,d}$ acts by translation on $\Rbb^{1,d}$. $\mathrm{SO}^+(1,d)$ is the (restricted) \textbf{Lorentz group}, \index{00@Lorentz group $\mathrm{SO}^+(1,d)$} the identity component of the (full) Lorentz group $\mathrm O(1,d)$ whose elements are invertible linear maps on $\Rbb^{1,d}$ preserving the Minkowski metric. 


\begin{rem}
Any $g\in \mathrm O(1,d)$ must have determinent $\pm1$. One can show that $\mathrm{SO^+}(1,d)$ is precisely the elements $g\in \mathrm O(1,d)$ such that $\det g=1$, and that $g$ \uwave{does not change the direction of time} (i.e., if $\mbf v=(v_0,\dots,v_d)\in\Rbb^{1,d}$ satisfies $v_0>0$, then the first component of $g\mbf v$ is $>0$). See \cite[Sec. I.2.1]{Haag}.
\end{rem}



\begin{df}
We say that $\xbf=(x_0,\dots,x_d),\ybf=(y_0,\dots,y_d)\in\Rbb^{1,d}$ are \textbf{spacelike (separated)} \index{00@Spacelike (separated)} if their Minkowski distance is negative, i.e.,
\begin{align*}
(x_0-y_0)^2<(x_1-y_1)^2+\cdots+(x_d-y_d)^2
\end{align*}
\end{df}




\subsection{}\label{lb1}






A Poincar\'e invariant QFT consists of the following data:
\begin{enumerate}[label=(\arabic*)]
\item We have a Hilbert space $\mc H$.
\item There is a (strongly continuous) projective unitary representation $U$ of $\Poid$ on $\mc H$. In particular, its restriction to the translation on the $k$-th component (where $k=0,1,\dots,d$) gives a one parameter unitary group $x^k\in\Rbb\mapsto \exp(\im x^k P_k)$ where $P_k$ is a self-adjoint operator on $\mc H$.
\item (Positive energy) The following are positive operators:
\begin{align*}
P_0\geq0\qquad (P_0)^2-(P_1)^2-\cdots-(P_d)^2\geq0
\end{align*}
The operator $P_0$ is called the \textbf{Hamiltonian} or the \textbf{energy operator}. $P_1,\dots,P_d$ are the momentum operators. $(P_0)^2-(P_1)^2-\cdots-(P_d)^2$ is the mass.
\item We have a collection of \textbf{(quantum) fields} $\scr Q$, where each $\Phi\in\scr Q$ is an operator-valued function on $\Rbb^{1,d}$. For each $\mbf x\in\Rbb^{1,d}$, $\Phi(x)$ is a ``linear operator on $\mc H$".  
\item {\textbf(Locality}) \index{00@Locality} If $\xbf_1,\xbf_2\in\Rbb^{1,d}$ are \uwave{spacelike} and $\Phi_1,\Phi_2\in\scr Q$, then
\begin{align}
[\Phi_1(x_1),\Phi_2(x_2)]=0
\end{align}
\item (*-invariance) For each $\Phi\in\scr Q$, there exists $\Phi^\dagger\in\scr Q$ such that
\begin{align}\label{eq3}
\Phi(\xbf)^\dagger=\Phi^\dagger(\xbf)
\end{align}
Moreover, $\Phi^{\dagger\dagger}=\Phi$.
\item (Poincar\'e invariance) There is a distinguished unit vector\footnote{A unit vector denotes a vector with length $1$} $\Omega$, called the \textbf{vacuum vector}, such that
\begin{align*}
U(g)\Omega=\Omega\qquad\forall g\in\Poid
\end{align*}
Moreover, for each $g\in\Poid$ and $\Phi\in\scr Q$, we have
\begin{align}\label{eq2}
U(g)\Phi(\xbf)U(g)^{-1}=\Phi(g\xbf)
\end{align}
\item (Cyclicity) Vectors of the form
\begin{align}\label{eq4}
\Phi_1(\xbf_1)\cdots\Phi_n(\xbf_n)\Omega
\end{align}
(where $n\in\Nbb$, $\xbf_1,\dots,\xbf_n\in\Rbb^{1,d}$ are mutually spacelike, and $\Phi_1,\dots,\Phi_n\in\scr Q$) span a dense subspace of $\mc H$.
\end{enumerate}



\begin{rem}
In some QFT, there is a factor (a function of $\xbf$) before $\Phi(g\xbf)$ in the Poincar\'e invariance relation \eqref{eq2}. Similarly, there is a factor before $\Phi^\dagger(\xbf)$ in the $*$-invariance formula \eqref{eq3}. We will encounter these more general covariance property later. In this section, we content ourselves with the simplest case that the factors are $1$.
\end{rem}

\begin{rem}\label{lb7}
By the Poincar\'e invariance and the cyclicity, the action of $\Poid$ is uniquely determined by $\scr Q$ by
\begin{align}
U(g)\Phi_1(\xbf_1)\cdots\Phi_n(\xbf_n)\Omega=\Phi_1(g\xbf_1)\cdots\Phi_n(g\xbf_n)\Omega
\end{align}
\end{rem}


\subsection{}

Technically speaking, $\Phi(\xbf)$ can not be viewed as a linear operator on $\mc H$. It  cannot be defined even on a sufficiently large subspace of $\mc H$. One should think about \textbf{smeared fields} \index{00@Smeared field}
\begin{align}
\Phi(f)=\int_{\Rbb^{1,d}}\Phi(\xbf)f(\xbf)d\xbf
\end{align}
where $f\in C_c^\infty(\Rbb^{1,d})$. (In contrast, we call $\Phi(\xbf)$ a \textbf{pointed field}.) Then $\Phi(f)$ is usually a closable unbounded operator on $\mc H$ with dense domain $\Dom(\Phi(f))$. Moreover, $\Dom(\Phi(f))$ is preserved by any smeared operator $\Psi(g)$. Therefore, for any $f_1,\dots,f_n\in C_c^\infty(\Rbb^{1,d})$ the following vector can be defined in $\mc H$:
\begin{align}\label{eq1}
\Phi_1(f_1)\cdots\Phi_n(f_n)\Omega
\end{align}
The precise meaning of cyclicity in Subsec. \ref{lb1} means that vectors of the form \eqref{eq1} span a dense subspace of $\mc H$. Locality means that for $f_1,f_2\in C_c^\infty(\Rbb^{1,d})$ compactly supported in \uwave{spacelike} regions, on a reasonable dense subspace of $\mc H$ (e.g., the subspace spanned by \eqref{eq1}) we have
\begin{align}
[\Phi_1(f_1),\Phi_2(f_2)]=0
\end{align}
The $*$-invariance means that
\begin{align}
\bk{\Phi(f)\xi|\eta}=\bk{\xi|\Phi^\dagger(f)\eta}
\end{align}
for each $\xi,\eta$ in the this good subspace. 




\subsection{}

In the remaining part of this section,  if possible, we also understand $\Phi(\xbf)$ as a smeared operator $\Phi(f)$ where $f\in C_c^\infty(\Rbb^{1,d})$ satisfies $\int f=1$ and is supported in a small region containing $\xbf$. Thus, $\Phi(\xbf)$ can almost be viewed as a closable operator. Hence the expression \eqref{eq4} makes sense in $\mc H$.


We now explore the consequences of positive energy. As we will see, it implies that $\Phi_1(\xbf_1)\cdots\Phi_n(\xbf_n)\Omega$, a function of $\xbf_\blt$, can be analytically continued.

The fact that $P_0\geq0$ implies that when $t\geq0$, $e^{tP_0}$ is a bounded linear operator with operator norm $\leq1$. Therefore, if $\tau$ belongs to
\begin{align*}
\fk I=\{\Imag\tau\geq0\}
\end{align*}
then $e^{\im \tau P_0}=e^{\im \Real\tau}\cdot e^{-\Imag\tau}$ is bounded. Indeed, $\tau\in\fk I\mapsto e^{\im\tau P_0}$ is continuous, and is holomorphic on $\Int\fk I$. 

Let $\mbf e_0=(1,0,\dots,0)$. Let $\xbf_1,\dots,\xbf_n\in\Rbb^{1,d}$ be distinct. By the Poincar\'e covariance, the relation
\begin{align}\label{eq5}
e^{\im\tau P_0}\Phi_1(\xbf_1)\cdots\Phi_n(\xbf_n)\Omega=\Phi_1(\xbf_1+\tau e_0)\cdots\Phi_n(\xbf_n+\tau e_0)\Omega
\end{align}
holds for all real $\tau$. Moreover, the LHS is continuous on $\fk I$ and holomorphic on $\Int\fk I$. This suggests that the RHS of \eqref{eq5} can also be defined as an element of $\mc H$ when $\tau\in\fk I$. 


\subsection{}\label{lb3}


We shall further explore the question: for which $\xbf_i$ is in $\Cbb^d$ can $\Phi_1(\xbf_1)\cdots\Phi_n(\xbf_n)\Omega$ be reasonably defined as an element of $\mc H$? 

\begin{rem}
We expect that the smeared fields should be defined on any \textbf{$P_0$-smooth vectors}, i.e., vectors in $\bigcap_{k\geq0}\Dom(P_0^k)$. For each $r>0$, since one can find $C_{k,r}\geq0$ such that $\lambda^{2k}\leq C_{k,r}e^{2r\lambda}$ for all $\lambda\geq0$, we conclude that
\begin{align}
\Rng(e^{-rP_0})\equiv \Dom(e^{rP_0})\subset \bigcap_{k\geq0}\Dom(P_0^k)
\end{align}
\end{rem}



The above remark shows that $\Phi_1(\xbf_1)$, viewed as a smeared operator localized on a small neighborhood of $\xbf_1$, is definable on $e^{\im\zeta_2P_0}\Phi_2(\xbf_2)\Omega=\Phi_2(\zeta_2\ebf_0+\xbf_2)\Omega$ whenever $\Imag\zeta_2>0$. Thus, heuristically, $(\zeta_1,\zeta_2)\mapsto e^{\im\zeta_1P_0}\Phi_1(\xbf_1)e^{\im\zeta_2P_0}\Phi_2(\xbf_2)\Omega$ should also be holomorphic on
\begin{align*}
\{(\zeta_1,\zeta_2)\in\Cbb^2:\Imag\zeta_1,\Imag\zeta_2>0\}
\end{align*}
Repeating this procedure, we see that the holomorphicity holds for
\begin{align*}
e^{\im\zeta_1 P_0}\Phi_1(\xbf_1)e^{\im\zeta_2P_0}\Phi_2(\xbf_2)\cdots e^{\im\zeta_nP_0}\Phi_n(\xbf_n)\Omega
\end{align*}
when $\Imag\zeta_i>0$. By Poincar\'e covariance, the above expression equals
\begin{align*}
\Phi_1(\xbf_1+\zeta_1\ebf_0)\Phi_2(\xbf_2+(\zeta_1+\zeta_2)\ebf_0)\cdots\Phi_n(\xbf_n+(\zeta_1+\cdots+\zeta_n)\ebf_0)\Omega
\end{align*}
Therefore,
\begin{align}\label{eq6}
(\zeta_1,\dots,\zeta_n)\mapsto \Phi_1(\xbf_1+\zeta_1\ebf_0)\cdots\Phi_n(\xbf_n+\zeta_n\ebf_0)\in\mc H
\end{align}
should be holomorphic on $\{\zeta_\blt\in\Cbb^n:0<\Imag\zeta_1<\cdots<\Imag\zeta_n\}$.

By the locality axiom, the order of products of quantum fields can be exchanged. Thus, our expectation for a reasonable QFT includes the following condition:
\begin{concl}
Let $\xbf_1,\dots,\xbf_n\in\Rbb^{1,d}$. Then \eqref{eq6} is holomorphic on
\begin{subequations}\label{eq11}
\begin{align}\label{eq11a}
\{(\zeta_1,\dots,\zeta_n)\in\Cbb^n:\Imag\zeta_i>0,\text{ and }\Imag\zeta_i\neq\Imag\zeta_j\text{ if }i\neq j\}
\end{align}
Moreover, since \eqref{eq6} is also definable and continuous on
\begin{align}\label{eq11b}
\{(\zeta_1,\dots,\zeta_n)\in\Rbb^n:\text{$\xbf_1+\zeta_1\ebf_1,\dots,\xbf_n+\zeta_n\ebf_0$ are mutually spacelike}\}
\end{align}
\end{subequations}
we expect that the function \eqref{eq6} is continuous on the union of \eqref{eq11a} and \eqref{eq11b}.
\end{concl}

\subsection{}\label{lb2}


We have (informally) derived some consequences from the positivity of $P_0$.  Note that since $P_0\geq0$, we have $U(g)P_0U(g)^{-1}\geq0$ for each $g\in\SO^+(1,d)$. Since $P_0$ is the generator of the flow $t\in\Rbb\mapsto t\ebf_0\in\Rbb^{1,d}\subset\Poid$, $U(g)P_0U(g)^{-1}$ is the generator of the flow
\begin{align}
t\in\Rbb\mapsto g(t\ebf_0)g^{-1}=t\cdot g\ebf_0
\end{align}
Therefore, if $g\mbf e_0=(a_0,\dots,a_n)$, then
\begin{align}
U(g)P_0U(g)^{-1}=a_0P_0+\cdots+a_nP_n
\end{align}
Hence the RHS must be positive. But what are all the possible $g\ebf_0$?

\begin{rem}
One can show that the orbit of $\mbf \ebf_0=(1,0,\dots,0)$ under $\mathrm{SO}^+(1,d)$ is the upper hyperbola with diameter $1$, i.e., the set of all $(a_0,\dots,a_n)\in\Rbb^{1,d}$ satisfying
\begin{align}
a_0>0\qquad (a_0)^2-(a_1)^2-\cdots-(a_n)^2=1
\end{align}
\end{rem}

Thus $\sum_i a_iP_i\geq0$ for all such $a_\blt$. What are the consequences of this positivity?



\subsection{}



To simplify the following discussions, we set $d=2$ and
\begin{align*}
t=x^0\qquad x=x^1
\end{align*}
We further set
\begin{align}
u=t-x\qquad v=t+x
\end{align}
so that
\begin{align}
t=\frac{u+v}2\qquad x=\frac{-u+v}2
\end{align}
The Minkowski metric becomes
\begin{align}
\boxed{~(dt)^2-(dx)^2=du\cdot dv~}
\end{align}
Then 
\begin{align}
(u,v)\text{ is spacelike to }(u',v')\qquad\Longleftrightarrow\qquad (u-u')(v-v')<0
\end{align}


\begin{figure}[h]
	\centering
	\includegraphics[height=2cm]{fig1.png}
	\caption{. The coordinates $u,v$}
\end{figure}


For each $\Phi\in\scr Q$, we write
\begin{align}
\wtd \Phi(u,v):=\Phi(t,x)=\Phi\big(\frac{u+v}2,\frac{-u+v}2\big)
\end{align}
We let $H_0$ and $H_1$ be the self-adjoint operators such that
\begin{align*}
H_0=P_0-P_1\qquad H_1=P_0+P_1
\end{align*}
so that they are the generators of the flow $t\mapsto (t,-t)$ and $t\mapsto (t,t)$.




\begin{rem}
Since $\Rbb^{1,d}$ is an abelian group, we know that $P_i$ commutes with $P_j$. Hence $H_0$ commutes with $H_1$. 
\end{rem}

\subsection{}\label{lb9}


The orbit of $\mbf e_0$ under $\SO^+(1,1)$ is the unit upper hyperbola $(x^0)^2-(x_1)^2=1,x^0>0$. Equivalently, it is $uv=1,u>0$. According to Subsec. \ref{lb2}, we conclude that $b_0H_0+b_1H_1\geq0$ for each $b_0,b_1$ satisfying $b_0b_1=1,b_0>0$ (equivalently, for each $b_0>0,b_1>0$). This implies
\begin{align}
H_0\geq0\qquad H_1\geq0
\end{align}
Therefore, similar to the argument in Subsec. \ref{lb3} (and specializing to the special case that $\xbf_1=\cdots=\xbf_n=0$), the holomorphicity of
\begin{align*}
(\zeta_\blt,\gamma_\blt)\mapsto e^{\im\zeta_1H_0+\im\gamma_1 H_1}\wtd\Phi_1(0)e^{\im\zeta_2H_0+\im\gamma_2 H_1}\wtd\Phi_2(0)\cdots e^{\im\zeta_nH_0+\im\gamma_nH_1}\wtd\Phi_n(0)\Omega
\end{align*}
on the region $\Imag\zeta_i>0,\Imag\gamma_i>0$, together with locality, implies:

\begin{concl}\label{lb4}
Let $\Phi_1,\dots,\Phi_n\in\scr Q$. Then
\begin{align}\label{eq8}
(u_1,v_1,\dots,u_n,v_n)\mapsto \wtd\Phi_1(u_1,v_1)\cdots\wtd\Phi(u_n,v_n)\Omega
\end{align}
is holomorphic on
\begin{subequations}\label{eq9}
\begin{align}\label{eq9a}
\{(u_\blt,v_\blt)\in\Cbb^{2n}:\Imag u_i>0 ,\Imag v_i>0,\Imag u_i\neq\Imag u_j,\Imag v_i\neq\Imag v_j\text{ if }i\neq j\}
\end{align}
and can be continuously extended to 
\begin{align}\label{eq9b}
\{(u_\blt,v_\blt)\in\Rbb^{2n}:(u_i-u_j)\cdot(v_i-v_j)<0\text{ if }i\neq j\}
\end{align}
\end{subequations}
\end{concl}

Rigorously speaking, the above mentioned ``continuity" of the extension should be understood in terms of distributions. Here, we ignore such subtlety and view pointed fields as smeared field in a small region.


\begin{comment}
\begin{rem}\label{lb5}
Recall that we have said that a pointed field should also be viewed as a smeared field localized on a small region. Otherwise, the RHS of \eqref{eq8} still belongs to $\mc H$ in the interior of \eqref{eq9}, but not so on the boundary. Thus, a more rigorous interpretation of Conc. \ref{lb4} should be as follows: If $O_1,\dots,O_n$ are spacelike separated regions of $\Rbb^2$ (i.e. for each $(u_i,v_i)\in O_i$ and $(u_j,v_j)\in O_j$ where $i\neq j$, we have \eqref{eq7}), and if $f_i\in C_c^\infty(O_i)$, then the function
\begin{align}\label{eq10}
\begin{aligned}
\int_{O_1}\cdots\int_{O_n}& \wtd\Phi_1(u_1+\tau_1,v_1+\sgm_1)\cdots\wtd\Phi(u_n+\tau_n,v_n+\sgm_n)\Omega \\
&\cdot f_1(u_1,v_1)\cdots f_n(u_n,v_n)\cdot  d(u_1,v_1)\cdots d(u_n,v_n)
\end{aligned}
\end{align}
of $(\tau_1,\sgm_1,\dots,\tau_n,\sgm_n)$ is continuous on \eqref{eq9} and holomorphic on its interior.
\end{rem}
\end{comment}




\subsection{}

%In the following, as in Rem. \ref{lb5}, the following claims about points fields should be viewed as claims about smeared fields supported in small regions, or should be translated into claims about smeared fields in a similar manner as in \eqref{eq10}.


We note that $\diag(-1,\pm 1)$ is not inside $\SO^+(1,1)$, since it reverses the time direction. Neither is $\diag(1,-1)$ in $\SO^+(1,1)$ because its determinant is negative. Consequently, the QFT is not necessarily symmetric under the following operations:
\begin{itemize}
\item \textbf{{\color{red}T}ime reversal} \index{00@Time reversal}  $t\mapsto -x$.
\item \textbf{{\color{red}P}arity transformation} \index{00@Parity transformation} $x\mapsto -x$.
\item \textbf{{\color{red}PT} transformation}  $(t,x)\mapsto (-t,-x)$, the combination of time and parity inversions.
\end{itemize} 
Mathematically, this means that the maps
\begin{gather*}
\Phi_1(t_1,x_1)\cdots \Phi_n(t_n,x_n)\Omega\quad\mapsto\quad \Phi_1(-t_1,x_1)\cdots \Phi_n(-t_n,x_n)\Omega\\
\Phi_1(t_1,x_1)\cdots \Phi_n(t_n,x_n)\Omega\quad\mapsto\quad \Phi_1(t_1,-x_1)\cdots \Phi_n(t_n,-x_n)\Omega\\
\Phi_1(t_1,x_1)\cdots \Phi_n(t_n,x_n)\Omega\quad\mapsto\quad \Phi_1(-t_1,-x_1)\cdots \Phi_n(-t_n,-x_n)\Omega
\end{gather*}
(where $(t_1,x_1),\dots,(t_n,x_n)$ are mutually spacelike) are not necessarily unitary. (Compare Rem. \ref{lb7}.) Simiarly, the QFT is not necessarily symmetric under \textbf{{\color{red}C}harge conjugation} $\Phi\mapsto\Phi^\dagger$, which means that the map
\begin{align*}
\Phi_1(t_1,x_1)\cdots \Phi_n(t_n,x_n)\Omega\quad\mapsto\quad &\Phi_n(t_n,x_n)^\dagger\cdots \Phi_1(t_1,x_1)^\dagger\Omega\\
=&\Phi_1^\dagger(t_1,x_1)\cdots\Phi_n^\dagger(t_n,x_n)\Omega
\end{align*}
is not necessarily (anti)unitary. However, as we shall explain, the combination of PCT transformations is actually unitary, and hence is a symmetry of the QFT. This is called the PCT theorem.

\subsection{}


To prove the PCT theorem, we shall first prove that the PT transformation, though not implemented by a unitary operator, is actually implemented by the analytic continuation of a one parameter unitary group.

\begin{df}
The one parameter group $s\mapsto \Lambda(s)\in\SO^+(1,1)$ defined by
\begin{align}
\Lambda(s)(u,v)=(e^{-s}u,e^sv) 
\end{align}
is called the \textbf{Lorentz boost}. \index{00@Lorentz boost $\Lambda$} Equivalently,
\begin{align}
\Lambda(s)\begin{bmatrix}
t\\
x
\end{bmatrix}
=\begin{bmatrix}
\cosh s&\sinh s\\
\sinh s&\cosh s
\end{bmatrix}
\begin{bmatrix}
t\\
x
\end{bmatrix}
\end{align}
\end{df}

Define the (open) \textbf{right wedge} $\mc W$ and \textbf{left wedge} $-\mc W$ by
\begin{align}
\mc W=\{(u,v)\in\Rbb^2:v>0,u<0\}=\{(t,x)\in\Rbb^{1,1}:-x<t<x\}
\end{align}


\begin{thm}[\textbf{PT theorem}]\label{lb8}
Let $(u_1,v_1),\dots,(u_n,v_n)\in \mc W$ be mutually spacelike (i.e. satisfying $(u_i-u_j)(v_i-v_j)<0$ if $i\neq j$), cf. Fig. \ref{lb6}. Let $\Phi_1,\dots,\Phi_n\in\scr Q$. Let $K$ be the self-adjoint generator of the Lorentz boost, i.e.,
\begin{align*}
U(\Lambda(s))=e^{\im sK}
\end{align*}
Then  $\Phi_1(\xbf_1)\cdots\Phi_n(\xbf_n)\Omega$ belongs to the domain of $e^{-\pi K}$, and
\begin{align}\label{eq16}
e^{-\pi K}\Phi_1(\xbf_1)\cdots\Phi_n(\xbf_n)\Omega=\Phi_1(-\xbf_1)\cdots\Phi_n(-\xbf_n)\Omega
\end{align}
\end{thm}


\begin{figure}[h]
	\centering
	\includegraphics[height=2cm]{fig2.png}
	\caption{.}\label{lb6}
\end{figure}


Equivalently, $\wtd\Phi_1(u_1,v_1)\cdots\wtd\Phi_n(u_n,v_n)\Omega$ belongs to the domain of $e^{-\pi K}$, and
\begin{align}\label{eq12}
e^{-\pi K}\wtd\Phi_1(u_1,v_1)\cdots\wtd\Phi_n(u_n,v_n)\Omega=\wtd\Phi_1(-u_1,-v_1)\cdots\wtd\Phi_n(-u_n,-v_n)\Omega
\end{align}
Note that the requirement that $(u_1,v_1),\dots,(u_n,v_n)\in\mc W$ are spacelike means, after relabeling the subscripts, that
\begin{align*}
0<v_1<\cdots<v_n\qquad 0<-u_1<\cdots<-u_n
\end{align*}
\begin{proof}
This theorem relies on the following fact that we shall prove rigorously in the future:
\begin{itemize}
\item[$\varstar$] Let $T\geq0$ be a self-adjoint operator on $\mc H$ with $\Ker(T)=0$. Let $r>0$. Then $\xi\in\mc H$ belongs to $\Dom(T^r)$ iff the function $s\in\Rbb\mapsto T^{\im s}\xi\in\mc H$ can be extended to a continuous function $F$ on 
\begin{align*}
\{z\in\Cbb:-r\leq\Imag z\leq0\}
\end{align*}
and holomorphic on its interior. Moreover, for such $\xi$ we have $F(-\im r)=T^r\xi$.
\end{itemize}
In fact, the function $F(z)$ is given by $z\mapsto T^z\xi$.

We shall apply this result to $T=e^{-K}$ and $r=\pi$. For that purpose, we must show that the $\mc H$-valued function of $s\in\Rbb$ defined by
\begin{align*}
e^{\im\pi s}\wtd\Phi_1(u_1,v_1)\cdots\wtd\Phi_n(u_n,v_n)\Omega=\wtd\Phi_1(e^{-s}u_1,e^sv_1)\cdots\wtd\Phi_n(e^{-s}u_n,e^sv_n)\Omega
\end{align*}
can be extended to a continuous function on
\begin{align*}
\{z\in\Cbb:0\leq\Imag z\leq \pi\}
\end{align*}
and holomorphic on its interior. 

In fact, we can construct this $\mc H$-valued function, which is
\begin{align*}
z\mapsto \wtd\Phi_1(e^{-z}u_1,e^zv_1)\cdots\wtd\Phi_n(e^{-z}u_n,e^zv_n)\Omega
\end{align*}
noting that the conditions in Conc. \ref{lb4} are fulfilled. In particular, the condition $0<\Imag<\pi$ is used to ensure that, since $u_i<0,v_i>0$, we have $\Imag (e^{-z}u_i)>0$ and $\Imag(e^zv_i)>0$ as required by \eqref{eq9a}. The value of this function at $z=\im\pi$ equals the RHS of \eqref{eq12}. Therefore the theorem is proved.
\end{proof}




\subsection{}




\begin{thm}[\textbf{PCT theorem}] \index{PCT}\label{lb10}
We have an antiunitary map $\Theta:\mc H\rightarrow\mc H$, called the \textbf{PCT operator}, \index{00@PCT operator} such that
\begin{align}
\Theta\cdot \Phi_1(\xbf_1)\cdots\Phi_n(\xbf_n)\Omega=\Phi_1(-\xbf_1)^\dagger\cdots\Phi_n(-\xbf_n)^\dagger\Omega
\end{align}
for any $\Phi_1,\dots,\Phi_n\in\scr Q$ and mutually spacelike $\xbf_1,\dots,\xbf_n$.
\end{thm}


Equivalently, $\Theta$ is defined by
\begin{align}\label{eq13}
\Theta\cdot\wtd\Phi_1(u_1,v_1)\cdots\wtd\Phi_n(u_n,v_n)=\wtd\Phi_1(-u_1,-v_1)^\dagger\cdots \wtd\Phi_n(-u_n,-v_n)^\dagger\Omega
\end{align}

\begin{proof}
The existence of an antilinear isometry $\Theta$ satisfying \eqref{eq13} is equivalent to showing that (cf. \eqref{eq14})
\begin{align*}
\begin{aligned}
&\bk{\wtd\Phi_1(\ubf_1)\cdots\wtd\Phi_n(\ubf_n)\Omega|\wtd\Psi_1(\ubf_1')\cdots\wtd\Psi_k(\ubf_k')\Omega}\\
=&\bk{\wtd\Psi_1(-\ubf_1')^\dagger\cdots\wtd\Psi_k(-\ubf_k')^\dagger\Omega|\wtd\Phi_1(-\ubf_1)^\dagger\cdots\wtd\Phi_n(-\ubf_n)^\dagger\Omega}
\end{aligned}\tag{$\star$}\label{eq17}
\end{align*} 
if $\ubf_1,\dots\ubf_n$ are spacelike, and $\ubf_1',\dots\ubf_k'$ are spacelike. (We do not assume that, say, $\ubf_1$ and $\ubf_1'$ are spacelike.)

It suffices to prove this in the special case that $\ubf_1,\dots,\ubf_n,\ubf_1',\dots,\ubf_k'$ are mutually spacelike. Then the general case will follow that both sides of the above relation can be analytically continued to suitable regions as functions of $\ubf_1,\dots,\ubf_n$. For example, the fact that $H_0,H_1\geq0$ implies that
\begin{align*}
e^{\im\zeta H_0+\im\gamma H_1}\wtd\Phi_1(\ubf_1)\cdots\wtd\Phi_n(\ubf_n)\Omega=\wtd\Phi_1(\ubf_1+(\zeta,\gamma))\cdots\wtd\Phi_n(\ubf_n+(\zeta,\gamma))\Omega
\end{align*}
is continuous on $\{(\zeta,\gamma)\in\Cbb^2:\Imag\zeta\geq0,\Imag\gamma\geq0\}$ and holomorphic on its interior.

Set $\Gamma_j=\Psi_j^\dagger$. Then \eqref{eq17} is equivalent to
\begin{align*}
&\bk{\wtd\Phi_1(\ubf_1)\cdots\wtd\Phi_1(\ubf_n)\wtd\Gamma_1(\ubf_1')\cdots\wtd\Gamma_k(\ubf_k')\Omega|\Omega}\\
=&\bk{\wtd\Phi_1(-\ubf_1)\cdots\wtd\Phi_1(-\ubf_n)\wtd\Gamma_1(-\ubf_1')\cdots\wtd\Gamma_k(-\ubf_k')\Omega|\Omega}
\end{align*}
By the PT Thm. \ref{lb8}, this relation is equivalent to
\begin{align*}
&\bk{\wtd\Phi_1(\ubf_1)\cdots\wtd\Phi_1(\ubf_n)\wtd\Gamma_1(\ubf_1')\cdots\wtd\Gamma_k(\ubf_k')\Omega|\Omega}\\
=&\bk{e^{-\pi K}\wtd\Phi_1(\ubf_1)\cdots\wtd\Phi_1(\ubf_n)\wtd\Gamma_1(\ubf_1')\cdots\wtd\Gamma_k(\ubf_k')\Omega|\Omega}
\end{align*}
But this of course holds since $e^{-\pi K}\Omega=\Omega$ by Poincar\'e invariance.
\end{proof}



\subsection{}\label{lb11}

Combining the PT Thm. \ref{lb8} with the PCT Thm. \ref{lb10}, we conclude that $e^{-\pi K}$ is an injective positive operator, $\Theta$ is antinitary, and
\begin{subequations}\label{eq15}
\begin{align}
\Theta e^{-\pi K}A\Omega=A^\dagger\Omega
\end{align}
where $A$ is a product of spacelike separated field in $\mc W$. The rigorous statement should be that
\begin{align*}
A=\Phi_1(f_1)\cdots\Phi_n(f_n)
\end{align*}
where $\Phi_1,\dots,\Phi_n\in\scr Q$, and $f_i\in C_c^\infty(O_i)$ where $O_1,\dots,O_n\subset\mc W$ are open and mutually spacelike. If we let $\scr A(\mc W)$ be the $*$-algebra generated by all such $A$, then by the Poincar\'e invariance, for each $g\in\Poid$ we have
\begin{align*}
U(g)\scr A(\mc W)U(g)^{-1}=\scr A(g\mc W)
\end{align*}
In particular, since for the Lorentz boost $\Lambda$ we have $\Lambda(s)\mc W=\mc W$, we therefore have
\begin{align}
e^{\im sK}\scr A(\mc W)e^{-\im sK}=\scr A(\mc W)
\end{align}
for all $s\in\Rbb$. Since the PT transformation sends $\mc W$ to $-\mc W$, the definition of $\Theta$ clearly also implies
\begin{align}\label{eq15c}
\Theta\scr A(\mc W)\Theta^{-1}=\scr A(-\mc W)
\end{align}
Note that since $\mc W$ is local to $-\mc W$, we have $[\scr A(\mc W),\scr A(-\mc W)]=0$. Therefore, $\Theta\scr A(\mc W)\Theta$ is a subset of the (in some sense) commutant of $\scr A(\mc W)$.
\end{subequations}


\subsection{}



The set of formulas \eqref{eq15} is reminiscent of the Tomita-Takesaki theory, one of the deepest theories in the area of operator algebras. The setting is as follows. 

Let $\mc M$ be a von Neumann algebra on a Hilbert space $\mc H$. Namely, $\mc M$ is a $*$-subalgebra of $\End(\mc H)$ closed under the ``strong operator topology". (We will formally introduce von Neumann algebras in a later section.) Let $\Omega\in\mc H$ be a unit vector. Assume that $\Omega$ is \textbf{cyclic} (i.e. $\mc M\Omega$ is dense) and \textbf{separating} (i.e., if $x\in\mc M$ and $x\Omega=0$ then $x=0$) under $\mc M$. Then the \textbf{Tomita-Takesaki theorem} says that the linear map
\begin{align*}
S:\mc M\Omega\rightarrow\mc M\Omega\qquad x\Omega\mapsto x^*\Omega
\end{align*}
is antilinear and closable. Denote its closure also by $S$, and consider its polar decomposition $S=J\Delta^{\frac 12}$ where $\Delta$ is a positive closed operator, and $J$ is an antiunitary map. Then $\Delta$ is injective, we have $J^{-1}=J^*=J$, and
\begin{align*}
\Delta^{\im s}\mc M\Delta^{-\im s}=\mc M\qquad J\mc MJ=\mc M'
\end{align*}
where $\mc M'$ is the commutant $\{y\in\End(\mc H):xy=yx~(\forall x\in\mc M)\}$.We call $\Delta$ and $J$ respectively the \textbf{modular operator} and the \textbf{modular conjugation}.



\subsection{}



To relate the Tomita-Takesaki theory to QFT, one takes $\mc M$ to be $\fk A(\mc W)$, the von Neumann algebra generated by $\scr A(\mc W)$. Note that the elements of $\scr A(\mc W)$ are typically unbounded operators, whereas those of $\fk A(\mc W)$ are bounded. Thus, the meaning of ``the von Neumann algebra generated by a set of closed/closable operators" should be clarified. This is an important notion, and we will study it in a later section.


To apply the setting of Tomita-Takesaki, one should first show that the vacuum vector is cyclic and separating under $\fk A(\mc W)$. This is not an easy task, although it is relatively easier to show that $\Omega$ is cyclic and separating under $\scr A(\mc W)$. Moreover, we have
\begin{thm}[\textbf{Bisognano-Wichmann}]\index{00@Bisognano-Wichmann theorem}
Let $\Delta$ and $J$ be the modular operator and the modular conjugation of $(\fk A(\mc W),\Omega)$. Then $J=\Theta$ and $\Delta^{\frac 12}=e^{-\pi K}$.
\end{thm}
Since \eqref{eq15c} easily implies $\Theta\fk A(\mc W)\Theta^{-1}=\fk A(-\mc W)$, together with $J\mc M J^{-1}=\mc M'$ we obtain
\begin{align}
\fk A(\mc W)'=\fk A(-\mc W)
\end{align}
a version of \textbf{Haag duality}.\index{00@Haag duality}



One of the main goals of this course is to give a rigorous and self-contained proof of the PCT theorem, the Bisognano-Wichmann theorem, and the Haag duality for 2d chiral conformal field theories.




\subsection{}

For a general odd number $d>0$, the above results should be modified as follows. Let $K$ be the generator of the \textbf{Lorentz boost} \index{00@Lorentz boost}
\begin{align*}
\Lambda(s)=\left(
\begin{array}{c|c}
\begin{matrix}
\cosh s&\sinh s\\
\sinh s&\cosh s
\end{matrix}
&0\\
\hline
0&
\begin{matrix}
1&&\\
&\ddots&\\
&&1
\end{matrix}
\end{array}
\right)
\end{align*}
Let $\Lambda(\im\pi)=\diag(-1,-1,1,\dots,1)$, which does not belong to $\Poid$ since it reverses the time direction (although it has positive determinant). Then the PT Thm. \ref{lb8} should be modified by replacing \eqref{eq16} with 
\begin{align}
e^{-\pi K}\Phi_1(\xbf_1)\cdots\Phi_n(\xbf_n)\Omega=\Phi_1(\Lambda(\im\pi)\xbf_1)\cdots\Phi_n(\Lambda(\im\pi)\xbf_n)\Omega
\end{align} 

Let $\rho=\diag(1,1,-1,\dots,-1)$, which has determinant $1$ (since $d$ is odd) and hence belongs to $\SO^+(1,d)$. Then the PCT Thm. \ref{lb10} still holds verbatim. Let
\begin{align}
\mc W=\{(a_0,\dots,a_n)\in\Rbb^{1,d}:-a_1<a_0<a_1\}
\end{align}
Then the \textbf{Bisognano-Wichmann theorem} says that $e^{-\pi K}$ is the modular operator of $(\fk A(\mc W),\Omega)$, and $\Theta U(\rho)$ is the modular conjugation. 


We refer the readers to \cite[Sec. V.4.1]{Haag} and the reference therein for a detailed study.
 























 
\hypertarget{beforeindex}{}




\newpage

\printindex


\begin{thebibliography}{999999}
\footnotesize	



\bibitem[Gui-S]{Gui-S}
Gui, B. (2021). Spectral Theory for Strongly Commuting Normal Closed Operators. See \href{https://binguimath.github.io/}{https://binguimath.github.io/}


\bibitem[Haag]{Haag}
Haag, G. Local quantum physics. Fields, particles, algebras. 2nd., rev. and enlarged ed. Berlin: Springer-Verlag (1996)







\end{thebibliography}
\noindent {\small \sc Yau Mathematical Sciences Center, Tsinghua University, Beijing, China.}

\noindent {\textit{E-mail}}: binguimath@gmail.com







\end{document}




