\documentclass[11pt,b5paper,notitlepage]{article}
\usepackage[b5paper, margin={0.5in,0.65in}]{geometry}

\usepackage{amsmath,amscd,amssymb,amsthm,mathrsfs,amsfonts,layout,indentfirst,graphicx,caption,mathabx, stmaryrd,appendix,calc,imakeidx,upgreek,appendix} % mathabx for \widecheck
%\usepackage{ulem} %wave underline
\usepackage[dvipsnames]{xcolor}
\usepackage{palatino}  %template
\usepackage{slashed} % Dirac operator
\usepackage{mathrsfs} % Enable using \mathscr
%\usepackage{eufrak}  another template/font
\usepackage{extarrows} % long equal sign, \xlongequal{blablabla}
\usepackage{enumitem} % enumerate label change e.g. [label=(\alph*)]  shows (a) (b) 

\usepackage{fancyhdr} % date in footer
\usepackage{verbatim}
\usepackage{halloweenmath}
\usepackage{simpler-wick}





\usepackage{tikz-cd}
\usepackage[nottoc]{tocbibind}  
\title{Errata to ``Convergence of Sewing Conformal Blocks"}
\author{Bin Gui}
\date{October 2024}

\begin{document}\sloppy

\maketitle


\begin{enumerate}
\item In Thm 6.3, one should add the extra assumption that $\mathfrak X$ admits local coordinates $\eta_\bullet$. This condition ensures that $\mathscr W_{\mathfrak X}(\mathbb W_\bullet)\simeq \mathbb W_\bullet\otimes\mathscr O_{\mathcal B}$, and hence each stalk of $\mathscr W_{\mathfrak X}(\mathbb W_\bullet)$ is generated by the global sections of $\mathscr W_{\mathfrak X}(\mathbb W_\bullet)$.

\item In Rem. 10.3, it is not correct to say that $(\mathbb M\otimes\mathbb M'\otimes R)((\xi))[\log q]\{q\}$ is an $R((\xi))[\log q]\{q\}$-module. (Not every two elements of $\mathbb C\{q\}$ can be multiplied.) The correct way to say this as follows. First, for each vector space $W$, we define $W\{q\}$ to be the set of formal series $\sum_{n\in\mathbb C}w_nq^n$ where $w_n\in W$ and $w_n=0$ when $\Re(n)<<0$. (This lower truncation property was originally not assumed in the paper, but it is sufficient for the purpose of the paper.) Then $(\mathbb M\otimes\mathbb M'\otimes R)((\xi))[\log q]\{q\}$ is an $R((\xi))[\log q][[q]]$-module. Thus, its elements can be multiplied by $f(\xi,q/\xi)\in R((\xi))[[q]]$.


\item In Prop. 11.12, a factor $\frac 1{2\mathrm{i}\pi}$ is missing in the contour integrals defining $A$ and $B$. The same can be said about Eq. (13.8).
\end{enumerate}


\end{document}
